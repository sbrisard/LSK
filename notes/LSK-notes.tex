\newcommand{\sbtitle}{Notes relatives à la méthode asymptotique de Lyapunov–Schmidt–Koiter}
\newcommand{\sbauthor}{Sébastien Brisard}
\newcommand{\sbemail}{sebastien.brisard@univ-eiffel.fr}
\newcommand{\sbaddress}{Univ Gustave Eiffel, Ecole des Ponts, IFSTTAR, CNRS, Navier, F-77454 Marne-la-Vall\'ee, France}
\newcommand{\sbsubject}{Note bibliographique}

\documentclass[12pt, final]{scrartcl}
\setkomafont{disposition}{\rmfamily}

\usepackage{polyglossia}
\setdefaultlanguage{french}

\usepackage{amsfonts}
\usepackage{amsmath}
\usepackage{amssymb}

\usepackage{amsthm}
\theoremstyle{definition}
\renewcommand{\qedsymbol}{}
\newtheorem{remark}{Remarque}
\newtheorem{theorem}{Théorème}

\usepackage[backend=biber,bibencoding=utf8,doi=false,giveninits=true,isbn=false,maxnames=10,minnames=5,sortcites=true,style=authoryear,texencoding=utf8,url=false]{biblatex}
\addbibresource{stab.bib}

\usepackage[breaklinks=true, colorlinks=true, pdftitle={\sbtitle}, pdfauthor={\sbauthor}, pdfsubject={\sbsubject}, urlcolor=blue]{hyperref}

\usepackage[color={1 1 0}]{pdfcomment}

\usepackage{unicode-math}
% \setmainfont{XITS}
% \setmathfont{XITS Math}
\setmainfont{Asana Math}
\setmathfont{Asana Math}

\newcommand{\reals}{\mathbb{R}}

\begin{document}
\title{\sbtitle}
\author{\sbauthor\thanks{\sbaddress~--- \sbemail}}
\maketitle

\begin{abstract}
 blabla
\end{abstract}


\section{Notations}

L'espace des champs cinématiquement admissibles est noté \(U\). On suppose qu'il
a la structure d'espace vectoriel. L'énergie du système est notée \(ℰ(u, λ)\),
où \(λ\) désigne un paramètre de chargement. Soit \(u^{\ast}(λ)\) la branche
fondamentale. Par définition
\begin{equation}
  ℰ_{,u}[u^{\ast}(λ), λ; \hat{u}]=0 \quad \text{pour tout} \quad \hat{u}∈U.
\end{equation}
Il sera commode d'introduire les notations suivantes
\begin{equation}
  ℰ₂(λ) = ℰ_{,uu}[u^{\ast}(λ), λ], \quad ℰ₃(λ) = ℰ_{,uuu}[u^{\ast}(λ), λ], \quad ℰ₄(λ) = ℰ_{,uuuu} [u^{\ast}(λ), λ].
\end{equation}
Noter que \(ℰ₂\), \(ℰ₃\) et \(ℰ₄\) sont des formes bi-, tri- et
quadri-linéaires, respectivement. L'application de ces formes à des éléments de
\(U\) sera notée \(ℰ₂(λ; u, v)\), \(ℰ₃(λ; u, v, w)\), etc. La dérivée de ces
formes par rapport à \(λ\) sera notée à l'aide d'un point supérieur
(\(\dot{ℰ}_2\), \(\dot{ℰ}_3\), \dots).

On suppose que l'équilibre est stable pour des valeurs suffisamment petites de
\(λ\). Plus précisément, on suppose que \(ℰ₂(λ)\) est définie positive pour tout
\(λ < λ₀\). Pour \(λ = λ₀\), la forme quadratique \(ℰ₂(λ₀)\) n'est plus que
positive. En notant \(u₀ = u^{\ast}(λ₀)\) la position d'équilibre obtenue pour
la valeur critique \(λ₀\) du paramètre de chargement \(λ\), on s'intéresse à
toutes les courbes d'équilibre qui passent par le point \((u₀, λ₀)\).

Noter que dans ce qui suit, on convient que les formes \(ℰ₂\), \(ℰ₃\) et \(ℰ₄\)
sont implicitement évaluées en \(λ₀\) lorsque \(λ\) n'est pas rappelé : ainsi,
on notera \(ℰ₂(•, •)\) plutôt que \(ℰ₂(λ₀ ; •, •)\).

Par hypothèse, \(ℰ₂(λ₀)\) est positive, sans être définie
positive~; soit \(V\) son noyau, qui forme un sous-espace vectoriel de \(U\). On
suppose que \(V\) est de dimension finie \(m = \dim V\). Soit
\((v₁, \ldots, v_m)\) une base orthonormée de ce noyau pour le produit scalaire
\(〈 •, • 〉\)(qui n'est pas précisé pour le moment). On introduit le
sous-espace supplémentaire orthogonal \(W\) de \(V\) dans \(U\)
\begin{equation}
  U = V \overset{\perp}{\otimes} W.
\end{equation}

The bilinear form \(ℰ₂\) being elliptic over \(W\), variational problems of the
type: find \(w ∈ W\) such that
\begin{equation}
  ℰ₂(w, \hat{w})+\ell(\hat{w}) = 0 \quad \text{for all} \quad \hat{w}∈W
\end{equation}
are well-posed for any linear form \(\ell\) over \(W\). For \(1 ≤ i, j ≤ m\), we
introduce the solutions \(w_i, w_{ij} ∈ W\) to the following variational
problems
\begin{gather}
  \label{eq:20220524134525}
  ℰ₂(λ₀; w_i, \hat{w}) + 2\dot{ℰ}₂(λ₀; v_i, \hat{w}) = 0,\\
  \label{eq:20220519164523}
  ℰ₂(λ₀; w_{i j}, \hat{w})+ℰ₃(λ₀; v_i, v_j, \hat{w}) = 0,
\end{gather}
for all \(\hat{w} ∈ W\). Since \(w_{i}\) and \(w_{ij}\) belong to \(W\), we have
\(〈 w_{i}, v 〉 = 〈 w_{ij}, v 〉 = 0\) for all \(v ∈ V\). Since \(ℰ₂(λ₀; •,
•)\) is symmetric, it can be verified that \(w_{ij}=w_{ji}\). We also introduce
the following tensors, defined in \(V\)
\begin{gather}
  E_{ijk} = ℰ₃(λ₀; v_i, v_j, v_k) + ℰ₂(λ₀; v_i, w_{jk}) + ℰ₂(λ₀; v_j, w_{ki}) + ℰ₂(λ₀; v_k, w_{ij}),\\
  E_{ijkl} = ℰ₄(λ₀ ; v_i, v_j, v_k, v_l) + ℰ₃(λ₀ ; v_i, v_j, w_{kl}) + ℰ₃(λ₀ ; v_i, v_k, w_{lj}) + ℰ₃(λ₀ ; v_i, v_l, w_{jk}),\\
  F_{ij} = \dot{ℰ}₂(λ₀; v_i, v_j) + \tfrac{1}{2} \bigl[ℰ₂(λ₀; v_i, w_j) + ℰ₂(λ₀; v_j, w_i)\bigr],
\end{gather}
as well as the derivatives
\begin{gather}
  \label{eq:20220615063626}
  \mathring{E}_{ijk} = \dot{ℰ}₃(λ₀; v_i, v_j, v_k) + \dot{ℰ₂}(λ₀; v_i, w_{jk}) + \dot{ℰ}₂(λ₀; v_j, w_{ki}) + \dot{ℰ}₂(λ₀; v_k, w_{ij}),\\
  \label{eq:20220615063633}
  \mathring{F}_{ij} = \ddot{ℰ}₂(λ₀; v_i, v_j) + \tfrac{1}{2} \bigl[\dot{ℰ}₂(λ₀; v_i, w_j) + \dot{ℰ}₂(λ₀; v_j, w_i)\bigr].
\end{gather}

Note that, since \(ℰ₂(λ₀; v_i, •) = 0\), the above expressions simplify as follows
\begin{gather}
  \label{eq:20220524135619}
  E_{ijk} = ℰ₃(λ₀; v_i, v_j, v_k),\\
  \label{eq:20220524135553}
  E_{ijkl} = ℰ₄(λ₀ ; v_i, v_j, v_k, v_l) + ℰ₃(λ₀ ; v_i, v_j, w_{kl}) + ℰ₃(λ₀ ; v_i, v_k, w_{jl}) + ℰ₃(λ₀ ; v_i, v_l, w_{jk}),\\
  \label{eq:20220524135643}
  F_{ij} = \dot{ℰ}₂(λ₀; v_i, v_j).
\end{gather}

The tensors \(E_{ijk}\), \(F_{ij}\), \(\mathring{E}_{ijk}\) and
\(\mathring{F}_{ij}\) are fully symmetric. Furthermore, the following expression
of \(E_{ijkl}\) result from Eq.~\eqref{eq:20220519164523}
\begin{equation}
  E_{ijkl} = ℰ₄(λ₀ ; v_i, v_j, v_k, v_l) - ℰ₂(λ₀ ; w_{ij}, w_{kl}) - ℰ₂(λ₀ ; w_{ik}, w_{jl}) - ℰ₂(λ₀ ; w_{il}, w_{jk}),
\end{equation}
which shows that \(E_{ijkl}\) is also fully symmetric. To close this section,
the following useful identities are derived from applications of
Eqs.~\eqref{eq:20220524134525} and \eqref{eq:20220519164523}
\begin{align}
  \notag
  \mathring{F}_{ij} ={} & \ddot{ℰ}₂(λ₀; v_i, v_j) + \tfrac{1}{2}\bigl[\dot{ℰ}₂(λ₀; v_i, w_j) + \dot{ℰ}₂(λ₀; w_i, v_j)\bigr] = \ddot{ℰ}₂(λ₀; v_i, v_j) + \dot{ℰ}₂(λ₀; v_j, w_i)\\
  \label{eq:20220617084433}
  ={} & \ddot{ℰ}₂(λ₀; v_i, v_j) -\tfrac{1}{2}ℰ₂(λ₀; w_i, w_j) = \ddot{ℰ}₂(λ₀; v_i, v_j) + \dot{ℰ}₂(λ₀; v_i, w_j),\\
  \notag
  \mathring{E}_{ijk} ={}& \dot{ℰ}₃(λ₀; v_i, v_j, v_k) + \dot{ℰ}₂(λ₀; v_i, w_{jk}) + \dot{ℰ}₂(λ₀; v_j, w_{ik}) + \dot{ℰ}₂(λ₀; v_k, w_{ij})\\
  \label{eq:20220617085256}
  ={}& \dot{ℰ}₃(λ₀; v_i, v_j, v_k) -\tfrac{1}{2}\bigl[ℰ₂(λ₀; w_i, w_{jk}) + ℰ₂(λ₀; w_j, w_{ik}) + ℰ₂(λ₀; w_k, w_{ij})\bigr].
\end{align}

\section{Analyse de la branche fondamentale}

On s'intéresse dans ce paragraphe à la stabilité du point critique \((u₀,
λ₀)\). À cet effet, on calcule l'énergie dans un état \(u₀ + ξ v + η w\) voisin
du point d'équilibre \(u₀\), avec \(ξ, η∈\reals\) {\guillemotleft} petits
{\guillemotright}, \(v \in V\) et \(w∈W\). On obtient alors, à l'ordre 4 en
\(ξ\) et \(η\)
\begin{equation}
  \begin{aligned}[b]
    Δℰ ={} &  ℰ(u₀ + ξv + ηw, λ₀) - ℰ(u₀, λ₀)\\
    ={} & \tfrac{1}{2} ℰ₂(ξv + ηw, ξv + ηw) + \tfrac{1}{6} ℰ₃(ξv + ηw, ξv + ηw, ξv + ηw)\\
    & + \tfrac{1}{24} ℰ₄(ξv + ηw, ξv + ηw, ξv + ηw, ξv + ηw) + \mathcal{O}\bigl[\bigl(ξ² + η²\bigr)²\bigr],
  \end{aligned}
\end{equation}
où le terme linéaire a été omis puisque \(u₀\) est un point critique de
l'énergie. En tenant compte de la multilinéarité et de la symétrie des
différentielles successives de l'énergie \(ℰ\), ainsi que du fait que
\(ℰ₂(v, •) = 0\) (puisque \(v∈V\)), l'expression précédente s'écrit
\begin{equation}
  \begin{aligned}[b]
    \Delta ℰ ={} & \tfrac{1}{2} η² ℰ₂(w, w) + \tfrac{1}{6} ξ³ ℰ₃(v, v, v) + \tfrac{1}{2} ξ² η ℰ₃(v, v, w)\\
    & + \tfrac{1}{2} ξ η² ℰ₃(v, w, w) + \tfrac{1}{6} η³ ℰ₃(w, w, w)\\
    & + \tfrac{1}{24} ξ⁴ ℰ₄(v, v, v, v) + \tfrac{1}{6} ξ³ η ℰ₄(v, v, v, w)\\
    & + \tfrac{1}{4} ξ² η² ℰ₄(v, v, w, w) + \tfrac{1}{6} ξ η³ ℰ₄(v, w, w, w)\\
    & + \tfrac{1}{24} η⁴ ℰ₄(w, w, w, w) +\mathcal{O}\bigl[\bigl(ξ² + η²\bigr)²\bigr],
  \end{aligned}
\end{equation}
où l'on convient que toutes les différentielles de \(ℰ\) sont évaluées au point
d'équilibre \(u₀\).

Pour que l'équilibre soit stable, il faut que cette expression soit positive ou
nulle pour tous \(ξ\) et \(η\) suffisamment petits. En prenant tout d'abord
\(η = 0\), on obtient les conditions nécessaires
\begin{equation}
  \label{eq:20211108164416}
  ℰ₃(v, v, v) = 0 \quad \text{et} \quad ℰ₄(v, v, v, v) \geq 0 \quad \text{pour tout} \quad v∈V.
\end{equation}

En d'autres termes, s'il existe \(v∈V\) tel que \(ℰ₃(v, v, v) \neq 0\) ou
\(ℰ₄(v, v, v, v) < 0\), alors l'équilibre est \emph{instable}. Les conditions
précédentes ne sont pas suffisantes pour assurer la stabilité. En effet,
supposant ces conditions remplies, on prend maintenant \(η = ξ²\)
\begin{equation}
  Δℰ = \tfrac{1}{2} ξ⁴ \bigl[ ℰ₂(w, w) + ℰ₃(v, v, w) + \tfrac{1}{12} ℰ₄(v, v, v, v) \bigr] + o(ξ⁴)
\end{equation}
et on obtient la condition nécessaire supplémentaire
\begin{equation}
  \label{eq:20211109145356}
  ℰ₂(w, w) + ℰ₃(v, v, w) + \tfrac{1}{12} ℰ₄(v, v, v, v) \geq 0,
\end{equation}
pour tous \(v∈V\) et \(w∈W\). Pour \(v∈V\) fixé, l'expression précédente est
minimale lorsque \(w\) satisfait le problème variationnel
\begin{equation}
  \label{eq:20211109145224}
  2ℰ₂(w, \hat{w}) +ℰ₃(v, v, \hat{w}) = 0 \quad \text{pour tout} \quad \hat{w}∈W.
\end{equation}

Alors, pour \(v = ξ^i v_i\), la solution du problème
variationnel~\eqref{eq:20211109145224} est \(w = \tfrac{1}{2} ξ^i ξ^j w_{ij}\)
où \(w_{ij}\) désigne la solution du problème variationnel
\eqref{eq:20220519164523}. Pour cette valeur de \(v\), la
condition~\eqref{eq:20211109145356} s'écrit
\begin{equation}
  \bigl[ℰ₄(v_i, v_j, v_k, v_l) - 3ℰ₂(w_{ij}, w_{kl})\bigr] ξ^i ξ^j ξ^k ξ^l \geq 0,
\end{equation}
pour tous \(ξ^i, ξ^j, ξ^k, ξ^l∈\reals\). On peut montrer que l'inégalité stricte
est une condition \emph{suffisante} de
stabilité. \pdfmargincomment{Est-ce vrai ? Essayer de préciser}

\section{Bifurcations}
\label{sec:20220617075558}

On écrit toute courbe d'équilibre passant par le point \((u₀, λ₀)\) sous la
forme paramétrique suivante
\begin{align}
  \label{eq:20211115075817}
  λ &=  λ₀ + η λ₁ + \tfrac{1}{2} η² λ₂ + \tfrac{1}{6} η³ λ₃ + \cdots,\\
  \label{eq:20211115075835}
  u &= u^{\ast}(λ) + η u₁ + \tfrac{1}{2} η² u₂ + \tfrac{1}{6} η³ u₃ + \cdots,
\end{align}
où \(η\) est un paramètre, non précisé pour le moment. Noter que, dans la
représentation paramétrique de \(u\), \(u^{\ast}\) est évalué en \(λ\) et pas en
\(λ₀\).

Les coefficients \(λ_k\) et \(u_k\) des développements~\eqref{eq:20211115075817}
et \eqref{eq:20211115075835} sont identifiés en écrivant que l'énergie est
stationnaire le long de la courbe d'équilibre, c'est-à-dire que le résidu
\(ℰ_{, u} [u(η), λ(η)]\) est nul. Le développement limité du résidu est établi
au voisinage de \(η = 0\) dans l'annexe~\ref{sec:20211112182000} [voir
Éq.~\eqref{eq:20220107080901}]. En écrivant que tous ses termes s'annulent, on
trouve successivement, pour tout \(\hat{u}∈U\)
\begin{equation}
  \label{eq:20211112182917}
  ℰ₂(λ₀; u₁, \hat{u}) = 0,
\end{equation}
\begin{equation}
  \label{eq:20220524133447}
  ℰ₃(λ₀; u₁, u₁, \hat{u}) + 2λ₁\dot{ℰ}₂(λ₀; u₁, \hat{u}) + ℰ₂(λ₀; u₂, \hat{u}) = 0,
\end{equation}
\begin{equation}
  \label{eq:20220708060436}
  \begin{aligned}[b]
    ℰ₄(λ₀; u₁, u₁, u₁, \hat{u}) + 3ℰ₃(λ₀; u₁, u₂, \hat{u}) + ℰ₂(λ₀; u₃, \hat{u})&\\
    + 3λ₁\dot{ℰ}₃(λ₀; u₁, u₁, \hat{u}) + 3λ₁\dot{ℰ}₂(λ₀;  u₂, \hat{u})&\\
    + 3λ₁²\ddot{ℰ}₂(λ₀; u₁, \hat{u}) + 3λ₂\dot{ℰ}₂(λ₀; u₁, \hat{u}) & = 0.
  \end{aligned}
\end{equation}
On déduit de l'équation~\eqref{eq:20211112182917} que \(u₁∈V\). En prenant la
fonction test également dans \(V\), on déduit de
l'équation~\eqref{eq:20220524133447} que \(u₁\) est solution du problème
suivant~: trouver \(u₁∈V\) tel que
\begin{equation}
  \label{eq:20220524133816}
  \tfrac{1}{2} ℰ₃(λ₀; u₁, u₁, \hat{v}) + λ₁\dot{ℰ}₂(λ₀; u₁, \hat{v}) = 0,
\end{equation}
pour tout \(\hat{v}∈V\). The above problem can be transformed into a system of
scalar equations. Indeed, expanding the \(u₁∈V\) in the basis
\((v_i)_{1 ≤ i ≤ m}\) as follows
\begin{equation}
  \label{eq:20220524133944}
  u₁ = ξ₁^i v_i
\end{equation}
and plugging the definitions~\eqref{eq:20220524135619} and
\eqref{eq:20220524135643} of \(E_{ijk}\) and \(F_{ij}\) into
Eq.~\eqref{eq:20220524133816}
\begin{equation}
  \label{eq:20220524135036}
  \tfrac{1}{2} E_{ijk} ξ₁^j ξ₁^k + λ₁ F_{ij} ξ₁^j = 0.
\end{equation}

On obtient ainsi un système de \(m\) équations quadratiques à \((m + 1)\)
inconnues, qui permet en général de déterminer les valeurs de \(λ₁\) et
\(u₁\)(\pdfmarkupcomment{voir discussion ci-après}{Compléter référence}).

Afin de déterminer les termes suivants du développement asymptotique de la
branche bifurquée, soit \(λ₂\) et \(u₂\), on introduit la décomposition
\begin{equation}
  u₂ = ξ₂^i v_i + \tilde{u}₂,
\end{equation}
où \(\tilde{u}₂∈W\) est la projection orthogonale de \(u₂\) sur \(W\). On a
alors \(ℰ₂(u₂, \hat{u}) =ℰ₂(\tilde{u}₂, \hat{u})\) et
l'équation~\eqref{eq:20220524133447} s'écrit
\begin{equation}
 ℰ₃(λ₀; u₁, u₁, \hat{u}) + 2λ₁ \dot{ℰ}₂(λ₀; u₁, \hat{u}) + ℰ₂(λ₀; \tilde{u}₂, \hat{u}) = 0,
\end{equation}
pour tout \(\hat{u}∈U\). En prenant cette fois-ci la fonction test dans l'espace
\(W\), on obtient le problème variationnel suivant~: trouver \(\tilde{u}₂∈W\)
tel que
\begin{equation}
  \label{eq:20211210131623}
  ℰ₂(λ₀; \tilde{u}₂, \hat{w}) + ξ₁^i ξ₁^j ℰ₃(λ₀; v_i, v_j, \hat{w}) + 2λ₁ ξ₁^i \dot{ℰ}₂(λ₀; v_i, \hat{w}) = 0,
\end{equation}
pour tout \(\hat{w}∈W\). The solution to the variational
problem~\eqref{eq:20211210131623} is expressed as a linear combination of the
\(w_i\) and \(w_{ij}\) [defined by the variational
problems~\eqref{eq:20220524134525} and \eqref{eq:20220519164523}]: \(\tilde{u}₂ = ξ₁^i ξ₁^j w_{ij} + λ₁ ξ₁^i w_i\) and
\begin{equation}
  \label{eq:20220524134613}
  u₂ = ξ₂^i v_i + ξ₁^i ξ₁^j w_{ij} + λ₁ ξ₁^i w_i.
\end{equation}

Plugging expressions~\eqref{eq:20220524133944} and \eqref{eq:20220524134613}
into Eq.~\eqref{eq:20220708060436} and taking further \(\hat{u} = v_i\)
[remember that \(ℰ₂(λ₀; v_i, •) = 0\)], we then get
% \begin{multline*}
%   ℰ₄(λ₀; v_i, ξ₁^j v_j, ξ₁^k v_k, ξ₁^l v_l) + 3ℰ₃(λ₀; v_i, ξ₁^j v_j, ξ₂^k v_k + ξ₁^k ξ₁^l w_{kl} + λ₁ ξ₁^k w_k)\\
%   + 3λ₁ \dot{ℰ}₃(λ₀; v_i, ξ₁^j v_j, ξ₁^k v_k) + 3λ₁ \dot{ℰ}₂(λ₀; v_i, ξ₂^j v_j + ξ₁^j ξ₁^k w_{jk} + λ₁ ξ₁^j w_j)\\
%   + 3λ₁² \ddot{ℰ}₂(λ₀; v_i, ξ₁^j v_j) + 3λ₂ \dot{ℰ}₂(λ₀; v_i, ξ₁^j v_j) = 0
% \end{multline*}
% \begin{multline*}
%   ℰ₄(λ₀; v_i, v_j, v_k, v_l) ξ₁^j ξ₁^k ξ₁^l + 3ℰ₃(λ₀; v_i, v_j, v_k) ξ₁^j ξ₂^k + 3ℰ₃(λ₀; v_i, v_j, w_{kl}) ξ₁^j ξ₁^k ξ₁^l\\
%   + 3λ₁ ℰ₃(λ₀; v_i, v_j, w_k) ξ₁^j ξ₁^k + 3λ₁ \dot{ℰ}₃(λ₀; v_i, v_j, v_k) ξ₁^j ξ₁^k + 3λ₁ \dot{ℰ}₂(λ₀; v_i, v_j) ξ₂^j\\
%   + 3λ₁ \dot{ℰ}₂(λ₀; v_i, w_{jk}) ξ₁^j ξ₁^k + 3λ₁² \dot{ℰ}₂(λ₀; v_i, w_j) ξ₁^j + 3λ₁² \ddot{ℰ}₂(λ₀; v_i, v_j) ξ₁^j + 3λ₂ \dot{ℰ}₂(λ₀; v_i, v_j) ξ₁^j = 0
% \end{multline*}
\begin{multline*}
  \bigl[ℰ₄(λ₀; v_i, v_j, v_k, v_l) + 3ℰ₃(λ₀; v_i, v_j, w_{kl})\bigr] ξ₁^j ξ₁^k ξ₁^l\\
  + 3λ₁ \bigl[ℰ₃(λ₀; v_i, v_j, w_k) + \dot{ℰ}₃(λ₀; v_i, v_j, v_k) + \dot{ℰ}₂(λ₀; v_i, w_{jk}) \bigr] ξ₁^j ξ₁^k\\
  + 3\bigl[λ₁² \dot{ℰ}₂(λ₀; v_i, w_j) + λ₁² \ddot{ℰ}₂(λ₀; v_i, v_j) + λ₂ \dot{ℰ}₂(λ₀; v_i, v_j)\bigr] ξ₁^j\\
  + 3\bigl[ℰ₃(λ₀; v_i, v_j, v_k) ξ₁^k + λ₁ \dot{ℰ}₂(λ₀; v_i, v_j)\bigr] ξ₂^j = 0
\end{multline*}
It results from the variational problems \eqref{eq:20220524134525} and
\eqref{eq:20220519164523} that
\begin{equation*}
  ℰ₃(λ₀; v_i, v_j, w_k) = -ℰ₂(λ₀ ; w_{ij}, w_k) = 2\dot{ℰ}₂(λ₀; v_k, w_{ij}),
\end{equation*}
therefore
\begin{equation*}
  \begin{aligned}[b]
    ℰ₃(λ₀; v_i, v_j, w_k) ξ₁^j ξ₁^k &= \tfrac{1}{2} \bigl[ ℰ₃(λ₀; v_i, v_j, w_k) + ℰ₃(λ₀; v_i, v_k, w_j)\bigr] ξ₁^j ξ₁^k\\
                                    &= \bigl[ \dot{ℰ}₂(λ₀; v_k, w_{ij}) + \dot{ℰ}₂(λ₀; v_j, w_{ik}) \bigr] ξ₁^j ξ₁^k.
  \end{aligned}
\end{equation*}
Similarly,
\begin{equation*}
  \begin{aligned}[b]
    \dot{ℰ}₂(λ₀; v_i, w_j) &= - \tfrac{1}{2} ℰ₂(λ₀; w_i, w_j) = - \tfrac{1}{2} ℰ₂(λ₀; w_j, w_i) = \dot{ℰ}₂(λ₀; v_j, w_i)\\
                           &= \tfrac{1}{2} \bigl[ \dot{ℰ}₂(λ₀; v_i, w_j) + \dot{ℰ}₂(λ₀; v_j, w_i) \bigr].
  \end{aligned}
\end{equation*}
% \begin{multline*}
%   \bigl[ℰ₄(λ₀; v_i, v_j, v_k, v_l) + ℰ₃(λ₀; v_i, v_j, w_{kl}) + ℰ₃(λ₀; v_i, v_k, w_{jl}) + ℰ₃(λ₀; v_i, v_l, w_{jk}) \bigr] ξ₁^j ξ₁^k ξ₁^l\\
%   + 3λ₁ \bigl[\dot{ℰ}₃(λ₀; v_i, v_j, v_k) + \dot{ℰ}₂(λ₀; v_i, w_{jk}) + \dot{ℰ}₂(λ₀; v_j, w_{ik}) + \dot{ℰ}₂(λ₀; v_k, w_{ij}) \bigr] ξ₁^j ξ₁^k\\
%   + 3λ₁² \bigl\{ \ddot{ℰ}₂(λ₀; v_i, v_j) + \tfrac{1}{2} \bigl[ \dot{ℰ}₂(λ₀; v_i, w_j) + \dot{ℰ}₂(λ₀; v_j, w_i) \bigr] \bigr\} ξ₁^j\\
%   + 3\bigl[ℰ₃(λ₀; v_i, v_j, v_k) ξ₁^k + λ₁ \dot{ℰ}₂(λ₀; v_i, v_j)\bigr] ξ₂^j + 3λ₂ \dot{ℰ}₂(λ₀; v_i, v_j) ξ₁^j = 0
% \end{multline*}

Finally, the definitions \eqref{eq:20220615063626}, \eqref{eq:20220615063633},
\eqref{eq:20220524135619}, \eqref{eq:20220524135553} and
\eqref{eq:20220524135643} of \(E_{ijk}\), \(E_{ijkl}\), \(F_{ij}\),
\(\mathring{E}_{ijk}\) and \(\mathring{F}_{ij}\) lead to the following compact
bifurcation equation
\begin{equation}
  \label{eq:20220601070917}
  \tfrac{1}{3} E_{ijkl} ξ₁^j ξ₁^k ξ₁^l + λ₁ \bigl( \mathring{E}_{ijk} ξ₁^k + λ₁ \mathring{F}_{ij} \bigr)ξ₁^j + \bigl(E_{ijk} ξ₁^k + λ₁ F_{ij}\bigr) ξ₂^j + λ₂ F_{ij} ξ₁^j = 0.
\end{equation}

In order to analyse the stability of the bifurcated branches thus found, one
must look at the Hessian of the energy. It is first observed that, on the
fundamental branch
\begin{equation}
 ℰ₂(λ; \hat{u}, \hat{v}) = ℰ₂(λ₀; \hat{u}, \hat{v}) + \bigl(λ - λ₀\bigr) \dot{ℰ}₂(λ₀; \hat{u}, \hat{v}) + o(λ - λ₀).
\end{equation}

In what follows, it will be assumed that \(\dot{ℰ}₂(λ₀)≠0\) and that \(ℰ₂(λ)\)
(which is positive definite over \(V\) for \(λ<λ₀\) and null for \(λ=λ₀\)) is
negative definite for \(λ>λ₀\) sufficiently small (the fundamental branch is
strictly unstable beyond the critical load). From the above expansion, it
results that \(\dot{ℰ}₂(λ₀)\) is negative definite over \(V\). In other words,
\(-F_{ij}\) is a positive definite tensor. The asymptotic expansion of the
Hessian of the energy along the bifurcated branch is derived in
appendix~\ref{sec:20220616055207}. For all \(\hat{u}, \hat{v}∈U\)
\begin{multline}
  \label{eq:20220531054247}
  ℰ_{, uu}[u(η), λ(η); \hat{u}, \hat{v}] = ℰ₂(λ₀ ; \hat{u}, \hat{v}) + η \bigl[ℰ₃(λ₀ ; u₁, \hat{u}, \hat{v})  + λ₁ \dot{ℰ}₂(λ₀; \hat{u}, \hat{v})\bigr]\\
  + \tfrac{1}{2} η² \bigl[ℰ₄(λ₀; u₁, u₁, \hat{u}, \hat{v}) + ℰ₃(λ₀; u₂, \hat{u}, \hat{v}) + 2λ₁ \dot{ℰ}₃(λ₀; u₁, \hat{u}, \hat{v})\\
  + λ₁² \ddot{ℰ}₂(λ₀; \hat{u}, \hat{v}) + λ₂ \dot{ℰ}₂(λ₀; \hat{u}, \hat{v}) \bigr] + o(η²).
\end{multline}

Stability analysis is performed by means of the eigenvalues \(α ∈ \reals\) and
eigenvectors \(x ∈ U\) of the Hessian
\begin{equation}
  \label{eq:20220617074949}
  ℰ_{, u u} [u(η), λ(η); x, \hat{u}] = α 〈 x, \hat{u} 〉 \quad \text{for all} \quad \hat{u} ∈ V,
\end{equation}
where \(α\) and \(x\) are expanded to second order in \(η\)
\begin{equation}
  \label{eq:20220617064633}
  α = α₀ + η α₁ + \tfrac{1}{2} η² α₂ + o(η²)
  \quad \text{and} \quad
  x = x₀ + η x₁ + \tfrac{1}{2} η² x₂ + o(η²).
\end{equation}

The following results are proved in Appendix~\ref{sec:20220616074108}: first,
\((α₀, x_0)\) is necessarily an eigenpair of \(ℰ₂(λ₀)\). Since \(ℰ₂ (λ₀)\) is
positive, \(α₀ ≥ 0\). If \(α₀>0\), then \(α>0\) in the neighborhood of
\(λ₀\). Potentially unstable modes are therefore such that \(α₀=0\). In other
words, \(x₀ ∈ V\); furthermore, \((α₁, χ₀^i)\) is an eigenpair of the symmetric
tensor \((E_{ijk} ξ₁^k + λ₁ F_{ij})\)
\begin{equation}
  \label{eq:20220609133608}
  x₀ = χ₀^i v_i
  \quad \text{and} \quad
  \bigl(E_{ijk} ξ₁^k + λ₁ F_{ij} \bigr) χ₀^j = α₁ χ₀^i.
\end{equation}
As for the higher order terms, it is also found that
\begin{equation}
  \label{eq:20220609133629}
  x₁ = χ₁^i v_i +  χ₀^i ξ₁^j w_{i j} + \tfrac{1}{2} λ₁ χ₀^i w_i
\end{equation}
and
\begin{multline}
  \label{eq:20220616082923}
  \bigl[E_{ijkl} ξ₁^k ξ₁^l + λ₁\bigl(2 \mathring{E}_{ijk} ξ₁^k + λ₁ \mathring{F}_{ij}\bigr) + E_{ijk} ξ₂^k + λ₂ F_{ij} \bigr] χ₀^j\\
  + 2\bigl(E_{ijk}  ξ₁^k + λ₁ F_{ij} \bigr) χ₁^j = 2α₁χ₁^i + α₂ χ₀^i.
\end{multline}

Finally, to close this analysis of the bifurcated branches, the following
asymptotic expansion of the energy is derived in
Appendix~\ref{sec:20220525053434}
\begin{multline}
  \label{eq:20220525053600}
    ℰ[u(η), λ(η)] = ℰ\{u^{\ast}[λ(η)], λ(η)\} + \tfrac{1}{6} λ₁ η³ F_{i j} ξ₁^i ξ₁^j\\
    +\tfrac{1}{24} η⁴ \bigl[E_{ijkl} ξ₁^i ξ₁^j ξ₁^k ξ₁^l + 4λ₁ \mathring{E}_{ijk} ξ₁^i ξ₁^j ξ₁^k + 6 \bigl(λ₁² \mathring{F}_{ij} + λ₂ F_{ij}\bigr) ξ₁^i ξ₁^j\bigr] + o(η⁴).
\end{multline}
Si \(λ₁ ≠ 0\), le premier terme non-nul du développement limité précédent est
d'ordre 3
\begin{equation}
 ℰ [u(η), λ(η)] = ℰ\{u^{\ast}[λ(η)], λ(η)\} - \tfrac{1}{6} λ₁ η³ G_{ij} ξ₁^i ξ₁^j + o(η³),
\end{equation}
tandis que si \(λ₁ = 0\), le premier terme est d'ordre 4 \pdfcomment{07/06/2022
  — Vérifier le coefficient de η⁴.}
\begin{equation}
 ℰ[u(η), λ(η)] = ℰ\{u^{\ast} [λ(η)], λ(η)\} - \tfrac{1}{8} λ₂ η⁴ G_{ij} ξ₁^i ξ₁^j + o(η⁴).
\end{equation}
\begin{center}
 ***
\end{center}

\section{Discussion}

In this section, we discuss the two main cases of bifurcations, namely
\emph{asymmetric} and \emph{symmetric}. In each case, we analyse the stability
of the bifurcated branch.

\subsection{Asymmetric bifurcations (\(λ₁ ≠ 0\))}

We first consider the situation where \(λ₁ ≠ 0\) on the bifurcated branch. The
bifurcation equation~\eqref{eq:20220524135036} shows that necessarily,
\(E_{ijk}\) is not identically nul. This equation has at most \((2^m - 1)\)
pairs of real solutions \((λ₁, u₁)\) et \((- λ₁, - u₁)\). \pdfmargincomment{Je
  ne sais pas démontrer ce résultat sur le nombre de solutions réelles.}

From Eq.~\eqref{eq:20220531054247}, we find that
\begin{equation}
  \begin{aligned}[b]
    ℰ_{, uu}[u(η), λ(η); u₁, u₁]
    &= η \bigl[ℰ₃(λ₀ ; u₁, u₁, u₁)  + λ₁ \dot{ℰ}₂(λ₀; u₁, u₁)\bigr] + o(η)\\
    &= - η λ₁ \dot{ℰ}₂(λ₀; u₁, u₁) + o(η),
  \end{aligned}
\end{equation}
where we have used~\eqref{eq:20220524133816} in the second line. Along the
bifurcated branch, we also have \(λ = λ₀ + η λ₁ + o(η)\), and the above equation
can also be written
\begin{equation}
  ℰ_{, uu}[u(η), λ(η); u₁, u₁] = -\bigl( λ - λ₀ \bigr) \dot{ℰ}₂(λ₀; u₁, u₁) + o(λ - λ₀).
\end{equation}

For \(λ < λ₀\), the above quantity is \emph{negative} (since \(\dot{ℰ}₂\) is
negative definite). In other words

\begin{center}
  \framebox{For asymmetric bifurcations, below the critical load, the bifurcated
    branch is unstable}
\end{center}

To investigate the stability above the critical load, we need to analyse the
sign of the eigenvalues \(α\) of the Hessian. At first order,
\(α = η α₁ + o(η)\), where \(α₁\) is an eigenvalue of
\((E_{ijk} ξ₁^k + λ₁ F_{ij})\). Let \(α_{\min}\) and \(α_{\max}\) be the minimum
and maximum eigenvalues of this second-order tensor. Three cases must be
discussed
\begin{enumerate}
\item If \(α_{\min} α_{\max} > 0\), then \((E_{ijk} ξ₁^k + λ₁ F_{ij})\) is
  positive or negative definite: all eigenvalues have the same sign,
  \(\epsilon ∈ \{-1, +1\}\). Then the sign of the eigenvalues \(α\) of the
  Hessian is \(\epsilon η\) and there is a stability switch at the critical
  load. Since the bifurcated branch is unstable \emph{below} the critical load,
  this means that it is \emph{stable} above the critical load.
\item If \(α_{\min} α_{\max} < 0\), then the extremal eigenvalues of the Hessian
  are \(η α_{\min}\) and \(η α_{\max}\), the product of which is
  \(η² α_{\min} α_{\max} < 0\). The bifurcated branch is \emph{unstable} for all
  values of \(λ\).
\item If \(α_{\min} α_{\max} = 0\), the analysis is inconclusive.
\end{enumerate}

\subsection{Symmetric bifurcations (\(λ₁ = 0\) and \(λ₂ ≠ 0\))}

We now consider the case \(λ₁ = 0\), the next term of the expansion of \(λ\)
being non-zero: \(λ₂ ≠ 0\). The bifurcation equations~\eqref{eq:20220524133447}
and \eqref{eq:20220708060436} therefore reduce to
\begin{gather}
  ℰ₃(λ₀; u₁, u₁, \hat{u}) + ℰ₂(λ₀; u₂, \hat{u}) = 0,\\
  ℰ₄(λ₀; u₁, u₁, u₁, \hat{u}) + 3ℰ₃(λ₀; u₁, u₂, \hat{u}) + ℰ₂(λ₀; u₃, \hat{u}) + 3λ₂\dot{ℰ}₂(λ₀; u₁, \hat{u}) = 0.
\end{gather}
for all \( \hat{u} ∈ U \). Substituting \(\hat{u} = u₂\) in the first equation
and \(\hat{u} = u₁\) in the second equation, we get
\begin{gather}
  ℰ₃(λ₀; u₁, u₁, u₂) + ℰ₂(λ₀; u₂, u₂) = 0,\\
  \label{eq:20220708062645}
  ℰ₄(λ₀; u₁, u₁, u₁, u₁) + 3ℰ₃(λ₀; u₁, u₁, u₂) + 3λ₂ \dot{ℰ}₂(λ₀; u₁, u₁) = 0.
\end{gather}

The quantity \(ℰ₂(λ₀; u₂, u₂)\) has been evaluated in
Appendix~\ref{sec:20220525053434}; with \(λ₁ = 0\), we have
\begin{equation}
  ℰ₂(λ₀; u₂, u₂) = -ℰ₃(v_i, v_j, w_{kl}) ξ₁^i ξ₁^j ξ₁^k ξ₁^l
\end{equation}
and we find
\begin{equation}
  \bigl[ℰ₄(λ₀; v_i, v_j, v_k, v_l) + 3 ℰ₃(v_i, v_j, w_{kl}) \bigr] ξ₁^i ξ₁^j ξ₁^k ξ₁^l + 3λ₂ \dot{ℰ}₂(λ₀; v_i, v_j) ξ₁^i ξ₁^j = 0
\end{equation}
which reduces to
\begin{equation}
  E_{ijkl} ξ₁^i ξ₁^j ξ₁^k ξ₁^l + 3λ₂ F_{ij} ξ₁^i ξ₁^j = 0.
\end{equation}

Plugging \(λ₁ = 0\), \(\hat{u} = \hat{v} = u₁\) and
Eq.~\eqref{eq:20220708062645} into Eq.~\eqref{eq:20220531054247}, we further
find
\begin{equation}
  ℰ_{, uu}[u(η), λ(η); u₁, u₁] = - η² \bigl[ ℰ₃(λ₀; u₁, u₁, u₂) + λ₂ \dot{ℰ}₂(λ₀; u₁, u₁) \bigr] + o(η²).
\end{equation}


\paragraph{Si la forme \(ℰ₃(λ₀)\) est nulle sur \(V\)} L'équation
\eqref{eq:20220524133816} conduit nécessairement à \(λ₁ = 0\), puisque
\(\dot{ℰ}₂(λ₀)\) est définie négative. Dès lors, l'équation
\eqref{eq:20220601070917} s'écrit \pdfmargincomment{Expliquer pourquoi la forme
  quadratique ℰ̇₂(λ₀) bien définie négative}
\begin{equation}
 \tfrac{1}{3} E_{ijkl}(λ₀) ξ₁^j ξ₁^k ξ₁^l + λ₂ F_{ij}(λ₀) ξ₁^j = 0.
\end{equation}
Cette équation admet cette fois au plus \(\frac{3^m - 1}{2}\) paires de
solutions réelles \((λ₂, u₁)\) et \((- λ₂, - u₁)\). \pdfmargincomment{Je ne sais
  pas non plus démontrer ce résultat sur le nombre de solutions réelles.}
\pdfmargincomment{Note du 29/04/2022 – J'ai relu tous les calculs précédents. Il
  reste à reprendre les calculs des développements asymptotiques de l'énergie et
  de sa hessienne, pour tenir compte en particulier des factorielles introduites
  maintenant dans les développements asymptotiques. Il faudrait également
  introduire les tenseurs précédents dans les expressions de l'énergie et de sa
  hessienne.}


\section{Propriétés des formes bilinéaires symétriques, positives}

Dans ce qui suit, \(\mathcal{B}\) désigne une forme bilinéaire symétrique et
positive sur l'espace vectoriel \(U\). On définit son noyau \(\ker \mathcal{B}\)
de la façon suivante
\begin{equation}
 \ker \mathcal{B}= \bigl\{ u ∈ U, \mathcal{B}(u, u) = 0 \bigr\} .
\end{equation}

\begin{theorem}
  Le noyau d'une forme bilinéaire, symétrique et positive est un sous-espace
  vectoriel.
\end{theorem}
\begin{proof}
  Soient \(u, v∈\ker \mathcal{B}\), \(α∈\reals\) et \(w = u + α v\). Montrons
  que \(w ∈ \ker\mathcal{B}\). Il suffit d'évaluer \(\mathcal{B}(w, w)\)
 \begin{equation}
   \mathcal{B}(w, w) = \mathcal{B}(u + α v, u + α v)
   = \mathcal{B}(u, u) + 2 α \mathcal{B}(u, v) + α² \mathcal{B}(v, v),
 \end{equation}
 où l'on a tenu compte de la symétrie de \(\mathcal{B}\) pour écrire que
 \(\mathcal{B}(u, v) =\mathcal{B}(v, u)\). Comme \(u, v ∈ \ker\mathcal{B}\), le
 premier et le dernier terme sont nuls, soit
 \(\mathcal{B}(w, w) = 2α \mathcal{B}(u, v)\). La forme bilinéaire étant
 positive, cette grandeur est positive, \emph{quelle que soit la valeur de
   \(α∈\reals\)}. On en déduit donc que \(\mathcal{B}(u, v) = 0\), puis que
 \(\mathcal{B}(w, w) = 0\) et donc que \(w ∈ \ker\mathcal{B}\).
\end{proof}

\begin{theorem}
 Soit \(u∈V\). Alors
 \begin{equation}
  u ∈ \ker\mathcal{B} \quad \text{ssi} \quad \text{pour tout } v ∈ V, \mathcal{B}(u, v) = 0.
 \end{equation}
\end{theorem}

\begin{proof}
  Soient \(u∈\ker \mathcal{B}\), \(v∈V\) et \(α∈\reals\). Comme précédemment, on
  écrit que \(\mathcal{B}(w, w) ≥ 0\), avec \(w = α u + v\)
 \begin{equation}
  \mathcal{B}(w, w) = 2 α \mathcal{B}(u, v) +\mathcal{B}(v, v) \geq
  0,
 \end{equation}
 où l'on a tenu compte de ce que \(\mathcal{B}(u, u) = 0\). L'expression
 précédente, affine en \(α\), a un signe constant. Le terme linéaire en \(α\)
 est donc nul, soit \(\mathcal{B}(u, v) = 0\).  Réciproquement, si
 \(\mathcal{B}(u, v) = 0\) pour tout \(v∈V\), alors \(\mathcal{B}(u, u) = 0\)(en
 prenant \(v = u\)).
\end{proof}

\appendix
\section{Développements limités le long d'une branche bifurquée du diagramme d'équilibre}

\subsection{Principe du calcul}
\label{sec:20220107121442}
% 02/06/2022 — 099042106e938251657847daca64c8fcbaa833c3
%
% Validation des calculs de ce paragraphe

On pose dans ce qui suit
\begin{align}
  \label{eq:20211112155446}
  Λ(η) & = λ(η) - λ₀ = η λ₁ + \tfrac{1}{2} η² λ₂ + \tfrac{1}{6} η³ λ₃ + \cdots,\\
  \label{eq:20211112113028}
  U(η) & = u(η) - u^{\ast}[λ(η)] = η u₁ + \tfrac{1}{2} η² u₂ + \tfrac{1}{6} η³ u₃ + \cdots.
\end{align}

On considère une fonctionnelle \(\mathcal{F}\) de \(u\) et \(λ\)~:
\(\mathcal{F}(u, λ)\). Cette fonctionnelle est évaluée le long de la branche
bifurquée. En d'autres termes, on considère
\begin{equation*}
  f(η) = F\{ u^{\ast} [λ₀ + Λ(η)] + U(η), λ₀ + Λ(η) \}.
\end{equation*}

On souhaite établir un développement limité de \(f\) au voisinage de \(η = 0\),
ce qui conduit à calculer les dérivées successives de \(f\) en \(η = 0\),
puisque
\begin{equation*}
  f(η) = f(0) + η f'(0) + \tfrac{1}{2} η² f''(0) + \cdots.
\end{equation*}

Pour calculer ces dérivées, il sera commode d'introduire la fonction auxiliaire
\(F\)
\begin{equation*}
  F(η, λ) =\mathcal{F}[u^{\ast}(λ) + U(η), λ],
\end{equation*}
dans laquelle les variables \(λ\) et \(η\) sont provisoirement considérées comme
indépendantes. On a \(f(η) = F[η, λ₀ + Λ(η)]\), d'où l'on déduit successivement
que
\begin{gather*}
  f'(η) = ∂_{η} F + Λ' ∂_{λ} F,\\
  f''(η) = ∂_{ηη}² F + 2Λ' ∂_{ηλ}²F + Λ'^2 ∂_{λλ}² F + Λ'' ∂_{λ} F,\\
  \begin{aligned}[b]
    f'''(η) ={}
    & ∂_{ηηη}³ F + 3Λ' ∂_{ηηλ}³F + 3Λ'^2 ∂_{ηλλ}³F + λ'^3 ∂_{λλλ}³ F\\
    & + 3Λ'' ∂_{ηλ}² F + 3Λ' Λ'' ∂_{λ λ}² F + Λ''' ∂_{λ} F,
  \end{aligned}\\
  \begin{aligned}[b]
    f''''(η) ={}
    & ∂_{ηηηη}⁴ F + 4Λ' ∂_{ηηηλ}⁴F + 6Λ'^2 ∂_{ηηλλ}⁴F + 4Λ'^3 ∂_{ηλλλ}⁴F + Λ'^4 ∂_{λλλλ}⁴ F\\
    & + 6Λ'' ∂_{ηηλ}³ F + 12Λ' Λ'' ∂_{ηλλ}³F + 6Λ'^2 Λ'' ∂_{λλλ}³ F\\
    & + 4 Λ''' ∂_{ηλ}² F + \bigl( 3Λ''^2 + 4 Λ' Λ''' \bigr) ∂_{λλ}² F + λ'''' ∂_{λ}F,
  \end{aligned}
\end{gather*}
où \(Λ\) et ses dérivées sont évaluées en \(η\), tandis que \(F\) et ses
dérivées partielles sont évaluées en \([η, λ₀ + Λ(η)]\). En \(η = 0\), les
relations précédentes s'écrivent
\begin{gather}
  \label{eq:20220107060454}
  f'(0) = ∂_{η} F + λ₁ ∂_{λ} F,\\
  \label{eq:20220107124311}
  f''(0) = ∂_{ηη}² F + 2 λ₁ ∂_{ηλ}² F + λ₁² ∂_{λλ}² F + λ₂ ∂_{λ} F,\\
  \label{eq:20220107060500}
  \begin{aligned}[b]
    f'''(0) ={}
    & ∂_{ηηη}³ F + 3 λ₁ ∂_{ηηλ}³ F + 3 λ₁² ∂_{ηλλ}³ F + λ₁³ ∂_{λλλ}³ F\\
    & + 3 λ₂ ∂_{ηλ}² F + 3 λ₁ λ₂ ∂_{λλ}² F + λ₃ ∂_{λ} F,
  \end{aligned}\\
  \label{eq:20220602185935}
  \begin{aligned}[b]
    f''''(0) ={}
    & ∂_{ηηηη}⁴F + 4 λ₁ ∂_{ηηηλ}⁴ F + 6 λ₁² ∂_{ηηλλ}⁴ F + 4 λ₁³ ∂_{ηλλλ}⁴ F + λ₁⁴ ∂_{λλλλ}⁴ F\\
    & + 6 λ₂ ∂_{ηηλ}³ F + 12 λ₁ λ₂ ∂_{ηλλ}³ F + 6 λ₁² λ₂ ∂_{λλλ}³ F\\
    & + 4 λ₃ ∂_{ηλ}² F + \bigl(3 λ₂² + 4 λ₁ λ₃\bigr) ∂_{λλ}² F + λ₄ ∂_{λ} F,
  \end{aligned}
\end{gather}
où \(F\) et ses dérivées sont maintenant évaluées en \((η = 0, λ = λ₀)\).

\subsection{Développement limité du résidu}
\label{sec:20211112182000}
% 03/06/2022 — b028b234970605720c9022c16c7fc3012997ced7
%
% Validation des calculs de ce paragraphe

On cherche un développement limité du résidu (c'est-à-dire de la première
variation de l'énergie). La fonction test \(\hat{u} ∈ U\) étant fixée, la
méthode précédente est donc appliquée avec
\begin{equation}
  \label{eq:20220107054629}
  f(η) = ℰ_{, u} [u(η), λ(η); \hat{u}]
  \quad \text{et} \quad
  F(η, λ) = ℰ_{, u}[u^{\ast}(λ) + U(η), λ; \hat{u}].
\end{equation}

On remarque tout d'abord que
\(F(0, λ) =ℰ_{, u} [u^{\ast} (λ), λ; \hat{u}] = 0\), puisque \(u^{\ast}(λ)\) est
un point d'équilibre. En dérivant par rapport à \(λ\), on obtient
\begin{equation*}
  \frac{∂^k F}{∂ λ^k}(0, λ) = 0 \quad \text{pour tout} \quad k ≥ 0.
\end{equation*}

En dérivant par rapport à \(η\) l'expression~\eqref{eq:20220107054629} de \(F\),
on obtient successivement
\begin{equation*}
  ∂_{η}F(η, λ) = ℰ_{, u u}[u^{\ast}(λ) + U(η), λ; U'(η), \hat{u}],
\end{equation*}
\begin{equation*}
  \begin{aligned}[b]
    ∂_{η η}² F(η, λ) ={}
    & ℰ_{, uuu}[u^{\ast}(λ) + U(η), λ; U'(η), U'(η), \hat{u}]\\
    & + ℰ_{, uu} [u^{\ast}(λ) + U(η), λ; U''(η), \hat{u}],
  \end{aligned}
\end{equation*}
\begin{equation*}
  \begin{aligned}[b]
    ∂_{ηηη}³ F(η, λ) ={}
    & ℰ_{, uuuu}[u^{\ast}(λ) + U(η), λ; U'(η), U'(η), U'(η), \hat{u}]\\
    & + 3ℰ_{, u u u}[u^{\ast}(λ) + U(η), λ; U'(η), U''(η), \hat{u}]\\
    & + ℰ_{, uu}[u^{\ast}(λ) + U(η), λ; U'''(η), \hat{u}],
  \end{aligned}
\end{equation*}
soit, en \(η = 0\)
\[∂_{η}F(0, λ) = ℰ₂(λ; u₁, \hat{u}),\]
\[∂_{ηη}² F(0, λ) = ℰ₃(λ; u₁, u₁, \hat{u}) + ℰ₂(λ; u₂, \hat{u}),\]
\[∂_{ηηη}³ F(0, λ) = ℰ₄(λ; u₁, u₁, u₁, \hat{u}) + 3ℰ₃(λ; u₁, u₂, \hat{u}) + ℰ₂(λ; u₃, \hat{u}).\]

Les dérivées croisées de \(F\) en \((0, λ)\) s'obtiennent par simple dérivation
des relations précédentes par rapport à \(λ\)
\[∂_{ηλ}² F(0, λ) = \dot{ℰ}₂(λ; u₁, \hat{u}), \quad ∂_{ηλλ}³ F(0, λ) = \ddot{ℰ}₂(λ; u₁, \hat{u}),\]
\[∂_{ηηλ}³ F(0, λ) = \dot{ℰ}₃(λ; u₁, u₁, \hat{u}) + \dot{ℰ₂}(λ; u₂, \hat{u}).\]

En insérant les résultats précédents dans les relations
générales~\eqref{eq:20220107060454}--\eqref{eq:20220602185935}, on trouve alors
les expressions suivantes des dérivées successives de \(f\) en \(η = 0\)
\begin{gather*}
  f'(0) = ℰ₂(λ₀; u₁, \hat{u}),\\
  f''(0) = ℰ₃(λ₀; u₁, u₁, \hat{u}) + ℰ₂(λ₀; u₂, \hat{u}) + 2 λ₁ \dot{ℰ}₂(λ₀; u₁, \hat{u}),\\
  \begin{aligned}[b]
    f'''(0) ={}
    & ℰ₄(λ₀; u₁, u₁, u₁, \hat{u}) + 3ℰ₃(λ₀; u₁, u₂, \hat{u}) + ℰ₂(λ₀ ; u₃, \hat{u})\\
    & + 3λ₁ \dot{ℰ}₃(λ₀; u₁, u₁, \hat{u}) + 3λ₁ \dot{ℰ}₂(λ₀; u₂, \hat{u})\\
    & + 3 λ₁² \ddot{ℰ}₂(λ₀; u₁, \hat{u}) + 3 λ₂ \dot{ℰ}₂(λ₀; u₁, \hat{u}).
  \end{aligned}
\end{gather*}

On en déduit finalement le développement limité à l'ordre 3 en \(η\) du résidu
\begin{equation}
  \label{eq:20220107080901}
  \begin{gathered}[b]
    ℰ_{, u}[u(η), λ(η)] ={} η ℰ₂(λ₀; u₁, \hat{u}) + \tfrac{1}{2} η² \bigl[ℰ₃(λ₀; u₁, u₁, \hat{u})  + ℰ₂(λ₀; u₂, \hat{u})\\
    {} + 2 λ₁ \dot{ℰ}₂(λ₀; u₁, \hat{u})\bigr] + \tfrac{1}{6} η³ \bigl[ ℰ₄(λ₀; u₁, u₁, u₁, \hat{u}) + 3ℰ₃(λ₀; u₁, u₂, \hat{u})\\
    {} + ℰ₂(λ₀; u₃, \hat{u}) + 3λ₁ \dot{ℰ}₃(λ₀; u₁, u₁, \hat{u}) + 3λ₁ \dot{ℰ}₂(λ₀; u₂, \hat{u})\\
    {} + 3 λ₁² \ddot{ℰ}₂(λ₀; u₁, \hat{u}) + 3 λ₂ \dot{ℰ}₂(λ₀ ; u₁, \hat{u}) \bigr] + o(η³).
  \end{gathered}
\end{equation}

\subsection{Développement limité de l'énergie}
\label{sec:20220525053434}
% 07/06/2022 — dd1a4abf18cd94861d754bf3e19a54b8974bb2e8
%
% Relecture de tous les calculs de ce paragraphe

On s'intéresse ici à l'écart d'énergie, pour un chargement \(λ\) donné, entre la
branche bifurquée et la branche fondamentale, soit
\begin{equation}
  F(η, λ) = ℰ[u^{\ast}(λ) + U(η), λ] - ℰ[u^{\ast}(λ), λ]
  \quad \text{et} \quad
  f(η) = F [η, λ₀ + Λ(η)].
\end{equation}

On observe tout d'abord que \(F(0, λ) = 0\) pour tout \(λ\), donc
\begin{equation*}
  \frac{∂^k F}{∂ λ^k}(0, λ) = 0 \quad \text{pour tout} \quad k ≥ 0,
\end{equation*}
tandis que les dérivées de \(F\) par rapport à \(η\) s'écrivent
\begin{gather*}
  ∂_{η} F(η, λ) = ℰ_{, u}(U'),\\
  ∂_{ηη}² F(η, λ) = ℰ_{, uu} (U', U') + ℰ_{, u} (U''),\\
  ∂_{ηηη}³ F(η, λ) = ℰ_{, uuu}(U', U', U') + 3ℰ_{, uu}(U', U'') + ℰ_{, u}(U'''),\\
  \begin{aligned}[b]
    ∂_{ηηηη}⁴ F ={}
    & ℰ_{, uuuu}(U', U', U', U') + 6ℰ_{,uuu}(U', U', U'')\\
    & + 3ℰ_{, uu}(U'', U'') + 4ℰ_{, uu}(U', U''') + ℰ_{, u}(U''''),
  \end{aligned}\\
\end{gather*}
où les différentielles successives de \(ℰ\) sont évaluées en
\([u^{\ast}(λ) + U(η), λ]\), tandis que les dérivées successives de \(U\) sont
évaluées en \(η\).  Les relations précédentes s'écrivent, en \(η = 0\), en
observant que \(ℰ_{, u}[u^{\ast}(λ), λ] = 0\)
\begin{gather*}
  ∂_{η} F(0, λ) = 0,\\
  ∂_{ηη}² F(0, λ) =ℰ₂(λ ; u₁, u₁),\\
  ∂_{ηηη}³ F(0, λ) = ℰ₃(λ; u₁, u₁, u₁) + 3ℰ₂(λ; u₁, u₂),\\
  ∂_{ηηηη}⁴ F(η, λ) = ℰ₄(λ; u₁, u₁, u₁, u₁) + 6ℰ₃(λ; u₁, u₁, u₂) + 3ℰ₂(λ; u₂, u₂) + 4ℰ₂(λ; u₁, u₃).
\end{gather*}

En dérivant alors par rapport à \(λ\), on en déduit que\\
\begin{equation*}
  \begin{gathered}
    ∂_{ηλ}² F(0, λ) = 0,\\
    ∂_{ηηλ}³ F(0, λ) = \dot{ℰ}₂(λ; u₁, u₁),\\
    ∂_{ηλλ}³ F(0, λ) = 0,\\
  \end{gathered}
  \qquad
  \begin{gathered}
    ∂_{ηηηλ}⁴ F(0, λ) = \dot{ℰ}₃(λ; u₁, u₁, u₁) + 3\dot{ℰ}₂(λ; u₁, u₂),\\
    ∂_{ηηλλ}⁴ F(0, λ) = \ddot{ℰ}₂(λ; u₁, u₁),\\
    ∂_{ηλλλ}⁴ F(0, λ) = 0
  \end{gathered}
\end{equation*}
et finalement
\begin{gather*}
    f'(0) = 0, \qquad f''(0) = ℰ₂(λ₀; u₁, u₁),\\
    f'''(0) =ℰ₃(λ₀; u₁, u₁, u₁) + 3ℰ₂(λ₀; u₁, u₂) + 3λ₁ \dot{ℰ}₂(λ₀; u₁, u₁),\\
    \begin{aligned}[b]
      f''''(0) ={}
      & ℰ₄(λ₀; u₁, u₁, u₁, u₁) + 6ℰ₃(λ₀; u₁, u₁, u₂) + 3ℰ₂(λ₀; u₂, u₂) + 4ℰ₂(λ₀; u₁, u₃)\\
      & + 4 λ₁ \dot{ℰ}₃(λ₀; u₁, u₁, u₁) + 12 λ₁ \dot{ℰ}₂(λ₀; u₁, u₂) + 6λ₁² \ddot{ℰ}₂(λ₀; u₁, u₁) + 6λ₂ \dot{ℰ}₂(λ₀; u₁, u₁).
    \end{aligned}
\end{gather*}

Les relations précédentes se simplifient notamment en tenant compte de ce que
\(u₁∈V\) : \(ℰ₂(λ₀; u₁, u_i) = 0\) pour \(i = 1, 2, 3\). On trouve ainsi
\(f''(0)=0\) et
\begin{equation}
  \label{eq:20220601055448}
  f'''(0) = -λ₁ G_{ij} ξ₁^i ξ₁^j,
\end{equation}
en utilisant l'équation de bifurcation~\eqref{eq:20220524133816}. En
introduisant les décompositions \eqref{eq:20220524133944} et
\eqref{eq:20220524134613} de \(u₁\) et \(u₂\), on trouve tout d'abord, pour
\(ℰ₃(λ₀; u₁, u₁, u₂)\)
\begin{equation*}
  \begin{aligned}[b]
    ℰ₃(λ₀; u₁, u₁, u₂)
    ={} & ℰ₃(v_i, v_j, v_k) ξ₁^i ξ₁^j ξ₂^k + ℰ₃(v_i, v_j, w_{k l}) ξ₁^i ξ₁^j ξ₁^k ξ₁^l + λ₁ ℰ₃(v_i, v_j, w_k) ξ₁^i ξ₁^j ξ₁^k \\
    ={} & ℰ₃(v_i, v_j, v_k) ξ₁^i ξ₁^j ξ₂^k + ℰ₃(v_i, v_j, w_{k l}) ξ₁^i ξ₁^j ξ₁^k ξ₁^l - λ₁ ℰ₂(w_{ij}, w_k) ξ₁^i ξ₁^j ξ₁^k,
  \end{aligned}
\end{equation*}
en tenant compte de la définition~\eqref{eq:20220519164523} des \(w_{ij}\). Dans
le dernier terme de l'expression précédente, les indices \(i\), \(j\) et \(k\)
sont muets, donc
\begin{equation*}
  \begin{aligned}[b]
    ℰ₃(λ₀; u₁, u₁, u₂)
    ={} & ℰ₃(v_i, v_j, v_k) ξ₁^i ξ₁^j ξ₂^k + ℰ₃(v_i, v_j, w_{kl}) ξ₁^i ξ₁^j ξ₁^k ξ₁^l - λ₁ ℰ₂(w_{i}, w_{jk}) ξ₁^i ξ₁^j ξ₁^k\\
    ={} & ℰ₃(v_i, v_j, v_k) ξ₁^i ξ₁^j ξ₂^k + ℰ₃(v_i, v_j, w_{kl}) ξ₁^i ξ₁^j ξ₁^k ξ₁^l + 2 λ₁ \dot{ℰ}₂(v_{i}, w_{jk}) ξ₁^i ξ₁^j ξ₁^k,
  \end{aligned}
\end{equation*}
en introduisant cette fois-ci la définition~\eqref{eq:20220524134525} de
\(w_i\). On procède de même pour le terme suivant, soit \(ℰ₂(u₂, u₂)\)
\begin{equation*}
  \begin{aligned}[b]
    ℰ₂(u₂, u₂)
    ={} & ℰ₂(ξ₂^i v_i + ξ₁^i ξ₁^j w_{i j} + λ₁ ξ₁^i w_i, ξ₂^k v_k + ξ₁^k ξ₁^l w_{k l} + λ₁ ξ₁^k w_k)\\
    ={} & ℰ₂(ξ₁^i ξ₁^j w_{i j} + λ₁ ξ₁^i w_i, ξ₁^k ξ₁^l w_{k l} + λ₁ ξ₁^k w_k)\\
    ={} & ℰ₂(w_{i j}, w_{k l}) ξ₁^i ξ₁^j ξ₁^k ξ₁^l + 2 λ₁ ℰ₂(w_i, w_{j k}) ξ₁^i ξ₁^j ξ₁^k + λ₁² ℰ₂(w_i, w_j) ξ₁^i ξ₁^j\\
    ={} & ℰ₂(w_{i j}, w_{k l}) ξ₁^i ξ₁^j ξ₁^k ξ₁^l + 2 λ₁ ℰ₂(w_i, w_{j k}) ξ₁^i ξ₁^j ξ₁^k + \tfrac{1}{2} λ₁² \bigl[ℰ₂(w_i, w_j) + ℰ₂(w_j, w_i)\bigr] ξ₁^i ξ₁^j\\
    ={} & -ℰ₃(v_i, v_j, w_{k l}) ξ₁^i ξ₁^j ξ₁^k ξ₁^l - 4 λ₁ \dot{ℰ}₂ (v_i, w_{j k}) ξ₁^i ξ₁^j ξ₁^k - λ₁² \bigl[\dot{ℰ}₂(v_i, w_j) + \dot{ℰ}₂(v_j, w_i)\bigr] ξ₁^i ξ₁^j
  \end{aligned}
\end{equation*}
et enfin
\begin{equation*}
  \begin{aligned}[b]
    \dot{ℰ}₂(u₁, u₂)
    ={} & \dot{ℰ}₂ (v_i, v_j) ξ₁^i ξ₂^j + \dot{ℰ}₂(v_i, w_{j k}) ξ₁^i ξ₁^j ξ₁^k + λ₁ \dot{ℰ}₂(v_i, w_j) ξ₁^i ξ₁^j\\
    ={} & \dot{ℰ}₂(v_i, v_j) ξ₁^i ξ₂^j + \dot{ℰ}₂(v_i, w_{j k}) ξ₁^i ξ₁^j ξ₁^k + \tfrac{1}{2} λ₁ [\dot{ℰ}₂(v_i, w_j) + \dot{ℰ}₂(v_j, w_i)] ξ₁^i ξ₁^j.
  \end{aligned}
\end{equation*}
En rassemblant les résultats précédents, on trouve pour \(f''''(0)\)
\begin{equation*}
  \begin{aligned}[b]
    f''''(0)
    ={} & \bigl[ ℰ₄(v_i, v_j, v_k , v_l) + 3ℰ₃(v_i, v_j, w_{k l}) \bigr] ξ₁^i ξ₁^j ξ₁^k ξ₁^l + 4 λ₁ \bigl[\dot{ℰ}₃(v_i, v_j, v_k) + 3 \dot{ℰ}₂(v_i, w_{j k})\bigr] ξ₁^i ξ₁^j ξ₁^k\\
    & + \bigl\{3 λ₁² \bigl[ 2\ddot{ℰ}₂ (v_i, v_j) + \dot{ℰ}₂(v_i, w_j) + \dot{ℰ}₂(v_j, w_i) \bigr] + 6λ₂ \dot{ℰ}₂(v_i, v_j) \bigr\} ξ₁^i ξ₁^j\\
    & + 6\bigl[ℰ₃(v_i, v_j, v_k) ξ₁^k + 2 λ₁ \dot{ℰ}₂(v_i, v_j)\bigr] ξ₁^i ξ₂^j,
  \end{aligned}
\end{equation*}
et on observe que le dernier terme(en \(ξ₁^i ξ₂^j\)) est nul, du fait de
l'équation de bifurcation~\eqref{eq:20220524135036}. On obtient donc
\begin{equation}
  \label{eq:20220601055512}
  f''''(0) = E_{i j k l} ξ₁^i ξ₁^j ξ₁^k ξ₁^l + 4 λ₁ \mathring{E}_{i j k} ξ₁^i ξ₁^j ξ₁^k - 6 \bigl(λ₁² \mathring{G}_{i j} + λ₂ G_{i j}\bigr) ξ₁^i ξ₁^j .
\end{equation}

Le développement limité~\eqref{eq:20220525053600} est alors obtenu en
rassemblant les expressions précédentes de \(f'(0)\), \(f''(0)\), \(f'''(0)\) et
\(f''''(0)\).

\subsection{Développement limité de la hessienne}
\label{sec:20220616055207}
% 08/06/2022 — aea0da72c80440d74d38d8ace59f381061f71c3e
%
% Relecture de tous les calculs de ce paragraphe

On cherche maintenant un développement limité de la hessienne de l'énergie. Les
fonctions test \(\hat{u}, \hat{v} ∈ U\) étant fixées, on applique la méthode du
\S\ref{sec:20220107121442} à la fonction \(f(η) = F [η, λ₀ + Λ(η)]\), avec
\begin{equation*}
  F(η, λ) = ℰ_{, u u} [u^{\ast}(λ) + U(η), λ; \hat{u}, \hat{v}].
\end{equation*}

On observe tout d'abord que \(F(0, λ) =ℰ₂(λ; \hat{u}, \hat{v})\), soit, en
dérivant par rapport à \(λ\)
\begin{equation*}
  ∂_{λ} F(0, λ) = \dot{ℰ}₂(λ; \hat{u}, \hat{v})
  \quad \text{et} \quad
  ∂_{λλ}² F(0, λ) = \ddot{ℰ}₂(λ; \hat{u}, \hat{v}).
\end{equation*}

On trouve de même successivement
\begin{gather*}
  ∂_{η} F(η, λ) = ℰ_{, uuu}(U', \hat{u}, \hat{v}),\\
  ∂_{ηη}² F(η, λ) = ℰ_{, uuuu}(U', U', \hat{u}, \hat{v}) + ℰ_{, uuu}(U'', \hat{u}, \hat{v}),
\end{gather*}
où les différentielles successives de \(ℰ\) sont évaluées en
\([u^{\ast}(λ) + U(η), λ]\), tandis que les dérivées successives de \(U\) sont
évaluées en \(η\). Les relations précédentes s'écrivent en \(η = 0\)
\begin{gather*}
  ∂_{η} F(0, λ) = ℰ₃(λ; u₁, \hat{u}, \hat{v}),\\
  ∂_{ηη}² F(0, λ) = ℰ₄(λ ; u₁, u₁, \hat{u}, \hat{v}) + ℰ₃(λ; u₂, \hat{u}, \hat{v}),
\end{gather*}
et en dérivant cette fois par rapport à \(λ\)
\begin{equation*}
  ∂_{η λ}² F(0, λ) = \dot{ℰ}₃(λ; u₁, \hat{u}, \hat{v}).
\end{equation*}

En insérant les résultats précédents dans les
expressions~\eqref{eq:20220107060454} et \eqref{eq:20220107124311}, on trouve
\begin{gather*}
  f'(0) = ℰ₃(u₁, \hat{u}, \hat{v}) + λ₁ \dot{ℰ}₂(\hat{u}, \hat{v}),\\
  f''(0) = ℰ₄(u₁, u₁, \hat{u}, \hat{v}) + ℰ₃(u₂, \hat{u}, \hat{v}) + 2λ₁ \dot{ℰ}₃(u₁, \hat{u}, \hat{v}) + λ₁² \ddot{ℰ}₂(\hat{u}, \hat{v}) + λ₂ \dot{ℰ}₂(\hat{u}, \hat{v}) .
\end{gather*}
qui conduisent finalement au développement limité~\eqref{eq:20220531054247}.

\subsection{Asymptotic expansions of the eigenvalues and eigenvectors of the Hessian}
\label{sec:20220616074108}

In this appendix, Eqs.~\eqref{eq:20220609133608}, \eqref{eq:20220609133629} and
\eqref{eq:20220616082923} are derived. The postulated
expansions~\eqref{eq:20220617064633} are plugged into the asymptotic expansion
\eqref{eq:20220531054247} of the Hessian on the one hand
\begin{multline*}
  ℰ_{, uu} [u(η), λ(η); x, \hat{u}] = ℰ₂(x₀, \hat{u}) + η \bigl[ ℰ₂(x₁, \hat{u}) + ℰ₃(u₁, x₀, \hat{u}) + λ₁ \dot{ℰ}₂(x₀, \hat{u})\bigr]\\
  + \tfrac{1}{2} η² \bigl[ℰ₂(x₂, \hat{u}) + 2ℰ₃(u₁, x₁, \hat{u}) + 2 λ₁ \dot{ℰ}₂(x₁, \hat{u}) + ℰ₄(u₁, u₁, x₀, \hat{u})\\
  + ℰ₃(u₂, x₀, \hat{u}) + 2λ₁ \dot{ℰ}₃(u₁, x₀, \hat{u}) + λ₁² \ddot{ℰ}₂(x₀, \hat{u}) + λ₂ \dot{ℰ}₂(x₀, \hat{u}) \bigr] + o(η²)
\end{multline*}
(where the \(ℰ_k\) and \(\dot{ℰ}_k\) are all evaluated at \(λ=λ₀\)) and into the
scalar product \(α 〈 x, \hat{u} 〉\) on the other hand
\begin{multline*}
    α 〈 x, \hat{u} 〉 = α₀ 〈 x₀, \hat{u} 〉 + η \bigl(α₁ 〈 x₀, \hat{u} 〉 + α₀ 〈 x₁, \hat{u} 〉\bigr)\\
    + \tfrac{1}{2} η² \bigl(α₀ 〈 x₂, \hat{u} 〉 + 2 α₁ 〈 x₁, \hat{u} 〉 + α₂ 〈 x₀, \hat{u} 〉\bigr) + o(η²).
\end{multline*}

Equating both expressions for all \(\hat{u} ∈ U\) [see
Eq.~\eqref{eq:20220617074949}] leads to three variational problems (for the
\(η⁰\), \(η¹\) and \(η²\) terms) that are discussed below.

\paragraph{Variational problem of order 0} Find \(x₀∈U\) and \(α₀∈\reals\) such
that, for all \(\hat{u}∈U\)
\begin{equation*}
  ℰ₂(x₀, \hat{u}) = α₀ 〈 x₀, \hat{u} 〉.
\end{equation*}

The above equation shows that \((α₀, x₀)\) is an eigenpair of \(ℰ₂(λ₀)\). As
discussed in Sec.~\ref{sec:20220617075558}, only the case \(α₀ = 0\) is
relevant. Then \(x₀ ∈ V\), which is expressed by the
expansion~\eqref{eq:20220609133608} of \(x₀\).

\paragraph{Variational problem of order 1} Find \(x₁∈U\) and \(α₁∈\reals\) such
that, for all \(\hat{u}∈U\)
\begin{equation}
  \label{eq:20220609131953}
  ℰ₂(x₁, \hat{u}) + ℰ₃(u₁, x₀, \hat{u}) + λ₁ \dot{ℰ}₂(x₀, \hat{u}) = α₁ 〈 x₀, \hat{u} 〉,
\end{equation}
or, equivalently, plugging the expansions~\eqref{eq:20220524133944} and
\eqref{eq:20220609133608} of \(u₁\) and \(x₀\) in the \(v_i\) basis
\begin{equation}
  \label{eq:20220617080547}
  ℰ₂(x₁, \hat{u}) + ℰ₃(v_j, v_k, \hat{u}) χ₀^j ξ₁^k + λ₁ \dot{ℰ}₂(v_j, \hat{u}) χ₀^j = α₁ χ₀^j 〈 v_j, \hat{u} 〉.
\end{equation}

For \(\hat{u} = v_i\), observing that \(〈 v_i, v_j 〉 = δ_{ij}\) since
\((v_i)\) is orthonormal, the above equation reads
\begin{equation}
  \bigl[ℰ₃(λ₀; v_i, v_j, v_k) ξ₁^k + λ₁ \dot{ℰ}₂(λ₀; v_i, v_j)\bigr] χ₀^j = α₁ χ₀^i,
\end{equation}
which reduces to Eq.~\eqref{eq:20220609133608}.

The test function is now picked in \(W = V^\perp\), and \(x₁\) is decomposed as
the sum of its projections onto \(V\) and \(W\): \(x₁ = χ₁^i v_i + y₁\), where
\(y₁ ∈ W\). Eq.~\eqref{eq:20220617080547} then delivers the following
variational problem: find \(y₁ ∈ W\) such that, for all \(\hat{w} ∈ W\),
\begin{equation}
  ℰ₂(y₁, \hat{w}) + ℰ₃(v_i, v_j, \hat{w}) χ₀^i ξ₁^j + λ₁ \dot{ℰ}₂(v_i, \hat{w}) χ₀^i = 0,
\end{equation}
(observe that \(〈 v_j, \hat{w} 〉\) since \(V\) and \(W\) are orthogonal
subspaces). The solution to the above problem is expressed as a linear
combination of the \(w_i\) and \(w_{ij}\) defined by the variational problems
\eqref{eq:20220524134525} and \eqref{eq:20220519164523}, respectively:
\(y₁ = χ₀^i ξ₁^j w_{i j} + \tfrac{1}{2} λ₁ χ₀^i w_i\), and the
decomposition~\eqref{eq:20220609133629} is retrieved.

\paragraph{Variational problem of order 2} For all \(\hat{u} ∈ U\),
\begin{multline*}
  ℰ₂(x₂, \hat{u}) + 2ℰ₃(u₁, x₁, \hat{u}) + 2 λ₁ \dot{ℰ}₂(x₁, \hat{u}) + ℰ₄(u₁, u₁, x₀, \hat{u}) + ℰ₃(u₂, x₀, \hat{u})\\
  + 2λ₁ \dot{ℰ}₃(u₁, x₀, \hat{u}) + λ₁² \ddot{ℰ}₂(x₀, \hat{u}) + λ₂ \dot{ℰ}₂(x₀, \hat{u}) = 2 α₁ 〈 x₁, \hat{u} 〉 + α₂ 〈 x₀, \hat{u} 〉.
\end{multline*}

For \(\hat{u} = \hat{v}_i\), plugging the decompositions
\eqref{eq:20220524133944}, \eqref{eq:20220524134613}, \eqref{eq:20220609133608}
and \eqref{eq:20220609133629} of \(u₁\), \(u₂\), \(x₀ \) et \(x₁\) delivers
% \begin{multline*}
%   2ℰ₃(v_i, x₁, u₁) + 2 λ₁ \dot{ℰ}₂(v_i, x₁) + ℰ₄(v_i, x₀, u₁, u₁) + ℰ₃(v_i, x₀, u₂)\\
%   + 2λ₁ \dot{ℰ}₃(v_i, x₀, u₁) + λ₁² \ddot{ℰ}₂(v_i, x₀) + λ₂ \dot{ℰ}₂(v_i, x₀) = 2α₁ 〈 v_i, x₁ 〉 + α₂ 〈 v_i, x₀ 〉,
% \end{multline*}
% \begin{multline*}
%   2ℰ₃(v_i, χ₁^jv_j + χ₀^jξ₁^kw_{jk}+\tfrac{1}{2} λ₁ χ₀^j w_j, ξ₁^l v_l) + 2 λ₁ \dot{ℰ}₂(v_i, χ₁^jv_j + χ₀^jξ₁^kw_{jk}+\tfrac{1}{2} λ₁ χ₀^j w_j)\\
%   + ℰ₄(v_i, χ₀^j v_j, ξ₁^k v_k, ξ₁^l v_l) + ℰ₃(v_i, χ₀^j v_j, ξ₂^k v_k + ξ₁^k ξ₁^l w_{kl} + λ₁ ξ₁^k w_k)\\
%   + 2λ₁ \dot{ℰ}₃(v_i, χ₀^j v_j, ξ₁^k v_k) + λ₁² \ddot{ℰ}₂(v_i, χ₀^j v_j) + λ₂ \dot{ℰ}₂(v_i, χ₀^j v_j)\\
%   = 2α₁ 〈 v_i,  χ₁^jv_j + χ₀^jξ₁^kw_{jk}+\tfrac{1}{2} λ₁ χ₀^j w_j 〉 + α₂ 〈 v_i, χ₀^j v_j〉,
% \end{multline*}
% \begin{multline*}
%   2ℰ₃(v_i, v_j,  v_k) χ₁^j ξ₁^k + 2ℰ₃(v_i, w_{jk}, v_l) χ₀^j ξ₁^k ξ₁^l + λ₁ ℰ₃(v_i,  w_j, v_k) χ₀^j ξ₁^k + 2 λ₁ \dot{ℰ}₂(v_i, v_j) χ₁^j\\
%   + 2 λ₁ \dot{ℰ}₂(v_i, w_{jk}) χ₀^j ξ₁^k+ λ₁² \dot{ℰ}₂(v_i, w_j) χ₀^j + ℰ₄(v_i, v_j,  v_k, v_l) χ₀^j ξ₁^k ξ₁^l\\
%   + ℰ₃(v_i, v_j, v_k) χ₀^j ξ₂^k + ℰ₃(v_i, v_j, w_{kl}) χ₀^j ξ₁^k ξ₁^l + λ₁ ℰ₃(v_i, v_j, w_k) χ₀^j ξ₁^k\\
%   + 2λ₁ \dot{ℰ}₃(v_i, v_j,  v_k) χ₀^j ξ₁^k + λ₁² \ddot{ℰ}₂(v_i, v_j) χ₀^j + λ₂ \dot{ℰ}₂(v_i, v_j) χ₀^j = 2α₁χ₁^i + α₂ χ₀^i.
% \end{multline*}
\begin{multline*}
  \bigl[ ℰ₄(v_i, v_j,  v_k, v_l) + 2ℰ₃(v_i, w_{jk}, v_l) + ℰ₃(v_i, v_j, w_{kl})\bigr] χ₀^j ξ₁^k ξ₁^l\\
  + λ₁ \bigl[ ℰ₃(v_i,  w_j, v_k) + 2 \dot{ℰ}₂(v_i, w_{jk}) + ℰ₃(v_i, v_j, w_k) + 2 \dot{ℰ}₃(v_i, v_j,  v_k) \bigr] χ₀^j ξ₁^k\\
  + λ₁² \bigl[\dot{ℰ}₂(v_i, w_j) + \ddot{ℰ}₂(v_i, v_j)\bigr] χ₀^j + \bigl[ℰ₃(v_i, v_j, v_k) ξ₂^k + λ₂ \dot{ℰ}₂(v_i, v_j)\bigr] χ₀^j \\
  +2\bigl[ℰ₃(v_i, v_j,  v_k)  ξ₁^k + λ₁ \dot{ℰ}₂(v_i, v_j)\bigr] χ₁^j = 2α₁χ₁^i + α₂ χ₀^i.
\end{multline*}

The \(χ₀^j ξ₁^k\) term is transformed with Eqs.~\eqref{eq:20220524134525} and
\eqref{eq:20220519164523}
\begin{multline*}
  \bigl[ ℰ₄(v_i, v_j,  v_k, v_l) + ℰ₃(v_i, w_{jk}, v_l) + ℰ₃(v_i, w_{jl}, v_k) + ℰ₃(v_i, v_j, w_{kl})\bigr] χ₀^j ξ₁^k ξ₁^l\\
  + λ₁ \bigl[ -ℰ₂(w_{ik},  w_j) - ℰ₂(w_i, w_{jk}) - ℰ₂(w_{ij}, w_k) + 2 \dot{ℰ}₃(v_i, v_j,  v_k) \bigr] χ₀^j ξ₁^k\\
  + λ₁² \bigl[\dot{ℰ}₂(v_i, w_j) + \ddot{ℰ}₂(v_i, v_j)\bigr] χ₀^j + \bigl[ℰ₃(v_i, v_j, v_k) ξ₂^k + λ₂ \dot{ℰ}₂(v_i, v_j)\bigr] χ₀^j \\
  +2\bigl[ℰ₃(v_i, v_j,  v_k)  ξ₁^k + λ₁ \dot{ℰ}₂(v_i, v_j)\bigr] χ₁^j = 2α₁χ₁^i + α₂ χ₀^i,
\end{multline*}
and Eq.~\eqref{eq:20220616082923} results from the application of
Eqs.~\eqref{eq:20220617084433} and \eqref{eq:20220617085256}.

\end{document}

%%% Local Variables:
%%% coding: utf-8
%%% fill-column: 80
%%% mode: latex
%%% TeX-engine: xetex
%%% TeX-master: t
%%% End:
