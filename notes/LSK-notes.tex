\newcommand{\sbtitle}{Notes relatives à la méthode asymptotique de Lyapunov–Schmidt–Koiter}
\newcommand{\sbauthor}{Sébastien Brisard}
\newcommand{\sbemail}{sebastien.brisard@univ-eiffel.fr}
\newcommand{\sbaddress}{Univ Gustave Eiffel, Ecole des Ponts, IFSTTAR, CNRS, Navier, F-77454 Marne-la-Vall\'ee, France}
\newcommand{\sbsubject}{Note bibliographique}

\documentclass[12pt, final]{scrartcl}
\setkomafont{disposition}{\rmfamily}

\usepackage{polyglossia}
\setdefaultlanguage{french}

\usepackage{amsfonts}
\usepackage{amsmath}
\usepackage{amssymb}

\usepackage{amsthm}
\theoremstyle{definition}
\renewcommand{\qedsymbol}{}
\newtheorem{remark}{Remarque}
\newtheorem{theorem}{Théorème}

\usepackage[backend=biber,bibencoding=utf8,doi=false,giveninits=true,isbn=false,maxnames=10,minnames=5,sortcites=true,style=authoryear,texencoding=utf8,url=false]{biblatex}
\addbibresource{stab.bib}

\usepackage[breaklinks=true, colorlinks=true, pdftitle={\sbtitle}, pdfauthor={\sbauthor}, pdfsubject={\sbsubject}, urlcolor=blue]{hyperref}

\usepackage[color={1 1 0}]{pdfcomment}

\usepackage{unicode-math}
% \setmainfont{XITS}
% \setmathfont{XITS Math}
\setmainfont{Asana Math}
\setmathfont{Asana Math}

\newcommand{\reals}{\mathbb{R}}

\begin{document}
\title{\sbtitle}
\author{\sbauthor\thanks{\sbaddress~--- \sbemail}}
\maketitle

\begin{abstract}
 blabla
\end{abstract}


\section{Notations}

L'espace des champs cinématiquement admissibles est noté \(U\). On suppose qu'il
a la structure d'espace vectoriel. L'énergie du système est notée \(ℰ(u, λ)\),
où \(λ\) désigne un paramètre de chargement. Soit \(u^{\ast}(λ)\) la branche
fondamentale. Par définition
\begin{equation}
  ℰ_{,u}[u^{\ast}(λ), λ; \hat{u}]=0 \quad \text{pour tout} \quad \hat{u}∈U.
\end{equation}
Il sera commode d'introduire les notations suivantes
\begin{equation}
  ℰ₂(λ) = ℰ_{,uu}[u^{\ast}(λ), λ], \quad ℰ₃(λ) = ℰ_{,uuu}[u^{\ast}(λ), λ], \quad ℰ₄(λ) = ℰ_{,uuuu} [u^{\ast}(λ), λ].
\end{equation}
Noter que \(ℰ₂\), \(ℰ₃\) et \(ℰ₄\) sont des formes bi-, tri- et
quadri-linéaires, respectivement. L'application de ces formes à des éléments de
\(U\) sera notée \(ℰ₂(λ; u, v)\), \(ℰ₃(λ; u, v, w)\), etc. La dérivée de ces
formes par rapport à \(λ\) sera notée à l'aide d'un point supérieur
(\(\dot{ℰ}_2\), \(\dot{ℰ}_3\), \dots).

On suppose que l'équilibre est stable pour des valeurs suffisamment petites de
\(λ\). Plus précisément, on suppose que \(ℰ₂(λ)\) est définie positive pour tout
\(λ < λ₀\). Pour \(λ = λ₀\), la forme quadratique \(ℰ₂(λ₀)\) n'est plus que
positive. En notant \(u₀ = u^{\ast}(λ₀)\) la position d'équilibre obtenue pour
la valeur critique \(λ₀\) du paramètre de chargement \(λ\), on s'intéresse à
toutes les courbes d'équilibre qui passent par le point \((u₀, λ₀)\).

Noter que dans ce qui suit, on convient que les formes \(ℰ₂\), \(ℰ₃\) et \(ℰ₄\)
sont implicitement évaluées en \(λ₀\) lorsque \(λ\) n'est pas rappelé : ainsi,
on notera \(ℰ₂(•, •)\) plutôt que \(ℰ₂(λ₀ ; •, •)\).

\section{Analyse de la branche fondamentale}

On s'intéresse dans ce paragraphe à la stabilité du point critique
\((u₀, λ₀) .\) Par hypothèse, \(ℰ₂(λ₀)\) est positive, sans être définie
positive~; soit \(V\) son noyau, qui forme un sous-espace vectoriel de \(U\). On
suppose que \(V\) est de dimension finie \(m = \dim V\). Soit
\((v₁, \ldots, v_m)\) une base orthonormée de ce noyau pour le produit scalaire
\(〈 •, • 〉\)(qui n'est pas précisé pour le moment). On introduit le
sous-espace supplémentaire orthogonal \(W\) de \(V\) dans \(U\)
\begin{equation}
  U = V \overset{\perp}{\otimes} W.
\end{equation}
Pour étudier la stabilité de l'équilibre, on calcule l'énergie dans un état
\(u₀ + ξ  v + η  w\) voisin du point d'équilibre \(u₀\), avec \(ξ, η∈\reals\)
{\guillemotleft} petits {\guillemotright}, \(v \in V\) et \(w∈W\). On obtient
alors, à l'ordre 4 en \(ξ\) et \(η\)
\begin{equation}
  \begin{aligned}[b]
    Δℰ ={} &  ℰ(u₀ + ξv + ηw, λ₀) - ℰ(u₀, λ₀)\\
    ={} & \tfrac{1}{2} ℰ₂(ξv + ηw, ξv + ηw) + \tfrac{1}{6} ℰ₃(ξv + ηw, ξv + ηw, ξv + ηw)\\
    & + \tfrac{1}{24} ℰ₄(ξv + ηw, ξv + ηw, ξv + ηw, ξv + ηw) + \mathcal{O}\bigl[\bigl(ξ^2 + η^2\bigr)^2\bigr],
  \end{aligned}
\end{equation}
où le terme linéaire a été omis puisque \(u₀\) est un point critique de
l'énergie. En tenant compte de la multilinéarité et de la symétrie des
différentielles successives de l'énergie \(ℰ\), ainsi que du fait que
\(ℰ₂(v, •) = 0\) (puisque \(v∈V\)), l'expression précédente s'écrit
\begin{equation}
  \begin{aligned}[b]
    \Delta ℰ ={} & \tfrac{1}{2} η^2 ℰ₂(w, w) + \tfrac{1}{6} ξ^3 ℰ₃(v, v, v) + \tfrac{1}{2} ξ^2 η ℰ₃(v, v, w)\\
    & + \tfrac{1}{2} ξ η^2 ℰ₃(v, w, w) + \tfrac{1}{6} η^3 ℰ₃(w, w, w)\\
    & + \tfrac{1}{24} ξ^4 ℰ₄(v, v, v, v) + \tfrac{1}{6} ξ^3 η ℰ₄(v, v, v, w)\\
    & + \tfrac{1}{4} ξ^2 η^2 ℰ₄(v, v, w, w) + \tfrac{1}{6} ξ η^3 ℰ₄(v, w, w, w)\\
    & + \tfrac{1}{24} η^4 ℰ₄(w, w, w, w) +\mathcal{O}\bigl[\bigl(ξ^2 + η^2\bigr)^2\bigr],
  \end{aligned}
\end{equation}
où l'on convient que toutes les différentielles de \(ℰ\) sont évaluées au point
d'équilibre \(u₀\).

Pour que l'équilibre soit stable, il faut que cette expression soit positive ou
nulle pour tous \(ξ\) et \(η\) suffisamment petits. En prenant tout d'abord
\(η = 0\), on obtient les conditions nécessaires
\begin{equation}
  \label{eq:20211108164416}
  ℰ₃(v, v, v) = 0 \quad \text{et} \quad ℰ₄(v, v, v, v) \geq 0 \quad \text{pour tout} \quad v∈V.
\end{equation}

En d'autres termes, s'il existe \(v∈V\) tel que \(ℰ₃(v, v, v) \neq 0\) ou
\(ℰ₄(v, v, v, v) < 0\), alors l'équilibre est \emph{instable}. Les conditions
précédentes ne sont pas suffisantes pour assurer la stabilité. En effet,
supposant ces conditions remplies, on prend maintenant \(η = ξ^2\)
\begin{equation}
  Δℰ = \tfrac{1}{2} ξ^4 \bigl[ ℰ₂(w, w) + ℰ₃(v, v, w) + \tfrac{1}{12} ℰ₄(v, v, v, v) \bigr] + o(ξ^4)
\end{equation}
et on obtient la condition nécessaire supplémentaire
\begin{equation}
  \label{eq:20211109145356}
  ℰ₂(w, w) + ℰ₃(v, v, w) + \tfrac{1}{12} ℰ₄(v, v, v, v) \geq 0,
\end{equation}
pour tous \(v∈V\) et \(w∈W\). Pour \(v∈V\) fixé, l'expression précédente est
minimale lorsque \(w\) satisfait le problème variationnel
\begin{equation}
  \label{eq:20211109145224}
  2ℰ₂(w, \hat{w}) +ℰ₃(v, v, \hat{w}) = 0 \quad \text{pour tout} \quad \hat{w}∈W.
\end{equation}
Soit \(w_{ij}∈W\) l'unique solution du problème variationnel suivant
\begin{equation}
  \label{eq:20220519164523}
  ℰ₂(w_{i j}, \hat{w})+ℰ₃(v_i, v_j, \hat{w}) = 0 \quad \text{pour tout} \quad \hat{w}∈W.
\end{equation}
Alors, pour \(v = ξ^i v_i\), la solution du problème
variationnel~\eqref{eq:20211109145224} est \(w = \tfrac{1}{2} ξ^i ξ^j
w_{ij}\). Pour cette valeur de \(v\), la condition~\eqref{eq:20211109145356}
s'écrit
\begin{equation}
  \bigl[ℰ₄(v_i, v_j, v_k, v_l) - 3ℰ₂(w_{ij}, w_{kl})\bigr] ξ^i ξ^j ξ^k ξ^l \geq 0,
\end{equation}
pour tous \(ξ^i, ξ^j, ξ^k, ξ^l∈\reals\). On peut montrer que l'inégalité stricte
est une condition \emph{suffisante} de
stabilité. \pdfmargincomment{Est-ce vrai ? Essayer de préciser}

\section{Bifurcations}

On écrit toute courbe d'équilibre passant par le point \((u₀, λ₀)\) sous la
forme paramétrique suivante
\begin{align}
  \label{eq:20211115075817}
  λ &=  λ₀ + η λ₁ + \tfrac{1}{2} η^2 λ₂ + \tfrac{1}{6} η^3 λ₃ + \cdots,\\
  \label{eq:20211115075835}
  u &= u^{\ast}(λ) + η u₁ + \tfrac{1}{2} η^2 u₂ + \tfrac{1}{6} η^3 u₃ + \cdots,
\end{align}
où \(η\) est un paramètre, non précisé pour le moment. Noter que, dans la
représentation paramétrique de \(u\), \(u^{\ast}\) est évalué en \(λ\) et pas en
\(λ₀\).

Les coefficients \(λ_k\) et \(u_k\) des développements~\eqref{eq:20211115075817}
et \eqref{eq:20211115075835} sont identifiés en écrivant que l'énergie est
stationnaire le long de la courbe d'équilibre, c'est-à-dire que le résidu
\(ℰ_{, u} [u(η), λ(η)]\) est nul. Le développement limité du résidu est établi
au voisinage de \(η = 0\) dans l'annexe~\ref{sec:20211112182000} [voir
Éq.~\eqref{eq:20220107080901}]. En écrivant que tous ses termes s'annulent, on
trouve successivement, pour tout \(\hat{u}∈U\)
\begin{equation}
  \label{eq:20211112182917}
  ℰ₂(λ₀; u₁, \hat{u}) = 0,
\end{equation}
\begin{equation}
  \label{eq:20220524133447}
  ℰ₃(λ₀; u₁, u₁, \hat{u}) + 2λ₁\dot{ℰ}₂(λ₀; u₁, \hat{u}) + ℰ₂(λ₀; u₂, \hat{u}) = 0,
\end{equation}
\begin{equation}
  \label{eq:res3}
  \begin{aligned}[b]
    ℰ₄(λ₀; u₁, u₁, u₁, \hat{u}) + 3ℰ₃(λ₀; u₁, u₂, \hat{u}) + ℰ₂(λ₀; u₃, \hat{u})&\\
    + 3λ₁\dot{ℰ}₃(λ₀; u₁, u₁, \hat{u}) + 3λ₁\dot{ℰ}₂(λ₀;  u₂, \hat{u})&\\
    + 3λ₁^2\ddot{ℰ}₂(λ₀; u₁, \hat{u}) + 3λ₂\dot{ℰ}₂(λ₀; u₁, \hat{u}) & = 0.
  \end{aligned}
\end{equation}
On déduit de l'équation~\eqref{eq:20211112182917} que \(u₁∈V\). En prenant la
fonction test également dans \(V\), on déduit de
l'équation~\eqref{eq:20220524133447} que \(u₁\) est solution du problème
suivant~: trouver \(u₁∈V\) tel que
\begin{equation}
  \label{eq:20220524133816}
  \tfrac{1}{2} ℰ₃(λ₀; u₁, u₁, \hat{v}) + λ₁\dot{ℰ}₂(λ₀; u₁, \hat{v}) = 0,
\end{equation}
pour tout \(\hat{v}∈V\). Il est commode de transformer l'équation de bifurcation
\eqref{eq:20220524133816}, intrinsèque, en un système d'équations scalaires. à
cet effet, on décompose \(u₁∈V\) dans la base \((v_i)_{1 ≤ i ≤ m}\)
\begin{equation}
  \label{eq:20220524133944}
  u₁ = ξ₁^i v_i.
\end{equation}
En prenant \(\hat{v} = v_i\), l'équation~\eqref{eq:20220524133816} s'écrit
\begin{equation}
  \label{eq:20220524134121}
  \tfrac{1}{2} ℰ₃(λ₀; v_i, v_j, v_k)ξ₁^j ξ₁^k + λ₁ \dot{ℰ}₂(λ₀; v_i, v_j) ξ₁^j = 0.
\end{equation}
On obtient ainsi un système de \(m\) équations quadratiques à \((m + 1)\)
inconnues, qui permet en général de déterminer les valeurs de \(λ₁\) et
\(u₁\)(\pdfmarkupcomment{voir discussion ci-après}{Compléter référence}).

Afin de déterminer les termes suivants du développement asymptotique de la
branche bifurquée, soit \(λ₂\) et \(u₂\), on introduit la décomposition
\begin{equation}
  u₂ = ξ₂^i v_i + \tilde{u}₂,
\end{equation}
où \(\tilde{u}₂∈W\) est la projection orthogonale de \(u₂\) sur \(W\). On a
alors \(ℰ₂(u₂, \hat{u}) =ℰ₂(\tilde{u}₂, \hat{u})\) et
l'équation~\eqref{eq:20220524133447} s'écrit
\begin{equation}
 ℰ₃(λ₀; u₁, u₁, \hat{u}) + 2λ₁ \dot{ℰ}₂(λ₀; u₁, \hat{u}) + ℰ₂(λ₀; \tilde{u}₂, \hat{u}) = 0,
\end{equation}
pour tout \(\hat{u}∈U\). En prenant cette fois-ci la fonction test dans l'espace
\(W\), on obtient le problème variationnel suivant~: trouver \(\tilde{u}₂∈W\)
tel que
\begin{equation}
  \label{eq:20211210131623}
  ℰ₂(λ₀; \tilde{u}₂, \hat{w}) + ξ₁^i ξ₁^j ℰ₃(λ₀; v_i, v_j, \hat{w}) + 2λ₁ ξ₁^i \dot{ℰ}₂(λ₀; v_i, \hat{w}) = 0,
\end{equation}
pour tout \(\hat{w}∈W\). Soient \(w_i∈W\) les solutions des problèmes
variationnels suivants
\begin{equation}
  \label{eq:20220524134525}
  ℰ₂(λ₀; w_i, \hat{w}) + 2\dot{ℰ}₂(λ₀; v_i, \hat{w}) = 0,
\end{equation}
pour tout \(\hat{w}∈W\). La solution du problème~\eqref{eq:20211210131623}
s'obtient par simple combinaison linéaire des \(w_i\) et \(w_{ij}\) [on rappelle
que ces derniers sont définis par le problème
variationnel~\eqref{eq:20220519164523}]
\begin{equation}
 \tilde{u}₂ = ξ₁^i ξ₁^j w_{ij} + λ₁ ξ₁^i w_i,
\end{equation}
de sorte que
\begin{equation}
  \label{eq:20220524134613}
  u₂ = ξ₂^i v_i + ξ₁^i ξ₁^j w_{ij} + λ₁ ξ₁^i w_i .
\end{equation}
En introduisant les expressions~\eqref{eq:20220524133944} et
\eqref{eq:20220524134613} dans l'équation~\eqref{eq:res3} et en prenant de plus
\(\hat{u} = v_i\), on obtient alors les équations suivantes
\begin{equation}
  \label{eq:20211112183220}
  \begin{aligned}[b]
    3[ℰ₃(λ₀; v_i, v_j, v_k) ξ₁^k + λ₁ \dot{ℰ}₂(λ₀; v_i, v_j)] ξ₂^j + 3λ₂ \dot{ℰ}₂(λ₀; v_i, v_j) ξ₁^j &\\
    + [ℰ₄(λ₀; v_i, v_j, v_k, v_l) + 3ℰ₃(λ₀; v_i, v_j, w_{kl})] ξ₁^j ξ₁^k ξ₁^l &\\
    + 3λ₁ [\dot{ℰ}₃(λ₀; v_i, v_j, v_k) + ℰ₃(λ₀; v_i, v_j, w_k) + \dot{ℰ₂}(λ₀; v_i, w_{jk})] ξ₁^j ξ₁^k&\\
    + 3λ₁^2 [\ddot{ℰ}₂(λ₀; v_i, v_j) + \dot{ℰ}₂(λ₀ ; v_i, w_j)] ξ₁^j &= 0,
  \end{aligned}
\end{equation}
qui permet en principe de déterminer \(λ₂\) ainsi que les \(ξ₂^i\). On montre
dans le paragraphe \ref{sec:20220524134954} que les équations
\eqref{eq:20220524134121} et \eqref{eq:20211112183220} peuvent s'écrire sous la
forme suivante
\begin{equation}
  \label{eq:20220524135036}
  \tfrac{1}{2} E_{ijk}(λ₀) ξ₁^j ξ₁^k + λ₁ F_{ij}(λ₀) ξ₁^j = 0,
\end{equation}
\begin{equation}
  \label{eq:20220601070917}
  \tfrac{1}{3} E_{ijkl}(λ₀) ξ₁^j ξ₁^k ξ₁^l + λ₂ F_{ij}(λ₀) ξ₁^j + A_{ij}(λ₀) ξ₂^j + λ₁ \dot{A}_{ij}(λ₀) ξ₁^j = 0,
\end{equation}
où les tenseurs \(E_{i j k}\), \(E_{i j k l}\), \(F_{i j}\) et \(A_{i j}\) sont
définis comme suit \pdfmargincomment{Je ne suis pas sûr du terme faisant
  intervenir Ȧᵢⱼ(λ₀)}
\begin{equation}
  \label{eq:20220524135619}
  E_{ijk}(λ) = ℰ₃(λ; v_i, v_j, v_k) + ℰ₂(λ; v_i, w_{jk}) + ℰ₂(λ; v_j, w_{ik}) + ℰ₂(λ ; v_k, w_{ij}),
\end{equation}
\begin{equation}
  \label{eq:20220524135553}
  E_{i j k l}(λ) = ℰ₄(λ ; v_i, v_j, v_k, v_l) + ℰ₃(λ ; v_i, v_j, w_{k l}) + ℰ₃(λ ; v_i, v_k, w_{l j}) + ℰ₃(λ ; v_i, v_l, w_{j k}),
\end{equation}
\begin{equation}
  \label{eq:20220524135643}
  F_{ij}(λ) = \dot{ℰ}₂(λ; v_i, v_j) + \tfrac{1}{2} [ℰ₂(λ; v_i, w_j) + ℰ₂(λ; v_j, w_i)],
\end{equation}
\begin{equation}
  \label{eq:20220524135705}
  A_{ij}(λ) = E_{ijk}(λ) ξ₁^k + λ₁ F_{ij}(λ) .
\end{equation}
Noter que tous ces tenseurs sont \emph{symétriques}. On remarque que, puisque
\(ℰ₂(λ₀ ; v_i, •) = 0\), on a les simplifications suivantes en \(λ = λ₀\) :
\(E_{ijk}(λ₀) = ℰ₃(λ₀; v_i, v_j, v_k)\) et
\(F_{ij}(λ₀) = \dot{ℰ}₂(λ₀; v_i, v_j)\).

\paragraph{Si la forme \(ℰ₃(λ₀)\) n'est pas nulle sur \(V\)} L'équation
\eqref{eq:20220524135036} admet au plus \((2^m - 1)\) paires de solutions
réelles \((λ₁, u₁)\) et \((- λ₁, - u₁)\). \pdfmargincomment{Je ne sais pas
  démontrer ce résultat sur le nombre de solutions réelles.}

\paragraph{Si la forme \(ℰ₃(λ₀)\) est nulle sur \(V\)} L'équation
\eqref{eq:20220524133816} conduit nécessairement à \(λ₁ = 0\), puisque
\(\dot{ℰ}₂(λ₀)\) est définie négative. Dès lors, l'équation
\eqref{eq:20220601070917} s'écrit \pdfmargincomment{Expliquer pourquoi la forme
  quadratique ℰ̇₂(λ₀) bien définie négative}
\begin{equation}
 \tfrac{1}{3} E_{ijkl}(λ₀) ξ₁^j ξ₁^k ξ₁^l + λ₂ F_{ij}(λ₀) ξ₁^j = 0.
\end{equation}
Cette équation admet cette fois au plus \(\frac{3^m - 1}{2}\) paires de
solutions réelles \((λ₂, u₁)\) et \((- λ₂, - u₁)\). \pdfmargincomment{Je ne sais
  pas non plus démontrer ce résultat sur le nombre de solutions réelles.}
\pdfmargincomment{Note du 29/04/2022 – J'ai relu tous les calculs précédents. Il
  reste à reprendre les calculs des développements asymptotiques de l'énergie et
  de sa hessienne, pour tenir compte en particulier des factorielles introduites
  maintenant dans les développements asymptotiques. Il faudrait également
  introduire les tenseurs précédents dans les expressions de l'énergie et de sa
  hessienne.}

Le développement limité suivant de l'énergie le long de la branche bifurquée est
établi dans l'annexe~\ref{sec:20220525053434}
\begin{equation}
  \label{eq:20220525053600}
  \begin{aligned}[b]
    ℰ[u(η), λ(η)] ={}
    & ℰ\{u^{\ast}[λ(η)], λ(η)\} + \tfrac{1}{6} λ₁ η^3 F_{i j}(λ₀) ξ₁^i ξ₁^j\\
    & + \tfrac{1}{24} η^4 \bigl\{E_{ijkl}(λ₀) ξ₁^i ξ₁^j ξ₁^k ξ₁^l + 4λ₁ \dot{E}_{ijk}(λ₀) ξ₁^i ξ₁^j ξ₁^k\\
    & + 6 \bigl[λ₁^2 \dot{F}_{ij}(λ₀) + λ₂ F_{ij}(λ₀)\bigr] ξ₁^i ξ₁^j\bigr\} + o(η^4).
  \end{aligned}
\end{equation}
Si \(λ₁ ≠ 0\), le premier terme non-nul du développement limité précédent est
d'ordre 3
\begin{equation}
 ℰ [u(η), λ(η)] = ℰ\{u^{\ast}[λ(η)], λ(η)\} + \tfrac{1}{6} λ₁ η^3 F_{ij}(λ₀) ξ₁^i ξ₁^j + o(η^3),
\end{equation}
tandis que si \(λ₁ = 0\), le premier terme est d'ordre 4
\begin{equation}
 ℰ[u(η), λ(η)] = ℰ\{u^{\ast} [λ(η)], λ(η)\} + \tfrac{1}{4} λ₂ η^4 F_{ij}(λ₀) ξ₁^i ξ₁^j + o(η^4).
\end{equation}
\begin{center}
 ***
\end{center}

Pour analyser la stabilité de la branche bifurquée ainsi trouvée, il faut
déterminer le signe de la hessienne de l'énergie. On peut d'ores et déjà
remarquer que, sur la branche fondamentale(\(u₁ = u₂ = 0\)), en prenant
\(η = λ - λ₀\) (\(λ₁ = 1\))
\begin{equation}
 ℰ₂(λ; \hat{u}, \hat{v}) = ℰ₂(λ₀; \hat{u}, \hat{v}) + \bigl(λ - λ₀\bigr) \dot{ℰ}₂(λ₀; \hat{u}, \hat{v}) + o(λ - λ₀).
\end{equation}

Dans ce qui suit, on supposera que \(\dot{ℰ}₂(λ₀) ≠ 0\). Pour \(\hat{v}∈V\),
l'égalité précédente s'écrit
\begin{equation}
 ℰ₂(λ₀; \hat{v}, \hat{v}) = \bigl(λ - λ₀\bigr)\dot{ℰ}₂(\hat{v}, \hat{v}) + o(λ - λ₀).
\end{equation}

Comme la branche fondamentale est stable pour \(λ < λ₀\), on doit avoir
\(\dot{ℰ}₂(λ₀ ; \hat{v}, \hat{v}) < 0\). La forme quadratique \(\dot{ℰ}₂(λ₀)\)
est donc définie négative sur \(V\). Le développement limité de la hessienne de
l'énergie le long de la branche bifurquée est établi \pdfmarkupcomment{dans
  l'annexe}{xref}. Pour tous \(\hat{u}, \hat{v}∈U\), on trouve
\begin{equation}
  \label{eq:20220531054247}
  \begin{aligned}[b]
    ℰ_{, uu}[u(η), λ(η); \hat{u}, \hat{v}] ={}
    & ℰ₂(λ₀ ; \hat{u}, \hat{v}) + η \bigl[ℰ₃(λ₀ ; u₁, \hat{u}, \hat{v})  + λ₁ \dot{ℰ}₂(λ₀; \hat{u}, \hat{v})\bigr]\\
    & + \tfrac{1}{2} η^2 \bigl[ℰ₄(λ₀; u₁, u₁, \hat{u}, \hat{v}) + ℰ₃(λ₀; u₂, \hat{u}, \hat{v}) + λ₂ \dot{ℰ}₂(λ₀; \hat{u}, \hat{v})\\
    & + 2 λ₁ \dot{ℰ}₃(λ₀; u₁, \hat{u}, \hat{v}) + λ₁^2 \ddot{ℰ}₂(λ₀; \hat{u}, \hat{v}) \bigr] + o(η^2).
  \end{aligned}
\end{equation}
Pour une analyse de stabilité, on doit prendre \(\hat{u} = \hat{v}\), soit
\begin{equation}
  \begin{aligned}[b]
    \label{eq:20220531054218}
    ℰ_{, uu}[u(η), λ(η) ; \hat{u}, \hat{u}] ={}
    & ℰ₂(λ₀; \hat{u}, \hat{u}) + η [ℰ(λ₀; u₁, \hat{u}, \hat{u}) + λ₁ \dot{ℰ}₂ (λ₀; \hat{u}, \hat{u})]\\
    & + \tfrac{1}{2} η² [ℰ₄(λ₀; u₁, u₁, \hat{u}, \hat{u}) +ℰ₃(λ₀; u₂, \hat{u}, \hat{u}) + λ₂ \dot{ℰ}₂(λ₀; \hat{u}, \hat{u})\\
    & + 2 λ₁ \dot{ℰ}₃(λ₀; u₁, \hat{u}, \hat{u}) + λ₁^2 \ddot{ℰ}₂(λ₀; \hat{u}, \hat{u})] + o(η^2) .
  \end{aligned}
\end{equation}

On peut décomposer le vecteur \(\hat{u}∈U\) de façon unique sous la forme
\(\hat{u} = \hat{v} + \hat{w}\), avec \(\hat{v}∈V\) et \(\hat{w}∈W\). Le terme
constant du développement précédent vaut alors \(ℰ₂(λ₀; \hat{w}, \hat{w})\). Si
\(\hat{w} ≠ 0\), alors ce terme constant est strictement positif, puisque la
hessienne est définie positive sur \(W\) en \(λ = λ₀\). Au voisinage du point de
bifurcation, la hessienne sur la branche bifurquée est donc positive pour tout
\(\hat{u}∈U\) ayant une composante dans \(W\). Il suffit donc d'étudier le signe
de la hessienne sur la branche bifurquée pour \(\hat{u}∈V\), soit
\(\hat{u} = \hat{ξ}^i v_i\). L'expression~\eqref{eq:20220531054218} se simplifie
alors sous la forme suivante Compte-tenu de
l'expression~\eqref{eq:20220524134613}
\begin{equation}
  \begin{aligned}[b]
    ℰ_{, uu}[u(η), λ(η); \hat{u}, \hat{u}] ={}
    & η [ℰ₃(λ₀; v_i, v_j, v_k) ξ₁^k + λ₁ \dot{ℰ}₂(λ₀; v_i, v_j)] \hat{ξ}^i \hat{ξ}^j\\
    & + \tfrac{1}{2} η^2 \bigl[ℰ₄(λ₀; v_i, v_j, v_k, v_l) ξ₁^k ξ₁^l + ℰ₃(λ₀; v_i, v_j, v_k) ξ₂^k\\
    & + ℰ₃(λ₀; v_i, v_j, w_{k l}) ξ₁^k ξ₁^l + λ₁ ℰ₃(λ₀; v_i, v_j, w_k) ξ₁^k + λ₂ \dot{ℰ}₂(λ₀; v_i, v_j)\\
    & + 2 λ₁ \dot{ℰ}₃(λ₀; v_i, v_j, v_k) ξ₁^k + λ₁^2 \ddot{ℰ}₂(λ₀; v_i, v_j)\bigr] \hat{ξ}^i \hat{ξ}^j + o(η^2).
  \end{aligned}
\end{equation}
Si \(λ₁ ≠ 0\), il suffit d'étudier le signe de la forme quadratique
\([E_{i j k}(λ₀) ξ₁^k + λ₁ F_{i j}(λ₀)].\) Si \(λ₁ = 0\) et que \(ℰ₃(λ₀) = 0\)
sur \(V\), alors le développement limité précédent s'écrit
\pdfmargincomment{12/05/2022 Relecture jusqu'à l'égalité précédente. Je suis un
  peu surpris, car je m'attendais à un terme en 3ℰ₃(λ₀; vᵢ, vⱼ, wₖₗ).}
\begin{equation}
  \begin{aligned}[b]
    ℰ_{, uu} [u(η), λ(η); \hat{u}, \hat{u}] ={}
    & \tfrac{1}{2} η^2 \bigl\{ \bigl[ℰ₄(λ₀; v_i, v_j, v_k, v_l) + ℰ₃(λ₀; v_i, v_j, w_{k l})\bigr] ξ₁^k ξ₁^l\\
    & + λ₂ \dot{ℰ}₂(λ₀; v_i, v_j) \bigr\} \hat{ξ}^i \hat{ξ}^j + o(η^2)
  \end{aligned}
\end{equation}

Compte-tenu de la relation~\eqref{eq:20211112183220}, on trouve pour
\(\hat{v} = u₁\) (\(\hat{ξ}^i = ξ₁^i\))
\begin{equation}
 ℰ_{, uu} [u(η), λ(η); u₁, u₁] = -λ₁ η \dot{ℰ}₂(λ₀; u₁, u₁) + o(η).
\end{equation}
Si \(λ₁ \neq 0\), l'expression précédente peut également s'écrire
\begin{equation}
 ℰ_{, uu} [u(η), λ(η); u₁, u₁] = -(λ - λ₀) \dot{ℰ}₂(λ₀; u₁, u₁) + o(λ - λ₀),
\end{equation}
qui est négative pour \(λ < λ₀\): la branche bifurquée est instable sous la
charge critique. Il reste alors à étudier le signe de la hessienne de la branche
bifurquée au-delà de la charge critique (\(λ > λ₀\)).

\section{Propriétés des formes bilinéaires symétriques, positives}

Dans ce qui suit, \(\mathcal{B}\) désigne une forme bilinéaire symétrique et
positive sur l'espace vectoriel \(U\). On définit son noyau \(\ker \mathcal{B}\)
de la façon suivante
\begin{equation}
 \ker \mathcal{B}= \bigl\{ u ∈ U, \mathcal{B}(u, u) = 0 \bigr\} .
\end{equation}

\begin{theorem}
  Le noyau d'une forme bilinéaire, symétrique et positive est un sous-espace
  vectoriel.
\end{theorem}
\begin{proof}
  Soient \(u, v∈\ker \mathcal{B}\), \(α∈\reals\) et \(w = u + α v\). Montrons
  que \(w ∈ \ker\mathcal{B}\). Il suffit d'évaluer \(\mathcal{B}(w, w)\)
 \begin{equation}
   \mathcal{B}(w, w) = \mathcal{B}(u + α v, u + α v)
   = \mathcal{B}(u, u) + 2 α \mathcal{B}(u, v) + α² \mathcal{B}(v, v),
 \end{equation}
 où l'on a tenu compte de la symétrie de \(\mathcal{B}\) pour écrire que
 \(\mathcal{B}(u, v) =\mathcal{B}(v, u)\). Comme \(u, v ∈ \ker\mathcal{B}\), le
 premier et le dernier terme sont nuls, soit
 \(\mathcal{B}(w, w) = 2α \mathcal{B}(u, v)\). La forme bilinéaire étant
 positive, cette grandeur est positive, \emph{quelle que soit la valeur de
   \(α∈\reals\)}. On en déduit donc que \(\mathcal{B}(u, v) = 0\), puis que
 \(\mathcal{B}(w, w) = 0\) et donc que \(w ∈ \ker\mathcal{B}\).
\end{proof}

\begin{theorem}
 Soit \(u∈V\). Alors
 \begin{equation}
  u ∈ \ker\mathcal{B} \quad \text{ssi} \quad \text{pour tout } v ∈ V, \mathcal{B}(u, v) = 0.
 \end{equation}
\end{theorem}

\begin{proof}
  Soient \(u∈\ker \mathcal{B}\), \(v∈V\) et \(α∈\reals\). Comme précédemment, on
  écrit que \(\mathcal{B}(w, w) ≥ 0\), avec \(w = α u + v\)
 \begin{equation}
  \mathcal{B}(w, w) = 2 α \mathcal{B}(u, v) +\mathcal{B}(v, v) \geq
  0,
 \end{equation}
 où l'on a tenu compte de ce que \(\mathcal{B}(u, u) = 0\). L'expression
 précédente, affine en \(α\), a un signe constant. Le terme linéaire en \(α\)
 est donc nul, soit \(\mathcal{B}(u, v) = 0\).  Réciproquement, si
 \(\mathcal{B}(u, v) = 0\) pour tout \(v∈V\), alors \(\mathcal{B}(u, u) = 0\)(en
 prenant \(v = u\)).
\end{proof}

\section{Développements limités le long d'une branche bifurquée du diagramme d'équilibre}

\subsection{Principe du calcul}
\label{sec:20220107121442}

On pose dans ce qui suit
\begin{align}
  \label{eq:20211112155446}
  λ(η) & = λ(η) - λ₀ = η λ₁ + \tfrac{1}{2} η^2 λ₂ + \tfrac{1}{6} η^3 λ₃ + \cdots,\\
 \label{eq:20211112113028}
  U(η) & = u(η) - u^{\ast}[λ(η)] = η u₁ + \tfrac{1}{2} η^2 u₂ + \tfrac{1}{6} η^3 u₃ + \cdots.
\end{align}

On considère une fonctionnelle \(\mathcal{F}\) de \(u\) et \(λ\)~:
\(\mathcal{F}(u, λ)\). Cette fonctionnelle est évaluée le long de la branche
bifurquée. En d'autres termes, on considère
\begin{equation}
  f(η) = F\{ u^{\ast} [λ₀ + λ(η)] + U(η), λ₀ + λ(η) \}.
\end{equation}

On souhaite établir un développement limité de \(f\) au voisinage de \(η = 0\),
ce qui conduit à calculer les dérivées successives de \(f\) en \(η = 0\),
puisque
\begin{equation}
  f(η) = f(0) + η f'(0) + \tfrac{1}{2} η^2 f''(0) + \cdots
\end{equation}

Pour calculer ces dérivées, il sera commode d'introduire la fonction auxiliaire
\(F\)
\begin{equation}
  F(η, λ) =\mathcal{F}[u^{\ast}(λ) + U(η), λ],
\end{equation}
dans laquelle les variables \(λ\) et \(η\) sont provisoirement considérées comme
indépendantes. On a
\begin{equation}
 f(η) = F[η, λ₀ + λ(η)],
\end{equation}
d'où l'on déduit successivement que
\begin{equation}
  \label{eq:20211112162417}
  f'(η) = ∂_{η} F + λ' ∂_{λ} F,
\end{equation}
\begin{equation}
  \label{eq:20211112165810}
  f''(η) = ∂_{ηη}^2 F + 2λ' ∂_{ηλ}^2 {F + λ'}^2 + ∂_{λλ}^2 F + λ'' ∂_{λ} F,
\end{equation}
\begin{equation*}
  \begin{aligned}[b]
    \label{eq:20211112173223}
    f'''(η) ={}
    & ∂_{ηηη}^3 F + 3 λ' ∂_{ηηλ}^3 {F + 3 λ'}^2 ∂_{ηλλ}^3 {F + λ'}^3 ∂_{λλλ}^3 F + 3 λ'' ∂_{ηλ}^2 F + 3 λ' λ'' ∂_{λ λ}^2 F\\
    & + λ''' ∂_{λ} F
  \end{aligned}
\end{equation*}
\begin{equation}
  \begin{aligned}[b]
    f''''(η) ={}
    & ∂_{ηηηη}^4 F + 4 λ' ∂_{ηηηλ}^4 {F + 6 λ'}^2 ∂_{ηηλλ}^4 {F + 4 λ'}^3 ∂_{ηλλλ}^4 {F + λ'}^4 ∂_{λλλλ}^4 F + 6 λ'' ∂_{ηηλ}^3 F\\
    & + 12 λ' λ'' ∂_{ηλλ}^3 {F + 6 λ'}^2 λ'' ∂_{λλλ}^3 F + 4 λ''' ∂_{ηλ}^2 F + \bigl( {3 λ''}^2 + 4 λ' λ''' \bigr) ∂_{λλ}^2 F\\
    & + λ'''' ∂_{λ}F
  \end{aligned}
\end{equation}
où \(λ\) et ses dérivées sont évaluées en \(η\), tandis que \(F\) et ses
dérivées partielles sont évaluées en \([η, λ₀ + λ(η)]\). En \(η = 0\), les
relations précédentes s'écrivent
\begin{equation}
  \label{eq:20220107060454}
  f'(0) = ∂_{η} F + λ₁ ∂_{λ} F,
\end{equation}
\begin{equation}
  \label{eq:20220107124311}
  f''(0) = ∂_{ηη}^2 F + 2 λ₁ ∂_{ηλ}^2 F + λ₂ ∂_{λ} F + λ₁^2 ∂_{λλ}^2 F,
\end{equation}
\begin{equation}
  \label{eq:20220107060500}
  \begin{aligned}[b]
    f'''(0) ={}
    & ∂_{ηηη}^3 F + 3 λ₁ ∂_{ηηλ}^3 F + 3 λ₁^2 ∂_{ηλλ}^3 F + λ₁^3 ∂_{λλλ}^3 F + 3 λ₂ ∂_{ηλ}^2 F + 3 λ₁ λ₂ ∂_{λλ}^2 F\\
    & + λ₃ ∂_{λ} F,
  \end{aligned}
\end{equation}
\begin{equation}
  \begin{aligned}[b]
    f''''(0) ={}
    & ∂_{ηηηη}^4F + 4 λ₁ ∂_{ηηηλ}^4 F + 6 λ₁^2 ∂_{ηηλλ}^4 F + 4 λ₁^3 ∂_{ηλλλ}^4 F + λ₁^4 ∂_{λλλλ}^4 F + 6 λ₂ ∂_{ηηλ}^3 F\\
    & + 12 λ₁ λ₂ ∂_{ηλλ}^3 F + 6 λ₁^2 λ₂ ∂_{λλλ}^3 F + 4 λ₃ ∂_{ηλ}^2 F + \bigl(3 λ₂^2 + 4 λ₁ λ₃\bigr) ∂_{λλ}^2 F \\
    & + λ₄ ∂_{λ} F,
  \end{aligned}
\end{equation}
où \(F\) et ses dérivées sont évaluées en \((0, λ₀)\).

\subsection{Développement limité du résidu}
\label{sec:20211112182000}

On cherche un développement limité du résidu(c'est-à-dire de la première
variation de l'énergie). La fonction test \(\hat{u} ∈ U\) étant fixée, la
méthode précédente est donc appliquée avec
\begin{equation}
  \label{eq:20220107054629}
  f(η) =ℰ_{, u} [u(η), λ(η); \hat{u}]
  \quad \text{et} \quad
  F(η, λ) = ℰ_{, u}[u^{\ast}(λ) + U(η), λ; \hat{u}].
\end{equation}

On remarque tout d'abord que
\(F(0, λ) =ℰ_{, u} [u^{\ast} (λ), λ; \hat{u}] = 0\), puisque \(u^{\ast}(λ)\) est
un point d'équilibre. En dérivant par rapport à \(λ\), on obtient
\begin{equation}
  \label{eq:20211112164204}
  \frac{∂^k F}{∂ λ^k}(0, λ) = 0.
\end{equation}

En dérivant par rapport à \(η\) l'expression~\eqref{eq:20220107054629} de \(F\),
on obtient successivement
\begin{gather}
  ∂_{η}F(η, λ) = ℰ_{, u u}[u^{\ast}(λ) + U(η), λ; U'(η), \hat{u}],\\
  \begin{aligned}[b]
    ∂_{η η}^2 F(η, λ) ={}
    & ℰ_{, uuu}[u^{\ast}(λ) + U(η), λ; U'(η), U'(η), \hat{u}]\\
    & + ℰ_{, uu} [u^{\ast}(λ) + U(η), λ; U''(η), \hat{u}],
  \end{aligned}\\
  \begin{aligned}[b]
    ∂_{ηηη}^3 F(η, λ) ={}
    & ℰ_{, uuuu}[u^{\ast}(λ) + U(η), λ; U'(η), U'(η), U'(η), \hat{u}]\\
    & + 3ℰ_{, u u u}[u^{\ast}(λ) + U(η), λ; U'(η), U''(η), \hat{u}]\\
    & + ℰ_{, uu}[u^{\ast}(λ) + U(η), λ; U'''(η), \hat{u}],
  \end{aligned}
\end{gather}
soit, en \(η = 0\)
\begin{gather}
  ∂_{η}F(0, λ) = ℰ₂(λ; u₁, \hat{u}),\\
  ∂_{ηη}^2 F(0, λ) = ℰ₃(λ; u₁, u₁, \hat{u}) + ℰ₂(λ; u₂, \hat{u}),\\
  ∂_{ηηη}^3 F(0, λ) = ℰ₄(λ; u₁, u₁, u₁, \hat{u}) + 3ℰ₃(λ; u₁, u₂, \hat{u}) + ℰ₂(λ; u₃, \hat{u}).
\end{gather}

Les dérivées croisées de \(F\) en \((0, λ)\) s'obtiennent par simple dérivation
des relations précédentes par rapport à \(λ\)
\begin{gather}
  ∂_{ηλ}^2 F(0, λ) = \dot{ℰ}₂(λ; u₁, \hat{u}),\\
  ∂_{ηηλ}^3 F(0, λ) = \dot{ℰ}₃(λ; u₁, u₁, \hat{u}) + \dot{ℰ₂}(λ; u₂, \hat{u}),\\
  ∂_{ηλλ}^3 F(0, λ) = \ddot{ℰ}₂(λ; u₁, \hat{u}).
\end{gather}

En insérant les résultats précédentes dans les relations
générales~\eqref{eq:20220107060454}--\eqref{eq:20220107060500}, on trouve alors
les expressions suivantes des dérivées successives de \(f\) en \(η = 0\)
\begin{gather}
  f'(0) = ℰ₂(λ₀; u₁, \hat{u}),\\
  f''(0) = ℰ₃(λ₀; u₁, u₁, \hat{u}) + 2 λ₁ \dot{ℰ}₂(λ₀; u₁, \hat{u}) +ℰ₂(λ₀; u₂, \hat{u}),\\
  \begin{aligned}[b]
    f'''(0) ={}
    & ℰ₄(λ₀; u₁, u₁, u₁, \hat{u}) + 3ℰ₃(λ₀; u₁, u₂, \hat{u}) + ℰ₂(λ₀ ; u₃, \hat{u})\\
    & + 3 λ₁ [\dot{ℰ}₃(λ₀; u₁, u₁, \hat{u}) + \dot{ℰ}₂(λ₀; u₂, \hat{u})]\\
    & + 3 λ₁^2 \ddot{ℰ}₂(λ₀; u₁, \hat{u}) + 3 λ₂ \dot{ℰ}₂(λ₀; u₁, \hat{u}) .
  \end{aligned}
\end{gather}

On en déduit finalement le développement limité à l'ordre 3 en \(η\) du résidu
\begin{equation}
  \label{eq:20220107080901}
  \begin{aligned}[b]
    ℰ_{, u}[u(η), λ(η)] ={} & η ℰ₂(λ₀; u₁, \hat{u}) \\
    & + \tfrac{1}{2} η^2 [ℰ₃(λ₀; u₁, u₁, \hat{u}) + 2 λ₁ \dot{ℰ}₂(λ₀; u₁, \hat{u}) + ℰ₂(λ₀; u₂, \hat{u})]\\
    & + \tfrac{1}{6} η^3 \bigl\{ ℰ₄(λ₀; u₁, u₁, u₁, \hat{u}) + 3ℰ₃(λ₀; u₁, u₂, \hat{u}) + ℰ₂(λ₀; u₃, \hat{u})\\
    & + 3 λ₁ \bigl[\dot{ℰ}₃(λ₀; u₁, u₁, \hat{u}) + \dot{ℰ}₂(λ₀; u₂, \hat{u})\bigr] + 3 λ₁^2 \ddot{ℰ}₂(λ₀; u₁, \hat{u})\\
    &  + 3 λ₂ \dot{ℰ}₂(λ₀ ; u₁, \hat{u}) \bigr\} + o(η^3).
  \end{aligned}
\end{equation}

\subsection{Développement limité de l'énergie}
\label{sec:20220525053434}

On s'intéresse ici à l'écart d'énergie, pour un chargement \(λ\) donné, entre la
branche bifurquée et la branche fondamentale, soit
\begin{equation}
  F(η, λ) = ℰ[u^{\ast}(λ) + U(η), λ] - ℰ[u^{\ast}(λ), λ]
\end{equation}
et
\begin{equation}
  f(η) = F [η, λ₀ + λ(η)].
\end{equation}

On observe tout d'abord que \(F(0, λ) = 0\) pour tout \(λ\), donc
\begin{equation}
  \frac{∂^k F}{∂ λ^k}(0, λ) = 0 \quad (k ≥ 0),
\end{equation}
tandis que les dérivées de \(F\) par rapport à \(η\) s'écrivent
\begin{equation}
  ∂_{η} F(η, λ) = ℰ_{, u}[u^{\ast}(λ) + U(η), λ; U'(η)],
\end{equation}
\begin{equation}
  ∂_{ηη}^2 F(η, λ) = ℰ_{, uu} [u^{\ast}(λ) + U(η), λ; U'(η), U'(η)] + ℰ_{, u} [u^{\ast}(λ) + U(η), λ; U''(η)],
\end{equation}
\begin{equation*}
  \begin{aligned}[b]
    ∂_{ηηη}^3 F(η, λ) ={}
    & ℰ_{, uuu} [u^{\ast}(λ) + U(η), λ; U'(η), U'(η), U'(η)]\\
    & + 3ℰ_{, uu}[u^{\ast}(λ) + U(η), λ; U'(η), U''(η)]\\
    & + ℰ_{, u}[u^{\ast}(λ) + U(η), λ; U'''(η)],
  \end{aligned}
\end{equation*}
\begin{equation}
  \begin{aligned}[b]
    ∂_{ηηηη}^4 F(η, λ) ={}
    & ℰ_{, uuuu}[u^{\ast}(λ) + U(η), λ; U'(η), U'(η), U'(η), U'(η)]\\
    & + 6ℰ_{,uuu}[u^{\ast}(λ) + U(η), λ; U'(η), U'(η), U''(η)]\\
    & + 3ℰ_{, uu}[u^{\ast}(λ) + U(η), λ; U''(η), U''(η)]\\
    & + 3ℰ_{, uu}[u^{\ast}(λ) + U(η), λ; U'(η), U'''(η)]\\
    & +ℰ_{, u}[u^{\ast}(λ) + U(η), λ ; U''''(η)],
  \end{aligned}
\end{equation}
soit, en \(η = 0\), en observant que \(ℰ_{, u}[u^{\ast}(λ), λ] = 0\)
\begin{gather}
  ∂_{η} F(0, λ) = 0,\\
  ∂_{ηη}^2 F(0, λ) =ℰ₂(λ ; u₁, u₁),\\
  ∂_{ηηη}^3 F(0, λ) = ℰ₃(λ; u₁, u₁, u₁) + 3ℰ₂(λ; u₁, u₂),\\
  \begin{aligned}[b]
    ∂_{ηηηη}^4 F(η, λ) ={}
    & ℰ₄(λ; u₁, u₁, u₁, u₁) + 6ℰ₃(λ; u₁, u₁, u₂)\\
    & + 3ℰ₂(λ; u₂, u₂) + 3ℰ₂(λ; u₁, u₃).
  \end{aligned}
\end{gather}

On en déduit que
\begin{gather}
  ∂_{ηλ}^2 F(0, λ) = 0,\\
  ∂_{ηηλ}^3 F(0, λ) = \dot{ℰ}₂(λ; u₁, u₁),\\
  ∂_{ηλλ}^3 F(0, λ) = 0,\\
  ∂_{ηηηλ}^4 F(0, λ) = \dot{ℰ}₃(λ; u₁, u₁, u₁) + 3\dot{ℰ}₂(λ; u₁, u₂),\\
  ∂_{ηηλλ}^4 F(0, λ) = \ddot{ℰ}₂(λ; u₁, u₁),\\
  ∂_{ηλλλ}^4 F(0, λ) = 0
\end{gather}
et finalement
\begin{gather}
  f'(0) = 0,\\
  f''(0) = ℰ₂(λ₀; u₁, u₁),\\
  f'''(0) =ℰ₃(λ₀; u₁, u₁, u₁) + 3ℰ₂(λ₀; u₁, u₂) + 3λ₁ \dot{ℰ}₂(λ₀; u₁, u₁),\\
  \begin{aligned}[b]
    f''''(0) ={}
    & ℰ₄(λ₀; u₁, u₁, u₁, u₁) + 6ℰ₃(λ₀; u₁, u₁, u₂)\\
    & + 3ℰ₂(λ₀; u₂, u₂) + 3ℰ₂(λ₀; u₁, u₃)\\
    & + 4 λ₁ \dot{ℰ}₃(λ₀; u₁, u₁, u₁) + 12 λ₁ \dot{ℰ}₂(λ₀; u₁, u₂)\\
    & + 6λ₁^2 \ddot{ℰ}₂(λ₀; u₁, u₁) + 6λ₂ \dot{ℰ}₂(λ₀; u₁, u₁).
  \end{aligned}
\end{gather}

Les relations précédentes se simplifient notamment en tenant compte de ce que
\(u₁∈V\) : \(ℰ₂(λ₀; u₁, u_i) = 0\) pour \(i = 1, 2, 3\). On trouve ainsi, pour
\(f''(0)\) et \(f'''(0)\)
\begin{gather}
  \label{eq:20220601055423}
  f''(0) = 0\\
  \intertext{et}
  \begin{aligned}[b]
    \label{eq:20220601055448}
    f'''(0) ={} &
    ℰ₃(λ₀; u₁, u₁, u₁) + 3λ₁ \dot{ℰ}₂(λ₀; u₁, u₁) \\
    ={} & - 2λ₁ \dot{ℰ₂}(λ₀; u₁, \hat{v}) + 3λ₁ \dot{ℰ}₂(λ₀; u₁, u₁) \\
    ={} & λ₁ F_{ij}(λ₀) ξ₁^i ξ₁^j,
  \end{aligned}
\end{gather}
en utilisant l'équation de bifurcation~\eqref{eq:20220524133816} dans la
deuxième ligne. En introduisant les décompositions \eqref{eq:20220524133944} et
\eqref{eq:20220524134613} de \(u₁\) et \(u₂\), on trouve tout d'abord, pour
\(ℰ₃(λ₀; u₁, u₁, u₂)\)
\begin{equation}
  \begin{aligned}[b]
    ℰ₃(λ₀; u₁, u₁, u₂)
    ={} & ℰ₃(λ₀ ; v_i, v_j, v_k) ξ₁^i ξ₁^j ξ₂^k + ℰ₃(λ₀; v_i, v_j, w_{k l}) ξ₁^i ξ₁^j ξ₁^k ξ₁^l\\
    & + λ₁ ℰ₃(λ₀; v_i, v_j, w_k) ξ₁^i ξ₁^j ξ₁^k\\
    ={} & ℰ₃(λ₀; v_i, v_j, v_k) ξ₁^i ξ₁^j ξ₂^k + ℰ₃(λ₀; v_i, v_j, w_{k l}) ξ₁^i ξ₁^j ξ₁^k ξ₁^l\\
    & - λ₁ ℰ₂(λ₀; w_{ij}, w_k) ξ₁^i ξ₁^j ξ₁^k,
  \end{aligned}
\end{equation}
en tenant compte de la définition~\eqref{eq:20220519164523} des \(w_{ij}\). Dans
le dernier terme de l'expression précédente, les indices \(i\), \(j\) et \(k\)
sont muets, donc
\begin{equation*}
  \begin{aligned}[b]
    ℰ₃(λ₀; u₁, u₁, u₂)
    ={} & ℰ₃(λ₀ ; v_i, v_j, v_k) ξ₁^i ξ₁^j ξ₂^k + ℰ₃(λ₀; v_i, v_j, w_{kl}) ξ₁^i ξ₁^j ξ₁^k ξ₁^l\\
    & - λ₁ ℰ₂(λ₀; w_{i}, w_{jk}) ξ₁^i ξ₁^j ξ₁^k\\
    ={} & ℰ₃(λ₀; v_i, v_j, v_k) ξ₁^i ξ₁^j ξ₂^k + ℰ₃(λ₀; v_i, v_j, w_{kl}) ξ₁^i ξ₁^j ξ₁^k ξ₁^l\\
    & + 2 λ₁ \dot{ℰ}₂(λ₀; v_{i}, w_{jk}) ξ₁^i ξ₁^j ξ₁^k,
  \end{aligned}
\end{equation*}
en introduisant cette fois-ci la définition~\eqref{eq:20220524134525} de
\(w_i\). On procède de même pour le terme suivant, soit \(ℰ₂(λ₀; u₂, u₂)\)
\begin{equation*}
  \begin{aligned}[b]
    ℰ₂(λ₀; u₂, u₂)
    ={} & ℰ₂(λ₀ ; ξ₁^i v_i + ξ₁^i ξ₁^j w_{i j} + λ₁ ξ₁^i w_i, ξ₁^i v_i + ξ₁^k ξ₁^l w_{k l} + λ₁ ξ₁^k w_k)\\
    ={} & ℰ₂(λ₀; ξ₁^i ξ₁^j w_{i j} + λ₁ ξ₁^i w_i, ξ₁^k ξ₁^l w_{k l} + λ₁ ξ₁^k w_k)\\
    ={} & ℰ₂(λ₀; w_{i j}, w_{k l}) ξ₁^i ξ₁^j ξ₁^k ξ₁^l + 2 λ₁ ℰ₂(λ₀ ; w_{i j}, w_k) ξ₁^i ξ₁^j ξ₁^k\\
    & + λ₁^2 ℰ₂(λ₀; w_i, w_j) ξ₁^i ξ₁^j\\
    ={} & ℰ₂(λ₀; w_{i j}, w_{k l}) ξ₁^i ξ₁^j ξ₁^k ξ₁^l + 2 λ₁ ℰ₂(λ₀ ; w_i, w_{j k}) ξ₁^i ξ₁^j ξ₁^k\\
    & + \tfrac{1}{2} λ₁^2 \bigl[ℰ₂(λ₀; w_i, w_j) + ℰ₂(λ₀; w_j, w_i)\bigr] ξ₁^i ξ₁^j\\
    ={} & ℰ₂(λ₀; w_{i j}, w_{k l}) ξ₁^i ξ₁^j ξ₁^k ξ₁^l - 4 λ₁ \dot{ℰ}₂ (λ₀; v_i, w_{j k}) ξ₁^i ξ₁^j ξ₁^k\\
    & - λ₁^2 \bigl[\dot{ℰ}₂(λ₀; v_i, w_j) + \dot{ℰ}₂(λ₀; v_j, w_i)\bigr] ξ₁^i ξ₁^j\\
    ={} & ℰ₃(λ₀; v_i, v_j, w_{k l}) ξ₁^i ξ₁^j ξ₁^k ξ₁^l - 4 λ₁ \dot{ℰ}₂(λ₀; v_i, w_{j k}) ξ₁^i ξ₁^j ξ₁^k\\
    & - λ₁^2 \bigl[\dot{ℰ}₂(λ₀; v_i, w_j) + \dot{ℰ}₂(λ₀; v_j, w_i)\bigr] ξ₁^i ξ₁^j
  \end{aligned}
\end{equation*}
et enfin
\begin{equation*}
  \begin{aligned}[b]
    \dot{ℰ}₂(λ₀; u₁, u₂)
    ={} & \dot{ℰ}₂ (λ₀; v_i, v_j) ξ₁^i ξ₂^j + \dot{ℰ}₂(λ₀ ; v_i, w_{j k}) ξ₁^i ξ₁^j ξ₁^k + λ₁ \dot{ℰ}₂(λ₀; v_i, w_j) ξ₁^i ξ₁^j\\
    ={} & \dot{ℰ}₂(λ₀; v_i, v_j) ξ₁^i ξ₂^j + \dot{ℰ}₂(λ₀; v_i, w_{j k}) ξ₁^i ξ₁^j ξ₁^k\\
    & + \tfrac{1}{2} λ₁ [\dot{ℰ}₂(λ₀ ; v_i, w_j) + \dot{ℰ}₂(λ₀; v_j, w_i)] ξ₁^i ξ₁^j.
  \end{aligned}
\end{equation*}
En rassemblant les résultats précédents, on trouve pour \(f''''(0)\)
\begin{equation*}
  \begin{aligned}[b]
    f''''(0)
    ={} & ℰ₄(λ₀; v_i, v_j, v_k , v_l) ξ₁^i ξ₁^j ξ₁^k ξ₁^l + 6ℰ₃(λ₀; v_i, v_j, v_k) ξ₁^i ξ₁^j ξ₂^k\\
    & + 6ℰ₃(λ₀; v_i, v_j, w_{k l}) ξ₁^i ξ₁^j ξ₁^k ξ₁^l + 12 λ₁ \dot{ℰ}₂ (λ₀; v_{i }, w_{j k}) ξ₁^i ξ₁^j ξ₁^k\\
    & - 3ℰ₃(λ₀; v_i, v_j, w_{k l}) ξ₁^i ξ₁^j ξ₁^k ξ₁^l - 12 λ₁ \dot{ℰ}₂ (λ₀; v_i, w_{j k}) ξ₁^i ξ₁^j ξ₁^k\\
    & - 3 λ₁^2 \bigl[\dot{ℰ}₂(λ₀; v_i, w_j) + \dot{ℰ}₂(λ₀; v_j, w_i)\bigr] ξ₁^i ξ₁^j + 4 λ₁ \dot{ℰ}₃(λ₀; v_i, v_j, v_k) ξ₁^i ξ₁^j ξ₁^k\\
    & + 12 λ₁ \dot{ℰ}₂(λ₀; v_i, v_j) ξ₁^i ξ₂^j + 12 λ₁ \dot{ℰ}₂(λ₀ ; v_i, w_{j k}) ξ₁^i ξ₁^j ξ₁^k\\
    & + 6 λ₁^2 \bigl[\dot{ℰ}₂(λ₀; v_i, w_j) + \dot{ℰ}₂(λ₀; v_j, w_i)\bigr] ξ₁^i ξ₁^j + 6 λ₁^2 \ddot{ℰ}₂(λ₀; v_i, v_j) ξ₁^i ξ₁^j\\
    & + 6 λ₂ \dot{ℰ}₂(λ₀; v_i, v_j) ξ₁^i ξ₁^j\\
    ={} & \bigl[ ℰ₄(λ₀; v_i, v_j, v_k , v_l) + 3ℰ₃(λ₀; v_i, v_j, w_{k l}) \bigr] ξ₁^i ξ₁^j ξ₁^k ξ₁^l\\
    & + 4 λ₁ \bigl[\dot{ℰ}₃(λ₀; v_i, v_j, v_k) + 3 \dot{ℰ}₂(λ₀; v_i, w_{j k})\bigr] ξ₁^i ξ₁^j ξ₁^k\\
    & + \bigl\{ 3 λ₁^2 \bigl[2 \ddot{ℰ}₂ (λ₀; v_i, v_j) + \dot{ℰ}₂(λ₀; v_i, w_j) + \dot{ℰ}₂(λ₀; v_j, w_i)\bigr] + 6 λ₂ \dot{ℰ}₂(λ₀; v_i, v_j) \bigr\} ξ₁^i ξ₁^j\\
    & + 6ℰ₃(λ₀; v_i, v_j, v_k) ξ₁^i ξ₁^j ξ₂^k + 12 λ₁ \dot{ℰ}₂(λ₀; v_i, v_j) ξ₁^i ξ₂^j\\
    ={} & E_{i j k l}(λ₀) ξ₁^i ξ₁^j ξ₁^k ξ₁^l + 4 λ₁ \dot{E}_{i j k} (λ₀) ξ₁^i ξ₁^j ξ₁^k + 6 \bigl[λ₁^2 \dot{F}_{i j}(λ₀) + λ₂ F_{i j}(λ₀)\bigr] ξ₁^i ξ₁^j\\
    & + 6 \bigl[E_{i j k}(λ₀) ξ₁^k + 2 λ₁ F_{i j}(λ₀)\bigr] ξ₁^i ξ₂^j,
  \end{aligned}
\end{equation*}
et on observe que le dernier terme(en \(ξ₁^i ξ₂^j\)) est nul, du fait de
l'équation de bifurcation~\eqref{eq:20220524135036}. On obtient donc
\begin{equation}
  \label{eq:20220601055512}
  f''''(0) = E_{i j k l}(λ₀) ξ₁^i ξ₁^j ξ₁^k ξ₁^l + 4 λ₁ \dot{E}_{i j k}(λ₀) ξ₁^i ξ₁^j ξ₁^k + 6 \bigl[λ₁^2 \dot{F}_{i j}(λ₀) + λ₂ F_{i j}(λ₀)\bigr] ξ₁^i ξ₁^j .
\end{equation}
Le développement limité~\eqref{eq:20220525053600} est alors obtenu en
rassemblant les résultats~\eqref{eq:20220601055423}, \eqref{eq:20220601055448}
et \eqref{eq:20220601055512}.

\begin{remark}
  On peut réécrire \(f''''(0)\) en tenant compte de l'équation de
  bifurcation~\eqref{eq:20220601070917}. En multipliant celle-ci par \(ξ₁^i\),
  on trouve en effet
 \begin{equation*}
   \begin{aligned}[b]
     E_{i j k l}(λ₀) ξ₁^i ξ₁^j ξ₁^k ξ₁^l
     ={} & - 3 λ₂ F_{i j}(λ₀) ξ₁^i ξ₁^j - 3 A_{i j}(λ₀) ξ₁^i ξ₂^j - 3 λ₁ \bigl[\dot{E}_{i j k} (λ₀) ξ₁^k + λ₁ \dot{F}_{i j}(λ₀)\bigr] ξ₁^i ξ₁^j\\
     ={} & - 3 λ₁ \dot{E}_{i j k}(λ₀) ξ₁^i ξ₁^j ξ₁^k - 3 [λ₁^2 \dot{F}_{i j} (λ₀) + λ₂ F_{i j}(λ₀)] ξ₁^i ξ₁^j - 3 A_{i j}(λ₀) ξ₁^i ξ₂^j,
   \end{aligned}
 \end{equation*}
 soit
 \begin{equation}
   f''''(0) = λ₁ \dot{E}_{i j k}(λ₀) ξ₁^i ξ₁^j ξ₁^k + 3 [λ₁^2 \dot{F}_{i j} (λ₀) + λ₂ F_{i j}(λ₀)] ξ₁^i ξ₁^j - 3 A_{i j}(λ₀) ξ₁^i ξ₂^j.
 \end{equation}
\end{remark}

\subsection{Développement limité de la hessienne}

On cherche maintenant un développement limité de la hessienne de l'énergie. Les
fonctions test \(\hat{u}, \hat{v} ∈ U\) étant fixées, on applique la méthode du
\S\ref{sec:20220107121442} à la fonction \(f(η) = F [η, λ₀ + λ(η)]\), avec
\begin{equation}
  F(η, λ) =ℰ_{, u u} [u^{\ast}(λ) + U(η), λ; \hat{u}, \hat{v}].
\end{equation}

On observe tout d'abord que \(F(0, λ) =ℰ₂(λ; \hat{u}, \hat{v})\), soit, en
dérivant par rapport à \(λ\)
\begin{equation}
  ∂_{λ} F(0, λ) = \dot{ℰ}₂(λ; \hat{u}, \hat{v})
  \quad \text{et} \quad
  ∂_{λ λ}^2 F(0, λ) = \ddot{ℰ}₂(λ; \hat{u}, \hat{v}).
\end{equation}
On trouve de même successivement
\begin{gather}
  ∂_{η} F(η, λ) = ℰ_{, u u u} [u^{\ast}(λ) + U(η), λ; U'(η), \hat{u}, \hat{v}],\\
  \begin{aligned}[b]
    ∂_{η η}^2 F(η, λ)
    ={} & ℰ_{, u u u u} [u^{\ast}(λ) + U(η), λ ; U'(η), U'(η), \hat{u}, \hat{v}] \\
    ={} & +ℰ_{, u u u} [u^{\ast}(λ) + U(η), λ; U''(η), \hat{u}, \hat{v}],
  \end{aligned}
\end{gather}
soit, en \(η = 0\)
\begin{equation}
  ∂_{η} F(0, λ) = ℰ₃(λ; u₁, \hat{u}, \hat{v})
\end{equation}
et
\begin{equation}
  ∂_{η η}^2 F(0, λ) = ℰ₄(λ ; u₁, u₁, \hat{u}, \hat{v}) + ℰ₃(λ; u₂, \hat{u}, \hat{v}),
\end{equation}
et en dérivant cette fois par rapport à \(λ\)
\begin{equation}
  ∂_{η λ}^2 F(0, λ) = \dot{ℰ}₃(λ; u₁, \hat{u}, \hat{v}).
\end{equation}

En insérant les résultats précédents dans les
expressions~\eqref{eq:20220107060454} et \eqref{eq:20220107124311}, on trouve
\begin{gather}
  f'(0) =ℰ₃(λ₀; u₁, \hat{u}, \hat{v}) + λ₁ \dot{ℰ}₂(λ₀; \hat{u}, \hat{v}),\\
  \begin{aligned}[b]
    f''(0)
    ={} & ℰ₄(λ₀; u₁, u₁, \hat{u}, \hat{v}) +ℰ₃(λ₀; u₂, \hat{u}, \hat{v}) + λ₂ \dot{ℰ}₂(λ₀; \hat{u}, \hat{v})\\
    & + 2 λ₁ \dot{ℰ}₃(λ₀ ; u₁, \hat{u}, \hat{v}) + λ₁^2 \ddot{ℰ}₂(λ₀ ; \hat{u}, \hat{v}).
  \end{aligned}
\end{gather}
qui conduisent finalement au développement limité~\eqref{eq:20220531054247}.

\subsection{Développement limité des valeurs propres et vecteurs
propres de la Hessienne}

On cherche les vecteurs propres \(x ∈ U\) et valeurs propres \(α ∈ \reals\) de
la hessienne de l'énergie. En d'autre terme, on cherche \(x\) et \(α\) tels que
\begin{equation}
 ℰ_{, u u} [u(η), λ(η); x, \hat{u}] = α 〈 x, \hat{u} 〉 \quad \text{pour tout} \quad \hat{u} ∈ V.
\end{equation}

On cherche les développements limités à l'ordre 2 en \(η\) de \(x\) et \(α\)
\begin{align*}
 x & = x₀ + η x₁ + \tfrac{1}{2} η^2 x₂ + o(η^2),\\
 α & = α₀ + η α₁ + \tfrac{1}{2} η^2 α₂ + o(η^2).
\end{align*}

Ces développements limités sont tout d'abord insérés dans le développement
limité \eqref{eq:20220531054247} de la hessienne de l'énergie
\begin{equation*}
  \begin{aligned}[b]
    ℰ_{, u u} [u(η), λ(η); x, \hat{u}]
    ={} & ℰ₂(λ₀; x₀, \hat{u}) + η ℰ₂(λ₀ ; x₁, \hat{u}) + \tfrac{1}{2} η^2 ℰ₂(λ₀; x₂, \hat{u})\\
    & + η ℰ₃(λ₀; u₁, x₀, \hat{u}) + η^2 ℰ₃(λ₀; u₁, x₁, \hat{u})\\
    & + η λ₁ \dot{ℰ}₂(λ₀; x₀, \hat{u}) + η^2 λ₁ \dot{ℰ}₂(λ₀; x₁, \hat{u})\\
    & + \tfrac{1}{2} η^2 \bigl[ℰ₄(λ₀; u₁, u₁, x₀, \hat{u}) + ℰ₃(λ₀; u₂, x₀, \hat{u})\\
    & + λ₂ \dot{ℰ}₂(λ₀; x₀, \hat{u}) + 2 λ₁ \dot{ℰ}₃(λ₀; u₁, x, \hat{u})\\
    & + λ₁^2 \ddot{ℰ}₂(λ₀; x, \hat{u}) \bigr] + o(η^2)\\
    ={} & ℰ₂(λ₀; x₀, \hat{u})\\
    & + η \bigl[ℰ₃(λ₀; u₁, x₀, \hat{u}) + ℰ₂ (λ₀; x₁, \hat{u}) + λ₁ \dot{ℰ}₂(λ₀ ; x₀, \hat{u})\bigr]\\
    & + \tfrac{1}{2} η^2 \bigl[ℰ₂(λ₀; x₂, \hat{u}) + 2ℰ₃(λ₀; u₁, x₁, \hat{u}) + 2 λ₁ \dot{ℰ}₂(λ₀; x₁, \hat{u})\bigr]\\
    & + \tfrac{1}{2} η^2 \bigl[ℰ₄(λ₀; u₁, u₁, x₀, \hat{u}) + ℰ₃(λ₀; u₂, x₀, \hat{u}) + λ₂ \dot{ℰ}₂(λ₀; x₀, \hat{u})\bigr]\\
    & + \tfrac{1}{2} η^2 \bigl[2 λ₁ \dot{ℰ}₃(λ₀ ; u₁, x, \hat{u}) + λ₁^2 \ddot{ℰ}₂(λ₀; x, \hat{u})\bigr] + o(η^2)
  \end{aligned}
\end{equation*}
et
\begin{equation*}
  \begin{aligned}[b]
    α 〈 x, \hat{u} 〉
    ={} & α₀ 〈 x₀, \hat{u} 〉 + η(α₁ 〈 x₀, \hat{u} 〉 + α₀ 〈 x₁, \hat{u} 〉)\\
    & + \tfrac{1}{2} η^2 (α₂ 〈 x₀, \hat{u} 〉 + 2 α₁ 〈 x₁, \hat{u} 〉 + α₀ 〈 x₂, \hat{u} 〉) + o(η^2) .
  \end{aligned}
\end{equation*}

\paragraph{Problème variationnel d'ordre 0} Trouver \(x₀∈U\) et \(α₀∈\reals\)
tels que, pour tout \(\hat{u}∈U\)
\begin{equation}
  ℰ₂(λ₀; x₀, \hat{u}) = α₀ 〈 x₀, \hat{u} 〉.
\end{equation}

On en déduit que \(x₀\) est le vecteur propre de \(ℰ₂(λ₀)\) associé à la valeur
propre \(α₀\). Si \(α₀ \neq 0\), \(ℰ₂ (λ₀)\) étant positive par hypothèse, on a
nécessairement \(α₀ > 0\), et la valeur propre de la hessienne est positive. On
considère donc dans ce qui suit le cas où \(α₀ = 0\), c'est-à-dire que \(x₀∈V\)
\begin{equation}
  x₀ = χ₀^i v_i.
\end{equation}

\paragraph{Problème variationnel d'ordre 1} Trouver \(x₁∈U\) et \(α₁∈\reals\)
tels que, pour tout \(\hat{u}∈U\)
\begin{equation}
  ℰ₃(λ₀; u₁, x₀, \hat{u}) + ℰ₂(λ₀ ; x₁, \hat{u}) + λ₁ \dot{ℰ}₂(λ₀; x₀, \hat{u}) = α₁ 〈 x₀, \hat{u} 〉,
\end{equation}
soit, en remplaçant \(u₁\) et \(x₀\) par leurs décompositions dans la base \(v_i\)
\begin{equation}
  ℰ₃(λ₀; v_j, v_k, \hat{u}) ξ₁^k χ₀^j + ℰ₂(λ₀; x₁, \hat{u}) + λ₁ \dot{ℰ}₂(λ₀; v_j, \hat{u}) χ₀^j = α₁ χ₀^j 〈 v_j, \hat{u} 〉.
\end{equation}
En prenant tout d'abord \(\hat{u} = v_i\), on obtient
\begin{equation}
  \bigl[ℰ₃(λ₀; v_i, v_j, v_k) ξ₁^k + λ₁ \dot{ℰ}₂(λ₀; v_i, v_j)\bigr] χ₀^j = α₁ χ₀^i,
\end{equation}
soit encore
\begin{equation}
  \bigl[E_{i j k}(λ₀) ξ₁^k + λ₁ F_{i j}(λ₀)\bigr] χ₀^j = α₁ χ₀^i.
\end{equation}
Ainsi, le vecteur \(χ₀^i\) apparaît comme le vecteur propre de la matrice
symétrique \([E_{i j k}(λ₀) ξ₁^k + λ₁ F_{i j}(λ₀)]\) associé à la valeur propre
\(α₁\). On doit alors discuter en fonction du type de bifurcation.

\paragraph{Cas d'une bifurcation asymétrique} Dans ce cas, la forme trilinéaire
\(E_{i j k}(λ₀)\) n'est pas nulle sur \(V\), et \(α₁ \neq 0\). Le terme dominant
de \(α\) est donc d'ordre 1, tandis que le terme dominant de \(x\) est d'ordre
0.

\paragraph{Cas d'une bifurcation symétrique} La forme trilinéaire
\(E_{i j k}(λ₀)\) est identiquement nulle sur \(V\); de plus, \(λ₁ = 0\). On
trouve alors que \(α₁ = 0\), et on ne peut déterminer les \(χ₀^i\). On prend
maintenant \(\hat{u} = \hat{w}∈W\) dans le problème variationnel d'ordre 1, et
on pose \(x₁ = χ₁^i v_i + y₁\), avec \(y₁∈W\). On obtient alors le problème
variationnel suivant : trouver \(y₁∈W\) tel que, pour tout \(\hat{w}∈W\),
\begin{equation}
  ℰ₃(λ₀; v_j, v_k, \hat{w}) ξ₁^k χ₀^j + ℰ₂(λ₀; y₁, \hat{w}) + λ₁ \dot{ℰ}₂(λ₀; v_j, \hat{w}) χ₀^j = 0.
\end{equation}

La solution de ce problème est exprimée à l'aide des \(w_{i j}\) et \(w_i\)
définis respectivement par les problèmes variationnels auxiliaires
\eqref{eq:20220519164523} et \eqref{eq:20220524134525}
\begin{equation}
  y₁ = ξ₁^i χ₀^j w_{i j} + λ₁ χ₀^i w_i,
\end{equation}
soit
\begin{equation}
  x₁ = χ₁^i v_i + ξ₁^i χ₀^j w_{i j} + λ₁ χ₀^i w_i.
\end{equation}

Dans le cas d'une bifurcation symétrique, le problème aux valeurs propres d'ordre 2 s'écrit quant à lui
\begin{equation}
  ℰ₂(λ₀; x₂, \hat{u}) + 2ℰ₃(λ₀; u₁, x₁, \hat{u}) +ℰ₄ (λ₀; u₁, u₁, x₀, \hat{u}) + ℰ₃(λ₀; u₂, x₀, \hat{u}) + λ₂ \dot{ℰ}₂(λ₀; x₀, \hat{u}) = α₂ 〈 x₀, \hat{u} 〉
\end{equation}
soit, en prenant \(\hat{u} = \hat{v}_i∈V\) et en introduisant les développements
de \(u₁\), \(u₂\), \(x₀ \) et \(x₁\)
\begin{equation}
  ℰ₄ (λ₀; v_i, v_j, v_k, v_l) χ₀^j ξ_{1}^k ξ₁^l + 2ℰ₃(λ₀; u₁, x₁, v_i) +ℰ₃(λ₀ ; u₂, x₀, \hat{u}) + λ₂ \dot{ℰ₂}(λ₀; x₀, \hat{u})
\end{equation}

\section{Simplification des équations de bifurcation}
\label{sec:20220524134954}

Dans ce paragraphe, on simplifie les équations de bifurcation
\eqref{eq:20220524134121} et \eqref{eq:20211112183220} pour obtenir les formes
\eqref{eq:20220524135036} et \eqref{eq:20220601070917}. On commence par
symétriser les termes cubique, quadratique et linéaire en \(ξ₁^i\) de l'équation
\eqref{eq:20220601070917}.

\paragraph{Terme cubique en \(ξ₁^i\)} On observe que
\begin{equation}
  ℰ₃(λ₀; v_i, v_j, w_{k l}) ξ₁^j ξ₁^k ξ₁^l = \tfrac{1}{3} \bigl[ℰ₃(λ₀; v_i, v_j, w_{k l}) + ℰ₃(λ₀; v_i, v_k, w_{j l}) + ℰ₃(λ₀; v_i, v_l, w_{j k})\bigr] ξ₁^j ξ₁^k ξ₁^l.
\end{equation}
On obtient donc l'expression suivante du terme cubique en \(ξ₁^i\) dans
l'équation de bifurcation \eqref{eq:20211112183220}
\begin{equation}
  ℰ₄(λ₀; v_i, v_j, v_k, v_l) + ℰ₃(λ₀ ; v_i, v_j, w_{k l}) + ℰ₃(λ₀; v_i, v_k, w_{j l}) + ℰ₃(λ₀; v_i, v_l, w_{j k}),
\end{equation}
qui suggère d'introduire le tenseur \(E_{ijkl}(λ)\) défini par l'équation
\eqref{eq:20220524135553}. Le terme cubique en \(ξ₁^i\) dans l'équation de
bifurcation \eqref{eq:20211112183220} est alors simplement~: \(E_{ijkl}(λ₀)\).

\paragraph{Terme quadratique en \(ξ₁^i\)} On observe de même que
\begin{equation}
  ℰ₃(λ₀; v_i, v_j, w_k) ξ₁^j ξ₁^k = \tfrac{1}{2} \bigl[ℰ₃(λ₀; v_i, v_j, w_k) + ℰ₃(λ₀; v_i, w_j, v_k)\bigr] ξ₁^j ξ₁^k.
\end{equation}

En prenant tout d'abord \(\hat{w} = w_k\) dans le problème variationnel
\eqref{eq:20220519164523}, on trouve
\begin{equation}
  ℰ₃(λ₀; v_i, v_j, w_k) = -ℰ₂(λ₀ ; w_{i j}, w_k),
\end{equation}
puis, en prenant cette fois \(\hat{w} = w_{i j}\) dans le problème variationnel
\eqref{eq:20220524134525}
\begin{equation}
  ℰ₂(λ₀; w_k, w_{i j}) = - 2 \dot{ℰ}₂(λ₀; v_k, w_{i j}),
\end{equation}
soit finalement
\begin{equation}
  ℰ₃(λ₀; v_i, v_j, w_k) ξ₁^j ξ₁^k = \bigl[\dot{ℰ}₂(λ₀; v_j, w_{i k}) + \dot{ℰ}₂(λ₀; v_k, w_{i j})\bigr] ξ₁^j ξ₁^k.
\end{equation}
On obtient donc l'expression suivante du terme quadratique en \(ξ₁^i\) dans
l'équation de bifurcation \eqref{eq:20211112183220}
\begin{equation}
  3 λ₁ [\dot{ℰ}₃(λ₀; v_i, v_j, v_k) + \dot{ℰ}₂(λ₀; v_i, w_{j k}) + \dot{ℰ}₂(λ₀; v_j, w_{i k}) + \dot{ℰ}₂(λ₀; v_k, w_{i j})],
\end{equation}
qui suggère d'introduire le tenseur \(E_{ijk}(λ)\) défini par l'équation
\eqref{eq:20220524135619}. Le terme quadratique en \(ξ₁^i\) dans l'équation de
bifurcation \eqref{eq:20211112183220} est alors simplement~:
\(3 λ₁ \dot{E}_{ijk}(λ₀)\).

\paragraph{Terme linéaire en \(ξ₁^i\)} Par des arguments similaires, on établit
également que
\begin{equation}
  \dot{ℰ}₂(λ₀; v_i, w_j) = - \tfrac{1}{2} ℰ₂(λ₀; w_i, w_j) = - \tfrac{1}{2} ℰ₂(λ₀; w_j, w_i) = \dot{ℰ}₂(λ₀; v_j, w_i).
\end{equation}
On obtient donc l'expression suivante du terme linéaire en \(ξ₁^i\) dans
l'équation de bifurcation \eqref{eq:20211112183220}
\begin{equation}
  \ddot{ℰ}₂(λ₀; v_i, v_j) + \tfrac{1}{2}\bigl[\dot{ℰ}₂(λ₀; v_i, w_j) + \dot{ℰ}₂(λ₀; v_j, w_i)\bigr],
\end{equation}
qui suggère d'introduire le tenseur \(F_{i j}(λ)\) défini par l'équation
\eqref{eq:20220524135643}. Le terme linéaire en \(ξ₁^i\) dans l'équation de
bifurcation \eqref{eq:20211112183220} est alors simplement~:
\(3 λ₁^2 \dot{F}_{i j}(λ₀)\).

\paragraph{Synthèse~: simplification des équations \eqref{eq:20220524133816} et
  \eqref{eq:20211112183220}} En rassemblant les résultats précédents, on obtient
tout d'abord pour l'équation \eqref{eq:20211112183220}
\begin{equation}
  3 [E_{i j k}(λ₀) + λ₁ F_{i j} (λ₀)] ξ₂^j + 3 λ₂ F_{i j}(λ₀) ξ₁^j + E_{i j k l}(λ₀) ξ₁^j ξ₁^k ξ₁^l + 3 λ₁ \dot{E}_{i j k} (λ₀) ξ₁^j ξ₁^k + 3 λ₁^2 \dot{F}_{i j}(λ₀) ξ₁^j = 0,
\end{equation}
qui suggère d'introduire le tenseur \(A_{i j}(λ)\) défini par l'équation
\eqref{eq:20220524135705}. On obtient alors finalement l'équation de bifurcation
\eqref{eq:20220601070917}. Les tenseurs \(F_{i j}\) et \(E_{i j k}\) ainsi
introduits permettent également de réécrire l'équation de bifurcation
\eqref{eq:20220524134121} sous la forme compacte \eqref{eq:20220524135036}.

\end{document}

%%% Local Variables:
%%% coding: utf-8
%%% fill-column: 80
%%% mode: latex
%%% TeX-engine: xetex
%%% TeX-master: t
%%% End:
