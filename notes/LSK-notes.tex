\documentclass{article}
\usepackage[french]{babel}
\usepackage{amsmath,amssymb,latexsym,accents}

%%%%%%%%%% Start TeXmacs macros
\newcommand{\nosymbol}{}
\newcommand{\tmaffiliation}[1]{\\ #1}
\newcommand{\tmem}[1]{{\em #1\/}}
\newcommand{\tmemail}[1]{\\ \textit{Email:} \texttt{#1}}
\newcommand{\tmop}[1]{\ensuremath{\operatorname{#1}}}
\newcommand{\tmstrong}[1]{\textbf{#1}}
\newcommand{\tmtextbf}[1]{\text{{\bfseries{#1}}}}
\newenvironment{proof}{\noindent\textbf{Proof\ }}{\hspace*{\fill}$\Box$\medskip}
\newcommand{\nonconverted}[1]{\mbox{}}
\newtheorem{theorem}{Theorem}
%%%%%%%%%% End TeXmacs macros

%


\begin{document}

\title{Notes relatives {\`a} la m{\'e}thode asymptotique de
Lyapunov--Schmidt--Koiter}

\author{
  S{\'e}bastien Brisard
  \tmaffiliation{Univ Gustave Eiffel, Ecole des Ponts, IFSTTAR, CNRS, Navier,
  F-77454 Marne-la-Vall{\'e}e, France}
  \tmemail{sebastien.brisard@univ-eiffel.fr}
}

\maketitle

\begin{abstract}
  blabla
\end{abstract}

\section{Notations}

L'espace des champs cin{\'e}matiquement admissibles est not{\'e} $U$. On
suppose qu'il a la structure d'espace vectoriel. L'{\'e}nergie du syst{\`e}me
est not{\'e}e $\mathcal{E} (u, \lambda)$, o{\`u} $\lambda$ d{\'e}signe un
param{\`e}tre de chargement. Soit $u^{\ast} (\lambda)$ la branche
fondamentale. Par d{\'e}finition
\begin{equation}
  \mathcal{E}_{, u} [u^{\ast} (\lambda), \lambda ; \hat{u}] = 0 \quad
  \text{pour tout} \quad \hat{u} \in U.
\end{equation}
Il sera commode d'introduire les notations suivantes
\begin{equation}
  \mathcal{E}_2 (\lambda) =\mathcal{E}_{, u  u}  [u^{\ast} (\lambda),
  \lambda], \quad \mathcal{E}_3 (\lambda) =\mathcal{E}_{, u  u
   u} [u^{\ast} (\lambda), \lambda], \quad \mathcal{E}_4 (\lambda)
  =\mathcal{E}_{, u  u  u  u} [u^{\ast} (\lambda),
  \lambda] .
\end{equation}
Noter que $\mathcal{E}_2$, $\mathcal{E}_3$ et $\mathcal{E}_4$ sont des formes
bi-, tri- et quadri-lin{\'e}aires, respectivement. L'application de ces formes
{\`a} des {\'e}l{\'e}ments de $U$ sera not{\'e}e $\mathcal{E}_2 (\lambda ; u,
v)$, $\mathcal{E}_3 (\lambda ; u, v, w)$, etc.... La d{\'e}riv{\'e}e de ces
formes par rapport {\`a} $\lambda$ sera not{\'e}e {\`a} l'aide d'un point
sup{\'e}rieur ($\dot{\mathcal{E}_2}$, $\dot{\mathcal{E}_3}$, ...).

On suppose que l'{\'e}quilibre est stable pour des valeurs suffisamment
petites de $\lambda$. Plus pr{\'e}cis{\'e}ment, on suppose que $\mathcal{E}_2
(\lambda)$ est d{\'e}finie positive pour tout $\lambda < \lambda_0$. Pour
$\lambda = \lambda_0$, la forme quadratique $\mathcal{E}_2 (\lambda_0)$ n'est
plus que positive. En notant $u_0 = u^{\ast} (\lambda_0)$ la position
d'{\'e}quilibre obtenue pour la valeur critique $\lambda_0$ du param{\`e}tre
de chargement $\lambda$, on s'int{\'e}resse {\`a} toutes les courbes
d'{\'e}quilibre qui passent par le point $(u_0, \lambda_0)$.

Noter que dans ce qui suit, on convient que les formes $\mathcal{E}_2$,
$\mathcal{E}_3$ et $\mathcal{E}_4$ sont implicitement {\'e}valu{\'e}es en
$\lambda_0$ lorsque $\lambda$ n'est pas rappel{\'e} : ainsi, on notera
$\mathcal{E}_2 (\bullet, \bullet)$ plut{\^o}t que $\mathcal{E}_2 (\lambda_0 ;
\bullet, \bullet)$.

\section{Analyse de la branche fondamentale}

On s'int{\'e}resse dans ce paragraphe {\`a} la stabilit{\'e} du point critique
$(u_0, \lambda_0) .$ Par hypoth{\`e}se, $\mathcal{E}_2 (\lambda_0)$ est
positive, sans {\^e}tre d{\'e}finie positive~; soit $V$ son noyau, qui forme
un sous-espace vectoriel de $U$. On suppose que $V$ est de dimension finie $m
= \dim V$. Soit $(v_1, \ldots, v_m)$ une base orthonorm{\'e}e de ce noyau pour
le produit scalaire $\langle \bullet, \bullet \rangle$ (qui n'est pas
pr{\'e}cis{\'e} pour le moment). On introduit le sous-espace
suppl{\'e}mentaire orthogonal $W$ de $V$ dans $U$
\begin{equation}
  U = V \overset{\perp}{\otimes} W.
\end{equation}
Pour {\'e}tudier la stabilit{\'e} de l'{\'e}quilibre, on calcule l'{\'e}nergie
dans un {\'e}tat $u_0 + \xi v + \eta w$ voisin du point d'{\'e}quilibre $u_0$,
avec $\xi, \eta \in \mathbb{R}$ {\guillemotleft} petits {\guillemotright}, $v
\in V$ and $w \in W$. On obtient alors, {\`a} l'ordre 4 en $\xi$ et $\eta$
\begin{eqnarray}
  \Delta \mathcal{E} & = & \mathcal{E} (u_0 + \xi v + \eta w, \lambda_0)
  -\mathcal{E} (u_0, \lambda_0) \nonumber\\
  & = & \tfrac{1}{2} \mathcal{E}_2 (\xi v + \eta w, \xi v + \eta w) +
  \tfrac{1}{6} \mathcal{E}_3 (\xi v + \eta w, \xi v + \eta w, \xi v + \eta w)
  \nonumber\\
  &  & \nosymbol + \tfrac{1}{24} \mathcal{E}_4 (\xi v + \eta w, \xi v + \eta
  w, \xi v + \eta w, \xi v + \eta w) +\mathcal{O} [(\xi^2 + \eta^2)^2],
\end{eqnarray}
o{\`u} le terme lin{\'e}aire a {\'e}t{\'e} omis puisque $u_0$ est un point
critique de l'{\'e}nergie. En tenant compte de la multilin{\'e}arit{\'e} et de
la sym{\'e}trie des diff{\'e}rentielles successives de l'{\'e}nergie
$\mathcal{E}$, ainsi que du fait que $\mathcal{E}_2 (v, \bullet) = 0$ (puisque
$v \in V$), l'expression pr{\'e}c{\'e}dente s'{\'e}crit
\begin{eqnarray}
  \Delta \mathcal{E} & = & \tfrac{1}{2} \eta^2 \mathcal{E}_2 (w, w) +
  \tfrac{1}{6} \xi^3 \mathcal{E}_3 (v, v, v) + \tfrac{1}{2} \xi^2 \eta
  \mathcal{E}_3 (v, v, w) \nonumber\\
  &  & + \tfrac{1}{2} \xi \eta^2 \mathcal{E}_3 (v, w, w) + \tfrac{1}{6}
  \eta^3 \mathcal{E}_3 (w, w, w) \nonumber\\
  &  & + \tfrac{1}{24} \xi^4 \mathcal{E}_4 (v, v, v, v) + \tfrac{1}{6} \xi^3
  \mathcal{E}_4 (v, v, v, w) \nonumber\\
  &  & + \tfrac{1}{4} \xi^2 \eta^2 \mathcal{E}_4 (v, v, w, w) + \tfrac{1}{6}
  \xi \eta^3 \mathcal{E}_4 (v, w, w, w) \nonumber\\
  &  & + \tfrac{1}{24} \eta^4 \mathcal{E}_4 (w, w, w, w) +\mathcal{O} [(\xi^2
  + \eta^2)^2],
\end{eqnarray}
o{\`u} l'on convient que toutes les diff{\'e}rentielles de $\mathcal{E}$ sont
{\'e}valu{\'e}es au point d'{\'e}quilibre $u_0$.

Pour que l'{\'e}quilibre soit stable, il faut que cette expression soit
positive ou nulle pour tous $\xi$ et $\eta$ suffisamment petits. En prenant
tout d'abord $\eta = 0$, on obtient les conditions n{\'e}cessaires
\begin{equation}
  \label{eq20211108164416} \mathcal{E}_3 (v, v, v) = 0 \quad \text{et} \quad
  \mathcal{E}_4 (v, v, v, v) \geq 0 \quad \text{pour tout} \quad v \in V.
\end{equation}
En d'autres termes, s'il existe $v \in V$ tel que $\mathcal{E}_3 (v, v, v)
\neq 0$ ou $\mathcal{E}_4 (v, v, v, v) < 0$, alors l'{\'e}quilibre est
{\tmem{instable}}. Les conditions pr{\'e}c{\'e}dentes ne sont pas suffisantes
pour assurer la stabilit{\'e}. En effet, supposant ces conditions remplies, on
prend maintenant $\eta = \xi^2$
\begin{equation}
  \Delta \mathcal{E}= \tfrac{1}{2} \xi^4  \left[ \mathcal{E}_2 (w, w)
  +\mathcal{E}_3 (v, v, w) + \tfrac{1}{12} \mathcal{E}_4 (v, v, v, v) \right]
  + o (\xi^4)
\end{equation}
et on obtient la condition n{\'e}cessaire suppl{\'e}mentaire
\begin{equation}
  \label{eq20211109145356} \mathcal{E}_2 (v, v) +\mathcal{E}_3 (v, v, w) +
  \tfrac{1}{12} \mathcal{E}_4 (v, v, v, v) \geq 0,
\end{equation}
pour tous $v \in V$ et $w \in W$. Pour $v \in V$ fix{\'e}, l'expression
pr{\'e}c{\'e}dente est minimale lorsque $w$ satisfait le probl{\`e}me
variationnel
\begin{equation}
  \label{eq20211109145224} 2\mathcal{E}_2 (w, \hat{w}) +\mathcal{E}_3 (v, v,
  \hat{w}) = 0 \quad \text{pour tout} \quad \hat{w} \in W.
\end{equation}
Soit $w_{i  j} \in W$ l'unique solution du probl{\`e}me variationnel
suivant
\begin{equation}
  \label{eq20211221155859} 2\mathcal{E}_2 (w_{i  j}, \hat{w})
  +\mathcal{E}_3 (v_i, v_j, \hat{w}) = 0 \quad \text{pour tout} \quad \hat{w}
  \in W.
\end{equation}
Alors, pour $v = \xi^i v_i$, la solution du probl{\`e}me
variationnel~\eqref{eq20211109145224} est $w = \xi^i \xi^j w_{i  j}$.
Pour cette valeur de $v$, la condition~\eqref{eq20211109145356} s'{\'e}crit
\begin{equation}
  \left[ \tfrac{1}{12} \mathcal{E}_4 (v_i, v_j, v_k, v_l) -\mathcal{E}_2 (w_{i
   j}, w_{k  l}) \right] \xi^i \xi^j \xi^k \xi^l \geq 0,
\end{equation}
pour tous $\xi_i, \xi_j, \xi_k, \xi_l \in \mathbb{R}$. On peut montrer que
l'in{\'e}galit{\'e} stricte est une condition {\tmem{suffisante}} de
stabilit{\'e}.

\section{Bifurcations}

On {\'e}crit toute courbe d'{\'e}quilibre passant par le point $(u_0,
\lambda_0)$ sous la forme param{\'e}trique suivantes
\begin{eqnarray}
  \lambda & = & \lambda_0 + \eta \lambda_1 + \eta^2 \lambda_2 + \eta^3
  \lambda_3 + \cdots,  \label{eq20211115075817}\\
  u & = & u^{\ast} (\lambda) + \eta u_1 + \eta^2 u_2 + \eta^3 u_3 + \cdots,
  \label{eq20211115075835}
\end{eqnarray}
o{\`u} $\eta$ est un param{\`e}tre, non pr{\'e}cis{\'e} pour le moment. Noter
que, dans la repr{\'e}sentation param{\'e}trique de $u$, $u^{\ast}$ est
{\'e}valu{\'e} en $\lambda$ et pas en $\lambda_0$.

On se restreindra dans ce qui suit au cas non-d{\'e}g{\'e}n{\'e}r{\'e} $u_1
\neq 0$. On peut alors toujours supposer que $\langle u_1, u_1 \rangle = 1$.
En effet, en posant $\theta = \lVert u_1 \rVert \eta$, les d{\'e}veloppements
pr{\'e}c{\'e}dents s'{\'e}crivent
\begin{eqnarray}
  \lambda & = & \lambda_0 + \theta \lVert u_1 \rVert^{- 1} \lambda_1 + \theta
  \nonconverted{twosuperior} \lVert u_1 \rVert^{- 2} \lambda_2 + \theta^3
  \lVert u_1 \rVert^{- 3} \lambda_3 + \cdots, \\
  u & = & u^{\ast} (\lambda) + \theta \lVert u_1 \rVert^{- 1} u_1 + \theta
  \nonconverted{twosuperior} \lVert u_1 \rVert^{- 2} u_2 + \theta^3 \lVert u_1
  \rVert^{- 3} u_3 +
\end{eqnarray}
et le terme lin{\'e}aire en $\theta$ du d{\'e}veloppement asymptotique de $u$
est bien de norme 1. {\tmstrong{Que se passe-t-il si $u_1 = u_2 = \ldots = 0$
?}}

Les coefficients $\lambda_k$ et $u_k$ des
d{\'e}veloppements~\eqref{eq20211115075817} et \eqref{eq20211115075835} sont
identifi{\'e}s en {\'e}crivant que l'{\'e}nergie est stationnaire le long de
la courbe d'{\'e}quilibre, c'est-{\`a}-dire que le r{\'e}sidu $\mathcal{E}_{,
u}  [u (\eta), \lambda (\eta)]$ est nul. Le d{\'e}veloppement limit{\'e} du
r{\'e}sidu est {\'e}tabli au voisinage de $\eta = 0$ dans
l'annexe~\ref{sec20211112182000} [voir {\'E}q.~\eqref{eq20220107080901}]. En
{\'e}crivant que tous ses termes s'annulent, on trouve successivement, pour
tout $\hat{u} \in U$
\begin{equation}
  \label{eq20211112182917} \mathcal{E}_2 (\lambda_0 ; u_1, \hat{u}) = 0,
\end{equation}
\begin{equation}
  \label{eq20211112183220} \mathcal{E}_3 (\lambda_0 ; u_1, u_1, \hat{u}) + 2
  \lambda_1  \dot{\mathcal{E}_2} (\lambda_0 ; u_1, \hat{u}) + 2\mathcal{E}_2
  (\lambda_0 ; u_2, \hat{u}) = 0,
\end{equation}
\begin{eqnarray}
  \mathcal{E}_4 (\lambda_0 ; u_1, u_1, u_1, \hat{u}) + 6\mathcal{E}_3
  (\lambda_0 ; u_1, u_2, \hat{u}) + 6\mathcal{E}_2 (\lambda_0 ; u_3, \hat{u})
  &  &  \nonumber\\
  + 3 \lambda_1  [\dot{\mathcal{E}_3} (\lambda_0 ; u_1, u_1, \hat{u}) + 2
  \dot{\mathcal{E}_2} (\lambda_0 ; u_2, \hat{u})] &  &  \nonumber\\
  + 3 \lambda_1^2  \ddot{\mathcal{E}_2} (\lambda_0 ; u_1, \hat{u}) + 6
  \lambda_2  \dot{\mathcal{E}_2} (\lambda_0 ; u_1, \hat{u}) & = & 0.
  \label{eq20220114135717}
\end{eqnarray}
On d{\'e}duit de l'{\'e}quation~\eqref{eq20211112182917} que $u_1 \in V$. On
pose alors
\begin{equation}
  \label{eq20220124135236} u_1 = \xi_1^i v_i .
\end{equation}
En prenant $\hat{u} = v_i$, l'{\'e}quation~\eqref{eq20211112183220}
s'{\'e}crit
\begin{equation}
  \label{eq20220216140121} \mathcal{E}_{i  j  k} (\lambda_0)
  \hspace{0.17em} \xi_1^j \xi_1^k + 2 \lambda_1  \dot{\mathcal{E}}_{i
  j} (\lambda_0)  \hspace{0.17em} \xi_1^j = 0.
\end{equation}
Pour le terme d'ordre 2, on introduit la d{\'e}composition~: $u_2 = \xi_2^i
v_i + u_2^W$, o{\`u} $u_2^W \in W$. On a alors $\mathcal{E}_2 (u_2, \hat{u})
=\mathcal{E}_2 (u_2^W, \hat{u})$ et l'{\'e}quation~\eqref{eq20211112183220}
s'{\'e}crit
\begin{equation}
  \mathcal{E}_3 (\lambda_0 ; u_1, u_1, \hat{u}) + 2 \lambda_1
  \dot{\mathcal{E}_2} (\lambda_0 ; u_1, \hat{u}) + 2\mathcal{E}_2 (\lambda_0 ;
  u_2^W, \hat{u}) = 0,
\end{equation}
pour tout $\hat{u} \in V$. En prenant la fonction test dans l'espace $W$, on
obtient le probl{\`e}me variationnel suivant~: trouver $u_2^W \in W$ tel que
\begin{equation}
  \label{eq20211210131623} \xi_1^i \xi_1^j \mathcal{E}_3 (\lambda_0 ; v_i,
  v_j, \hat{w}) + 2 \lambda_1 \xi_1^i  \dot{\mathcal{E}_2} (\lambda_0 ; v_i,
  \hat{w}) + 2\mathcal{E}_2 (\lambda_0 ; u_2^W, \hat{w}) = 0,
\end{equation}
pour tout $\hat{w} \in W$. Soient $w_i \in W$ les solutions des probl{\`e}mes
variationnels suivants
\begin{equation}
  \label{eq20220208143055} \mathcal{E}_2 (\lambda_0 ; w_i, \hat{w}) +
  \dot{\mathcal{E}_2} (\lambda_0 ; v_i, \hat{w}) = 0,
\end{equation}
pour tout $\hat{w} \in W$. La solution du
probl{\`e}me~\eqref{eq20211210131623} s'obtient par simple combinaison
lin{\'e}aire des $w_i$ et $w_{ij}$ introduits pr{\'e}c{\'e}demment par le
probl{\`e}me variationnel~\eqref{eq20211221155859}, de sorte que
\begin{equation}
  \label{eq20220124135324} u_2^W = \xi_1^i \xi_1^j w_{i  j} +
  \lambda_1 \xi_1^i w_i \quad \text{et} \quad u_2 = \xi_2^i v_i + \xi_1^i
  \xi_1^j w_{i  j} + \lambda_1 \xi_1^i w_i .
\end{equation}
En prenant $\hat{u} = v_i$ dans l'{\'e}quation~\eqref{eq20220114135717}, on
obtient l'{\'e}quation de bifurcation suivante
\begin{eqnarray}
  6 \xi_2^j  [\xi_1^k \mathcal{E}_{i  j  k} (\lambda_0) +
  \lambda_1  \dot{\mathcal{E}}_{i  j} (\lambda_0)] &  &  \nonumber\\
  + \xi_1^j \xi_1^k \xi_1^l  [\mathcal{E}_{i  j  k  l}
  (\lambda_0) + 6\mathcal{E}_3 (\lambda_0 ; v_i, v_j, w_{k  l})] &  &
  \nonumber\\
  + 3 \lambda_1 \xi_1^j \xi_1^k  [\dot{\mathcal{E}}_{i  j  k}
  (\lambda_0) + 2\mathcal{E}_3 (\lambda_0 ; v_i, v_j, w_k) + 2
  \dot{\mathcal{E}_2} (\lambda_0 ; v_i, w_{j  k})] &  &  \nonumber\\
  + 3 \lambda_1^2 \xi_1^j  [\ddot{\mathcal{E}}_{i  j} (\lambda_0) + 2
  \dot{\mathcal{E}_2} (v_i, w_j)] + 6 \lambda_2 \xi_1^j  \dot{\mathcal{E}}_{i
   j} (\lambda_0) & = & 0.  \label{eq20220210143805}
\end{eqnarray}
On remarque que certains termes peuvent {\^e}tre sym{\'e}tris{\'e}s. Ainsi
\begin{eqnarray}
  \xi_1^j \xi_1^k \xi_1^l \mathcal{E}_3 (\lambda_0 ; v_i, v_j, w_{k
  l}) & = & \tfrac{1}{3} \xi_1^j \xi_1^k \xi_1^l  [\mathcal{E}_3 (\lambda_0 ;
  v_i, v_j, w_{k  l})  +\mathcal{E}_3 (\lambda_0 ; v_i, v_k,
  w_{l  j}) \nonumber\\
  &  &  \nosymbol +\mathcal{E}_3 (\lambda_0 ; v_i, v_l, w_{j
   k})],
\end{eqnarray}
de m{\^e}me
\begin{equation}
  2 \xi_1^j \xi_1^k \mathcal{E}_3 (\lambda_0 ; v_i, v_j, w_k) = \xi_1^j
  \xi_1^k  [\mathcal{E}_3 (\lambda_0 ; v_i, v_j, w_k) +\mathcal{E}_3
  (\lambda_0 ; v_i, w_j, v_k)]
\end{equation}
et l'{\'e}quation~\eqref{eq20220210143805} s'{\'e}crit
\begin{equation}
  \label{eq20220216141706} 6 A_{i  j} \xi_2^j + E_{i  j
   k  l}  \hspace{0.17em} \xi_1^j \xi_1^k \xi_1^l + 3
  \lambda_1 F_{i  j  k} \xi_1^j \xi_1^k + 3 \lambda_1^2 G_{i
   j} \xi_1^j + 6 \lambda_2  \ring{E}_{i  j} \xi_1^j = 0,
\end{equation}
en posant \tmtextbf{C'est l'expression de $B_{ij}$ de Nick, voir {\'E}q.
(AC-5.14) p. 74}
\begin{equation}
  A_{i  j} = \xi_1^k \mathcal{E}_{i  j  k} (\lambda_0)
  + \lambda_1  \dot{\mathcal{E}}_{i  j} (\lambda_0)
\end{equation}
\tmtextbf{Cette expression co{\"i}ncide avec l'expression (AC-5.11), page 71,
de $\mathcal{E}_{i  j  k  l}$ de Nick. Le facteur 2
provient du fait que dans le probl{\`e}me variationnel (AC-5.9) qui
d{\'e}finit les $v_{i  j}$ de Nick, le facteur 2 du
probl{\`e}me~\eqref{eq20211221155859} n'est pas pr{\'e}sent.}
\begin{equation}
  E_{i  j  k  l} =\mathcal{E}_4  (\lambda_0 ; v_i,
  v_j, v_k, v_l) + 2 [\mathcal{E}_3 (\lambda_0 ; v_i, v_j, w_{k  l})
  +\mathcal{E}_3 (\lambda_0 ; v_i, v_k, w_{l  j}) +\mathcal{E}_3
  (\lambda_0 ; v_i, v_l, w_{j  k})] .
\end{equation}
\begin{equation}
  F_{i  j  k} = \dot{\mathcal{E}}_3  (\lambda_0 ; v_i, v_j,
  v_k) +\mathcal{E}_3  (\lambda_0 ; v_i, v_j, w_k) +\mathcal{E}_3  (\lambda_0
  ; v_i, w_j, v_k) + 2 \dot{\mathcal{E}_2} (\lambda_0 ; v_i, w_{j  k})
\end{equation}
\begin{equation}
  G_{i  j} = \ddot{\mathcal{E}}_{i  j} (\lambda_0) + 2
  \dot{\mathcal{E}_2} (v_i, w_j)
\end{equation}
On supposera satisfaite la condition suivante, qui assure que ce syst{\`e}me
est r{\'e}gulier
\begin{equation}
  \det (\xi_1^k \mathcal{E}_{i  j  k} + \lambda_1
  \dot{\mathcal{E}}_{i  j}) \neq 0.
\end{equation}
Les $\xi_2^i$ sont donc d{\'e}termin{\'e}s de fa{\c c}on unique si
$\lambda_1$, $\lambda_2$ et $\xi_1^i$ sont connus.

Le d{\'e}veloppement limit{\'e} suivant de l'{\'e}nergie le long de la branche
bifurqu{\'e}e est {\'e}tabli dans l'annexe~\ref{sec20220121172919}
\begin{eqnarray}
  \mathcal{E} [u (\eta), \lambda (\eta)] & = & \mathcal{E} [u^{\ast} [\lambda
  (\eta)], \lambda (\eta)] + \tfrac{1}{2} \eta^2 \mathcal{E}_2 (\lambda_0 ;
  u_1, u_1) \nonumber\\
  &  & + \tfrac{1}{6} \eta^3  [\mathcal{E}_3 (\lambda_0 ; u_1, u_1, u_1) +
  6\mathcal{E}_2 (\lambda_0 ; u_1, u_2)  \nonumber\\
  &  &  + 3 \lambda_1  \dot{\mathcal{E}}_2 (\lambda_0 ; u_1, u_1)]
  + \tfrac{1}{24} \eta^4  \{ \mathcal{E}_4 (\lambda_0 ; u_1, u_1, u_1, u_1)
   \nonumber\\
  &  & + 12\mathcal{E}_3 (\lambda_0 ; u_1, u_1, u_2) + 12\mathcal{E}_2
  (\lambda_0 ; u_2, u_2) \nonumber\\
  &  & + 18\mathcal{E}_2 (\lambda_0 ; u_1, u_3) + 4 \lambda_1
  [\dot{\mathcal{E}}_3 (\lambda_0 ; u_1, u_1, u_1)  \nonumber\\
  &  &  + 6 \dot{\mathcal{E}}_2 (\lambda_0 ; u_1, u_2)] + 6
  \lambda_1^2  \ddot{\mathcal{E}}_2 (\lambda_0 ; u_1, u_1) \nonumber\\
  &  &  + 12 \lambda_2  \dot{\mathcal{E}}_2 (\lambda_0 ; u_1, u_1)
  \} + o (\eta^4) .  \label{eq20220121172753}
\end{eqnarray}
La relation~\eqref{eq20211112182917} montre tout d'abord que les termes en
$\mathcal{E}_2 (\lambda_0 ; u_1, u_i)$ sont nuls pour $i = 1, 2, 3$. Le
premier terme non-nul du d{\'e}veloppement limit{\'e}~\eqref{eq20220121172753}
est donc le terme d'ordre 3. En prenant de plus $\hat{u} = u_1$ dans la
relation~\eqref{eq20211112183220}, on trouve finalement \tmtextbf{Cette
expression co{\"i}ncide avec l'{\'E}q. (AC-5.29) de Tryantafyllidis.}
\begin{equation}
  \mathcal{E} [u (\eta), \lambda (\eta)] =\mathcal{E} (u^{\ast} [\lambda
  (\eta)], \lambda (\eta)) + \tfrac{1}{6} \lambda_1 \eta^3
  \dot{\mathcal{E}}_2 (u_1, u_1) + o (\eta^3) .
\end{equation}
Si $\lambda_1 = 0$, le premier terme non-nul du d{\'e}veloppement
limit{\'e}~\eqref{eq20220121172753} est d'ordre 4. En prenant cette fois
$\hat{u} = u_2$ dans la relation~\eqref{eq20211112183220} et $\hat{u} = u_1$
dans la relation~\eqref{eq20220114135717}, on obtient \tmtextbf{Cette
expression co{\"i}ncide avec l'{\'E}q. (AC-5.30) de Tryantafyllidis.}
\begin{equation}
  \mathcal{E} [u (\eta), \lambda (\eta)] =\mathcal{E} (u^{\ast} [\lambda
  (\eta)], \lambda (\eta)) + \tfrac{1}{4} \lambda_2 \eta^4
  \dot{\mathcal{E}}_2 (\lambda_0 ; u_1, u_1) + o (\eta^4) .
\end{equation}
\begin{center}
  ***
\end{center}

Pour analyser la stabilit{\'e} de la branche bifurqu{\'e}e ainsi trouv{\'e}e,
il faut d{\'e}terminer le signe de la hessienne de l'{\'e}nergie. On peut
d'ores et d{\'e}j{\`a} remarquer que, sur la branche fondamentale ($u_1 = u_2
= 0$), en prenant $\eta = \lambda - \lambda_0$ ($\lambda_1 = 1$)
\begin{equation}
  \mathcal{E}_2 (\lambda ; \hat{u}, \hat{v}) =\mathcal{E}_2 (\lambda_0 ;
  \hat{u}, \hat{v}) + (\lambda - \lambda_0)  \dot{\mathcal{E}}_2 (\lambda_0 ;
  \hat{u}, \hat{v}) + o (\lambda - \lambda_0) .
\end{equation}
Dans ce qui suit, on supposera que $\dot{\mathcal{E}}_2 (\lambda_0) \neq 0$.
Pour $\hat{v} \in V$, l'{\'e}galit{\'e} pr{\'e}c{\'e}dente s'{\'e}crit
\begin{equation}
  \mathcal{E}_2 (\lambda_0 ; \hat{v}, \hat{v}) = (\lambda - \lambda_0)
  \dot{\mathcal{E}}_2 (\hat{v}, \hat{v}) + o (\lambda - \lambda_0) .
\end{equation}
Comme la branche fondamentale est stable pour $\lambda < \lambda_0$, on doit
avoir $\dot{\mathcal{E}}_2 (\lambda_0 ; \hat{v}, \hat{v}) < 0$. La forme
quadratique $\dot{\mathcal{E}}_2 (\lambda_0)$ est donc d{\'e}finie
n{\'e}gative sur $V$. Le d{\'e}veloppement limit{\'e} de la hessienne de
l'{\'e}nergie le long de la branche bifurqu{\'e}e est {\'e}tabli dans
l'annexe~\ref{sec20211115081016}. Pour tout $\hat{u} \in U$, on trouve
\begin{eqnarray}
  \mathcal{E}_{, u  u} [u (\eta), \lambda (\eta) ; \hat{u}, \hat{u}] &
  = & \mathcal{E}_2 (\lambda_0 ; \hat{u}, \hat{u}) + \eta [\mathcal{E}_3
  (\lambda_0 ; u_1, \hat{u}, \hat{u}) + \lambda_1  \dot{\mathcal{E}}_2
  (\lambda_0 ; \hat{u}, \hat{u})] \nonumber\\
  &  & + \tfrac{1}{2} \eta^2  [\mathcal{E}_4 (\lambda_0 ; u_1, u_1, \hat{u},
  \hat{u}) + 2 \lambda_1  \dot{\mathcal{E}}_3 (\lambda_0 ; u_1, \hat{u},
  \hat{u})  \nonumber\\
  &  & + \lambda_1^2  \ddot{\mathcal{E}}_2 (\lambda_0 ; \hat{u}, \hat{u})
  +\mathcal{E}_3 (\lambda_0 ; u_2, \hat{u}, \hat{u})  + \lambda_2
  \dot{\mathcal{E}}_2 (\lambda_0 ; \hat{u}, \hat{u})] + o (\eta^2) .
  \label{eq20211115082025}
\end{eqnarray}
On peut d{\'e}composer le vecteur $\hat{u} \in U$ de fa{\c c}on unique sous la
forme $\hat{u} = \hat{v} + \hat{w}$, avec $\hat{v} \in V$ et $\hat{w} \in W$.
Le terme constant du d{\'e}veloppement pr{\'e}c{\'e}dent vaut alors
$\mathcal{E}_2 (\lambda_0 ; \hat{w}, \hat{w})$. Si $\hat{w} \neq 0$, alors ce
terme constant est strictement positif, puisque la hessienne est d{\'e}finie
positive sur $W$ en $\lambda = \lambda_0$. La hessienne sur la branche
bifurqu{\'e}e est donc positive pour tout $\hat{u} \in U$ ayant une composante
dans $W$. Il suffit donc d'{\'e}tudier le signe de la hessienne sur la branche
bifurqu{\'e}e pour $\hat{u} \in V$, soit $\hat{u} = \hat{\xi}^i v_i$. Dans ce
cas, compte-tenu de l'expression~\eqref{eq20220124135324} de $u_2$
\begin{eqnarray}
  \mathcal{E}_3 (\lambda_0 ; u_2, \hat{u}, \hat{u}) & = & \xi_1^i \xi_1^j
  \hat{\xi}^k  \hat{\xi}^l \mathcal{E}_3 (\lambda_0 ; w_{i  j}, v_k,
  v_l) + \xi_2^i  \hat{\xi}^j  \hat{\xi}^k \mathcal{E}_3 (\lambda_0 ; v_i,
  v_j, v_k) \nonumber\\
  &  & + \lambda_1 \xi_1^i  \hat{\xi}^j  \hat{\xi}^k \mathcal{E}_3 (\lambda_0
  ; w_i, v_j, v_k) .
\end{eqnarray}
On peut compl{\`e}tement sym{\'e}triser le premier terme
\begin{eqnarray}
  \xi_1^i \xi_1^j  \hat{\xi}^k  \hat{\xi}^l \mathcal{E}_3 (\lambda_0 ; w_{i
   j}, v_k, v_l) & = & \tfrac{1}{3} [\xi_1^i \xi_1^j  \hat{\xi}^k
  \hat{\xi}^l \mathcal{E}_3 (\lambda_0 ; w_{i  j}, v_k, v_l)
   \nonumber\\
  &  & + \xi_1^i \xi_1^j  \hat{\xi}^k  \hat{\xi}^l \mathcal{E}_3 (\lambda_0 ;
  w_{i  j}, v_k, v_l) \nonumber\\
  &  &  + \xi_1^i \xi_1^j  \hat{\xi}^k  \hat{\xi}^l \mathcal{E}_3
  (\lambda_0 ; w_{i  j}, v_k, v_l)]
\end{eqnarray}
\begin{eqnarray}
  \mathcal{E}_{, u  u} [u (\eta), \lambda (\eta) ; \hat{u}, \hat{u}] &
  = & \eta \hat{\xi}^i  \hat{\xi}^j  [\xi_1^k \mathcal{E}_{i  j
   k} (\lambda_0) + \lambda_1  \dot{\mathcal{E}}_{i  j}
  (\lambda_0)] \nonumber\\
  &  & + \tfrac{1}{2} \eta \nonconverted{twosuperior} \hat{\xi}^i
  \hat{\xi}^j  \{ \xi_1^k \xi_1^l  [\mathcal{E}_{i  j  k
   l} (\lambda_0) - 2\mathcal{E}_2 (\lambda_0 ; w_{i  j}, w_{k
   l})]  \nonumber\\
  &  & + \lambda_1 \xi_1^k  [\mathcal{E}_3 (\lambda_0 ; v_i, v_j, w_k) +
  \dot{\mathcal{E}}_{i  j  k} (\lambda_0)] \nonumber\\
  &  & + \lambda_1^2  \ddot{\mathcal{E}}_{i  j} (\lambda_0) + \xi_2^k
  \mathcal{E}_{i  j  k} (\lambda_0) + \lambda_2
  \dot{\mathcal{E}}_{i  j} (\lambda_0 \} + o (\eta^2) .
  \label{eq20220203144500}
\end{eqnarray}
Compte-tenu de la relation~\eqref{eq20211112183220}, on trouve pour $\hat{v} =
u_1$ ($\hat{\xi}^i = \xi_1^i$)
\begin{equation}
  \mathcal{E}_{, u  u} [u (\eta), \lambda (\eta) ; u_1, u_1] = -
  \lambda_1 \eta \dot{\mathcal{E}}_2 (\lambda_0 ; u_1, u_1) + o (\eta) .
\end{equation}
Si $\lambda_1 \neq 0$, l'expression pr{\'e}c{\'e}dente peut {\'e}galement
s'{\'e}crire
\begin{equation}
  \mathcal{E}_{, u  u} [u (\eta), \lambda (\eta) ; u_1, u_1] = -
  (\lambda - \lambda_0)  \dot{\mathcal{E}}_2 (\lambda_0 ; u_1, u_1) + o
  (\lambda - \lambda_0),
\end{equation}
qui est n{\'e}gative pour $\lambda < \lambda_0$: la branche bifurqu{\'e}e est
instable sous la charge critique. Il reste alors {\`a} {\'e}tudier le signe de
la hessienne de la branche bifurqu{\'e}e au-del{\`a} de la charge critique
($\lambda > \lambda_0$).

\section{Cas d'un mode de flambement simple ($m = 1$)}

Lorsque $m = \dim V = 1$, la base $v_1, \ldots, v_m$ est r{\'e}duite au seul
vecteur $v_1$ et $u_1$ est parall{\`e}le {\`a} ce vecteur. Comme $\lVert u_1
\rVert = 1$, on a donc n{\'e}cessairement $u_1 = v_1$ (quitte {\`a} changer
$\eta$ en $- \eta$). L'{\'e}quation de bifurcation~\eqref{eq20220216140121}
s'{\'e}crit alors
\begin{equation}
  \label{eq20220203144712} \mathcal{E}_{1  1  1} (\lambda_0) +
  2 \lambda_1  \dot{\mathcal{E}}_{1  1} (\lambda_0) = 0, \quad
  \text{soit} \quad \lambda_1 = - \frac{\mathcal{E}_{1  1  1}
  (\lambda_0)}{2 \dot{\mathcal{E}}_{1  1} (\lambda_0)},
\end{equation}
o{\`u} on remarque que le quotient a un sens, puisque $\dot{\mathcal{E}_2}
(\lambda_0)$ est d{\'e}finie n{\'e}gative sur $V$. On trouve donc les
d{\'e}veloppements limit{\'e}s
\begin{equation}
  \lambda = \lambda_0 + \lambda_1 \eta + o (\eta)  \quad \text{et} \quad u =
  u^{\ast} (\lambda) + \eta v_1 + o (\eta),
\end{equation}
soit finalement, en {\'e}liminant $\eta$
\begin{equation}
  \lambda = \lambda_0 - \frac{\xi \mathcal{E}_{1  1  1}
  (\lambda_0)}{2 \dot{\mathcal{E}}_{1  1} (\lambda_0)} + o (\xi),
  \quad \text{avec} \quad \xi = \langle u (\lambda) - u^{\ast} (\lambda), v_1
  \rangle .
\end{equation}
Pour d{\'e}terminer la stabilit{\'e} de la branche bifurqu{\'e}e, on calcule
la hessienne en $(v_1, v_1)$. L'{\'e}quation~\eqref{eq20220203144500}
s'{\'e}crit
\begin{equation}
  \mathcal{E}_{, u  u} [u (\eta), \lambda (\eta) ; v_1, v_1] = \eta
  [\mathcal{E}_{1  1  1} (\lambda_0) + \lambda_1
  \dot{\mathcal{E}}_{1  1} (\lambda_0)] + o (\eta),
\end{equation}
soit, en substituant l'{\'e}quation~\eqref{eq20220203144712}
\begin{equation}
  \mathcal{E}_{, u  u} [u (\eta), \lambda (\eta) ; v_1, v_1] = -
  \lambda_1 \eta \dot{\mathcal{E}}_{1  1} (\lambda_0) + o (\eta) .
\end{equation}
Ce d{\'e}veloppement ne permet de conclure que si le terme lin{\'e}aire est
non-nul, soit $\mathcal{E}_{1  1  1} (\lambda_0) \neq 0$ [voir
{\'E}q.~\eqref{eq20220203144712}]. Dans ce cas, le d{\'e}veloppement
asymptotique pr{\'e}c{\'e}dent s'{\'e}crit {\'e}galement
\begin{equation}
  \mathcal{E}_{, u  u} [u (\eta), \lambda (\eta) ; v_1, v_1] = -
  (\lambda - \lambda_0)  \dot{\mathcal{E}}_{1  1} (\lambda_0) + o
  (\lambda - \lambda_0) .
\end{equation}
Comme $\dot{\mathcal{E}}_2 (\lambda_0)$ est d{\'e}finie n{\'e}gative, la
branche bifurqu{\'e}e est {\tmem{instable}} pour $\lambda < \lambda_0$ et
{\tmem{stable}} pour $\lambda > \lambda_0$ lorsque $\mathcal{E}_{1  1
 1} (\lambda_0) \neq 0$.

Supposons maintenant que $\mathcal{E}_{1  1  1} (\lambda_0) =
0$~; alors $\lambda_1 = 0$ et il faut calculer au moins un terme
suppl{\'e}mentaire dans le d{\'e}veloppement limit{\'e} de la Hessienne.
L'{\'e}quation de bifurcation~\eqref{eq20220216141706} s'{\'e}crit
\begin{equation}
  \label{eq20220217164528} \mathcal{E}_{1  1  1  1}
  (\lambda_0) + 6\mathcal{E}_3 (\lambda_0 ; v_1, v_1, u_2) + 6 \lambda_2
  \dot{\mathcal{E}}_{1  1} (\lambda_0) = 0.
\end{equation}
En introduisant le d{\'e}veloppement~\eqref{eq20220124135324} de $u_2$ et en
utilisant le probl{\`e}me variationnel~\eqref{eq20211221155859}
\begin{equation}
  u_2 = \xi_2 v_1 + w_{1  1} + \lambda_1 w_1,
\end{equation}
donc
\begin{equation}
  \mathcal{E}_3 (\lambda_0 ; v_1, v_1, u_2) =\mathcal{E}_3 (\lambda_0 ; v_1,
  v_1, w_{1  1}) = - 2\mathcal{E}_2 (\lambda_0 ; w_{11}, w_{11})
\end{equation}
soit finalement
\[ \lambda_2 = - \frac{\mathcal{E}_{1  1  1  1}
   (\lambda_0) - 12\mathcal{E}_2 (\lambda_0 ; w_{11}, w_{11})}{6
   \dot{\mathcal{E}}_{1  1} (\lambda_0)}, \]
le quotient ayant une nouvelle fois un sens. Le d{\'e}veloppement
asymptotique~\eqref{eq20211115082025} de la Hessienne s'{\'e}crit alors, en
tenant compte de l'{\'E}q.~\eqref{eq20220217164528}
\begin{eqnarray}
  \mathcal{E}_{, u  u} [u (\eta), \lambda (\eta) ; v_1, v_1] & = &
  \tfrac{1}{2} \eta^2  [\mathcal{E}_{1  1  1  1}
  (\lambda_0) + 2\mathcal{E}_3 (\lambda_0 ; v_1, v_1, u_2) + 2 \lambda_2
  \dot{\mathcal{E}}_{1  1} (\lambda_0)] + o (\eta^2) \nonumber\\
  & = & \tfrac{5}{12} \eta^2 \mathcal{E}_{1  1  1  1}
  (\lambda_0) + o (\eta^2) .
\end{eqnarray}

\section{Propri{\'e}t{\'e}s des formes bilin{\'e}aires sym{\'e}triques,
positives}

Dans ce qui suit, $\mathcal{B}$ d{\'e}signe une forme bilin{\'e}aire
sym{\'e}trique et positive sur l'espace vectoriel $U$. On d{\'e}finit son
noyau $\ker \mathcal{B}$ de la fa{\c c}on suivante
\begin{equation}
  \ker \mathcal{B}= \{u \in U, \mathcal{B}(u, u) = 0\} .
\end{equation}
\begin{theorem}
  Le noyau d'une forme bilin{\'e}aire, sym{\'e}trique et positive est un
  sous-espace vectoriel.
\end{theorem}

\begin{proof}
  Soient $u, v \in \ker \mathcal{B}$, $\alpha \in \mathbb{R}$ et $w = u +
  \alpha v$. Montrons que $w \in \ker \mathcal{B}$. Il suffit d'{\'e}valuer
  $\mathcal{B} (w, w)$
  \begin{equation}
    \mathcal{B} (w, w) =\mathcal{B} (u + \alpha v, u + \alpha v) =\mathcal{B}
    (u, u) + 2 \alpha \mathcal{B} (u, v) + \alpha^2 \mathcal{B} (v, v),
  \end{equation}
  o{\`u} l'on a tenu compte de la sym{\'e}trie de $\mathcal{B}$ pour
  {\'e}crire que $\mathcal{B} (u, v) =\mathcal{B} (v, u)$. Comme $u, v \in
  \ker \mathcal{B}$, le premier et le dernier terme sont nuls, soit
  $\mathcal{B} (w, w) = 2 \alpha \mathcal{B} (u, v)$. La forme bilin{\'e}aire
  {\'e}tant positive, cette grandeur est positive, {\tmem{quelle que soit la
  valeur de $\alpha \in \mathbb{R}$}}. On en d{\'e}duit donc que $\mathcal{B}
  (u, v) = 0$, puis que $\mathcal{B} (w, w) = 0$ et donc que $w \in \ker
  \mathcal{B}.$
\end{proof}

\begin{theorem}
  Soit $u \in V$. Alors
  \begin{equation}
    u \in \ker \mathcal{B} \quad \text{ssi} \quad \text{pour tout } v \in V,
    \mathcal{B} (u, v) = 0.
  \end{equation}
\end{theorem}

\begin{proof}
  Soient $u \in \ker \mathcal{B}$, $v \in V$ et $\alpha \in \mathbb{R}$. Comme
  pr{\'e}c{\'e}demment, on {\'e}crit que $\mathcal{B} (w, w) \geq 0$, avec $w
  = \alpha u + v$
  \begin{equation}
    \mathcal{B} (w, w) = 2 \alpha \mathcal{B} (u, v) +\mathcal{B} (v, v) \geq
    0,
  \end{equation}
  o{\`u} l'on a tenu compte de ce que $\mathcal{B} (u, u) = 0$. L'expression
  pr{\'e}c{\'e}dente, affine en $\alpha$, a un signe constant. Le terme
  lin{\'e}aire en $\alpha$ est donc nul, soit $\mathcal{B} (u, v) = 0$.
  R{\'e}ciproquement, si $\mathcal{B} (u, v) = 0$ pour tout $v \in V$, alors
  $\mathcal{B} (u, u) = 0$ (en prenant $v = u$).
\end{proof}

\section{D{\'e}veloppements limit{\'e}s le long d'une branche bifurqu{\'e}e du
diagramme d'{\'e}quilibre}

\subsection{Principe du calcul}\label{sec20220107121442}

On pose dans ce qui suit
\begin{eqnarray}
  \Lambda (\eta) & = & \lambda (\eta) - \lambda_0 = \eta \lambda_1 + \eta^2
  \lambda_2 + \eta^3 \lambda_3 + \cdots,  \label{eq20211112155446}\\
  U (\eta) & = & u (\eta) - u^{\ast} [\lambda (\eta)] = \eta u_1 + \eta^2 u_2
  + \eta^3 u_3 + \cdots .  \label{eq20211112113028}
\end{eqnarray}
On consid{\`e}re une fonctionnelle $\mathcal{F}$ de $u$ et $\lambda$~:
$\mathcal{F} (u, \lambda)$. Cette fonctionnelle est {\'e}valu{\'e}e le long de
la branche bifurqu{\'e}e. En d'autres termes, on consid{\`e}re
\begin{equation}
  f (\eta) = F \{ u^{\ast} [\lambda_0 + \Lambda (\eta)] + U (\eta), \lambda_0
  + \Lambda (\eta) \} .
\end{equation}
On souhaite {\'e}tablir un d{\'e}veloppement limit{\'e} de $f$ au voisinage de
$\eta = 0$, ce qui conduit {\`a} calculer les d{\'e}riv{\'e}es successives de
$f$ en $\eta = 0$, puisque
\begin{equation}
  f (\eta) = f (0) + \eta f' (0) + \tfrac{1}{2} \eta^2 f'' (0) + \cdots
\end{equation}
Pour calculer ces d{\'e}riv{\'e}es, il sera commode d'introduire la fonction
auxiliaire $F$
\begin{equation}
  F (\eta, \lambda) =\mathcal{F} [u^{\ast} (\lambda) + U (\eta), \lambda],
\end{equation}
dans laquelle les variables $\lambda$ et $\eta$ sont provisoirement
consid{\'e}r{\'e}es comme ind{\'e}pendantes. On a
\begin{equation}
  f (\eta) = F [\eta, \lambda_0 + \Lambda (\eta)],
\end{equation}
d'o{\`u} l'on d{\'e}duit successivement que
\begin{equation}
  \label{eq20211112162417} f' (\eta) = \partial_{\eta} F + \Lambda'
  \partial_{\lambda} F,
\end{equation}
\begin{equation}
  \label{eq20211112165810} f'' (\eta) = \partial_{\eta  \eta}^2 F + 2
  \Lambda' \partial_{\eta  \lambda}^2 {F + \Lambda'}^2
  \partial_{\lambda  \lambda}^2 F + \Lambda'' \partial_{\lambda} F,
\end{equation}
\begin{eqnarray}
  \label{eq20211112173223} f''' (\eta) & = & \partial_{\eta  \eta
   \eta}^3 F + 3 \Lambda' \partial_{\eta  \eta
  \lambda}^3 {F + 3 \Lambda'}^2 \partial_{\eta  \lambda
  \lambda}^3 {F + \Lambda'}^3 \partial_{\lambda  \lambda
  \lambda}^3 F + 3 \Lambda'' \partial_{\eta  \lambda}^2 F + 3 \Lambda'
  \Lambda'' \partial_{\lambda  \lambda}^2 F \nonumber\\
  &  & \nosymbol + \Lambda''' \partial_{\lambda} F
\end{eqnarray}
\begin{eqnarray}
  f'''' (\eta) & = & \partial_{\eta  \eta  \eta
  \eta}^4 F + 4 \Lambda' \partial_{\eta  \eta  \eta
  \lambda}^4 {F + 6 \Lambda'}^2 \partial_{\eta  \eta  \lambda
   \lambda}^4 {F + 4 \Lambda'}^3 \partial_{\eta  \lambda
   \lambda  \lambda}^4 {F + \Lambda'}^4 \partial_{\lambda
   \lambda  \lambda  \lambda}^4 F + 6 \Lambda''
  \partial_{\eta  \eta  \lambda}^3 F \nonumber\\
  &  & + 12 \Lambda' \Lambda'' \partial_{\eta  \lambda
  \lambda}^3 {F + 6 \Lambda'}^2 \Lambda'' \partial_{\lambda  \lambda
   \lambda}^3 F + 4 \Lambda''' \partial_{\eta  \lambda}^2 F +
  \left( {3 \Lambda''}^2 + 4 \Lambda' \Lambda''' \right) \partial_{\lambda
   \lambda}^2 F \\
  &  & + \Lambda'''' \partial_{\lambda} F
\end{eqnarray}
o{\`u} $\Lambda$ et ses d{\'e}riv{\'e}es sont {\'e}valu{\'e}es en $\eta$,
tandis que $F$ et ses d{\'e}riv{\'e}es partielles sont {\'e}valu{\'e}es en
$[\eta, \lambda_0 + \Lambda (\eta)]$. En $\eta = 0$, les relations
pr{\'e}c{\'e}dentes s'{\'e}crivent
\begin{equation}
  \label{eq20220107060454} f' (0) = \partial_{\eta} F + \lambda_1
  \partial_{\lambda} F,
\end{equation}
\begin{equation}
  \label{eq20220107124311} f'' (0) = \partial_{\eta  \eta}^2 F + 2
  \lambda_1 \partial_{\eta  \lambda}^2 F + 2 \lambda_2
  \partial_{\lambda} F + \lambda_1^2 \partial_{\lambda  \lambda}^2 F,
\end{equation}
\begin{eqnarray}
  f''' (0) & = & \partial_{\eta  \eta  \eta}^3 F + 3 \lambda_1
  \partial_{\eta  \eta  \lambda}^3 F + 3 \lambda_1^2
  \partial_{\eta  \lambda  \lambda}^3 F + \lambda_1^3
  \partial_{\lambda  \lambda  \lambda}^3 F + 6 \lambda_2
  \partial_{\eta  \lambda}^2 F + 6 \lambda_1 \lambda_2
  \partial_{\lambda  \lambda}^2 F \nonumber\\
  &  & \nosymbol + 6 \lambda_3 \partial_{\lambda} F,
  \label{eq20220107060500}
\end{eqnarray}
\begin{eqnarray}
  f'''' (0) & = & \partial_{\eta  \eta  \eta  \eta}^4
  F + 4 \lambda_1 \partial_{\eta  \eta  \eta
  \lambda}^4 F + 6 \lambda_1^2 \partial_{\eta  \eta  \lambda
   \lambda}^4 F + 4 \lambda_1^3 \partial_{\eta  \lambda
   \lambda  \lambda}^4 F + \lambda_1^4 \partial_{\lambda
   \lambda  \lambda  \lambda}^4 F + 12 \lambda_2
  \partial_{\eta  \eta  \lambda}^3 F \nonumber\\
  &  & + 24 \lambda_1 \lambda_2 \partial_{\eta  \lambda
  \lambda}^3 F + 12 \lambda_1^2 \lambda_2 \partial_{\lambda  \lambda
   \lambda}^3 F + 24 \lambda_3 \partial_{\eta  \lambda}^2 F +
  (12 \lambda_2^2 + 24 \lambda_1 \lambda_3) \partial_{\lambda
  \lambda}^2 F \nonumber\\
  &  & + 24 \lambda_4 \partial_{\lambda} F,
\end{eqnarray}
o{\`u} $F$ et ses d{\'e}riv{\'e}es sont {\'e}valu{\'e}es en $(0, \lambda_0)$.

\subsection{D{\'e}veloppement limit{\'e} du
r{\'e}sidu}\label{sec20211112182000}

On cherche un d{\'e}veloppement limit{\'e} du r{\'e}sidu (c'est-{\`a}-dire de
la premi{\`e}re variation de l'{\'e}nergie). La fonction test $\hat{u} \in U$
{\'e}tant fix{\'e}e, la m{\'e}thode pr{\'e}c{\'e}dente est donc appliqu{\'e}e
avec
\begin{equation}
  \label{eq20220107054629} f (\eta) =\mathcal{E}_{, u} [u (\eta), \lambda
  (\eta) ; \hat{u}] \quad \text{et} \quad F (\eta, \lambda) =\mathcal{E}_{, u}
  [u^{\ast} (\lambda) + U (\eta), \lambda ; \hat{u}] .
\end{equation}
On remarque tout d'abord que $F (0, \lambda) =\mathcal{E}_{, u} [u^{\ast}
(\lambda), \lambda ; \hat{u}] = 0$, puisque $u^{\ast} (\lambda)$ est un point
d'{\'e}quilibre. En d{\'e}rivant par rapport {\`a} $\lambda$, on obtient
\begin{equation}
  \label{eq20211112164240} \frac{\partial^k F}{\partial \lambda^k} (0,
  \lambda) = 0.
\end{equation}
En d{\'e}rivant par rapport {\`a} $\eta$ l'expression~\eqref{eq20220107054629}
de $F$, on obtient successivement
\begin{equation}
  \partial_{\eta} F (\eta, \lambda) =\mathcal{E}_{, u  u} [u^{\ast}
  (\lambda) + U (\eta), \lambda ; U' (\eta), \hat{u}],
\end{equation}
\begin{eqnarray}
  \partial_{\eta  \eta}^2 F (\eta, \lambda) & = & \mathcal{E}_{, u
   u  u} [u^{\ast} (\lambda) + U (\eta), \lambda ; U' (\eta),
  U' (\eta), \hat{u}] \nonumber\\
  &  & \nosymbol +\mathcal{E}_{, u  u} [u^{\ast} (\lambda) + U
  (\eta), \lambda ; U'' (\eta), \hat{u}],
\end{eqnarray}
\begin{eqnarray}
  \partial_{\eta  \eta  \eta}^3 F (\eta, \lambda) & = &
  \mathcal{E}_{, u  u  u  u} [u^{\ast} (\lambda) + U
  (\eta), \lambda ; U' (\eta), U' (\eta), U' (\eta), \hat{u}] \nonumber\\
  &  & \nosymbol + 3\mathcal{E}_{, u  u  u} [u^{\ast}
  (\lambda) + U (\eta), \lambda ; U' (\eta), U'' (\eta), \hat{u}] \nonumber\\
  &  & \nosymbol +\mathcal{E}_{, u  u} [u^{\ast} (\lambda) + U
  (\eta), \lambda ; U''' (\eta), \hat{u}],
\end{eqnarray}
soit, en $\eta = 0$
\begin{equation}
  \partial_{\eta} F (0, \lambda) =\mathcal{E}_2 (\lambda ; u_1, \hat{u}),
\end{equation}
\begin{equation}
  \partial_{\eta  \eta}^2 F (0, \lambda) =\mathcal{E}_3 (\lambda ;
  u_1, u_1, \hat{u}) + 2\mathcal{E}_2 (\lambda ; u_2, \hat{u}),
\end{equation}
\begin{equation}
  \partial_{\eta  \eta  \eta}^3 F (0, \lambda) =\mathcal{E}_4
  (\lambda ; u_1, u_1, u_1, \hat{u}) + 6\mathcal{E}_3 (\lambda ; u_1, u_2,
  \hat{u}) + 6\mathcal{E}_2 (\lambda ; u_3, \hat{u}) .
\end{equation}
Les d{\'e}riv{\'e}es crois{\'e}es de $F$ en $(0, \lambda)$ s'obtiennent par
simple d{\'e}rivation des relations pr{\'e}c{\'e}dentes par rapport {\`a}
$\lambda$
\begin{equation}
  \partial_{\eta  \lambda}^2 F (0, \lambda) = \dot{\mathcal{E}_2}
  (\lambda ; u_1, \hat{u}),
\end{equation}
\begin{equation}
  \partial_{\eta  \eta  \lambda}^3 F (0, \lambda) =
  \dot{\mathcal{E}_3} (\lambda ; u_1, u_1, \hat{u}) + 2 \dot{\mathcal{E}_2}
  (\lambda ; u_2, \hat{u}),
\end{equation}
\begin{equation}
  \partial_{\eta  \lambda  \lambda}^3 F (0, \lambda) =
  \ddot{\mathcal{E}_2} (\lambda ; u_1, \hat{u}) .
\end{equation}
En ins{\'e}rant les r{\'e}sultats pr{\'e}c{\'e}dentes dans les relations
g{\'e}n{\'e}rales~\eqref{eq20220107060454}--\eqref{eq20220107060500}, on
trouve alors les expressions suivantes des d{\'e}riv{\'e}es successives de $f$
en $\eta = 0$
\begin{equation}
  f' (0) =\mathcal{E}_2 (\lambda_0 ; u_1, \hat{u}),
\end{equation}
\begin{equation}
  f'' (0) =\mathcal{E}_3 (\lambda_0 ; u_1, u_1, \hat{u}) + 2 \lambda_1
  \dot{\mathcal{E}_2} (\lambda_0 ; u_1, \hat{u}) + 2\mathcal{E}_2 (\lambda_0 ;
  u_2, \hat{u}),
\end{equation}
\begin{eqnarray}
  f''' (0) & = & \mathcal{E}_4 (\lambda_0 ; u_1, u_1, u_1, \hat{u}) +
  6\mathcal{E}_3 (\lambda_0 ; u_1, u_2, \hat{u}) + 6\mathcal{E}_2 (\lambda_0 ;
  u_3, \hat{u}) \nonumber\\
  &  & \nosymbol + 3 \lambda_1  [\dot{\mathcal{E}_3} (\lambda_0 ; u_1, u_1,
  \hat{u}) + 2 \dot{\mathcal{E}_2} (\lambda_0 ; u_2, \hat{u})] \nonumber\\
  &  & \nosymbol + 3 \lambda_1   \ddot{\mathcal{E}_2} (\lambda_0 ; u_1,
  \hat{u}) + 6 \lambda_2  \dot{\mathcal{E}_2} (\lambda_0 ; u_1, \hat{u}) .
\end{eqnarray}
On en d{\'e}duit finalement le d{\'e}veloppement limit{\'e} {\`a} l'ordre 3 en
$\eta$ du r{\'e}sidu
\begin{eqnarray}
  \mathcal{E}_{, u} [u (\eta), \lambda (\eta)] & = & \eta \mathcal{E}_2
  (\lambda_0 ; u_1, \hat{u}) \nonumber\\
  &  & \nosymbol + \tfrac{1}{2} \eta^2  [\mathcal{E}_3 (\lambda_0 ; u_1, u_1,
  \hat{u}) + 2 \lambda_1  \dot{\mathcal{E}_2} (\lambda_0 ; u_1, \hat{u}) +
  2\mathcal{E}_2 (\lambda_0 ; u_2, \hat{u})] \nonumber\\
  &  & \nosymbol + \tfrac{1}{6} \eta^3  \{ \mathcal{E}_4 (\lambda_0 ; u_1,
  u_1, u_1, \hat{u}) + 6\mathcal{E}_3 (\lambda_0 ; u_1, u_2, \hat{u})
   + 6\mathcal{E}_2 (\lambda_0 ; u_3, \hat{u}) \nonumber\\
  &  & \nosymbol + 3 \lambda_1  [\dot{\mathcal{E}_3} (\lambda_0 ; u_1, u_1,
  \hat{u}) + 2 \dot{\mathcal{E}_2} (\lambda_0 ; u_2, \hat{u})] + 3 \lambda_1^2
  \ddot{\mathcal{E}_2} (\lambda_0 ; u_1, \hat{u}) \nonumber\\
  &  &  \nosymbol + 6 \lambda_2  \dot{\mathcal{E}_2} (\lambda_0 ;
  u_1, \hat{u}) \} + o (\eta^3) .  \label{eq20220107080901}
\end{eqnarray}
\subsection{D{\'e}veloppement limit{\'e} de
l'{\'e}nergie}\label{sec20220121172919}

On s'int{\'e}resse ici {\`a} l'{\'e}cart d'{\'e}nergie, pour un chargement
$\lambda$ donn{\'e}, entre la branche bifurqu{\'e}e et la branche
fondamentale, soit
\begin{equation}
  F (\eta, \lambda) =\mathcal{E} [u^{\ast} (\lambda) + U (\eta), \lambda]
  -\mathcal{E} [u^{\ast} (\lambda), \lambda]
\end{equation}
et
\begin{equation}
  f (\eta) = F [\eta, \lambda_0 + \Lambda (\eta)] .
\end{equation}
On observe tout d'abord que $F (0, \lambda) = 0$ pour tout $\lambda$, donc
\begin{equation}
  \frac{\partial^k F}{\partial \lambda^k} (0, \lambda) = 0 \quad (k \geq 0),
\end{equation}
tandis que les d{\'e}riv{\'e}es de $F$ par rapport {\`a} $\eta$ s'{\'e}crivent
\begin{equation}
  \partial_{\eta} F (\eta, \lambda) =\mathcal{E}_{, u} [u^{\ast} (\lambda) + U
  (\eta), \lambda ; U' (\eta)],
\end{equation}
\begin{equation}
  \partial_{\eta  \eta}^2 F (\eta, \lambda) =\mathcal{E}_{, u
  u} [u^{\ast} (\lambda) + U (\eta), \lambda ; U' (\eta), U' (\eta)]
  +\mathcal{E}_{, u} [u^{\ast} (\lambda) + U (\eta), \lambda ; U'' (\eta)],
\end{equation}
\begin{eqnarray}
  \partial_{\eta  \eta  \eta}^3 F (\eta, \lambda) & = &
  \mathcal{E}_{, u  u  u} [u^{\ast} (\lambda) + U (\eta),
  \lambda ; U' (\eta), U' (\eta), U' (\eta)] \nonumber\\
  &  & \nosymbol + 3\mathcal{E}_{, u  u} [u^{\ast} (\lambda) + U
  (\eta), \lambda ; U' (\eta), U'' (\eta)] \nonumber\\
  &  & \nosymbol +\mathcal{E}_{, u} [u^{\ast} (\lambda) + U (\eta), \lambda ;
  U''' (\eta)],
\end{eqnarray}
\begin{eqnarray}
  \partial_{\eta  \eta  \eta  \eta}^4 F (\eta,
  \lambda) & = & \mathcal{E}_{, u  u  u  u} [u^{\ast}
  (\lambda) + U (\eta), \lambda ; U' (\eta), U' (\eta), U' (\eta), U' (\eta)]
  \nonumber\\
  &  & \nosymbol + 6\mathcal{E}_{, u  u  u} [u^{\ast}
  (\lambda) + U (\eta), \lambda ; U' (\eta), U' (\eta), U'' (\eta)]
  \nonumber\\
  &  & \nosymbol + 3\mathcal{E}_{, u  u} [u^{\ast} (\lambda) + U
  (\eta), \lambda ; U'' (\eta), U'' (\eta)] \nonumber\\
  &  & \nosymbol + 3\mathcal{E}_{, u  u} [u^{\ast} (\lambda) + U
  (\eta), \lambda ; U' (\eta), U''' (\eta)] \nonumber\\
  &  & \nosymbol +\mathcal{E}_{, u} [u^{\ast} (\lambda) + U (\eta), \lambda ;
  U'''' (\eta)],
\end{eqnarray}
soit, en $\eta = 0$, en observant que $\mathcal{E}_{, u} [u^{\ast} (\lambda),
\lambda] = 0$
\begin{equation}
  \partial_{\eta} F (0, \lambda) = 0,
\end{equation}
\begin{equation}
  \partial_{\eta  \eta}^2 F (0, \lambda) =\mathcal{E}_2 (\lambda ;
  u_1, u_1),
\end{equation}
\begin{equation}
  \partial_{\eta  \eta  \eta}^3 F (0, \lambda) =\mathcal{E}_3
  (\lambda ; u_1, u_1, u_1) + 6\mathcal{E}_2 (\lambda ; u_1, u_2),
\end{equation}
\begin{eqnarray}
  \partial_{\eta  \eta  \eta  \eta}^4 F (\eta,
  \lambda) & = & \mathcal{E}_4 (\lambda ; u_1, u_1, u_1, u_1) +
  12\mathcal{E}_3 (\lambda ; u_1, u_1, u_2) + 12\mathcal{E}_2 (\lambda ; u_2,
  u_2) \nonumber\\
  &  & + 18\mathcal{E}_2 (\lambda ; u_1, u_3) .
\end{eqnarray}
On en d{\'e}duit que
\begin{equation}
  \partial_{\eta  \lambda}^2 F (0, \lambda) = 0, \qquad \partial_{\eta
   \eta  \lambda}^3 F (0, \lambda) = \dot{\mathcal{E}}_2
  (\lambda ; u_1, u_1), \qquad \partial_{\eta  \lambda
  \lambda}^3 F (0, \lambda) = 0,
\end{equation}
\begin{equation}
  \partial_{\eta  \eta  \eta  \lambda}^4 F (0,
  \lambda) = \dot{\mathcal{E}}_3 (\lambda ; u_1, u_1, u_1) + 6
  \dot{\mathcal{E}}_2 (\lambda ; u_1, u_2), \qquad \partial_{\eta
  \eta  \lambda  \lambda}^4 F (0, \lambda) =
  \ddot{\mathcal{E}}_2 (\lambda ; u_1, u_1),
\end{equation}
\begin{equation}
  \partial_{\eta  \lambda  \lambda  \lambda}^4 F (0,
  \lambda) = 0
\end{equation}
et finalement
\begin{equation}
  f' (0) = 0, \qquad f'' (0) =\mathcal{E}_2 (\lambda_0 ; u_1, u_1),
\end{equation}
\begin{equation}
  f''' (0) =\mathcal{E}_3 (\lambda_0 ; u_1, u_1, u_1) + 6\mathcal{E}_2
  (\lambda_0 ; u_1, u_2) + 3 \lambda_1  \dot{\mathcal{E}}_2 (\lambda_0 ; u_1,
  u_1),
\end{equation}
\begin{eqnarray}
  f'''' (0) & = & \mathcal{E}_4 (\lambda_0 ; u_1, u_1, u_1, u_1) +
  12\mathcal{E}_3 (\lambda_0 ; u_1, u_1, u_2) \nonumber\\
  &  & + 12\mathcal{E}_2 (\lambda_0 ; u_2, u_2) + 18\mathcal{E}_2 (\lambda_0
  ; u_1, u_3) \nonumber\\
  &  & + 4 \lambda_1  [\dot{\mathcal{E}}_3 (\lambda_0 ; u_1, u_1, u_1) + 6
  \dot{\mathcal{E}}_2 (\lambda_0 ; u_1, u_2)] \nonumber\\
  &  & + 6 \lambda_1^2  \ddot{\mathcal{E}}_2 (\lambda_0 ; u_1, u_1) + 12
  \lambda_2  \dot{\mathcal{E}}_2 (\lambda_0 ; u_1, u_1)
\end{eqnarray}
On en d{\'e}duit finalement le d{\'e}veloppement limit{\'e} de
l'{\'e}nergie~\eqref{eq20220121172753}.

\subsection{D{\'e}veloppement limit{\'e} de la
hessienne}\label{sec20211115081016}

On cherche maintenant un d{\'e}veloppement limit{\'e} de la hessienne de
l'{\'e}nergie. Les fonctions test $\hat{u}, \hat{v} \in U$ {\'e}tant
fix{\'e}es, on applique la m{\'e}thode du
{\textsection}\ref{sec20220107121442} {\`a} la fonction $f (\eta) = F [\eta,
\lambda_0 + \Lambda (\eta)]$, avec
\begin{equation}
  F (\eta, \lambda) =\mathcal{E}_{, u  u} [u^{\ast} (\lambda) + U
  (\eta), \lambda ; \hat{u}, \hat{v}] .
\end{equation}
On observe tout d'abord que $F (0, \lambda) =\mathcal{E}_2 (\lambda ; \hat{u},
\hat{v})$, soit, en d{\'e}rivant par rapport {\`a} $\lambda$
\begin{equation}
  \partial_{\lambda} F (0, \lambda) = \dot{\mathcal{E}_2} (\lambda ; \hat{u},
  \hat{v}) \quad \text{et} \quad \partial_{\lambda  \lambda}^2 F (0,
  \lambda) = \ddot{\mathcal{E}_2} (\lambda ; \hat{u}, \hat{v}) .
\end{equation}
On trouve de m{\^e}me successivement
\begin{equation}
  \partial_{\eta} F (\eta, \lambda) =\mathcal{E}_{, u  u  u}
  [u^{\ast} (\lambda) + U (\eta), \lambda ; U' (\eta), \hat{u}, \hat{v}],
\end{equation}
\begin{eqnarray}
  \partial_{\eta  \eta}^2 F (\eta, \lambda) & = & \mathcal{E}_{, u
   u  u  u} [u^{\ast} (\lambda) + U (\eta), \lambda ;
  U' (\eta), U' (\eta), \hat{u}, \hat{v}] \nonumber\\
  &  & +\mathcal{E}_{, u  u  u} [u^{\ast} (\lambda) + U
  (\eta), \lambda ; U'' (\eta), \hat{u}, \hat{v}],
\end{eqnarray}
soit, en $\eta = 0$
\begin{equation}
  \partial_{\eta} F (0, \lambda) =\mathcal{E}_3 (\lambda ; u_1, \hat{u},
  \hat{v}) \quad \tmop{et} \quad \partial_{\eta  \eta}^2 F (0,
  \lambda) =\mathcal{E}_4 (\lambda ; u_1, u_1, \hat{u}, \hat{v}) +
  2\mathcal{E}_3 (\lambda ; u_2, \hat{u}, \hat{v}),
\end{equation}
et en d{\'e}rivant cette fois par rapport {\`a} $\lambda$
\begin{equation}
  \partial_{\eta  \lambda}^2 F (0, \lambda) = \dot{\mathcal{E}_3}
  (\lambda ; u_1, \hat{u}, \hat{v}) .
\end{equation}
En ins{\'e}rant les r{\'e}sultats pr{\'e}c{\'e}dents dans les
expressions~\eqref{eq20220107060454} et \eqref{eq20220107124311}, on trouve
\begin{equation}
  f' (0) =\mathcal{E}_3 (\lambda_0 ; u_1, \hat{u}, \hat{v}) + \lambda_1
  \dot{\mathcal{E}_2} (\lambda_0 ; \hat{u}, \hat{v}),
\end{equation}
\begin{eqnarray}
  f'' (0) & = & \mathcal{E}_4 (\lambda_0 ; u_1, u_1, \hat{u}, \hat{v}) + 2
  \lambda_1  \dot{\mathcal{E}_3} (\lambda_0 ; u_1, \hat{u}, \hat{v}) +
  \lambda_1^2  \ddot{\mathcal{E}_2} (\lambda_0 ; \hat{u}, \hat{v}) \nonumber\\
  &  & \nosymbol + 2\mathcal{E}_3 (\lambda_0 ; u_2, \hat{u}, \hat{v}) + 2
  \lambda_2  \dot{\mathcal{E}_2} (\lambda_0 ; \hat{u}, \hat{v}) .
\end{eqnarray}
qui conduisent finalement au d{\'e}veloppement limit{\'e} suivant, {\`a}
l'ordre 2 en $\eta$
\begin{eqnarray}
  \mathcal{E}_{, u  u} [u (\eta), \lambda (\eta) ; \hat{u}, \hat{v}] &
  = & \mathcal{E}_2 (\lambda_0 ; \hat{u}, \hat{v}) + \eta [\mathcal{E}_3
  (\lambda_0 ; u_1, \hat{u}, \hat{v})   + \lambda_1
  \dot{\mathcal{E}_2} (\lambda_0 ; \hat{u}, \hat{v})] \nonumber\\
  &  & \nosymbol + \tfrac{1}{2} \eta^2  [\mathcal{E}_4 (\lambda_0 ; u_1, u_1,
  \hat{u}, \hat{v})  + 2 \lambda_1  \dot{\mathcal{E}_3} (\lambda_0 ;
  u_1, \hat{u}, \hat{v}) + \lambda_1^2  \ddot{\mathcal{E}_2} (\lambda_0 ;
  \hat{u}, \hat{v}) \nonumber\\
  &  & \nosymbol  + 2\mathcal{E}_3 (\lambda_0 ; u_2, \hat{u},
  \hat{v}) + 2 \lambda_2  \dot{\mathcal{E}_2} (\lambda_0 ; \hat{u}, \hat{v})]
  + o (\eta^2) .
\end{eqnarray}

\subsection{D{\'e}veloppement limit{\'e} des valeurs propres et vecteurs
propres de la Hessienne}

On cherche les vecteurs propres $v \in V$ et valeurs propres $\alpha \in
\mathbb{R}$ de la Hessienne
\begin{equation}
  \label{eq20211115082122} \mathcal{E}_{, u  u} [u (\eta), \lambda
  (\eta)] (v, \hat{u}) = \alpha \langle v, \hat{u} \rangle \quad \text{pour
  tout} \quad \hat{u} \in V.
\end{equation}
On cherche les d{\'e}veloppements limit{\'e}s {\`a} l'ordre 1 en $\eta$ de $v$
et $\alpha$
\begin{equation}
  \label{eq20211115082037} v = v_0 + \eta v_1 + o (\eta) \quad \text{et} \quad
  \alpha = \alpha_0 + \eta \alpha_1 + o (\eta)
\end{equation}
Les d{\'e}veloppements limit{\'e}s~\eqref{eq20211115082025} et
\eqref{eq20211115082037} sont ins{\'e}r{\'e}s dans le
probl{\`e}me~\eqref{eq20211115082122}
\begin{eqnarray}
  \mathcal{E}_{, u  u} [u (\eta), \lambda (\eta)] (v, \hat{w}) & = &
  \mathcal{E}_2  (\lambda_0 ; v_0, \hat{w}) + \eta [\mathcal{E}_3  (\lambda_0
  ; u_1, v_0, \hat{w})  \nonumber\\
  &  & \nosymbol + \lambda_1  \dot{\mathcal{E}_2} (\lambda_0 ; v_0, \hat{w})
   +\mathcal{E}_2  (\lambda_0 ; v_1, \hat{w})] + o (\eta)
\end{eqnarray}
\begin{equation}
  \alpha \langle v, \hat{w} \rangle = \alpha_0  \langle v_0, \hat{w} \rangle +
  \eta (\alpha_1 \langle v_0, \hat{w} \rangle + \alpha_0 \langle v_1, \hat{w}
  \rangle) + o (\eta) .
\end{equation}
On obtient le probl{\`e}me variationnel d'ordre 0
\begin{equation}
  \mathcal{E}_2 (\lambda_0 ; v_0, \hat{w}) = \alpha_0  \langle v_0, \hat{w}
  \rangle \quad \text{pour tout} \quad \hat{w} \in V,
\end{equation}
qui montre que $v_0$ est le vecteur propre de $\mathcal{E}_2  (\lambda_0)$
associ{\'e} {\`a} la valeur propre $\alpha_0$. Si $\alpha_0 \neq 0$,
$\mathcal{E}_2  (\lambda_0)$ {\'e}tant positive par hypoth{\`e}se, on a
n{\'e}cessairement $\alpha_0 > 0$, et la valeur propre de la hessienne est
positive. On consid{\`e}re maintenant le cas o{\`u} $\alpha_0$,
c'est-{\`a}-dire que $v_0 \in \ker \mathcal{E}_2  (\lambda_0)$. En prenant
$\hat{w} \in \ker \mathcal{E}_2 (\lambda_0)$, on obtient alors le probl{\`e}me
variationnel d'ordre 1
\begin{equation}
  \mathcal{E}_3 (\lambda_0 ; u_1, v_0, \hat{w}) + \lambda_1
  \dot{\mathcal{E}_2} (\lambda_0 ; v_0, \hat{w}) = \alpha_1  \langle v_0,
  \hat{w} \rangle \quad \text{pour tout} \quad \hat{w} \in \ker \mathcal{E}_2
  (\lambda_0) .
\end{equation}
En posant $u_1 = \xi_i a_i$ et $v_0 = \chi_j a_j$, on obtient l'{\'e}quation
\begin{equation}
  [\mathcal{E}_{i  j  k}  (\lambda_0) \xi_k + \lambda_1
  \dot{\mathcal{E}}_{i  j} (\lambda_0)] \chi_j = \alpha_1 \chi_i,
\end{equation}
qui est un probl{\`e}me aux valeurs propres pour la matrice sym{\'e}trique
$[\mathcal{E}_{i  j  k}  (\lambda_0) \xi_k + \lambda_1
\dot{\mathcal{E}}_{i  j} (\lambda_0)]_{1 \leq i, j \leq m}$.

\end{document}
