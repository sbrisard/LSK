\newcommand{\sbtitle}{Notes on the Lyapunov--Schmidt--Koiter asymptotic method}
\newcommand{\sbauthor}{Sébastien Brisard}
\newcommand{\sbemail}{sebastien.brisard@univ-eiffel.fr}
\newcommand{\sbaddress}{Univ Gustave Eiffel, Ecole des Ponts, IFSTTAR, CNRS, Navier, F-77454 Marne-la-Vall\'ee, France}
\newcommand{\sbsubject}{Note bibliographique}

\documentclass[12pt, final]{scrartcl}
%\setkomafont{disposition}{\rmfamily}

\usepackage{polyglossia}
\setdefaultlanguage{english}

\usepackage{amsfonts}
\usepackage{amsmath}
\usepackage{amssymb}

\usepackage{amsthm}
\theoremstyle{definition}
\renewcommand{\qedsymbol}{}
\newtheorem{remark}{Remark}
\newtheorem{theorem}{Theorem}

\usepackage[backend=biber,bibencoding=utf8,doi=false,giveninits=true,isbn=false,maxnames=10,minnames=5,sortcites=true,style=authoryear,texencoding=utf8,url=false]{biblatex}
\addbibresource{LSK-notes.bib}

\usepackage[breaklinks=true, colorlinks=true, pdftitle={\sbtitle}, pdfauthor={\sbauthor}, pdfsubject={\sbsubject}, urlcolor=blue]{hyperref}

\usepackage[color={1 1 0}]{pdfcomment}

\usepackage{unicode-math}
% \setmainfont{XITS}
% \setmathfont{XITS Math}
\setmainfont{Asana Math}
\setmathfont{Asana Math}

\usepackage{xcolor}

\newcommand{\E}{\mathcal E}
\newcommand{\EE}{\mathcal E ^ \dagger}
\newcommand{\D}{\mathrm{d}}
\newcommand{\order}[2][1]{#2^{(#1)}}
\newcommand{\reals}{\mathbb{R}}
\newcommand{\uu}{u^\dagger}

\begin{document}
\title{\sbtitle}
\author{\sbauthor\thanks{\sbaddress~--- \sbemail}}
\maketitle

\color{gray}

\begin{abstract}
  These are my notes on the LSK method for the analysis of the stability and bifurcation(s) of a conservative
  system. These notes are based on several references: Koiter's initial PhD thesis~\parencite{koit1945} as well as some
  graphical illustrations from his lecture notes~\parencite{koit2009}. I enjoyed the concise presentation of
  \textcite{nguy2000} as well as the lecture notes of \textcite{tria2017}. Finally, the chapter by \textcite{poti1987}
  helped me clear some issues.

  These notes by Sébastien Brisard are licensed under a Creative Commons Attribution 4.0 International License. To view
  a copy of this license, visit \url{http://creativecommons.org/licenses/by/4.0/}.

  I hope the reader will find these notes useful, even though there are still a few points which I do not fully
  understand (they are clearly indicated in the text).
\end{abstract}

\section{Problem setting}

The space of admissible states of the system under consideration is denoted \(U\). It has the structure of a real vector
space. The energy of the system is \(\E(u, \lambda)\), where \(\lambda\) denotes a loading parameter. It is assumed that the
fundamental branch of the equilibrium diagram, \(u^\ast(\lambda)\) is known. Then the energy is stationary with respect to the
state \(u\) along the whole branch. In other words, for all \(\hat{u} \in U\)
\begin{equation}
  \label{eq:20221227184138}
  \E_{,u}[u^{\ast}(\lambda), \lambda; \hat{u}]=0,
\end{equation}
where \(\E_{,u}(u, \lambda; \hat{u})\) denotes the (real) value of the differential of the energy \(\E\) with respect to the
state \(u\), evaluated at \((u, \lambda)\), for the test function \(\hat{u}\). Similarly, evaluation of the second-, third-,
etc., order differential of the energy will be denoted \(\E_{,uu}(u, \lambda; \hat{u}, \hat{v})\),
\(\E_{,uuu}(u, \lambda; \hat{u}, \hat{v}, \hat{w})\), etc. It is assumed that a finite value \(\lambda_0 > 0\) of
\(\lambda\) can be found (critical load), such that
\begin{enumerate}
\item \label{item:20221017185946} \(\E_{,uu}[u(\lambda), \lambda] > 0\) for all \(0 < \lambda < \lambda_0\),
\item \label{item:20221017190357} \(\E_{,uu}(u_0, \lambda_0) \geq 0\) but \(\E_{,uu}(u_0, \lambda_0) \ngtr 0\),
\item \label{item:20221017190003} \(\E_{,uu}[u(\lambda), \lambda] < 0\) for \(\lambda > \lambda_0\), close enough to \(\lambda_0\).
\end{enumerate}

We introduce the following notations
\begin{equation}
  u_0 = u^\ast(\lambda_0), \quad \dot{u}_0 = \frac{\D u^\ast}{\D \lambda} \biggr \rvert_{\lambda=\lambda_0}, \quad \ddot{u}_0 = \frac{\D^2 u^\ast}{\D \lambda^2} \biggr \rvert_{\lambda=\lambda_0}, \quad \dddot{u}_0 = \ldots, \quad \ddddot{u}_0 = \ldots
\end{equation}
and
\begin{equation}
  \E_2 = \E_{,uu}(u_0, \lambda_0), \quad \E_3 = \E_{,uuu}(u_0, \lambda_0), \quad \E_4 = \E_{,uuuu}(u_0, \lambda_0).
\end{equation}

Note that \(\E_2\), \(\E_3\) and \(\E_4\) thus defined are bi-, tri- and quadrilinear forms, respectively. The following
derivatives are also introduced
\begin{equation}
  \label{eq:20221227184821}
  \dot{\E}_2(\hat{u}, \hat{v}) = \frac{\D}{\D\lambda} \E_{,uu}[ u^\ast(\lambda), \lambda; \hat{u}, \hat{v}] \biggr \rvert_{\lambda = \lambda_0}
  = \E_{,uuu}(u_0, \lambda_0; \dot{u}_0, \hat{u}, \hat{v}) + \E_{,uu\lambda}(u_0, \lambda_0; \hat{u}, \hat{v})
\end{equation}
\begin{equation}
  \begin{aligned}[b]
    \ddot{\E}_2(\hat{u}, \hat{v})
    ={} & \frac{\D^2}{\D\lambda^2} \E_{,uu}[ u^\ast(\lambda), \lambda; \hat{u}, \hat{v}] \biggr \rvert_{\lambda = \lambda_0} = \E_{,uuuu}(u_0, \lambda_0; \dot{u}_0, \dot{u}_0, \hat{u}, \hat{v}) + 2\E_{,uuu\lambda}(u_0, \lambda_0; \dot{u}_0, \hat{u}, \hat{v})\\
    & + \E_{,uu\lambda\lambda}( u_0, \lambda_0; \hat{u}, \hat{v}) + \E_{,uuu}(u_0, \lambda_0, \ddot{u}_0),
  \end{aligned}
\end{equation}
and, similarly, \(\dot{\E}_3\), \(\ddot{\E}_3\), etc.

The load \( \lambda_0\) will be referred to as the \emph{critical load}; similarly, the state \(u_0\) of the system at the
critical load will be referred to as the \emph{critical state}; finally, the pair \((u_0, \lambda_0)\) is the \emph{critical
  point} of the system. Assumption~\ref{item:20221017185946} implies that equilibria along the fundamental branch are
\emph{stable below the critical load}. Conversely, it results from assumption~\ref{item:20221017190003} that equilibrium
points on the fundamental branch are \emph{unstable above the critical load}. Stability at the critical load is yet
undetermined.

The goal of these notes is to analyze \emph{all} equilibrium paths that pass through the critical point \((u_0, \lambda_0)\).

\section{Stationarity of the energy along the fundamental branch}

Eq.~\eqref{eq:20221227184138} holds for all \(\lambda\). Upon successive derivations with respect to \(\lambda\), it is found that,
for all \(\hat{u} \in U\) and \(\lambda \in \reals\)
\begin{equation}
  \label{eq:20220901143843}
  \E_{,uu}[u^\ast(\lambda), \lambda; \dot{u}^\ast(\lambda), \hat{u}] + \E_{,u\lambda}[u^\ast(\lambda), \lambda; \hat{u}] = 0
\end{equation}
and
\begin{equation}
  \label{eq:20220901143902}
  \begin{aligned}[b]
    \E_{,uuu}[u^\ast(\lambda), \lambda; \dot{u}^\ast(\lambda), \dot{u}^\ast(\lambda), \hat{u}] + 2\E_{,uu\lambda}[u^\ast(\lambda), \lambda; \dot{u}^\ast(\lambda), \hat{u}] &\\
    + \E_{,u\lambda\lambda}[u^\ast(\lambda), \lambda; \hat{u}] + \E_{,uu}[u^\ast(\lambda), \lambda; \ddot{u}^\ast(\lambda), \hat{u}] &= 0
  \end{aligned}
\end{equation}
and, at the critical point
\begin{gather}
  \label{eq:20220901144331}
  \E_2(\dot{u}_0, \bullet) + \E_{,u\lambda}(u_0, \lambda_0; \bullet) = 0,\\
  \label{eq:20220901144335}
  \E_3(\dot{u}_0, \dot{u}_0, \bullet) + 2\E_{,uu\lambda}(u_0, \lambda_0; \dot{u}_0, \bullet) + \E_{,u\lambda\lambda}(u_0, \lambda_0; \bullet) + \E_2(\ddot{u}_0, \bullet) = 0.
\end{gather}

Combining with Eq.~\eqref{eq:20221227184821}, the mixed derivatives \(\E_{,u\lambda}(u_0, \lambda_0; \bullet)\),
\(\E_{,uu\lambda}(u_0, \lambda_0; \bullet, \bullet)\) and \(\E_{,u\lambda\lambda}(u_0, \lambda_0; \bullet)\) can be expressed as functions of
\(u_0\), \(\dot{u}_0\), \(\ddot{u}_0\), \(\E_2\) and \(\E_3\). More generally, it is shown in
\pdfmarkupcomment{Appendix}{} that the following mixed derivatives can be thus expressed
\begin{align}
  \E_{,u\lambda}(u_0, \lambda_0; \bullet) ={} & -\E_2(\dot{u}_0, \bullet),\\
  \E_{,uu\lambda}(u_0, \lambda_0; \bullet, \bullet) ={} & -\E_3(\dot{u}_0, \bullet, \bullet) + \dot{\E}_2(\bullet, \bullet),\\
  \E_{,u\lambda\lambda}(u_0, \lambda_0; \bullet) ={} &  - \E_2(\ddot{u}_0, \bullet) - 2\dot{\E}_2(\dot{u}_0, \bullet) + \E_3(\dot{u}_0, \dot{u}_0, \bullet),\\
  \E_{,uuu\lambda}(u_0, \lambda_0; \bullet, \bullet, \bullet) ={} & -\E_4(\dot{u}_0, \bullet, \bullet, \bullet) + \dot{\E}_3(\bullet, \bullet, \bullet),\\
  \E_{,uu\lambda\lambda}(u_0, \lambda_0; \bullet, \bullet) ={} & - \E_3(\ddot{u}_0, \bullet, \bullet) + \ddot{\E}_2(\bullet, \bullet) - 2\dot{\E}_3(\dot{u}_0, \bullet, \bullet) + \E_4(\dot{u}_0, \dot{u}_0, \bullet, \bullet),\\
  \notag
  \E_{,u\lambda\lambda\lambda}(u_0, \lambda_0; \bullet) ={} & -\E_2(\dddot{u}_0, \bullet) + 3\E_3(\dot{u}_0, \ddot{u}_0, \bullet) - \E_4(\dot{u}_0, \dot{u}_0, \dot{u}_0, \bullet) - 3\ddot{\E}_2(\dot{u}_0, \bullet)\\
  &-3\dot{\E}_2(\ddot{u}_0, \bullet)+ 3\dot{\E}_3(\dot{u}_0, \dot{u}_0, \bullet).
\end{align}

From assumption~\ref{item:20221017190357}, $\E_2(\lambda_0)$ is positive indefinite. Let $V = \ker \E_2(\lambda_0)$ be
its kernel; $V$ is a vector subspace of $U$ (see \S \ref{sec:20221013223647}) and we define the supplementary subspace
$W$, orthogonal to $V$ for the (yet unspecified) scalar product $\langle \bullet, \bullet \rangle$
\begin{gather}
  \E_2(\lambda_0; \hat{u}, \hat{v}) = 0 \quad \text{for all} \quad \hat{u} \in U \quad \text{and} \quad \hat{v} \in V.\\
  U = V \overset{\perp}{\otimes} W \quad \text{and} \quad \langle \hat{v}, \hat{w} \rangle = 0 \quad \text{for all} \quad \hat{v} \in V \quad \text{and} \quad \hat{w} \in W.
\end{gather}

\begin{remark}
  \label{rem:20220902095055}
  The bilinear form $\E_2$ being elliptic over $W$, variational problems of the type: find $w \in W$ such that
  \begin{equation}
    \E_2(w, \hat{w})+\ell(\hat{w}) = 0 \quad \text{for all} \quad \hat{w} \in W
  \end{equation}
  are well-posed for any linear form $\ell$ over $W$. In particular, for $\ell=0$, the unique solution to the
  variational problem
  \begin{equation}
    \E_2(w, \hat{w}) = 0 \quad \text{for all} \quad \hat{w} \in W
  \end{equation}
  is $w = 0$.
\end{remark}

It is assumed that the dimension of $V$ is finite: $m = \dim V < +\infty$; $m$ is the \emph{multiplicity} of the
critical point. A (finite) basis $(v_1, \ldots, v_m)$ of $V$ can therefore be introduced, that is orthonormal in the
sense of $\langle \bullet, \bullet \rangle$
\begin{equation}
  \langle v_i, v_j \rangle = \delta_{ij} \quad \text{for all} \quad i, j = 1, \ldots, m.
\end{equation}

\color{black}

For $1 \leq i, j \leq m$, we introduce the solutions $w_i, w_{ij} \in W$ to the following variational problems
\begin{gather}
  \label{eq:20220524134525}
  \E_2(\lambda_0; w_i, \hat{w}) + \dot{\E}_2(\lambda_0; v_i, \hat{w}) = 0,\\
  \label{eq:20220519164523}
  \E_2(\lambda_0; w_{i j}, \hat{w})+\E_3(\lambda_0; v_i, v_j, \hat{w}) = 0,
\end{gather}
for all $\hat{w} \in W$. Since $w_{i}$ and $w_{ij}$ belong to $W$, we have
$\langle w_{i}, v \rangle = \langle w_{ij}, v \rangle = 0$ for all $v \in V$. Since $\E_2(\lambda_0; \bullet, \bullet)$
is symmetric, it can be verified that $w_{ij}=w_{ji}$. We also introduce the following tensors, defined in $V$
\begin{gather}
  E_{ijk} = \E_3(\lambda_0; v_i, v_j, v_k) + \E_2(\lambda_0; v_i, w_{jk}) + \E_2(\lambda_0; v_j, w_{ki}) + \E_2(\lambda_0; v_k, w_{ij}),\\
  \label{eq:20221116155507}
  E_{ijkl} = \E_4(\lambda_0 ; v_i, v_j, v_k, v_l) + \E_3(\lambda_0 ; v_i, v_j, w_{kl}) + \E_3(\lambda_0 ; v_i, v_k, w_{lj}) + \E_3(\lambda_0 ; v_i, v_l, w_{jk}),\\
  F_{ij} = \dot{\E}_2(\lambda_0; v_i, v_j) + \E_2(\lambda_0; v_i, w_j) + \E_2(\lambda_0; v_j, w_i),
\end{gather}
as well as the derivatives
\begin{gather}
  \label{eq:20220615063626}
  \mathring{E}_{ijk} = \dot{\E}_3(\lambda_0; v_i, v_j, v_k) + \dot{\E_2}(\lambda_0; v_i, w_{jk}) + \dot{\E}_2(\lambda_0; v_j, w_{ki}) + \dot{\E}_2(\lambda_0; v_k, w_{ij}),\\
  \label{eq:20220615063633}
  \mathring{F}_{ij} = \ddot{\E}_2(\lambda_0; v_i, v_j) + \dot{\E}_2(\lambda_0; v_i, w_j) + \dot{\E}_2(\lambda_0; v_j, w_i).
\end{gather}

Note that, since $\E_2(\lambda_0; v_i, \bullet) = 0$, the above expressions simplify as follows
\begin{gather}
  \label{eq:20220524135619}
  E_{ijk} = \E_3(\lambda_0; v_i, v_j, v_k),\\
  \label{eq:20220524135553}
  E_{ijkl} = \E_4(\lambda_0 ; v_i, v_j, v_k, v_l) + \E_3(\lambda_0 ; v_i, v_j, w_{kl}) + \E_3(\lambda_0 ; v_i, v_k, w_{jl}) + \E_3(\lambda_0 ; v_i, v_l, w_{jk}),\\
  \label{eq:20220524135643}
  F_{ij} = \dot{\E}_2(\lambda_0; v_i, v_j).
\end{gather}

The tensors $E_{ijk}$, $F_{ij}$, $\mathring{E}_{ijk}$ and $\mathring{F}_{ij}$ are fully symmetric. Furthermore, the
following expression of $E_{ijkl}$ result from Eq.~\eqref{eq:20220519164523}
\begin{equation}
  \label{eq:20220802081116}
  E_{ijkl} = \E_4(\lambda_0 ; v_i, v_j, v_k, v_l) - \E_2(\lambda_0 ; w_{ij}, w_{kl}) - \E_2(\lambda_0 ; w_{ik}, w_{jl}) - \E_2(\lambda_0 ; w_{il}, w_{jk}),
\end{equation}
which shows that $E_{ijkl}$ is also fully symmetric. We close this section, with two useful identities. First, starting
from the definition of $\mathring{F}_{ij}$, and using Eq.~\eqref{eq:20220524134525}, with $v_i = v_j$ and
$\hat{w} = w_i$
\begin{equation}
  \label{eq:20220617084433}
  \begin{aligned}[b]
    \mathring{F}_{ij} ={} & \ddot{\E}_2(\lambda_0; v_i, v_j) + \dot{\E}_2(\lambda_0; v_i, w_j) + \dot{\E}_2(\lambda_0; v_j, w_i)\\
    ={} & \ddot{\E}_2(\lambda_0; v_i, v_j) + \dot{\E}_2(\lambda_0; v_i, w_j) - \E_2(\lambda_0; w_j, w_i),
  \end{aligned}
\end{equation}
then, using Eq.~\eqref{eq:20220524134525}, with $\hat{w} = w_j$
\begin{equation}
  \mathring{F}_{ij} = \ddot{\E}_2(\lambda_0; v_i, v_j)  + 2\dot{\E}_2(\lambda_0; v_i, w_j),
\end{equation}
which delivers, from the symmetry with respect to $i$ and $j$
\begin{equation}
  \mathring{F}_{ij} = \ddot{\E}_2(\lambda_0; v_i, v_j) + 2\dot{\E}_2(\lambda_0; v_j, w_i).
\end{equation}

We also have the following identity on $\mathring{E}_{ijk}$, which results from Eq.~\eqref{eq:20220519164523}
\begin{equation}
  \label{eq:20220617085256}
  \begin{aligned}[b]
  \mathring{E}_{ijk} ={}& \dot{\E}_3(\lambda_0; v_i, v_j, v_k) + \dot{\E}_2(\lambda_0; v_i, w_{jk}) + \dot{\E}_2(\lambda_0; v_j, w_{ik}) + \dot{\E}_2(\lambda_0; v_k, w_{ij})\\
  ={}& \dot{\E}_3(\lambda_0; v_i, v_j, v_k) - \bigl[\E_2(\lambda_0; w_i, w_{jk}) + \E_2(\lambda_0; w_j, w_{ik}) + \E_2(\lambda_0; w_k, w_{ij})\bigr].
  \end{aligned}
\end{equation}

\section{Analysis of the critical point}
\label{sec:20220802061621}

In this section, we discuss the stability of the critical point $(u_0, \lambda_0)$. To this end, we evaluate the
potential energy in a neighboring state $u_0 + u$, where $u \in U$ is ``small''. We have, to the fourth order
\begin{equation}
  \begin{aligned}[b]
    \E(u_0 + u, \lambda_0) - \E(u_0, \lambda_0) ={}
    &\tfrac{1}{2} \E_2(\lambda_0; u, u) + \tfrac{1}{6} \E_3(\lambda_0; u, u, u)\\
    &+ \tfrac{1}{24} \E_4(\lambda_0; u, u, u, u) + o(\langle u , u \rangle^2),
  \end{aligned}
\end{equation}
where the linear term has been omitted, $u_0$ being a critical point of the energy. Since $v \in V$, we have
$\E_2(\lambda_0; v, \bullet) = 0$. We now expand $u$ as $u = \xi v + \eta w$, with $\xi, \eta \in \reals$ and $v \in V$
and $w \in W$ are fixed, orthogonal directions. Owing to the multi-linearity and symmetry of the successive differential
of $\E$, the above expression expands as follows
\begin{equation}
  \begin{aligned}[b]
    \E(u_0 + u, \lambda_0) - \E(u_0, \lambda_0) ={}
    & \tfrac{1}{2} \eta^2 \E_2(\lambda_0; w, w) + \tfrac{1}{6} \xi^3 \E_3(\lambda_0; v, v, v)\\
    & + \tfrac{1}{2} \xi^2 \eta \E_3(\lambda_0; v, v, w) + \tfrac{1}{2} \xi \eta^2 \E_3(\lambda_0; v, w, w)\\
    & + \tfrac{1}{6} \eta^3 \E_3(\lambda_0; w, w, w) + \tfrac{1}{24} \xi^4 \E_4(\lambda_0; v, v, v, v)\\
    & + \tfrac{1}{6} \xi^3 \eta \E_4(\lambda_0; v, v, v, w) + \tfrac{1}{4} \xi^2 \eta^2 \E_4(\lambda_0; v, v, w, w)\\
    & + \tfrac{1}{6} \xi \eta^3 \E_4(\lambda_0; v, w, w, w) + \tfrac{1}{24} \eta^4 \E_4(\lambda_0; w, w, w, w)\\
    & + o\bigl[\bigl(\xi^2 + \eta^2\bigr)^2\bigr].
  \end{aligned}
\end{equation}

For the equilibrium to be stable, the above expression must be $\geq 0$ for all $\xi$ et $\eta$ small enough. Taking
first $\eta = 0$, we get the following necessary conditions
\begin{equation}
  \label{eq:20211108164416}
  \E_3(\lambda_0; v, v, v) = 0 \quad \text{and} \quad \E_4(\lambda_0; v, v, v, v) \geq 0 \quad \text{for all} \quad v \in V.
\end{equation}

\begin{remark}
  Note that, from Theorem~\ref{thr:20220802112835}, the first of these two conditions is equivalent to $E_{ijk}=0$, for
  all $i, j, k = 1, \ldots m$.
\end{remark}

In other words, if there exists $v \in V$ such that $\E_3(\lambda_0; v, v, v) \neq 0$ or $\E_4(v, v, v, v) < 0$, then
the equilibrium is \emph{unstable} at the critical point. The above conditions are not sufficient. Indeed, assuming
conditions~\eqref{eq:20211108164416} to hold, we now take $\eta = \xi^2$
\begin{equation}
  \begin{aligned}[b]
    \E(u_0 + u, \lambda_0) - \E(u_0, \lambda_0) ={} & \tfrac{1}{2} \xi^4 \bigl[ \E_2(\lambda_0; w, w) + \E_3(\lambda_0; v, v, w)\\
    & + \tfrac{1}{12} \E_4(\lambda_0; v, v, v, v) \bigr] + o(\xi^4)
  \end{aligned}
\end{equation}
and we get the further necessary condition
\begin{equation}
  \label{eq:20211109145356}
  \E_2(w, w) + \E_3(v, v, w) + \tfrac{1}{12} \E_4(v, v, v, v) \geq 0 \quad \text{for all} \quad v \in V \quad \text{and} \quad w \in W.
\end{equation}

The direction $v \in V$ being fixed, the above expression is minimal when $w$ satisfies the following variational
problem
\begin{equation}
  \label{eq:20211109145224}
  2\E_2(w, \hat{w}) +\E_3(v, v, \hat{w}) = 0 \quad \text{for all} \quad \hat{w} \in W.
\end{equation}

Expanding $v \in V$ in the $(v_i)$ basis as follows: $v = \xi_i v_i$, it is observed that the solution to the above
variational problem is $w = \tfrac{1}{2} \xi_i \xi_j w_{ij}$, where $w_{ij}$ is the solution to the elementary
variational problem \eqref{eq:20220519164523}. For this value of $w$, condition~\eqref{eq:20211109145356} reads
\begin{equation}
  \bigl[\E_4(v_i, v_j, v_k, v_l) - 3\E_2(w_{ij}, w_{kl})\bigr] \xi_i \xi_j \xi_k \xi_l \geq 0 \quad \text{for all} \quad \xi_1, \ldots, \xi_m \in \reals,
\end{equation}
which, in view of definition~\eqref{eq:20220802081116} of $E_{ijkl}$, is equivalent to
\begin{equation}
  E_{ijkl} \xi_i \xi_j \xi_k \xi_l \geq 0 \quad \text{for all} \quad \xi_m, \ldots, \xi_m \in \reals.
\end{equation}

Note that Eq.~\eqref{eq:20211109145224} implies $\E_4(\lambda_0; v, v, v, v) \geq 0$, which becomes a redundant
necessary condition. Indeed, plugging $w= \xi_i \xi_j w_{ij}$ into Eq.~\eqref{eq:20211109145224} cancels the first two
terms. To sum up, we have the following \emph{necessary} conditions for stability
\begin{equation}
  E_{ijk} \xi_i \xi_j \xi_k = 0 \quad \text{and} \quad E_{ijkl} \xi_i \xi_j \xi_k \xi_l \geq 0 \quad \text{for all} \quad \xi_m, \ldots, \xi_m \in \reals.
\end{equation}

\pdfmarkupcomment{Conversely, the following condition is \emph{sufficient} to ensure stability of the critical
  point}{\`A d\'emontrer}
\begin{equation}
  E_{ijk} \xi_i \xi_j \xi_k = 0 \quad \text{and} \quad E_{ijkl} \xi_i \xi_j \xi_k \xi_l > 0 \quad \text{for all} \quad \xi_m, \ldots, \xi_m \in \reals.
\end{equation}

\section{Analysis of bifurcated branches}
\label{sec:20220617075558}

\color{gray}

In this section, we show that, besides the fundamental branch $u^\ast(\lambda)$, other (bifurcated) equilibrium branches
may pass through the critical point $(u_0, \lambda_0)$. The starting point is the characterization of an equilibrium by
the stationarity of the energy, which defines all equilibrium branches as implicit functions, which can be expanded with
respect to some perturbation parameter.

Two approaches are possible. The first approach relies on the Lyapunov--Schmidt decomposition of the equilibrium branch
over $V$ and $W$. However, this approach leads to tedious derivations. Still, this approach has historical and
pedagogical value: in particular, it provides a meaning to $w_i$ and $w_{ij}$ defined by Eqs.~\eqref{eq:20220524134525}
and \eqref{eq:20220519164523}. The Lyapunov--Schmidt decomposition will be used in Sec.~\ref{sec:20221018083712} for the
analysis of imperfect systems.

\color{black}

In the present section, a more systematic approach is followed, that I first found in the paper by \textcite[][Appendix
A]{chak2018}. Following this paper, we postulate the following parametrization of the bifurcated branch
\begin{align}
  \label{eq:20211115075817}
  \lambda &=  \lambda_0 + \eta \order[1]{\lambda} + \tfrac{1}{2} \eta^2 \order[2]{\lambda} + \tfrac{1}{6} \eta^3 \order[3]{\lambda} + \cdots,\\
  \label{eq:20211115075835}
  u &= u^{\ast}(\lambda) + \eta \order[1]{u} + \tfrac{1}{2} \eta^2 \order[2]{u} + \tfrac{1}{6} \eta^3 \order[3]{u} + \cdots,
\end{align}
where the parameter $\eta$ is not specified, but for the fact that $\eta = 0$ corresponds to the critical point
$(u_0, \lambda_0)$. Note that, in Eq.~\eqref{eq:20211115075835}, $u^\ast$ is evaluated at $\lambda$ rather than
$\lambda_0$.

Expressing that the energy is stationary along the bifurcated equilibrium path leads to the identification of the
coefficients $\order[k]\lambda$ and $\order[k]u$ of the expansions~\eqref{eq:20211115075817} and
\eqref{eq:20211115075835}. In other words, the residual $\E_{, u} [u(\eta), \lambda(\eta)]$ vanishes for all $\eta$
close to $0$. The residual is expanded with respect to the powers of $\eta$ in Appendix~\ref{sec:20211112182000} [see
Eq.~\eqref{eq:20220107080901}]. Since all the terms of this expansion must vanish, we get successively, for all
$\hat{u} \in U$
\begin{equation}
  \label{eq:20211112182917}
  \E_2(\lambda_0; \order[1]u, \hat{u}) = 0,
\end{equation}
\begin{equation}
  \label{eq:20220524133447}
  \E_3(\lambda_0; \order[1]u, \order[1]u, \hat{u}) + 2\order[1]\lambda\dot{\E}_2(\lambda_0; \order[1]u, \hat{u}) + \E_2(\lambda_0; \order[2]u, \hat{u}) = 0,
\end{equation}
\begin{equation}
  \label{eq:20220708060436}
  \begin{aligned}[b]
    \E_4(\lambda_0; \order[1]u, \order[1]u, \order[1]u, \hat{u}) + 3\E_3(\lambda_0; \order[1]u, \order[2]u, \hat{u}) + \E_2(\lambda_0; \order[3]u, \hat{u})&\\
    + 3\order[1]\lambda\dot{\E}_3(\lambda_0; \order[1]u, \order[1]u, \hat{u}) + 3\order[1]\lambda\dot{\E}_2(\lambda_0;  \order[2]u, \hat{u})&\\
    + 3(\order[1]\lambda)^2\ddot{\E}_2(\lambda_0; \order[1]u, \hat{u}) + 3\order[2]\lambda\dot{\E}_2(\lambda_0; \order[1]u, \hat{u}) & = 0.
  \end{aligned}
\end{equation}
It results from Eq.~\eqref{eq:20211112182917} that $\order[1]u \in V$. Testing with $\hat{v} \in V$ (rather than
$\hat{u} \in U$), Eq.~\eqref{eq:20220524133447} shows that $\order[1]u$ est solves the following variational problem:
find $\order[1]u \in V$ such that
\begin{equation}
  \label{eq:20220524133816}
  \tfrac{1}{2} \E_3(\lambda_0; \order[1]u, \order[1]u, \hat{v}) + \order[1]\lambda\dot{\E}_2(\lambda_0; \order[1]u, \hat{v}) = 0,
\end{equation}
for all $\hat{v} \in V$. The above problem can be transformed into a system of scalar equations. Indeed, expanding the
$\order[1]u \in V$ in the basis $(v_i)_{1 \leq i \leq m}$ as follows
\begin{equation}
  \label{eq:20220524133944}
  \order[1]u = \order[1]{\xi_i} v_i
\end{equation}
and plugging the definitions~\eqref{eq:20220524135619} and
\eqref{eq:20220524135643} of $E_{ijk}$ and $F_{ij}$ into
Eq.~\eqref{eq:20220524133816}
\begin{equation}
  \label{eq:20220524135036}
  \tfrac{1}{2} E_{ijk} \order[1]{\xi_j} \order[1]{\xi_k} + \order[1]\lambda F_{ij} \order[1]{\xi_j} = 0.
\end{equation}

In order to find the higher-order terms, namely $\order[2]\lambda$ et $\order[2]u$, we postulate the following
decomposition
\begin{equation}
  \order[2]u = \order[2]{\xi_i} v_i + \order[2]w,
\end{equation}
where $\order[2]w \in W$ is the orthogonal projection of $\order[2]u$ onto $W$. Then
$\E_2(\order[2]u, \hat{u}) = \E_2(\order[2]{w}, \hat{u})$ and Eq.~\eqref{eq:20220524133447} reads
\begin{equation}
 \E_3(\lambda_0; \order[1]u, \order[1]u, \hat{u}) + 2\order[1]\lambda \dot{\E}_2(\lambda_0; \order[1]u, \hat{u}) + \E_2(\lambda_0; \order[2]w, \hat{u}) = 0,
\end{equation}
for all $\hat{u} \in U$. Testing now with $\hat{w} \in W$ (rather than $\hat{u} \in U$), we get the following
variational problem: find $\order[2]w \in W$ such that
\begin{equation}
  \label{eq:20211210131623}
  \E_2(\lambda_0; \order[2]w, \hat{w}) + \order[1]{\xi_i} \order[1]{\xi_j} \E_3(\lambda_0; v_i, v_j, \hat{w}) + 2\order[1]\lambda \order[1]{\xi_i} \dot{\E}_2(\lambda_0; v_i, \hat{w}) = 0,
\end{equation}
for all $\hat{w} \in W$. The solution to the variational problem~\eqref{eq:20211210131623} is expressed as a linear
combination of the $w_i$ and $w_{ij}$ [defined by the variational problems~\eqref{eq:20220524134525} and
\eqref{eq:20220519164523}]:
$\order[2]w = \order[1]{\xi_i} \order[1]{\xi_j} w_{ij} + 2\order[1]\lambda \order[1]{\xi_i} w_i$ and
\begin{equation}
  \label{eq:20220524134613}
  \order[2]u = \order[2]{\xi_i} v_i + \order[1]{\xi_i} \order[1]{\xi_j} w_{ij} + 2\order[1]\lambda \order[1]{\xi_i} w_i.
\end{equation}

Plugging expressions~\eqref{eq:20220524133944} and \eqref{eq:20220524134613} into Eq.~\eqref{eq:20220708060436} and
taking further $\hat{u} = v_i$ [remember that $\E_2(\lambda_0; v_i, \bullet) = 0$], we then get
% \begin{equation*}
%   \begin{aligned}[b]
%     \E_4(\lambda_0; v_i, \order[1]{\xi_j} v_j, \order[1]{\xi_k} v_k, \order[1]{\xi_l} v_l)
%     + 3\E_3(\lambda_0; v_i, \order[1]{\xi_j} v_j, \order[2]{\xi_k} v_k + \order[1]{\xi_k} \order[1]{\xi_l} w_{kl}
%     + 2\order[1]\lambda \order[1]{\xi_k} w_k)&\\
%   + 3\order[1]\lambda \dot{\E}_3(\lambda_0; v_i, \order[1]{\xi_j} v_j, \order[1]{\xi_k} v_k)
%     + 3\order[1]\lambda \dot{\E}_2(\lambda_0; v_i, \order[2]{\xi_j} v_j + \order[1]{\xi_j} \order[1]{\xi_k} w_{jk} + 2\order[1]\lambda \order[1]{\xi_j} w_j)&\\
%     + 3( \order[1]\lambda )^2 \ddot{\E}_2(\lambda_0; v_i, \order[1]{\xi_j} v_j) + 3\order[2]\lambda \dot{\E}_2(\lambda_0; v_i, \order[1]{\xi_j} v_j) &= 0
%   \end{aligned}
% \end{equation*}
% \begin{equation*}
%   \begin{aligned}[b]
%     \E_4(\lambda_0; v_i, v_j, v_k, v_l) \order[1]{\xi_j} \order[1]{\xi_k} \order[1]{\xi_l}
%     + 3\E_3(\lambda_0; v_i, v_j, v_k) \order[1]{\xi_j} \order[2]{\xi_k}
%     + 3\E_3(\lambda_0; v_i, v_j, w_{kl}) \order[1]{\xi_j} \order[1]{\xi_k} \order[1]{\xi_l}&\\
%     + 6\order[1]\lambda \E_3(\lambda_0; v_i, v_j, w_k) \order[1]{\xi_j} \order[1]{\xi_k}
%     + 3\order[1]\lambda \dot{\E}_3(\lambda_0; v_i, v_j, v_k) \order[1]{\xi_j} \order[1]{\xi_k}
%     + 3\order[1]\lambda \dot{\E}_2(\lambda_0; v_i, v_j) \order[2]{\xi_j}&\\
%     + 3\order[1]\lambda \dot{\E}_2(\lambda_0; v_i, w_{jk}) \order[1]{\xi_j} \order[1]{\xi_k}
%     + 6( \order[1]\lambda )^2 \dot{\E}_2(\lambda_0; v_i, w_j) \order[1]{\xi_j}
%     + 3( \order[1]\lambda )^2 \ddot{\E}_2(\lambda_0; v_i, v_j) \order[1]{\xi_j}&\\
%     + 3\order[2]\lambda \dot{\E}_2(\lambda_0; v_i, v_j) \order[1]{\xi_j} &= 0
%   \end{aligned}
% \end{equation*}
\begin{equation*}
  \begin{aligned}[b]
    \bigl[\E_4(\lambda_0; v_i, v_j, v_k, v_l) + 3\E_3(\lambda_0; v_i, v_j, w_{kl})\bigr] \order[1]{\xi_j} \order[1]{\xi_k} \order[1]{\xi_l}&\\
    + 3\order[1]\lambda \bigl[2\E_3(\lambda_0; v_i, v_j, w_k) + \dot{\E}_3(\lambda_0; v_i, v_j, v_k) + \dot{\E}_2(\lambda_0; v_i, w_{jk}) \bigr] \order[1]{\xi_j} \order[1]{\xi_k}&\\
    + 3 \bigl\{ ( \order[1]\lambda )^2 \bigl[ 2 \dot{\E}_2(\lambda_0; v_i, w_j) + \ddot{\E}_2(\lambda_0; v_i, v_j) \bigr] + \order[2]\lambda \dot{\E}_2(\lambda_0; v_i, v_j) \bigr\}\order[1]{\xi_j}&\\
    + 3\bigl[\E_3(\lambda_0; v_i, v_j, v_k) \order[1]{\xi_k} + \order[1]\lambda \dot{\E}_2(\lambda_0; v_i, v_j)\bigr] \order[2]{\xi_j} &= 0
  \end{aligned}
\end{equation*}
It results from the variational problems \eqref{eq:20220524134525} and
\eqref{eq:20220519164523} that
\begin{equation*}
  \E_3(\lambda_0; v_i, v_j, w_k) = -\E_2(\lambda_0 ; w_{ij}, w_k) = \dot{\E}_2(\lambda_0; v_k, w_{ij}),
\end{equation*}
therefore
\begin{equation*}
  \begin{aligned}[b]
    \E_3(\lambda_0; v_i, v_j, w_k) \order[1]{\xi_j} \order[1]{\xi_k} &= \tfrac{1}{2} \bigl[ \E_3(\lambda_0; v_i, v_j, w_k) + \E_3(\lambda_0; v_i, v_k, w_j)\bigr] \order[1]{\xi_j} \order[1]{\xi_k}\\
                                    &= \tfrac{1}{2} \bigl[ \dot{\E}_2(\lambda_0; v_k, w_{ij}) + \dot{\E}_2(\lambda_0; v_j, w_{ik}) \bigr] \order[1]{\xi_j} \order[1]{\xi_k}.
  \end{aligned}
\end{equation*}
Similarly,
\begin{equation*}
  \begin{aligned}[b]
    \dot{\E}_2(\lambda_0; v_i, w_j) &= -\E_2(\lambda_0; w_i, w_j) = -\E_2(\lambda_0; w_j, w_i) = \dot{\E}_2(\lambda_0; v_j, w_i)\\
                           &= \tfrac{1}{2} \bigl[ \dot{\E}_2(\lambda_0; v_i, w_j) + \dot{\E}_2(\lambda_0; v_j, w_i) \bigr].
  \end{aligned}
\end{equation*}
% \begin{equation*}
%   \begin{aligned}[b]
%     \bigl[\E_4(\lambda_0; v_i, v_j, v_k, v_l) + \E_3(\lambda_0; v_i, v_j, w_{kl}) + \E_3(\lambda_0; v_i, v_k, w_{jl}) + \E_3(\lambda_0; v_i, v_l, w_{jk}) \bigr] \order[1]{\xi_j} \order[1]{\xi_k} \order[1]{\xi_l}&\\
%   + 3\order[1]\lambda \bigl[\dot{\E}_3(\lambda_0; v_i, v_j, v_k) + \dot{\E}_2(\lambda_0; v_i, w_{jk}) + \dot{\E}_2(\lambda_0; v_j, w_{ik}) + \dot{\E}_2(\lambda_0; v_k, w_{ij}) \bigr] \order[1]{\xi_j} \order[1]{\xi_k}&\\
%   + 3( \order[1]\lambda )^2 \bigl[ \ddot{\E}_2(\lambda_0; v_i, v_j) + \dot{\E}_2(\lambda_0; v_i, w_j) + \dot{\E}_2(\lambda_0; v_j, w_i) \bigr] \order[1]{\xi_j}&\\
%   + 3\bigl[\E_3(\lambda_0; v_i, v_j, v_k) \order[1]{\xi_k} + \order[1]\lambda \dot{\E}_2(\lambda_0; v_i, v_j)\bigr] \order[2]{\xi_j} + 3\order[2]\lambda \dot{\E}_2(\lambda_0; v_i, v_j) \order[1]{\xi_j} &= 0
%   \end{aligned}
% \end{equation*}

Finally, the definitions \eqref{eq:20220615063626}, \eqref{eq:20220615063633}, \eqref{eq:20220524135619},
\eqref{eq:20220524135553} and \eqref{eq:20220524135643} of $E_{ijk}$, $E_{ijkl}$, $F_{ij}$, $\mathring{E}_{ijk}$ and
$\mathring{F}_{ij}$ lead to the following compact bifurcation equation
\begin{equation}
  \label{eq:20220601070917}
  \tfrac{1}{3} E_{ijkl} \order[1]{\xi_j} \order[1]{\xi_k} \order[1]{\xi_l} + \order[1]\lambda \bigl( \mathring{E}_{ijk} \order[1]{\xi_k} + \order[1]\lambda \mathring{F}_{ij} \bigr)\order[1]{\xi_j} + \bigl(E_{ijk} \order[1]{\xi_k} + \order[1]\lambda F_{ij}\bigr) \order[2]{\xi_j} + \order[2]\lambda F_{ij} \order[1]{\xi_j} = 0.
\end{equation}

In order to analyse the stability of the bifurcated branches thus found, one must look at the Hessian of the energy. It
is first observed that, on the fundamental branch
\begin{equation}
 \E_2(\lambda; \hat{u}, \hat{v}) = \E_2(\lambda_0; \hat{u}, \hat{v}) + \bigl(\lambda - \lambda_0\bigr) \dot{\E}_2(\lambda_0; \hat{u}, \hat{v}) + o(\lambda - \lambda_0).
\end{equation}

In what follows, it will be assumed that $\dot{\E}_2(\lambda_0)\neq0$ and that $\E_2(\lambda)$ (which is positive
definite over $V$ for $\lambda<\lambda_0$ and null for $\lambda=\lambda_0$) is negative definite for $\lambda>\lambda_0$
sufficiently small (the fundamental branch is strictly unstable beyond the critical load). From the above expansion, it
results that $\dot{\E}_2(\lambda_0)$ is negative definite over $V$. In other words, $-F_{ij}$ is a positive definite
tensor. The asymptotic expansion of the Hessian of the energy along the bifurcated branch is derived in
appendix~\ref{sec:20220616055207}. For all $\hat{u}, \hat{v} \in U$
\begin{equation}
  \label{eq:20220531054247}
  \begin{aligned}[b]
    \E_{, uu}[u(\eta), \lambda(\eta); \hat{u}, \hat{v}] ={}
    & \E_2(\lambda_0 ; \hat{u}, \hat{v}) + \eta \bigl[\E_3(\lambda_0 ; \order[1]u, \hat{u}, \hat{v})  + \order[1]\lambda \dot{\E}_2(\lambda_0; \hat{u}, \hat{v})\bigr]\\
    &+ \tfrac{1}{2} \eta^2 \bigl[\E_4(\lambda_0; \order[1]u, \order[1]u, \hat{u}, \hat{v}) + \E_3(\lambda_0; \order[2]u, \hat{u}, \hat{v})\\
    & + 2\order[1]\lambda \dot{\E}_3(\lambda_0; \order[1]u, \hat{u}, \hat{v}) + ( \order[1]\lambda )^2 \ddot{\E}_2(\lambda_0; \hat{u}, \hat{v})\\
    & + \order[2]\lambda \dot{\E}_2(\lambda_0; \hat{u}, \hat{v}) \bigr] + o(\eta^2).
  \end{aligned}
\end{equation}

Stability analysis is performed by means of the eigenvalues $\alpha \in \reals$ and eigenvectors $x \in U$ of the
Hessian
\begin{equation}
  \label{eq:20220617074949}
  \E_{, u u} [u(\eta), \lambda(\eta); x, \hat{u}] = \alpha \langle x, \hat{u} \rangle \quad \text{for all} \quad \hat{u} \in V,
\end{equation}
where $\alpha$ and $x$ are expanded to second order in $\eta$
\begin{equation}
  \label{eq:20220617064633}
  \alpha = \order[0]\alpha + \eta \order[1]\alpha + \tfrac{1}{2} \eta^2 \order[2]\alpha + o(\eta^2)
  \quad \text{and} \quad
  x = \order[0]x + \eta \order[1]x + \tfrac{1}{2} \eta^2 \order[2]x + o(\eta^2).
\end{equation}

The following results are proved in Appendix~\ref{sec:20220616074108}: first, $(\order[0]\alpha, x_0)$ is necessarily an
eigenpair of $\E_2(\lambda_0)$. Since $\E_2 (\lambda_0)$ is positive, $\order[0]\alpha \geq 0$. If $\order[0]\alpha>0$,
then $\alpha>0$ in the neighborhood of $\lambda_0$. Potentially unstable modes are therefore such that
$\order[0]\alpha=0$. In other words, $\order[0]x \in V$ and
\begin{equation}
  \label{eq:20220904160057}
  \order[0]x = \order[0]{\chi_i} v_i
\end{equation}
furthermore, $(\order[1]\alpha, \order[0]{\chi_i})$ is an eigenpair of the symmetric
tensor $(E_{ijk} \order[1]{\xi_k} + \order[1]\lambda F_{ij})$
\begin{equation}
  \label{eq:20220609133608}
  \bigl(E_{ijk} \order[1]{\xi_k} + \order[1]\lambda F_{ij} \bigr) \order[0]{\chi_j} = \order[1]\alpha \order[0]{\chi_i}.
\end{equation}
As for the higher order terms, it is also found that
\begin{equation}
  \label{eq:20220609133629}
  \order[1]x = \order[1]{\chi_i} v_i +  \order[0]{\chi_i} \order[1]{\xi_j} w_{i j} + \order[1]\lambda \order[0]{\chi_i} w_i
\end{equation}
and
\begin{equation}
  \label{eq:20220616082923}
  \begin{aligned}[b]
    \bigl[E_{ijkl} \order[1]{\xi_k} \order[1]{\xi_l} + \order[1]\lambda\bigl(2 \mathring{E}_{ijk} \order[1]{\xi_k} + \order[1]\lambda \mathring{F}_{ij}\bigr) + E_{ijk} \order[2]{\xi_k}
    + \order[2]\lambda F_{ij} \bigr] \order[0]{\chi_j} &\\
    + 2\bigl(E_{ijk}  \order[1]{\xi_k} + \order[1]\lambda F_{ij} \bigr) \order[1]{\chi_j}
    & = 2\order[1]\alpha\order[1]{\chi_i} + \order[2]\alpha \order[0]{\chi_i}.
  \end{aligned}
\end{equation}

Finally, to close this analysis of the bifurcated branches, the following asymptotic expansion of the energy is derived
in Appendix~\ref{sec:20220525053434}
\begin{equation}
  \label{eq:20220525053600}
  \begin{aligned}[b]
    \E[u(\eta), \lambda(\eta)] ={} & \E\{u^{\ast}[\lambda(\eta)], \lambda(\eta)\} + \tfrac{1}{6} \order[1]\lambda \eta^3 F_{i j} \order[1]{\xi_i} \order[1]{\xi_j} + \tfrac{1}{24} \eta^4 \bigl\{E_{ijkl} \order[1]{\xi_i} \order[1]{\xi_j} \order[1]{\xi_k} \order[1]{\xi_l}\\
    & + 4\order[1]\lambda \mathring{E}_{ijk} \order[1]{\xi_i} \order[1]{\xi_j} \order[1]{\xi_k} + 6 \bigl[( \order[1]\lambda )^2 \mathring{F}_{ij} + \order[2]\lambda F_{ij}\bigr] \order[1]{\xi_i} \order[1]{\xi_j}\bigr\} + o(\eta^4).
  \end{aligned}
\end{equation}

\section{Discussion}

In this section, we discuss the two main cases of bifurcations, namely \emph{asymmetric} and \emph{symmetric}. In each
case, we analyse the stability of the bifurcated branch.

\begin{remark}
  The boundary case is unclear to me. I think that whether a bifurcation is symmetric or asymmetric should depend on the
  value of $\order[1]\lambda$ only. If $\order[1]\lambda \neq 0$, the bifurcated branch is
  \emph{asymmetric}. Conversely, if $\order[1]\lambda = 0$ and $\order[2]\lambda \neq 0$, then the bifurcated branch is
  \emph{symmetric}.

  In the literature, the discussion is placed on $E_{ijk}$. If $\order[1]\lambda \neq 0$, surely one of the $E_{ijk}$ is
  non-zero also. However, I believe it is \emph{not} a sufficient condition: one of the bifurcated branches could be
  symmetric $(\order[1]\lambda = 0)$, even if all $E_{ijk}$ are not null. It is true however that \emph{all} bifurcated
  branches are symmetric if, and only if, $E_{ijk}=0$ for all $i, j, k = 1, \ldots, m$. Therefore, the two cases that
  will be discussed below are: (1) one of the bifurcated branches is asymmetric and (2) all bifurcated branches are
  symmetric. The mixed case ``one of the bifurcated branches is symmetric'' will \emph{not} be discussed.
\end{remark}

\subsection{Asymmetric bifurcated branch}

We first consider the situation where $\order[1]\lambda \neq 0$ on the bifurcated branch. The bifurcation
equation~\eqref{eq:20220524135036} shows that necessarily, $E_{ijk}$ is not identically nul. This equation has at most
$(2^m - 1)$ pairs of real solutions $(\order[1]\lambda, \order[1]u)$ et $(- \order[1]\lambda, - \order[1]u)$;
furthermore, multiplication by $\order[1]{\xi_i}$ shows that
\begin{equation}
  \label{eq:20220801085236}
  \order[1]\lambda = -\frac{E_{ijk} \order[1]{\xi_i} \order[1]{\xi_j} \order[1]{\xi_k}}{2 F_{ij} \order[1]{\xi_i} \order[1]{\xi_j}}.
\end{equation}

\begin{remark}
  I can't prove that the bifurcation equation~\eqref{eq:20220524135036} has at most $(2^m - 1)$ pairs of real solutions.
\end{remark}

Along the bifurcated branch, we have $\lambda = \lambda_0 + \eta \order[1]\lambda + o(\eta)$, and $\eta$ can be
eliminated. In other words, $\eta=\lambda$ ($\order[1]\lambda=1$ and $\order[2]\lambda = \order[3]\lambda = \cdots = 0$)
can be selected as a parameter. It is therefore possible to express the bifurcated branch as a function of $\lambda$:
$u(\lambda)$. For example, combining Eqs.~\eqref{eq:20220524133816} and \eqref{eq:20220531054247}, we find that
\begin{equation}
  \begin{aligned}[b]
    \E_{, uu}[u(\eta), \lambda(\eta); \order[1]u, \order[1]u]
    &= \eta \bigl[\E_3(\lambda_0 ; \order[1]u, \order[1]u, \order[1]u)  + \order[1]\lambda \dot{\E}_2(\lambda_0; \order[1]u, \order[1]u)\bigr] + o(\eta)\\
    &= - \eta \order[1]\lambda \dot{\E}_2(\lambda_0; \order[1]u, \order[1]u) + o(\eta),
  \end{aligned}
\end{equation}
or
\begin{equation}
  \label{eq:20220819160235}
  \E_{, uu}[u(\lambda), \lambda; \order[1]u, \order[1]u] = -\bigl( \lambda - \lambda_0 \bigr) \dot{\E}_2(\lambda_0; \order[1]u, \order[1]u) + o(\lambda - \lambda_0).
\end{equation}

For $\lambda < \lambda_0$, the above quantity is \emph{negative} (since $\dot{\E}_2$ is negative definite). In other
words

\begin{center}
  \framebox{For asymmetric bifurcations, below the critical load, the bifurcated branch is unstable}
\end{center}

To investigate the stability above the critical load, we need to analyse the sign of the eigenvalues $\alpha$ of the
Hessian. At first order, $\alpha = \eta \order[1]\alpha + o(\eta)$, where $\order[1]\alpha$ is an eigenvalue of
$(E_{ijk} \order[1]{\xi_k} + \order[1]\lambda F_{ij})$. Let $\alpha_{\min}$ and $\alpha_{\max}$ be the minimum and
maximum eigenvalues of this second-order tensor. Three cases must be discussed
\begin{enumerate}
\item If $\alpha_{\min} \alpha_{\max} > 0$, then $(E_{ijk} \order[1]{\xi_k} + \order[1]\lambda F_{ij})$ is positive or
  negative definite: all eigenvalues have the same sign, $\epsilon \in \{-1, +1\}$. Then the sign of the eigenvalues
  $\alpha$ of the Hessian is $\epsilon \eta$ and there is a stability switch at the critical load. Since the bifurcated
  branch is unstable \emph{below} the critical load, this means that it is \emph{stable} above the critical load.
\item If $\alpha_{\min} \alpha_{\max} < 0$, then the extremal eigenvalues of the Hessian are $\eta \alpha_{\min}$ and
  $\eta \alpha_{\max}$, the product of which is $\eta^2 \alpha_{\min} \alpha_{\max} < 0$. The bifurcated branch is
  \emph{unstable} for all values of $\lambda$.
\item If $\alpha_{\min} \alpha_{\max} = 0$, the analysis is inconclusive.
\end{enumerate}

To close this section, it is observed that the dominant term of the expansion~\eqref{eq:20220525053600} of the potential
energy along the bifurcated branch is of the third order in $\eta$
\begin{equation}
  \E[u(\eta), \lambda(\eta)] = \E\{u^{\ast}[\lambda(\eta)], \lambda(\eta)\} + \tfrac{1}{6} \order[1]\lambda \eta^3 F_{i j} \order[1]{\xi_i} \order[1]{\xi_j} + o(\eta^3).
\end{equation}

Eliminating $\lambda$ and plugging expression~\eqref{eq:20220801085236} of $\order[1]\lambda$ delivers the expression of
the potential energy, where $\lambda$ is the parameter
\begin{equation}
  \begin{aligned}[b]
    \E[u(\lambda), \lambda] &= \E[u^{\ast}(\lambda), \lambda] + \frac{\bigl(\lambda - \lambda_0\bigr)^3}{6\bigl( \order[1]\lambda \bigr)^2} F_{i j} \order[1]{\xi_i} \order[1]{\xi_j} + o(\lambda^3)\\
    &= \E[u^{\ast}(\lambda), \lambda] + \frac{2 \bigl( F_{i j} \order[1]{\xi_i} \order[1]{\xi_j} \bigr)^3}{3 \bigl( E_{ijk} \order[1]{\xi_i} \order[1]{\xi_j} \order[1]{\xi_k} \bigr)^2} \bigl(\lambda - \lambda_0\bigr)^3 + o(\lambda^3).
  \end{aligned}
\end{equation}

Recalling that $F_{i j} \order[1]{\xi_i} \order[1]{\xi_j} < 0$, it is found that, above the critical load, the potential
energy is \emph{smaller} along the bifurcated branch than along the fundamental branch.

\begin{remark}
  As expected, the above expression does not depend on the scaling of $\order[1]u$ (of the $\order[1]{\xi_i}$).
\end{remark}
\begin{remark}
  It has been shown in Sec.~\ref{sec:20220802061621} that, when $E_{ijk}$ is not identically null, the bifurcation point
  is \emph{unstable}.
\end{remark}

\subsection{A particular case of symmetric bifurcation}

We now consider the case $E_{ijk}=0$ for all $i, j, k = 1, \ldots, m$. Then [see Eq.~\eqref{eq:20220524135036}]
$\order[1]\lambda = 0$ on \emph{all} bifurcated branches. It is assumed that, on the bifurcated branch under
consideration, the next term of the expansion of $\lambda$ is non-zero: $\order[2]\lambda \neq 0$. The bifurcation is
\emph{symmetric}, and the bifurcation equation~\eqref{eq:20220601070917} reduces to
\begin{equation}
  \label{eq:20220801092222}
  \tfrac{1}{3} E_{ijkl} \order[1]{\xi_j} \order[1]{\xi_k} \order[1]{\xi_l}  + \order[2]\lambda F_{ij} \order[1]{\xi_j} = 0,
\end{equation}
which has at most $(3^m - 1) / 2$ pairs of real solutions
$(\order[2]\lambda, \order[1]u)$ and $(- \order[2]\lambda, - \order[1]u)$. Upon
multiplication by $\order[1]{\xi_i}$, the above equation delivers the following
expression of $\order[2]\lambda$
\begin{equation}
  \label{eq:20220801093236}
  \order[2]\lambda = -\frac{E_{ijkl} \order[1]{\xi_i} \order[1]{\xi_j} \order[1]{\xi_k} \order[1]{\xi_l}}{3 F_{ij} \order[1]{\xi_i} \order[1]{\xi_j}}.
\end{equation}

Since $F_{ij} \order[1]{\xi_i} \order[1]{\xi_j} < 0$, $\order[2]\lambda$ has the same sign as
$E_{ijkl}\order[1]{\xi_i} \order[1]{\xi_j} \order[1]{\xi_k} \order[1]{\xi_l}$. In other words, if
$E_{ijkl}\order[1]{\xi_i} \order[1]{\xi_j} \order[1]{\xi_k} \order[1]{\xi_l} > 0$, (resp. $<0$) then the bifurcated
branch exists above (resp. below) the critical load $\lambda_0$ only.

\begin{remark}
  I can't prove that the bifurcation equation~\eqref{eq:20220801092222} has at most $(3^m - 1) / 2$ pairs of real
  solutions.
\end{remark}

Turning now to the eigenpairs of the Hessian of the energy along the bifurcated branch, Eq.~\eqref{eq:20220609133608}
shows that $\order[1]\alpha = 0$. Then $\alpha = \order[2]\alpha \eta^2 / 2 + o(\eta^2)$ and, from
Eq.~\eqref{eq:20220616082923}
\begin{equation}
  \bigl(E_{ijkl} \order[1]{\xi_k} \order[1]{\xi_l} + \order[2]\lambda F_{ij} \bigr) \order[0]{\chi_j} = \order[2]\alpha \order[0]{\chi_i}.
\end{equation}

If $(E_{ijkl} \order[1]{\xi_k} \order[1]{\xi_l} + \order[2]\lambda F_{ij} )$ is positive definite, then the bifurcated
branch is stable (note that, in that case, the bifurcated branch exists above the critical load only). If one of the
eigenvalues of this tensor is $<0$, then the bifurcated branch is unstable. The stability is undecided when all
eigenvalues are $\geq 0$.

\begin{remark}
  Note that, from Eq.~\eqref{eq:20220801092222},
  \begin{equation}
    E_{ijkl} \order[1]{\xi_i} \order[1]{\xi_j} \order[1]{\xi_k} \order[1]{\xi_l} + \order[2]\lambda F_{ij} \order[1]{\xi_i} \order[1]{\xi_j} = \tfrac{2}{3} E_{ijkl} \order[1]{\xi_i} \order[1]{\xi_j} \order[1]{\xi_k} \order[1]{\xi_l}
  \end{equation}

\end{remark}

To conclude this section, it is observed that, when $\order[1]\lambda = 0$, the dominant term of the potential energy
along the bifurcated branch is of the fourth order [see Eq.~\eqref{eq:20220525053600}]. Combining with
Eq.~\eqref{eq:20220801093236},
\begin{equation}
  \label{eq:20220801094437}
  \begin{aligned}[b]
    \E[u(\eta), \lambda(\eta)]
    &= \E\{u^{\ast}[\lambda(\eta)], \lambda(\eta)\} + \tfrac{1}{24} \eta^4 \bigl(E_{ijkl} \order[1]{\xi_i} \order[1]{\xi_j} \order[1]{\xi_k} \order[1]{\xi_l}  + 6  \order[2]\lambda F_{ij} \order[1]{\xi_i} \order[1]{\xi_j}\bigr) + o(\eta^4)\\
    &= \E\{u^{\ast}[\lambda(\eta)], \lambda(\eta)\} - \tfrac{1}{24} \eta^4 E_{ijkl} \order[1]{\xi_i} \order[1]{\xi_j} \order[1]{\xi_k} \order[1]{\xi_l} + o(\eta^4).
  \end{aligned}
\end{equation}

The expansion $\lambda = \lambda_0 + \order[2]\lambda \eta^2 / 2 + o(\eta^2)$ can be inverted as follows
\begin{equation}
  \eta^4 = \frac{4 \bigl(\lambda - \lambda_0\bigr)^2}{\bigl( \order[2]\lambda \bigr)^2} + o(\lambda^2) = \frac{36 \bigl( F_{ij} \order[1]{\xi_i} \order[1]{\xi_j} \bigr)^2}{\bigl( E_{ijkl} \order[1]{\xi_i} \order[1]{\xi_j} \order[1]{\xi_k} \order[1]{\xi_l} \bigr)^2} \bigl( \lambda - \lambda_0 \bigr)^2
\end{equation}
and expression~\eqref{eq:20220801094437} reads
\begin{equation}
  \E[u(\eta), \lambda(\eta)] = \E\{u^{\ast}[\lambda(\eta)], \lambda(\eta)\} - \frac{3 \bigl( F_{ij} \order[1]{\xi_i} \order[1]{\xi_j} \bigr)^2}{2 E_{ijkl} \order[1]{\xi_i} \order[1]{\xi_j} \order[1]{\xi_k} \order[1]{\xi_l} } \bigl( \lambda - \lambda_0 \bigr)^2  + o(\lambda^2).
\end{equation}

Again, the above expression does not depend on the scaling of $\order[1]u$ (of the $\order[1]{\xi_i}$). Note that, if
$E_{ijkl} \order[1]{\xi_i} \order[1]{\xi_j} \order[1]{\xi_k} \order[1]{\xi_l} > 0$, then only loads that are greater
than the critical load can be reached on the bifurcated branch, where the energy is lower than the fundamental branch.

\medskip

The above discussion simplifies considerably when there is only one buckling mode ($m = 1$). This is addressed in the
next section.

\section{The case of a single mode}

In this section, we discuss the case $m = 1$; all tensors considered above ($F_{ij}$, $E_{ijk}$, $E_{ijkl}$) then reduce
to simple scalars. To avoid ambiguity, indices are kept: $F_{11}$, $E_{11}$, $E_{11}$. Since $\dot{\E}_2(\lambda_0)$ is
negative definite over $V$, we have $F_{11} < 0$.

It is first observed that the following conditions are \emph{necessary} to ensure stability of the critical point
\begin{equation}
  E_{111} = 0 \quad \text{and} \quad E_{1111} \geq 0,
\end{equation}
which shows that \emph{asymmetric} bifurcation points are always
\emph{unstable}.

\subsection{Asymmetric bifurcations}

We first consider the case $E_{111} \neq 0$. Owing to the discussion above, the bifurcation point is unstable. Setting
$\order[1]\lambda = 1$, Eq.~\eqref{eq:20220524135036} delivers
\begin{equation}
  E_{111} \order[1]\xi_1 + 2F_{11} = 0 \quad \text{and} \quad u(\lambda) = u^\ast(\lambda) - \frac{2F_{11}}{E_{111}} \bigl( \lambda - \lambda_0 \bigr) v_1 + o(\lambda - \lambda_0).
\end{equation}

Furthermore, the hessian of the energy along the bifurcated branch is retrieved from Eq.~\eqref{eq:20220819160235}
\begin{equation}
  \begin{aligned}[b]
    \E_{, uu}[u(\eta), \lambda(\eta), v_1, v_1] &= \eta \bigl(E_{111} \order[1]{\xi_1} + \order[1]\lambda F_{11}\bigr) + o(\eta) = -2 \eta F_{11} + o(\eta)\\
    &= -2 F_{11} \bigl( \lambda - \lambda_0 \bigr) + o(\lambda - \lambda_0).
  \end{aligned}
\end{equation}

Asymmetric bifurcations branches are \emph{unstable} for $\lambda \leq \lambda_0$ and \emph{stable} for
$\lambda > \lambda_0$ (stability switch).

\subsection{Symmetric bifurcations}

We now consider the case $E_{111}=0$. From the general discussion of Sec.~\ref{sec:20220802061621}, the bifurcation
point is \emph{stable} if $E_{1111} > 0$ and \emph{unstable} if $E_{1111} < 0$. The bifurcation
equation~\eqref{eq:20220801092222} reduces to
\begin{equation}
  E_{1111} \bigl( \order[1]{\xi_1} \bigr)^2 + 3\order[2]\lambda F_{11} = 0,
\end{equation}
which in particular shows that $\order[2]\lambda$ has the same sign as $E_{1111}$. Since the expansion of $\lambda$
reads: $\lambda = \lambda_0 + \order[2]\lambda \eta^2 / 2$, the bifurcation branch exists only for loads \emph{above}
the critical load ($\lambda \geq \lambda_0$) if $E_{1111} > 0$ and only for loads \emph{below} the critical load
($\lambda \leq \lambda_0$) if $E_{1111} < 0$.

From Eq.~\eqref{eq:20220531054247}, the hessian of the energy along the bifurcated branch reads
\begin{equation*}
  \E_{, uu}[u(\eta), \lambda(\eta); v_1, v_1] = \tfrac{1}{2} \eta^2 \bigl[ E_{1111}\bigl(\xi_1^1\bigr) ^2 + \order[2]\lambda F_{11} \bigr] + o(\eta^2) = - \eta^2 \order[2]\lambda F_{11} + o(\eta^2),
\end{equation*}
which has the sign of $\order[2]\lambda$. Therefore the Hessian is positive (resp. negative) definite if $E_{1111} > 0$
(resp $< 0$).

To sum up, if $E_{1111} > 0$, then the bifurcation branch (including the critical point) is \emph{stable} and exists
only for loads greater than the critical load. Conversely, if $E_{1111} < 0$, then the bifurcation branch (including the
critical point) is \emph{unstable} and exists only for loads lower than the critical load.

\section{Imperfect systems}
\label{sec:20221018083712}

This section is devoted to \emph{imperfect systems}. Such a system is described by the energy
\(\EE(u, \lambda, \epsilon)\), where \(\epsilon\) is the \emph{imperfection parameter}, assumed to be small. It is
assumed that for \(\epsilon = 0\), the previous energy is recovered
\begin{equation}
  \EE(u, \lambda, 0) = \E(u, \lambda).
\end{equation}

Our goal in this section is to analyze equilibria of the imperfect system close to the critical point
\((u_0, \lambda_0)\) of the perfect system. In other words, we aim to find an asymptotic expansion of the equilibrium
path \(u(\lambda, \epsilon)\) such that
\begin{equation}
  \label{eq:20221018085038}
  \EE_{,u}(u(\lambda, \epsilon), \lambda, \epsilon; \hat{u}) = 0 \quad \text{for all} \quad \hat{u} \in U.
\end{equation}

\color{gray}

To this end, we follow the Lyapunov--Schmidt approach, where the equilibrium point of the imperfect system is projected
onto the orthogonal subspaces \(V\) and \(W\).
\begin{equation}
  \label{eq:20220902174235}
  u = u^\ast(\lambda) + \xi_i \, v_i + w, \quad \text{with} \quad w \in W.
\end{equation}

It follows from the orthogonality of $V$ and $W$ that $\langle v_i, w \rangle = 0$ for all $i=1, \ldots,
m$. Eq.~\eqref{eq:20221018085038} is then expressed as follows
\begin{equation}
  \label{eq:20220901120544}
  \EE_{,u}[u^\ast(\lambda) + \xi_i v_i + w, \lambda, \epsilon; \hat{v}] = 0, \quad \text{for all} \quad \hat{v} \in V
\end{equation}
and
\begin{equation}
  \label{eq:20220825143616}
  \EE_{,u}[u^\ast(\lambda) + \xi_i v_i + w, \lambda, \epsilon; \hat{w}] = 0, \quad \text{for all} \quad \hat{w} \in W.
\end{equation}

The method proceeds in three steps. In \S~\ref{sec:20221020140204}, Eq.~\eqref{eq:20220825143616} is used to define $w$
as an implicit function of $\xi_1$, \dots, $\xi_m$, $\lambda$ and \(\epsilon\). Then, in \textbf{Step 2},
Eq.~\eqref{eq:20220825143616} is used to define $\lambda$ as an implicit function of $\xi_1$, \dots, $\xi_m$ and
\(\epsilon\). Finally, a parametrization $\eta$ of $\xi_1$, \dots $\xi_m$ is introduced in \textbf{Step 3} and the
Taylor expansion of $u$ and $\lambda$ with respect to $\eta$ is derived. These steps are presented below. In particular,
it will be shown that the method in general fails to deliver the seeked asymptotic expansion, put for two particular
cases, that will be discussed more thoroughly: simple modes (\(m = 1\)) and geometric imperfections
\(\EE(u, \lambda, \epsilon) = \E(u - \epsilon \, \uu, \lambda, \epsilon)\), where \(\uu\) denotes the shape of the
geometric imperfection.

In the remainder of this paragraph, the \emph{critical point} refers to
\((\xi_1, \ldots, \xi_m, \lambda, \epsilon) = (0, \ldots, 0, \lambda_0, 0)\) (bifurcation point of the perfect system).

\color{black}

\subsection{Elimination of \(w\)}
\label{sec:20221020140204}

In this paragraph, $\hat{w}$ denotes an arbitrary test function in $W$. Eq.~\eqref{eq:20220825143616} defines a function
$(\xi_1, \ldots, \xi_m, \lambda, \epsilon) \mapsto w(\xi_1, \ldots, \xi_m, \lambda)$ in the neighborhood of the critical point
$(\xi_1, \ldots, \xi_m, \lambda, \epsilon) = (0, \ldots, 0, \lambda_0, 0)$. A formal proof of this result is proposed by \textcite{poti1987}; in the
present note, existence of such mapping will be assumed, and its first- and second-order partial derivatives are
evaluated at the critical point from the derivatives of Eq.~\eqref{eq:20220825143616}.

\paragraph{Derivative of Eq.~\eqref{eq:20220825143616} with respect to \(\xi_i\)}
\begin{equation}
  \label{eq:20220826140926}
  \EE_{,uu}(u^\ast + \xi_k \, v_k + w, \lambda, \epsilon; v_i + w_{,i}, \hat{w}) = 0,
\end{equation}
and, at the critical point
\begin{equation}
  \label{eq:20220825150219}
  0 = \E_2(\lambda_0; v_i + w_{,i}, \hat{w}) = \E_2(\lambda_0; w_{,i}, \hat{w}),
\end{equation}
where it was observed that \(\E_2(\lambda_0; v_i, \hat{w}) = 0\) since \(v_i \in V = \ker \E_2\). Since $w \in W$ for all
$\xi^i$, $\lambda$ and \(\epsilon\) (in a neighborhood of the critical point), we have $w_{,i} \in W$ and,
Remark~\ref{rem:20220902095055} leads to $w_{,i}(0, \ldots, 0, \lambda_0, 0) = 0$ at the critical point.

\paragraph{Derivative of Eq.~\eqref{eq:20220825143616} with respect to $\lambda$}
\begin{equation}
  \label{eq:20220830145945}
  \EE_{,uu}(u^\ast + \xi_i v_i + w, \lambda, \epsilon; \dot{u}^\ast + w_{,\lambda}, \hat{w}) + \EE_{,u\lambda}(u^\ast + \xi_i v_i + w, \lambda, \epsilon; \hat{w}) = 0,
\end{equation}
and, at \(\xi_1 = \cdots = \xi_m = \epsilon = 0\), using Eq.~\eqref{eq:20220901143843}
\begin{equation}
  \label{eq:20220830151513}
  0 = \E_{,uu}(u^\ast, \lambda; w_{,\lambda}, \hat{w})
  + \E_{,uu}(u^\ast, \lambda; \dot{u}^\ast, \hat{w}) + \E_{,u\lambda}(u^\ast, \lambda; \hat{w})
  = \E_2(\lambda; w_{,\lambda}, \hat{w}),
\end{equation}
which proves similarly that \(w_{,\lambda}(0, \ldots, 0, \lambda_0, 0) = 0\).

\paragraph{Derivative of Eq.~\eqref{eq:20220825143616} with respect to $\epsilon$}
\begin{equation}
  \label{eq:20221020142808}
  \EE_{,uu}(u^\ast + \xi_i v_i + w, \lambda, \epsilon; w_{,\epsilon}, \hat{w}) + \EE_{,u\epsilon}(u^\ast + \xi_i v_i + w, \lambda, \epsilon; \hat{w}) = 0.
\end{equation}
Let \(w_\epsilon \in W\) the solution to the following variational problem
\begin{equation}
    \label{eq:20221030175016}
    \E_{,uu}(u_0, \lambda_0, w_\epsilon, \hat{w}) + \EE_{,u\epsilon}(u_0, \lambda_0, 0; \hat{w}) = 0.
\end{equation}
Evaluating Eq.~\eqref{eq:20221020142808} at the critical point delivers: \(w_{,\epsilon}(0, \ldots, 0, \lambda_0, 0) = w_\epsilon\).

\bigskip

To express the second-order derivatives of $w$, Eqs.~\eqref{eq:20220826140926}, \eqref{eq:20220830145945} and
\eqref{eq:20221020142808} are again derived with respect to $\xi_j$, $\lambda$ and \(\epsilon\).

\bigskip

\paragraph{Derivative of Eq.~\eqref{eq:20220826140926} with respect to \(\xi_j\)}
\begin{equation}
  \EE_{,uuu}(u^\ast + \xi_k v_k + w, \lambda, \epsilon; v_i + w_{,i}, v_j + w_{,j}, \hat{w}) + \EE_{,uu}(u^\ast + \xi_k v_k + w, \lambda, \epsilon; w_{,ij}, \hat{w}) = 0
\end{equation}
and, at the critical point, the variational problem~\eqref{eq:20220519164523} is recognized:
\(w_{,ij}(0, \ldots, 0, \lambda_0, \epsilon = 0) = w_{ij}\).

\paragraph{Derivative of Eq.~\eqref{eq:20220826140926} with respect to \(\lambda\)}
\begin{equation}
  \begin{aligned}[b]
    \EE_{,uuu}(u^\ast + \xi_k v_k + w, \lambda, \epsilon; v_i + w_{,i}, \dot{u}^\ast + w_{,\lambda}, \hat{w}) &\\
    + \EE_{,uu\lambda}(u^\ast + \xi_k v_k + w, \lambda, \epsilon; v_i + w_{,i}, \hat{w}) + \EE_{,uu}(u^\ast + \xi_k v_k + w, \lambda, \epsilon; w_{,i\lambda}, \hat{w}) &= 0.
  \end{aligned}
\end{equation}
Recalling that $w_{,1} = \cdots = w_{,m} = w_{,\lambda} = 0$ at the critical point, the variational
problem~\eqref{eq:20220524134525} is recognized: \(w_{,i\lambda}(0, \ldots, 0, \lambda_0, 0)=w_i\).

\paragraph{Derivative of Eq.~\eqref{eq:20220826140926} with respect to \(\epsilon\)}
\begin{equation}
  \label{eq:20221108141729}
  \begin{aligned}[b]
    \EE_{,uuu}(u^\ast + \xi_k v_k + w, \lambda, \epsilon; w_{,\epsilon}, v_i + w_{,i}, \hat{w}) + \EE_{,uu\epsilon}(u^\ast + \xi_k v_k + w, \lambda, \epsilon; v_i + w_{,i}, \hat{w}) &\\
    + \EE_{,uu}(u^\ast + \xi_k v_k + w, \lambda, \epsilon; v_i + w_{,i\epsilon}, \hat{w}) &= 0.
  \end{aligned}
\end{equation}
Introducing the solution \(w_{i\epsilon} \in W\) to the following variational problem
\begin{equation}
  \label{eq:20221030180004}
  \E_2(\lambda_0; w_{i\epsilon}, \hat{w}) + \E_3(\lambda_0; v_i, w_\epsilon, \hat{w}) + \E_2'(\lambda_0; v_i, \hat{w}) = 0
\end{equation}
and evaluating Eq.~\eqref{eq:20221108141729} at the critical point delivers: \(w_{,i\epsilon}(0, \ldots, 0, \lambda_0, 0)=w_{i\epsilon}\).

\paragraph{Derivative of Eq.~\eqref{eq:20220830151513} with respect to \(\lambda\)}
In order to find \(w_{,\lambda\lambda}(0, \ldots, 0, \lambda_0, 0)\), it is sufficient to differentiate Eq.~\eqref{eq:20220830151513} [rather
than Eq.~\eqref{eq:20220830145945}] with respect to \(\lambda\), delivering \(w_{,\lambda\lambda}(0, \ldots, 0, \lambda_0, 0)=0\).

\paragraph{Derivative of Eq.~\eqref{eq:20220830145945} with respect to \(\epsilon\)}
\begin{equation}
  \label{eq:20221108142026}
  \begin{aligned}[b]
    \EE_{,uuu}(u^\ast + \xi_i \, v_i + w, \lambda, \epsilon; w_{,\epsilon}, \dot{u}^\ast + w_{,\lambda}, \hat{w}) + \EE_{,uu\epsilon}(u^\ast + \xi_i \, v_i + w, \lambda, \epsilon; \dot{u}^\ast + w_{,\lambda}, \hat{w}) &\\
    + \EE_{,uu}(u^\ast + \xi_i \, v_i + w, \lambda, \epsilon; w_{,\lambda\epsilon}, \hat{w}) + \EE_{,uu\lambda}(u^\ast + \xi_i \, v_i + w, \lambda, \epsilon; w_{,\epsilon}\hat{w})&\\
    + \EE_{,u\lambda\epsilon}(u^\ast + \xi_i \, v_i + w, \lambda, \epsilon; \hat{w}) &= 0
  \end{aligned}
\end{equation}
Introducing the solution \(w_{\lambda\epsilon} \in W\) to the following variational problem
\begin{equation}
  \label{eq:20221030180909}
  \E_2(\lambda_0; w_{\lambda\epsilon}, \hat{w}) + \dot{\E}_2(\lambda_0; w_\epsilon, \hat{w})  + \E_2'(\lambda_0; \mathring{u}_0, \hat{w}) + \dot{\E}_1'(\lambda_0; \hat{w}) = 0,\\
\end{equation}
and evaluating Eq.~\eqref{eq:20221108142026} at the critical point delivers: \(w_{,\lambda\epsilon}(0, \ldots, 0, \lambda_0, 0) = w_{\lambda\epsilon}\).

\paragraph{Derivative of Eq.~\eqref{eq:20221020142808} with respect to \(\epsilon\)}
\begin{equation}
  \label{eq:20221108142405}
  \begin{aligned}[b]
    \EE_{,uuu}(u^\ast + \xi_i \, v_i + w, \lambda, \epsilon; w_{,\epsilon}, w_{,\epsilon}, \hat{w}) + 2\EE_{,uu\epsilon}(u^\ast + \xi_i \, v_i + w, \lambda, \epsilon; w_{,\epsilon}, \hat{w}) &\\
    + \EE_{,uu}(u^\ast + \xi_i \, v_i + w, \lambda, \epsilon; w_{,\epsilon\epsilon}, \hat{w}) + \EE_{,u\epsilon\epsilon}(u^\ast + \xi_i v_i + w, \lambda, \epsilon; \hat{w}) &= 0.
  \end{aligned}
\end{equation}
Introducing the solution \(w_{\epsilon\epsilon}\) to the following variational problem
\begin{gather}
  \label{eq:20221030181711}
  \E_2(\lambda_0; w_{\epsilon\epsilon}, \hat{w}) + \E_3(\lambda_0; w_\epsilon, w_\epsilon, \hat{w}) + 2\E_2'(\lambda_0; w_\epsilon, \hat{w}) + \E_1''(\lambda_0; \hat{w}) = 0,
\end{gather}
and evaluating Eq.~\eqref{eq:20221108142405} at the critical point delivers: \(w_{,\epsilon\epsilon}(0, \ldots, 0, \lambda_0, 0) = w_{\epsilon\epsilon}\).

\paragraph{Conclusion} To summarize the above derivation, we have obtained the following Taylor expansion of the
component $w$ of the LSK expansion of $u$
\begin{equation}
  \begin{aligned}[b]
    w(\xi_1, \ldots, \xi_m, \lambda) ={}& \epsilon \, w_\epsilon + \tfrac{1}{2} \xi_i \, \xi_j \, w_{ij} + \bigl( \lambda - \lambda_0 \bigr) \xi_i \, w_i + \epsilon \, \xi_i \, w_{i\epsilon} + \epsilon \bigl( \lambda - \lambda_0\bigr) w_{\lambda\epsilon}\\
    &+ \tfrac{1}{2} \epsilon^2 \, w_{\epsilon\epsilon} + o\Bigl(\xi_1^2 + \cdots + \xi_m^2 + \bigl(\lambda - \lambda_0\bigr)^2\Bigr).
  \end{aligned}
\end{equation}

\subsection{Elimination of \(λ\)}
\label{sec:20221020140252}

We now turn to Eq.~\eqref{eq:20220901120544}. Since $w$ is a function of \(\xi_1\), \dots, \(\xi_m\), \(\lambda\) and
\(\epsilon\), this equation implicitly defines $\lambda$ as a function of \(\xi_1\), \dots, \(\xi_m\) and \(\epsilon\), the derivatives of which
can be evaluated at $\xi_1 = \cdots = \xi_m = 0$ and \(\epsilon = 0\). In this paragraph, $\hat{v}$ denotes an arbitrary element of $V$.

\paragraph{Derivative of Eq.~\eqref{eq:20220901120544} with respect to \(\xi_i\)}
\begin{equation}
  \label{eq:20220901121940}
  \begin{aligned}[b]
    \EE_{,uu}[u^\ast(\lambda) + \xi_k \, v_k + w, \lambda, \epsilon; v_i + w_{,i} + \lambda_{,i} \, \dot{u}^\ast + \lambda_{,i} \, w_{,\lambda}, \hat{v}] &\\
    + \lambda_{, i} \, \EE_{,u\lambda}[u^\ast(\lambda) + \xi_k \, v_k + w, \lambda, \epsilon; \hat{v}] &= 0,
  \end{aligned}
\end{equation}
and at the critical point
\begin{equation}
  \E_{,uu}[u_0, λ_0; v_i + w_{,i} + \lambda_{,i} \, \dot{u}^\ast + \lambda_{,i} \, w_{,\lambda}, \hat{v}] + \lambda_{, i} \, \E_{,u\lambda}(u_0, \lambda_0; \hat{v}) = 0,
\end{equation}
which reduces to
\begin{equation}
  \E_2(\lambda_0; v_i + w_{,i} + \lambda_{,i} \, w_{,\lambda}, \hat{v}) + \lambda_{,i} \, \dot{\E}
\end{equation}
which is identically satisfied. The derivatives of \(\lambda\) with respect to \(\xi_i\) can therefore not be retrieved from
Eq.~\eqref{eq:20220901121940}.

\paragraph{Derivative of Eq.~\eqref{eq:20220901120544} with respect to \(\epsilon\)}
\begin{equation}
  \begin{aligned}[b]
    \EE_{,uu}[u^\ast(\lambda) + \xi_k \, v_k + w, \lambda, \epsilon; \lambda_{,\epsilon} \, \dot{u}^\ast(\lambda) + \lambda_{,\epsilon} \, w_{,\lambda} + w_{,\epsilon}, \hat{v}] &\\
    + \lambda_{,\epsilon} \, \EE_{,u\lambda}[u^\ast(\lambda) + \xi_k \, v_k + w, \lambda, \epsilon; \hat{v}] + \EE_{,u\epsilon}[u^\ast(\lambda) + \xi_k \, v_k + w, \lambda, \epsilon; \hat{v}] &= 0.
  \end{aligned}
\end{equation}
At the critical point
\begin{equation}
  \E_2(\lambda_0; w_\epsilon, \hat{v}) + \lambda_{,\epsilon} \, \dot{\E}_1(\lambda_0; \hat{v}) + \E_1'(\lambda_0; \hat{v}) = 0.
\end{equation}

In the above equation, the first term vanishes since \(\hat{v} \in V = \ker \E_2(\lambda_0)\), while the second term also
vanishes because \(\E_1(\lambda; \hat{v}) = 0\) for all \(\lambda\). We therefore have a contradiction if \(\E_1'(\lambda_0) \neq 0\).

\begin{remark}
  This contradiction probably invalidates the above determination of \(\lambda\) as a function of \(\xi_k\) and
  \(\epsilon\). In \S~\ref{sec:20221102134138}, it will be shown that it is possible to express \(\epsilon\) as a function of
  \(\xi_k\) and \(\lambda\). Still the present derivation is instructive (and valid!) for a perfect system. In the remainder of
  this section, the analysis will therefore be restricted to \(\epsilon=0\).
\end{remark}

\paragraph{Derivative of Eq.~\eqref{eq:20220901121940} with respect to \(\xi_j\)}
\begin{equation}
  \label{eq:20220901125230}
  \begin{gathered}[b]
    \E_{,uuu}[u^\ast + \xi_k \, v_k + w, \lambda; v_i + w_{,i} + \lambda_{,i} \, ( \dot{u}^\ast + w_{,\lambda} ), v_j + w_{,j} + \lambda_{,j} \, ( \dot{u}^\ast + \lambda_{,j} \, w_{,\lambda} ), \hat{v}]\\
    + \lambda_{,j} \, \E_{,uu\lambda}[u^\ast + \xi_k \, v_k + w, \lambda; v_i + w_{,i} + \lambda_{,i} \, ( \dot{u}^\ast + w_{,\lambda} ), \hat{v}]\\
    + \E_{,uu}[u^\ast + \xi_k \, v_k + w, \lambda; w_{,ij} + \lambda_{, j} \, w_{,i\lambda} + \lambda_{, i} \, w_{,j\lambda} + \lambda_{,ij} \, ( \dot{u}^\ast + w_{,\lambda} ) + \lambda_{,i}\, \lambda_{,j} \, ( \ddot{u}^\ast + w_{,\lambda\lambda} ), \hat{v}]\\
    + \lambda_{, ij} \, \E_{,u\lambda}(u^\ast + \xi_k \, v_k + w, \lambda; \hat{v}) + \lambda_{, i} \, \E_{,uu\lambda}[u^\ast + \xi_k \, v_k + w, \lambda; v_j + w_{,j} + \lambda_{,j} \, (\dot{u}^\ast + w_{,\lambda}), \hat{v}]\\
    + \lambda_{,i} \, \lambda_{,j} \, \E_{,u\lambda\lambda}(u^\ast + \xi_k \, v_k + w, \lambda; \hat{v})= 0.
  \end{gathered}
\end{equation}
At the critical point
% \begin{equation*}
%   \begin{gathered}[b]
%     \E_{,uuu}(u_0, \lambda_0; v_i + \lambda_{,i} \, \mathring{u}_0, v_j + \lambda_{,j} \, \mathring{u}_0, \hat{v}) + \lambda_{,j} \, \E_{,uu\lambda}(u_0, \lambda_0; v_i + \lambda_{,i} \, \mathring{u}_0, \hat{v})\\
%     + \E_{,uu}(u_0, \lambda_0; w_{ij} + \lambda_{,i} \, w_j + \lambda_{,j} \, w_i + \lambda_{,ij} \, \mathring{u}_0 + w_{,\lambda} + \lambda_{,i} \, \lambda_{,j} \, \ddot{u}_0, \hat{v})\\
%     + \lambda_{, ij} \, \E_{,u\lambda}(u_0, \lambda_0; \hat{v}) + \lambda_{, i} \, \E_{,uu\lambda}(u_0, \lambda_0; v_j + \lambda_{,j} \, \mathring{u}_0, \hat{v}) + \lambda_{,i} \, \lambda_{,j} \, \E_{,u\lambda\lambda}(u_0, \lambda_0; \hat{v}) = 0
%   \end{gathered}
% \end{equation*}

% \begin{equation*}
%   \begin{gathered}[b]
%     \E_{,uuu}(u_0, \lambda_0; v_i , v_j, \hat{v}) + \E_{,uu}(u_0, \lambda_0; w_{ij}, \hat{v})\\
%     +\lambda_{,i} \bigl[\E_{,uuu}(u_0, \lambda_0; v_j , \mathring{u}_0, \hat{v}) + \E_{,uu\lambda}[u_0, \lambda_0; v_j, \hat{v}]\bigr]\\
%     +\lambda_{,j} \bigl[\E_{,uuu}(u_0, \lambda_0; v_i , \mathring{u}_0, \hat{v}) + \E_{,uu\lambda}(u_0, \lambda_0; v_i, \hat{v})\bigr]\\
%     +\lambda_{,ij} \bigl[ \E_{,uu}(u_0, \lambda_0;  \mathring{u}_0, \hat{v}) + \E_{,u\lambda}(u_0, \lambda_0; \hat{v}) \bigr]\\
%     +\lambda_{,i} \lambda_{,j}\bigl[ \E_{,uuu}(u_0, \lambda_0; \mathring{u}_0 , \mathring{u}_0, \hat{v}) + 2\E_{,uu\lambda}(u_0, \lambda_0; \mathring{u}_0, \hat{v}) + \E_{,u\lambda\lambda}(u_0, \lambda_0; \hat{v}) + \E_{,uu}(u_0, \lambda_0; \ddot{u}_0, \hat{v}) \bigr] = 0
%   \end{gathered}
% \end{equation*}

% \begin{equation*}
%   \begin{gathered}[b]
%     \E_3(\lambda_0; v_i , v_j, \hat{v}) + \E_2(\lambda_0; w_{ij}, \hat{v}) + \lambda_{,i} \, \dot{\E}_2(\lambda_0; v_j, \hat{v}) + \lambda_{,j} \, \dot{\E}_2(\lambda_0; v_i, \hat{v})\\
%     +\lambda_{,ij} \, \dot{\E}_1(\lambda_0; \hat{v}) + \lambda_{,i} \, \lambda_{,j} \, \ddot{\E}_1(\lambda_0; \hat{v}) = 0
%   \end{gathered}
% \end{equation*}

\begin{equation}
    \E_3(\lambda_0; v_i , v_j, \hat{v}) + \lambda_{,i} \, \dot{\E}_2(\lambda_0; v_j, \hat{v}) + \lambda_{,j} \, \dot{\E}_2(\lambda_0; v_i, \hat{v}) = 0.
\end{equation}
Testing with $v_k \in V$, the above equation reads
\begin{equation}
  \E_3(\lambda_0; v_i , v_j, v_k) + \lambda_{,i} \dot{\E}_2(\lambda_0; v_j, v_k) + \lambda_{,j} \dot{\E}_2(\lambda_0; v_i, v_k) = 0,
\end{equation}
or, with Eqs.~\eqref{eq:20220524135619} and \eqref{eq:20220524135643}
\begin{equation}
  \label{eq:20220902125031}
  E_{ijk} +  F_{jk} \frac{\partial\lambda}{\partial\xi_i} \biggr\rvert_{\xi_1 = \cdots = \xi_m = 0} + F_{ik} \frac{\partial\lambda}{\partial\xi_j} \biggr\rvert_{\xi_1 = \cdots = \xi_m = 0} = 0.
\end{equation}

In order to evaluate the second order partial derivatives of $\lambda$, Eq.~\eqref{eq:20220901125230} should be further
differentiated with respect to $\xi_k$. This leads to extremely tedious derivations, that will not be pursued here.

To close this paragraph, a parametrization of the bifurcated branch is introduced. This branch is a curve
$(u, \lambda) \in \reals ^ {m + 1}$, which is parametrized by $\eta$: $[u(\eta), \lambda(\eta)]$, with $u(0) = u_0$ and
$\lambda(0) = \lambda_0$; primed quantities denoting derivatives with respect to $\eta$, we introduce
\begin{equation}
  \order[1]{\xi_i} = \xi_i'(0), \quad
  \order[2]{\xi_i} = \xi_i''(0), \quad \ldots, \quad
  \order[1]{\lambda} = \lambda'(0), \quad \ldots
\end{equation}
and first observe that
\begin{equation}
  \order[1]{\lambda} = \order[1]{\xi_i} \, \lambda_{,i}(\xi_1 = 0, \ldots, \xi_ m = 0)
\end{equation}

Multiplying both sides of Eq.~\eqref{eq:20220902125031} by $\order[1]{\xi_i} \order[1]{\xi_j}$ therefore results in the
following identity
\begin{equation}
  \begin{aligned}[b]
    0 &= E_{ijk} \, \order[1]{\xi_i} \, \order[1]{\xi_j} +  F_{jk} \, \order[1]{\xi_i} \, \order[1]{\xi_j} \, \lambda_{, i}(0, \ldots, 0) + F_{ik} \order[1]{\xi_i} \, \order[1]{\xi_j} \, \lambda_{, j}(0, \ldots, 0)\\
    &= E_{ijk} \order[1]{\xi_i} \order[1]{\xi_j} +  F_{jk} \order[1]{\lambda} \order[1]{\xi_j} + F_{ik} \order[1]{\xi_i} \order[1]{\lambda}
  \end{aligned}
\end{equation}
and, rearranging
\begin{equation}
  E_{ijk} \, \order[1]{\xi_j} \, \order[1]{\xi_k} +  2 \order[1]{\lambda} \, F_{ij}  \, \order[1]{\xi_j} = 0,
\end{equation}
to be compared with Eq.~\eqref{eq:20220524135036}. We now turn to $w$
\begin{equation}
  w'(\eta) = w_{,i} \, \xi_i' + w_{,\lambda} \, \lambda'
  \quad \text{and} \quad
  w''(\eta) = w_{,ij} \, \xi_i' \, \xi_j' + 2w_{,i\lambda} \, \xi_i' \, \lambda' + w_{,i} \, \xi_i'' + w_{,\lambda\lambda} \, \lambda^{'2} + w_{,\lambda} \, \lambda''
\end{equation}
and, at $\eta = 0$
\begin{equation}
  w'(0) = 0 \quad \text{and} \quad w''(0) = \order[1]{\xi_i} \, \order[1]{\xi_j} \, w_{ij}  + 2 \order[1]{\lambda} \, \order[1]{\xi_i} \, w_i
\end{equation}
and we get the Taylor expansion of the bifurcated branch as $\eta \to 0$
\begin{equation}
  u(\eta) = u^\ast[\lambda(\eta)] + \order[1]{\xi_i} \, v_i + \tfrac{1}{2} \bigl( \order[2]{\xi_i} \, v_i + \order[1]{\xi_i} \, \order[1]{\xi_j} \, w_{ij}  + 2\order[1]{\lambda} \, \order[1]{\xi_i} \, w_i\bigr) + o(\eta^2),
\end{equation}
to be compared with Eq.~\eqref{eq:20220524134613}.

\subsection{Elimination of}
\label{sec:20221102134138}

We again turn to Eq.~\eqref{eq:20220901120544}; \(\epsilon\) is now expressed as an implicit function of \(\xi_1\), \dots,
\(\xi_m\) and \(\lambda\). In this paragraph, $\hat{v}$ denotes an arbitrary element of $V$.

\paragraph{Derivative of Eq.~\eqref{eq:20220901120544} with respect to \(\xi_i\)}
\begin{equation}
  \label{eq:20221102205016}
  \EE_{,uu}(u^\ast + \xi_k \, v_k + w, \lambda, \epsilon; v_i + w_{,i} + \epsilon_{,i} \, w_{,\epsilon}, \hat{v}) + \epsilon_{,i} \, \EE_{,u\epsilon}(u^\ast + \xi_k \, v_k + w, \lambda, \epsilon; \hat{v}) = 0.
\end{equation}
At the critical point
\begin{equation}
  \E_{,uu}(u_0, \lambda_0; v_i + \epsilon_{,i} \, w_\epsilon, \hat{v}) + \epsilon_{,i} \, \EE_{,u\epsilon}(u_0, \lambda_0, 0; \hat{v}) = 0,
\end{equation}
and the first term vanishes. We assume that \(\EE_{,u\epsilon}(u_0, \lambda_0, 0)\) is not uniformly null. Then
\(\epsilon_{,i}(0, \ldots, 0, \lambda_0) = 0\).

\paragraph{Derivative of Eq.~\eqref{eq:20220901120544} with respect to \(\lambda\)}
\begin{equation}
  \label{eq:20221103054732}
  \begin{aligned}[b]
    \EE_{,uu}(u + \xi_k \, v_k + w, \lambda, \epsilon; \dot{u}^\ast + w_{,\lambda} + \epsilon_{,\lambda} \, w_{,\epsilon}, \hat{v})
    + \EE_{,u\lambda}(u + \xi_k \, v_k + w, \lambda, \epsilon; \hat{v})&\\
    + \epsilon_{,\lambda} \, \EE_{,u\epsilon}(u + \xi_k \, v_k + w, \lambda, \epsilon; \hat{v}) &= 0
  \end{aligned}
\end{equation}
At the critical point
% \begin{equation}
%   \E_{,uu}(u_0, \lambda_0; \dot{u}_0 + \epsilon_{,\lambda} \, w_\epsilon, \hat{v}) + \E_{,u\lambda}(u_0, \lambda_0; \hat{v}) + \epsilon_{,\lambda} \, \EE_{,u\epsilon}(u_0, \lambda_0; \hat{v}) = 0
% \end{equation}

% \begin{equation}
%   \dot{\E}_1(\lambda_0; \hat{v}) + \epsilon_{,\lambda} \bigl[ \E_2(\lambda_0; w_\epsilon, \hat{v}) + \EE_{,u\epsilon}(u_0, \lambda_0; \hat{v}) \bigr] = 0
% \end{equation}
\begin{equation}
  \epsilon_{,\lambda} \EE_{,u\epsilon}(u_0, \lambda_0; \hat{v}) = 0,
\end{equation}
which delivers \(\epsilon_{,\lambda}(0, \ldots, 0, \lambda_0) = 0\).

\paragraph{Derivative of Eq.~\eqref{eq:20221102205016} with respect to \(\xi_j\)}
\begin{equation}
  \label{eq:20221109062643}
  \begin{aligned}[b]
    \EE_{,uuu}(u^\ast + \xi_k \, v_k + w, \lambda, \epsilon; v_i + w_{,i} + \epsilon_{,i} \,  w_{,\epsilon}, v_j + w_{,j} + \epsilon_{,j} \, w_{,\epsilon}, \hat{v})&\\
    + \epsilon_{,j} \, \EE_{,uu\epsilon}(u^\ast + \xi_k \, v_k + w, \lambda, \epsilon; v_i + w_{,i} + w_{,\epsilon} \, \epsilon_{,i}, \hat{v})&\\
    + \EE_{,uu}(u^\ast + \xi_k \, v_k + w, \lambda, \epsilon; w_{,ij} + \epsilon_{,j} \, w_{,i\epsilon} + \epsilon_{,ij} \, w_{,\epsilon} + \epsilon_{,i} \, w_{,j\epsilon} + \epsilon_{,i} \, \epsilon_{,j} \, w_{,\epsilon\epsilon} , \hat{v})&\\
    + \epsilon_{,ij} \, \EE_{,u\epsilon}(u^\ast + \xi_k \, v_k + w, \lambda, \epsilon; \hat{v})&\\
    + \epsilon_{,i} \, \EE_{,uu\epsilon}(u^\ast + \xi_k \, v_k + w, \lambda, \epsilon; v_j + w_{,j} + \epsilon_{,j} \, w_{,\epsilon}, \hat{v})&\\
    + \epsilon_{,i} \, \epsilon_{,j} \, \EE_{,u\epsilon\epsilon}(u^\ast + \xi_k \, v_k + w, \lambda, \epsilon; \hat{v}) &= 0.
  \end{aligned}
\end{equation}
At the critical point, the above equation reduces to Eq.~\eqref{eq:20221116151527}.

\paragraph{Derivative of Eq.~\eqref{eq:20221102205016} with respect to \(\lambda\)}
\begin{equation}
  \label{eq:20221116061057}
  \begin{aligned}[b]
    % \EE_{,uu}(u^\ast + \xi_k \, v_k + w, \lambda, \epsilon; v_i + w_{,i} + \epsilon_{,i} \, w_{,\epsilon}, \hat{v})
    \EE_{,uuu}(u^\ast + \xi_k \, v_k + w, \lambda, \epsilon; v_i + w_{,i} + \epsilon_{,i} \, w_{,\epsilon}, \dot{u}^\ast + w_{,\lambda} + \epsilon_{,\lambda} \, w_{,\epsilon}, \hat{v})&\\
    + \EE_{,uu\lambda}(u^\ast + \xi_k \, v_k + w, \lambda, \epsilon; v_i + w_{,i} + \epsilon_{,i} \, w_{,\epsilon}, \hat{v})&\\
    + \epsilon_{,\lambda} \, \EE_{,uu\epsilon}(u^\ast + \xi_k \, v_k + w, \lambda, \epsilon; v_i + w_{,i} + \epsilon_{,i} \, w_{,\epsilon}, \hat{v})&\\
    + \EE_{,uu}(u^\ast + \xi_k \, v_k + w, \lambda, \epsilon; w_{,i\lambda} + \epsilon_{,\lambda} \, w_{,i\epsilon} + \epsilon_{,i\lambda} \, w_{,\epsilon} + \epsilon_{,i} \, w_{,\lambda\epsilon} + \epsilon_{,i} \, \epsilon_{,\lambda} \, w_{,\epsilon\epsilon}, \hat{v}) &\\
    % + \epsilon_{,i} \, \EE_{,u\epsilon}(u^\ast + \xi_k \, v_k + w, \lambda, \epsilon; \hat{v})
    + \epsilon_{,i\lambda} \, \EE_{,u\epsilon}(u^\ast + \xi_k \, v_k + w, \lambda, \epsilon; \hat{v}) &\\
    + \epsilon_{,i} \, \EE_{,uu\epsilon}(u^\ast + \xi_k \, v_k + w, \lambda, \epsilon; \dot{u}^\ast + w_{,\lambda} + \epsilon_{,\lambda} \, w_{,\epsilon}, \hat{v}) &\\
    + \epsilon_{,i} \, \EE_{,u\lambda\epsilon}(u^\ast + \xi_k \, v_k + w, \lambda, \epsilon; \hat{v}) &\\
    + \epsilon_{,i} \, \epsilon_{,\lambda} \, \EE_{,u\epsilon\epsilon}(u^\ast + \xi_k \, v_k + w, \lambda, \epsilon; \hat{v}) &= 0.
  \end{aligned}
\end{equation}
At the critical point, the above equation reduces to Eq.~\eqref{eq:20221116151638}.
% \begin{equation}
%   \begin{aligned}[b]
%     \E_{,uuu}(u_0, \lambda_0; v_i, \dot{u}_0, \hat{v})  + \E_{,uu\lambda}(u_0, \lambda_0; v_i, \hat{v}) + \E_{,uu}(u_0, \lambda_0; w_i + \epsilon_{,i\lambda} \, w_{\epsilon}, \hat{v}) &\\
%     + \epsilon_{,i\lambda} \, \EE_{,u\epsilon}(u_0, \lambda_0, 0; \hat{v}) &= 0\\
%   \end{aligned}
% \end{equation}

\paragraph{Derivative of Eq.~\eqref{eq:20221103054732} with respect to \(\lambda\)}
\begin{equation}
  \label{eq:20221116070255}
  \begin{aligned}[b]
    \EE_{,uuu}(u + \xi_k \, v_k + w, \lambda, \epsilon; \dot{u}^\ast + w_{,\lambda} + \epsilon_{,\lambda} \, w_{,\epsilon}, \dot{u}^\ast + w_{,\lambda} + \epsilon_{,\lambda} \, w_{,\epsilon}, \hat{v}) &\\
    + 2\EE_{,uu\lambda}(u + \xi_k \, v_k + w, \lambda, \epsilon; \dot{u}^\ast + w_{,\lambda} + \epsilon_{,\lambda} \, w_{,\epsilon}, \hat{v}) &\\
    + 2\epsilon_{,\lambda} \, \EE_{,uu\epsilon}(u + \xi_k \, v_k + w, \lambda, \epsilon; \dot{u}^\ast + w_{,\lambda} + \epsilon_{,\lambda} \, w_{,\epsilon}, \hat{v}) &\\
    + \EE_{,uu}(u + \xi_k \, v_k + w, \lambda, \epsilon; \ddot{u}^\ast + w_{,\lambda\lambda} + 2\epsilon_{,\lambda} \, w_{,\lambda\epsilon} + \epsilon_{,\lambda\lambda} \, w_{,\epsilon} + \epsilon_{,\lambda}^2 \, w_{,\epsilon\epsilon}, \hat{v}) &\\
    + \EE_{,u\lambda\lambda}(u + \xi_k \, v_k + w, \lambda, \epsilon; \hat{v})&\\
    + 2\epsilon_{,\lambda} \, \EE_{,u\lambda\epsilon}(u + \xi_k \, v_k + w, \lambda, \epsilon; \hat{v}) &\\
    + \epsilon_{,\lambda\lambda} \, \EE_{,u\epsilon}(u + \xi_k \, v_k + w, \lambda, \epsilon; \hat{v}) &\\
    + \epsilon_{,\lambda}^2 \, \EE_{,u\epsilon\epsilon}(u + \xi_k \, v_k + w, \lambda, \epsilon; \hat{v}) &= 0.
  \end{aligned}
\end{equation}
At the critical point
\begin{equation*}
  \begin{aligned}[b]
    \E_{,uuu}(u_0, \lambda_0; \dot{u}_0, \dot{u}_0, \hat{v}) + 2\E_{,uu\lambda}(u_0, \lambda_0; \dot{u}_0, \hat{v}) + \E_{,uu}(u_0, \lambda_0; \ddot{u}_0 + \epsilon_{,\lambda\lambda} \, w_\epsilon + \epsilon_{,\lambda}^2 \, w_{\epsilon\epsilon}, \hat{v}) &\\
    + \E_{,u\lambda\lambda}(u_0, \lambda_0; \hat{v}) + \epsilon_{,\lambda\lambda} \, \EE_{,u\epsilon}(u_0, \lambda_0, 0; \hat{v}) &= 0
  \end{aligned}
\end{equation*}
\begin{equation*}
  \underbrace{\ddot{\E}_1(\lambda_0; \hat{v})}_{=0}
  + \underbrace{\E_2(\lambda_0; \epsilon_{,\lambda\lambda} \, w_\epsilon + \epsilon_{,\lambda}^2 \, w_{\epsilon\epsilon}, \hat{v})}_{=0}
  + \epsilon_{,\lambda\lambda} \, \EE_{,u\epsilon}(u_0, \lambda_0, 0; \hat{v}) = 0
\end{equation*}
\begin{equation}
  \epsilon_{,\lambda\lambda} \, \EE_{,u\epsilon}(u_0, \lambda_0, 0; \hat{v}) = 0,
\end{equation}
which delivers \(\epsilon_{,\lambda\lambda}(0, \ldots, 0, \lambda_0) = 0\).

\paragraph{Derivative of Eq.~\eqref{eq:20221109062643} with respect to \(\xi_k\)}

\begin{equation}
  \begin{aligned}[b]
    \EE_{,uuuu}(u^\ast + \xi_l \, v_l + w, \lambda, \epsilon; v_i + w_{,i} + \epsilon_{,i} \,  w_{,\epsilon}, v_j + w_{,j} + \epsilon_{,j} \, w_{,\epsilon}, v_k + w_{,k} + \epsilon_{,k} \, w_{,\epsilon}, \hat{v})&\\
    + \epsilon_{,k} \, \EE_{,uuu\epsilon}(u^\ast + \xi_l \, v_l + w, \lambda, \epsilon; v_i + w_{,i} + \epsilon_{,i} \,  w_{,\epsilon}, v_j + w_{,j} + \epsilon_{,j} \, w_{,\epsilon}, \hat{v})&\\
    + \EE_{,uuu}(u^\ast + \xi_l \, v_l + w, \lambda, \epsilon; w_{,ik} + \epsilon_{,k} \, w_{,i\epsilon} + \epsilon_{,ik} \,  w_{,\epsilon} + \epsilon_{,i} \,  w_{,k\epsilon} + \epsilon_{,i} \, \epsilon_{,k}\,  w_{,\epsilon\epsilon}, v_j + w_{,j} + \epsilon_{,j} \, w_{,\epsilon}, \hat{v})&\\
    + \EE_{,uuu}(u^\ast + \xi_l \, v_l + w, \lambda, \epsilon; v_i + w_{,i} + \epsilon_{,i} \,  w_{,\epsilon}, w_{,jk} + \epsilon_{,k} \, w_{,j\epsilon} + \epsilon_{,jk} \,  w_{,\epsilon} + \epsilon_{,j} \,  w_{,k\epsilon} + \epsilon_{,j} \, \epsilon_{,k}\,  w_{,\epsilon\epsilon}, \hat{v})&\\
    + \epsilon_{,jk} \, \EE_{,uu\epsilon}(u^\ast + \xi_l \, v_l + w, \lambda, \epsilon; v_i + w_{,i} + w_{,\epsilon} \, \epsilon_{,i}, \hat{v})&\\
    + \epsilon_{,j} \, \EE_{,uuu\epsilon}(u^\ast + \xi_l \, v_l + w, \lambda, \epsilon; v_i + w_{,i} + w_{,\epsilon} \, \epsilon_{,i},  v_k + w_{,k} + \epsilon_{,k} \, w_{,\epsilon}, \hat{v})&\\
    + \epsilon_{,j} \, \epsilon_{,k} \, \EE_{,uu\epsilon\epsilon}(u^\ast + \xi_l \, v_l + w, \lambda, \epsilon; v_i + w_{,i} + w_{,\epsilon} \, \epsilon_{,i}, \hat{v})&\\
    + \epsilon_{,j} \, \EE_{,uu\epsilon}(u^\ast + \xi_l \, v_l + w, \lambda, \epsilon; w_{,ik} + \epsilon_{,k} \, w_{,i\epsilon} + \epsilon_{,ik} \,  w_{,\epsilon} + \epsilon_{,i} \,  w_{,k\epsilon} + \epsilon_{,i} \, \epsilon_{,k}\,  w_{,\epsilon\epsilon}, \hat{v})&\\
    + \EE_{,uuu}(u^\ast + \xi_l \, v_l + w, \lambda, \epsilon; w_{,ij} + \epsilon_{,j} \, w_{,i\epsilon} + \epsilon_{,ij} \, w_{,\epsilon} + \epsilon_{,i} \, w_{,j\epsilon} + \epsilon_{,i} \, \epsilon_{,j} \, w_{,\epsilon\epsilon},  v_k + w_{,k} + \epsilon_{,k} \, w_{,\epsilon}, \hat{v})&\\
    + \epsilon_{,k} \, \EE_{,uu\epsilon}(u^\ast + \xi_l \, v_l + w, \lambda, \epsilon; w_{,ij} + \epsilon_{,j} \, w_{,i\epsilon} + \epsilon_{,ij} \, w_{,\epsilon} + \epsilon_{,i} \, w_{,j\epsilon} + \epsilon_{,i} \, \epsilon_{,j} \, w_{,\epsilon\epsilon} , \hat{v})&\\
    + \EE_{,uu}(u^\ast + \xi_l \, v_l + w, \lambda, \epsilon; w_{,ijk} + \epsilon_{,k} \, w_{,ij\epsilon}+ \epsilon_{,jk} \, w_{,i\epsilon} + \epsilon_{,j} \, w_{,ik\epsilon} + \epsilon_{,j} \, \epsilon_{,k} \, w_{,i\epsilon\epsilon} + \epsilon_{,ijk} \, w_{,\epsilon} &\\
    + \epsilon_{,ij} \, w_{,k\epsilon} + \epsilon_{,ij} \, \epsilon_{,k} \, w_{,\epsilon\epsilon} + \epsilon_{,ik} \, w_{,j\epsilon} + \epsilon_{,i} \, w_{,jk\epsilon} + \epsilon_{,i} \, \epsilon_{,k} \, w_{,j\epsilon\epsilon} &\\+ \epsilon_{,ik} \, \epsilon_{,j} \, w_{,\epsilon\epsilon} + \epsilon_{,i} \, \epsilon_{,jk} \, w_{,\epsilon\epsilon} + \epsilon_{,i} \, \epsilon_{,j} \, w_{,k\epsilon\epsilon} + \epsilon_{,i} \, \epsilon_{,j} \, \epsilon_{,k} \, w_{,\epsilon\epsilon\epsilon}, \hat{v})&\\
    + \epsilon_{,ijk} \, \EE_{,u\epsilon}(u^\ast + \xi_l \, v_l + w, \lambda, \epsilon; \hat{v})&\\
    + \epsilon_{,ij} \, \EE_{,uu\epsilon}(u^\ast + \xi_l \, v_l + w, \lambda, \epsilon; v_k + w_{,k} + \epsilon_{,k} \, w_{,\epsilon}, \hat{v})&\\
    + \epsilon_{,ij} \, \epsilon_{,k} \, \EE_{,u\epsilon\epsilon}(u^\ast + \xi_l \, v_l + w, \lambda, \epsilon; \hat{v})&\\
    + \epsilon_{,ik} \, \EE_{,uu\epsilon}(u^\ast + \xi_l \, v_l + w, \lambda, \epsilon; v_j + w_{,j} + \epsilon_{,j} \, w_{,\epsilon}, \hat{v})&\\
    + \epsilon_{,i} \, \EE_{,uuu\epsilon}(u^\ast + \xi_l \, v_l + w, \lambda, \epsilon; v_j + w_{,j} + \epsilon_{,j} \, w_{,\epsilon}, v_k + w_{,k} + \epsilon_{,k} \, w_{,\epsilon}, \hat{v})&\\
    + \epsilon_{,i} \, \epsilon_{,k} \, \EE_{,uu\epsilon\epsilon}(u^\ast + \xi_l \, v_l + w, \lambda, \epsilon; v_j + w_{,j} + \epsilon_{,j} \, w_{,\epsilon}, \hat{v})&\\
    + \epsilon_{,i} \, \EE_{,uu\epsilon}(u^\ast + \xi_k \, v_k + w, \lambda, \epsilon; w_{,jk} + \epsilon_{,k} \, w_{,j\epsilon} + \epsilon_{,jk} \,  w_{,\epsilon} + \epsilon_{,j} \,  w_{,k\epsilon} + \epsilon_{,j} \, \epsilon_{,k}\,  w_{,\epsilon\epsilon}, \hat{v}, \hat{v})&\\
    + \epsilon_{,ik} \, \epsilon_{,j} \, \EE_{,u\epsilon\epsilon}(u^\ast + \xi_l \, v_l + w, \lambda, \epsilon; \hat{v}) &\\
    + \epsilon_{,i} \, \epsilon_{,jk} \, \EE_{,u\epsilon\epsilon}(u^\ast + \xi_l \, v_l + w, \lambda, \epsilon; \hat{v}) &\\
    + \epsilon_{,i} \, \epsilon_{,j} \, \EE_{,uu\epsilon\epsilon}(u^\ast + \xi_l \, v_l + w, \lambda, \epsilon;  v_k + w_{,k} + \epsilon_{,k} \, w_{,\epsilon}, \hat{v}) &\\
    + \epsilon_{,i} \, \epsilon_{,j} \, \epsilon_{,k} \, \EE_{,u\epsilon\epsilon\epsilon}(u^\ast + \xi_l \, v_l + w, \lambda, \epsilon; \hat{v}) &= 0
  \end{aligned}
\end{equation}
At the critical point, the above equation reduces to Eq.~\eqref{eq:20221116152151}.
% \begin{equation}
%   \begin{aligned}[b]
%     \E_{,uuuu}(u_0, \lambda_0; v_i, v_j, v_k, \hat{v}) + \E_{,uuu}(u_0, \lambda_0; w_{ik} + \epsilon_{,ik} \,  w_{,\epsilon}, v_j, \hat{v}) + \E_{,uuu}(u_0, \lambda_0; v_i, w_{jk} + \epsilon_{,jk} \,  w_{,\epsilon}, \hat{v})&\\
%     + \epsilon_{,jk} \, \EE_{,uu\epsilon}(u_0, \lambda_0, 0; v_i, \hat{v}) + \E_{,uuu}(u_0, \lambda_0; w_{ij} + \epsilon_{,ij} \, w_{,\epsilon},  v_k, \hat{v}) + \E_{,uu}(u_0, \lambda_0; \ldots, \hat{v}) &\\
%     + \epsilon_{,ijk} \, \EE_{,u\epsilon}(u_0, \lambda_0, 0; \hat{v}) + \epsilon_{,ij} \, \EE_{,uu\epsilon}(u_0, \lambda_0, 0; v_k, \hat{v}) + \epsilon_{,ik} \, \EE_{,uu\epsilon}(u_0, \lambda_0, 0; v_j, \hat{v}) &= 0
%   \end{aligned}
% \end{equation}

\paragraph{Derivative of Eq.~\eqref{eq:20221109062643} with respect to \(\lambda\)}

\begin{equation}
  \begin{aligned}[b]
    %\EE_{,uuu}(u^\ast + \xi_k \, v_k + w, \lambda, \epsilon; v_i + w_{,i} + \epsilon_{,i} \,  w_{,\epsilon}, v_j + w_{,j} + \epsilon_{,j} \, w_{,\epsilon}, \hat{v})
    \EE_{,uuuu}(u^\ast + \xi_k \, v_k + w, \lambda, \epsilon; \dot{u}^\ast + w_{,\lambda} + \epsilon_{,\lambda} \, w_{,\epsilon}, v_i + w_{,i} + \epsilon_{,i} \,  w_{,\epsilon}, v_j + w_{,j} + \epsilon_{,j} \, w_{,\epsilon}, \hat{v}) &\\
    + \EE_{,uuu\lambda}(u^\ast + \xi_k \, v_k + w, \lambda, \epsilon; v_i + w_{,i} + \epsilon_{,i} \,  w_{,\epsilon}, v_j + w_{,j} + \epsilon_{,j} \, w_{,\epsilon}, \hat{v}) + \epsilon_{,\lambda} \ldots &\\
    + \EE_{,uuu}(u^\ast + \xi_k \, v_k + w, \lambda, \epsilon; w_{,i\lambda} + \epsilon_{,\lambda} \, w_{,i\epsilon} + \epsilon_{,i\lambda} \,  w_{,\epsilon}  + \epsilon_{,i} \,  w_{,\lambda\epsilon} + \epsilon_{,i} \, \epsilon_{,\lambda} \, w_{,\epsilon\epsilon}, v_j + w_{,j} + \epsilon_{,j} \, w_{,\epsilon}, \hat{v})\\
    + \EE_{,uuu}(u^\ast + \xi_k \, v_k + w, \lambda, \epsilon; v_i + w_{,i} + \epsilon_{,i} \, w_{,\epsilon}, w_{,j\lambda} + \epsilon_{,\lambda} \, w_{,j\epsilon} + \epsilon_{,j\lambda} \,  w_{,\epsilon}  + \epsilon_{,j} \,  w_{,\lambda\epsilon} + \epsilon_{,j} \, \epsilon_{,\lambda} \, w_{,\epsilon\epsilon}, \hat{v})\\
    % + \epsilon_{,j} \, \EE_{,uu\epsilon}(u^\ast + \xi_k \, v_k + w, \lambda, \epsilon; v_i + w_{,i} + w_{,\epsilon} \, \epsilon_{,i}, \hat{v})
    + \epsilon_{,j\lambda} \, \EE_{,uu\epsilon}(u^\ast + \xi_k \, v_k + w, \lambda, \epsilon; v_i + w_{,i} + w_{,\epsilon} \, \epsilon_{,i}, \hat{v}) + \epsilon_{,j} \ldots &\\
    % + \EE_{,uu}(u^\ast + \xi_k \, v_k + w, \lambda, \epsilon; w_{,ij} + \epsilon_{,j} \, w_{,i\epsilon} + \epsilon_{,ij} \, w_{,\epsilon} + \epsilon_{,i} \, w_{,j\epsilon} + \epsilon_{,i} \, \epsilon_{,j} \, w_{,\epsilon\epsilon} , \hat{v})
    + \EE_{,uuu}(u^\ast + \xi_k \, v_k + w, \lambda, \epsilon; \dot{u}^\ast + w_{,\lambda} + \epsilon_{,\lambda} \, w_{,\epsilon}, w_{,ij} + \epsilon_{,j} \, w_{,i\epsilon} + \epsilon_{,ij} \, w_{,\epsilon} + \epsilon_{,i} \, w_{,j\epsilon} + \epsilon_{,i} \, \epsilon_{,j} \, w_{,\epsilon\epsilon} , \hat{v}) &\\
    + \EE_{,uu\lambda}(u^\ast + \xi_k \, v_k + w, \lambda, \epsilon; w_{,ij} + \epsilon_{,j} \, w_{,i\epsilon} + \epsilon_{,ij} \, w_{,\epsilon} + \epsilon_{,i} \, w_{,j\epsilon} + \epsilon_{,i} \, \epsilon_{,j} \, w_{,\epsilon\epsilon} , \hat{v}) &\\
    + \epsilon_{,\lambda} \, \ldots + \EE_{,uu}(u^\ast + \xi_k \, v_k + w, \lambda, \epsilon; \ldots, \hat{v}) &\\
    % + \epsilon_{,ij} \, \EE_{,u\epsilon}(u^\ast + \xi_k \, v_k + w, \lambda, \epsilon; \hat{v})
    + \epsilon_{,ij\lambda} \, \EE_{,u\epsilon}(u^\ast + \xi_k \, v_k + w, \lambda, \epsilon; \hat{v}) + \epsilon_{,ij} \, \EE_{,uu\epsilon}(u^\ast + \xi_k \, v_k + w, \lambda, \epsilon; \dot{u}^\ast + w_{,\lambda} + \epsilon_{,\lambda} \, w_{,\epsilon}, \hat{v}) &\\
    + \epsilon_{,ij} \, \EE_{,u\lambda\epsilon}(u^\ast + \xi_k \, v_k + w, \lambda, \epsilon; \hat{v}) + \epsilon_{,ij} \, \epsilon_{,\lambda} \ldots &\\
    % + \epsilon_{,i} \, \EE_{,uu\epsilon}(u^\ast + \xi_k \, v_k + w, \lambda, \epsilon; v_j + w_{,j} + \epsilon_{,j} \, w_{,\epsilon}, \hat{v})
    + \epsilon_{,i\lambda} \, \EE_{,uu\epsilon}(u^\ast + \xi_k \, v_k + w, \lambda, \epsilon; v_j + w_{,j} + \epsilon_{,j} \, w_{,\epsilon}, \hat{v}) + \epsilon_{,i} \ldots &\\
    % + \epsilon_{,i} \, \epsilon_{,j} \, \EE_{,u\epsilon\epsilon}(u^\ast + \xi_k \, v_k + w, \lambda, \epsilon; \hat{v})
    + \epsilon_{,i\lambda} \, \epsilon_{,j} \ldots + \epsilon_{,i} \, \epsilon_{,j\lambda} \ldots + \epsilon_{,i} \, \epsilon_{,j} \ldots &= 0
  \end{aligned}
\end{equation}
At the critical point, the above equation reduces to Eq.~\eqref{eq:20221116152457}.
% \begin{equation}
%   \begin{aligned}[b]
%     \E_{,uuuu}(u_0, \lambda_0; \dot{u}_0, v_i, v_j, \hat{v}) + \E_{,uuu\lambda}(u_0, \lambda_0; v_i, v_j, \hat{v}) &\\
%     + \E_{,uuu}(u_0, \lambda_0; w_i + \epsilon_{,i\lambda} \,  w_\epsilon, v_j, \hat{v}) + \E_{,uuu}(u_0, \lambda_0; v_i, w_j + \epsilon_{,j\lambda} \,  w_\epsilon, \hat{v}) &\\
%     + \E_{,uuu}(u_0, \lambda_0; \dot{u}_0, w_{ij} + \epsilon_{,ij} \, w_\epsilon , \hat{v}) + \E_{,uu\lambda}(u_0, \lambda_0; w_{ij} + \epsilon_{,ij} \, w_\epsilon, \hat{v}) &\\
%     + \epsilon_{,ij\lambda} \, \EE_{,u\epsilon}(u_0, \lambda_0, 0; \hat{v}) + \epsilon_{,ij} \, \EE_{,uu\epsilon}(u_0, \lambda_0, 0; \dot{u}_0, \hat{v}) + \epsilon_{,ij} \, \EE_{,u\lambda\epsilon}(u_0, \lambda_0, 0; \hat{v}) &\\
%     + \epsilon_{,i\lambda} \, \EE_{,uu\epsilon}(u_0, \lambda_0, 0; v_j, \hat{v}) + \epsilon_{,j\lambda} \, \EE_{,uu\epsilon}(u_0, \lambda_0, 0; v_i, \hat{v}) &= 0
%   \end{aligned}
% \end{equation}

\paragraph{Derivative of Eq.~\eqref{eq:20221116061057} with respect to \(\lambda\)}
\begin{equation}
  \begin{aligned}[b]
    % \EE_{,uuu}(u^\ast + \xi_k \, v_k + w, \lambda, \epsilon; v_i + w_{,i} + \epsilon_{,i} \, w_{,\epsilon}, \dot{u}^\ast + w_{,\lambda} + \epsilon_{,\lambda} \, w_{,\epsilon}, \hat{v})
    \EE_{,uuuu}(u^\ast + \xi_k \, v_k + w, \lambda, \epsilon; v_i + w_{,i} + \epsilon_{,i} \, w_{,\epsilon}, \dot{u}^\ast + w_{,\lambda} + \epsilon_{,\lambda} \, w_{,\epsilon}, \dot{u}^\ast + w_{,\lambda} + \epsilon_{,\lambda} \, w_{,\epsilon}, \hat{v}) &\\
    + \EE_{,uuu\lambda}(u^\ast + \xi_k \, v_k + w, \lambda, \epsilon; v_i + w_{,i} + \epsilon_{,i} \, w_{,\epsilon}, \dot{u}^\ast + w_{,\lambda} + \epsilon_{,\lambda} \, w_{,\epsilon}, \hat{v}) &\\
    + \epsilon_{,\lambda} \, \EE_{,uuu\epsilon}(u^\ast + \xi_k \, v_k + w, \lambda, \epsilon; v_i + w_{,i} + \epsilon_{,i} \, w_{,\epsilon}, \dot{u}^\ast + w_{,\lambda} + \epsilon_{,\lambda} \, w_{,\epsilon}, \hat{v}) &\\
    + \EE_{,uuu}(u^\ast + \xi_k \, v_k + w, \lambda, \epsilon; w_{,i\lambda} + \epsilon_{,\lambda} \, w_{,i\epsilon} + \epsilon_{,i\lambda} \, w_{,\epsilon} + \epsilon_{,i} \, w_{,\lambda\epsilon} + \epsilon_{,i} \, \epsilon_{,\lambda} \, w_{,\epsilon}, \dot{u}^\ast + w_{,\lambda} + \epsilon_{,\lambda} \, w_{,\epsilon}, \hat{v}) &\\
    + \EE_{,uuu}(u^\ast + \xi_k \, v_k + w, \lambda, \epsilon; v_i + w_{,i} + \epsilon_{,i} \, w_{,\epsilon}, \ddot{u}^\ast + w_{,\lambda\lambda} + \epsilon_{,\lambda} \, w_{,\lambda\epsilon} + \epsilon_{,\lambda\lambda} \, w_{, \epsilon} + \epsilon_{,\lambda} \, w_{,\lambda\epsilon} + \epsilon_{,\lambda}^2 \, w_{,\epsilon\epsilon}, \hat{v}) &\\
    % + \EE_{,uu\lambda}(u^\ast + \xi_k \, v_k + w, \lambda, \epsilon; v_i + w_{,i} + \epsilon_{,i} \, w_{,\epsilon}, \hat{v})
    + \EE_{,uuu\lambda}(u^\ast + \xi_k \, v_k + w, \lambda, \epsilon; \dot{u}^\ast + w_{,\lambda} + \epsilon_{,\lambda} \, w_{,\epsilon}, v_i + w_{,i} + \epsilon_{,i} \, w_{,\epsilon}, \hat{v}) &\\
    + \EE_{,uu\lambda\lambda}(u^\ast + \xi_k \, v_k + w, \lambda, \epsilon; v_i + w_{,i} + \epsilon_{,i} \, w_{,\epsilon}, \hat{v}) &\\
    + \epsilon_{,\lambda} \, \EE_{,uu\lambda\epsilon}(u^\ast + \xi_k \, v_k + w, \lambda, \epsilon; v_i + w_{,i} + \epsilon_{,i} \, w_{,\epsilon}, \hat{v}) &\\
    + \EE_{,uu\lambda}(u^\ast + \xi_k \, v_k + w, \lambda, \epsilon; v_i + w_{,i\lambda} + \epsilon_{,\lambda} \, w_{,i\epsilon} + \epsilon_{,i\lambda} \, w_{,\epsilon} + \epsilon_{,i} \, w_{,\lambda\epsilon}  + \epsilon_{,i} \, \epsilon_{,\lambda} \, w_{,\epsilon\epsilon}, \hat{v}) &\\
    % + \epsilon_{,\lambda} \, \EE_{,uu\epsilon}(u^\ast + \xi_k \, v_k + w, \lambda, \epsilon; v_i + w_{,i} + \epsilon_{,i} \, w_{,\epsilon}, \hat{v})
    + \epsilon_{,\lambda\lambda} \, \EE_{,uu\epsilon}(u^\ast + \xi_k \, v_k + w, \lambda, \epsilon; v_i + w_{,i} + \epsilon_{,i} \, w_{,\epsilon}, \hat{v}) &\\
    + \epsilon_{,\lambda} \ldots &\\
    % + \EE_{,uu}(u^\ast + \xi_k \, v_k + w, \lambda, \epsilon; w_{,i\lambda} + \epsilon_{,\lambda} \, w_{,i\epsilon} + \epsilon_{,i\lambda} \, w_{,\epsilon} + \epsilon_{,i} \, w_{,\lambda\epsilon} + \epsilon_{,i} \, \epsilon_{,\lambda} \, w_{,\epsilon\epsilon}, \hat{v})
    + \EE_{,uuu}(u^\ast + \xi_k \, v_k + w, \lambda, \epsilon; \dot{u}^\ast + w_{,\lambda} + \epsilon_{,\lambda} \, w_{,\epsilon}, w_{,i\lambda} + \epsilon_{,\lambda} \, w_{,i\epsilon} + \epsilon_{,i\lambda} \, w_{,\epsilon} + \epsilon_{,i} \, w_{,\lambda\epsilon} + \epsilon_{,i} \, \epsilon_{,\lambda} \, w_{,\epsilon\epsilon}, \hat{v}) &\\
    + \EE_{,uu\lambda}(u^\ast + \xi_k \, v_k + w, \lambda, \epsilon; w_{,i\lambda} + \epsilon_{,\lambda} \, w_{,i\epsilon} + \epsilon_{,i\lambda} \, w_{,\epsilon} + \epsilon_{,i} \, w_{,\lambda\epsilon} + \epsilon_{,i} \, \epsilon_{,\lambda} \, w_{,\epsilon\epsilon}, \hat{v}) &\\
    + \epsilon_{,\lambda} \ldots &\\
    + \EE_{,uu}(u^\ast + \xi_k \, v_k + w, \lambda, \epsilon;\ldots, \hat{v})
    % + \epsilon_{,i\lambda} \, \EE_{,u\epsilon}(u^\ast + \xi_k \, v_k + w, \lambda, \epsilon; \hat{v})
    + \epsilon_{,i\lambda\lambda} \, \EE_{,u\epsilon}(u^\ast + \xi_k \, v_k + w, \lambda, \epsilon; \hat{v}) &\\
    + \epsilon_{,i\lambda} \, \EE_{,uu\epsilon}(u^\ast + \xi_k \, v_k + w, \lambda, \epsilon; \dot{u}^\ast + w_{,\lambda} + \epsilon_{,\lambda} \, w_{,\epsilon}, \hat{v}) &\\
    + \epsilon_{,i\lambda} \, \EE_{,u\lambda\epsilon}(u^\ast + \xi_k \, v_k + w, \lambda, \epsilon; \hat{v}) &\\
    + \epsilon_{,\lambda} \ldots &\\
    % + \epsilon_{,i} \, \EE_{,uu\epsilon}(u^\ast + \xi_k \, v_k + w, \lambda, \epsilon; \dot{u}^\ast + w_{,\lambda} + \epsilon_{,\lambda} \, w_{,\epsilon}, \hat{v})
    + \epsilon_{,i\lambda} \, \EE_{,uu\epsilon}(u^\ast + \xi_k \, v_k + w, \lambda, \epsilon; \dot{u}^\ast + w_{,\lambda} + \epsilon_{,\lambda} \, w_{,\epsilon}, \hat{v}) &\\
    + \epsilon_{,i} \ldots &\\
    % + \epsilon_{,i} \, \EE_{,u\lambda\epsilon}(u^\ast + \xi_k \, v_k + w, \lambda, \epsilon; \hat{v})
    + \epsilon_{,i\lambda} \, \EE_{,u\lambda\epsilon}(u^\ast + \xi_k \, v_k + w, \lambda, \epsilon; \hat{v}) &\\
    + \epsilon_{,i} \ldots &\\
    % + \epsilon_{,i} \, \epsilon_{,\lambda} \, \EE_{,u\epsilon\epsilon}(u^\ast + \xi_k \, v_k + w, \lambda, \epsilon; \hat{v})
    + \epsilon_{,i\lambda} \, \epsilon_{,\lambda} \ldots &\\
    + \epsilon_{,i} \ldots
  \end{aligned}
\end{equation}
At the critical point, the above equation reduces to Eq.~\eqref{eq:20221116152647}.

\paragraph{Derivative of Eq.~\eqref{eq:20221116070255} with respect to \(\lambda\)}
\begin{equation}
  \begin{aligned}[b]
    % \EE_{,uuu}(u + \xi_k \, v_k + w, \lambda, \epsilon; \dot{u}^\ast + w_{,\lambda} + \epsilon_{,\lambda} \, w_{,\epsilon}, \dot{u}^\ast + w_{,\lambda} + \epsilon_{,\lambda} \, w_{,\epsilon}, \hat{v})
    \EE_{,uuuu}(u + \xi_k \, v_k + w, \lambda, \epsilon; \dot{u}^\ast + w_{,\lambda} + \epsilon_{,\lambda} \, w_{,\epsilon}, \dot{u}^\ast + w_{,\lambda} + \epsilon_{,\lambda} \, w_{,\epsilon}, \dot{u}^\ast + w_{,\lambda} + \epsilon_{,\lambda} \, w_{,\epsilon}, \hat{v}) &\\
    + \EE_{,uuu\lambda}(u + \xi_k \, v_k + w, \lambda, \epsilon; \dot{u}^\ast + w_{,\lambda} + \epsilon_{,\lambda} \, w_{,\epsilon}, \dot{u}^\ast + w_{,\lambda} + \epsilon_{,\lambda} \, w_{,\epsilon}, \hat{v}) &\\
    + \epsilon_{,\lambda} \ldots &\\
    + 2\EE_{,uuu}(u + \xi_k \, v_k + w, \lambda, \epsilon; \ddot{u}^\ast + w_{,\lambda\lambda} + \epsilon_{,\lambda} w_{,\lambda\epsilon} + \epsilon_{,\lambda\lambda} \, w_{,\epsilon} + \epsilon_{,\lambda} \ldots, \dot{u}^\ast + w_{,\lambda} + \epsilon_{,\lambda} \, w_{,\epsilon}, \hat{v}) &\\
    % + 2\EE_{,uu\lambda}(u + \xi_k \, v_k + w, \lambda, \epsilon; \dot{u}^\ast + w_{,\lambda} + \epsilon_{,\lambda} \, w_{,\epsilon}, \hat{v})
    + 2\EE_{,uuu\lambda}(u + \xi_k \, v_k + w, \lambda, \epsilon; \dot{u}^\ast + w_{,\lambda} + \epsilon_{,\lambda} \, w_{,\epsilon}, \dot{u}^\ast + w_{,\lambda} + \epsilon_{,\lambda} \, w_{,\epsilon}, \hat{v}) &\\
    + 2\EE_{,uu\lambda\lambda}(u + \xi_k \, v_k + w, \lambda, \epsilon; \dot{u}^\ast + w_{,\lambda} + \epsilon_{,\lambda} \, w_{,\epsilon}, \hat{v}) &\\
    + 2\epsilon_{,\lambda} \EE_{,uu\lambda\epsilon}(\ldots) &\\
     + 2\EE_{,uu\lambda}(u + \xi_k \, v_k + w, \lambda, \epsilon; \ddot{u}^\ast + w_{,\lambda\lambda} + \epsilon_{,\lambda} \, w_{,\lambda\epsilon} + \epsilon_{,\lambda\lambda} \, w_{,\epsilon} + \epsilon_{,\lambda} \ldots, \hat{v}) &\\
     % + 2\epsilon_{,\lambda} \, \EE_{,uu\epsilon}(u + \xi_k \, v_k + w, \lambda, \epsilon; \dot{u}^\ast + w_{,\lambda} + \epsilon_{,\lambda} \, w_{,\epsilon}, \hat{v})
    + 2\epsilon_{,\lambda\lambda} \ldots + 2\epsilon_{,\lambda} \ldots &\\
    % + \EE_{,uu}(u + \xi_k \, v_k + w, \lambda, \epsilon; \ddot{u}^\ast + w_{,\lambda\lambda} + 2\epsilon_{,\lambda} \, w_{,\lambda\epsilon} + \epsilon_{,\lambda\lambda} \, w_{,\epsilon} + \epsilon_{,\lambda}^2 \, w_{,\epsilon\epsilon}, \hat{v})
    + \EE_{,uuu}(u + \xi_k \, v_k + w, \lambda, \epsilon; \dot{u}^\ast + w_{,\lambda} + \epsilon_{,\lambda} \, w_{,\epsilon}, \ddot{u}^\ast + w_{,\lambda\lambda} + 2\epsilon_{,\lambda} \, w_{,\lambda\epsilon} + \epsilon_{,\lambda\lambda} \, w_{,\epsilon} + \epsilon_{,\lambda}^2 \, w_{,\epsilon\epsilon}, \hat{v}) &\\
    + \EE_{,uu\lambda}(u + \xi_k \, v_k + w, \lambda, \epsilon; \ddot{u}^\ast + w_{,\lambda\lambda} + 2\epsilon_{,\lambda} \, w_{,\lambda\epsilon} + \epsilon_{,\lambda\lambda} \, w_{,\epsilon} + \epsilon_{,\lambda}^2 \, w_{,\epsilon\epsilon}, \hat{v}) &\\
    + \epsilon_{,\lambda} \ldots &\\
     + \EE_{,uu}(u + \xi_k \, v_k + w, \lambda, \epsilon; \ldots, \hat{v}) &\\
     % + \EE_{,u\lambda\lambda}(u + \xi_k \, v_k + w, \lambda, \epsilon; \hat{v})
    + \EE_{,uu\lambda\lambda}(u + \xi_k \, v_k + w, \lambda, \epsilon; \dot{u}^\ast + w_{,\lambda} + \epsilon_{,\lambda} \, w_{,\epsilon}, \hat{v}) &\\
    + \EE_{,u\lambda\lambda\lambda}(u + \xi_k \, v_k + w, \lambda, \epsilon; \hat{v}) &\\
    + \epsilon_{,\lambda} \ldots &\\
    % + 2\epsilon_{,\lambda} \, \EE_{,u\lambda\epsilon}(u + \xi_k \, v_k + w, \lambda, \epsilon; \hat{v})
    +2\epsilon_{,\lambda\lambda} \ldots + 2\epsilon_{,\lambda} \ldots &\\
    % + \epsilon_{,\lambda\lambda} \, \EE_{,u\epsilon}(u + \xi_k \, v_k + w, \lambda, \epsilon; \hat{v})
    + \epsilon_{,\lambda\lambda\lambda} \, \EE_{,u\epsilon}(u + \xi_k \, v_k + w, \lambda, \epsilon; \hat{v})
    + \epsilon_{,\lambda\lambda} \ldots &\\
    % + \epsilon_{,\lambda}^2 \, \EE_{,u\epsilon\epsilon}(u + \xi_k \, v_k + w, \lambda, \epsilon; \hat{v})
    + 2 \epsilon_{,\lambda} \, \epsilon_{,\lambda\lambda} \ldots
    + \epsilon_{,\lambda}^2 \ldots &= 0
  \end{aligned}
\end{equation}
At the critical point
% \begin{equation}
%   \begin{aligned}[b]
%     \E_{,uuuu}(u_0, \lambda_0; \dot{u}_0, \dot{u}_0, \dot{u}_0, \hat{v}) + \E_{,uuu\lambda}(u_0, \lambda_0; \dot{u}_0, \dot{u}_0, \hat{v}) + 2\E_{,uuu}(u_0, \lambda_0; \ddot{u}_0, \dot{u}_0, \hat{v}) &\\
%     %
%     + 2\E_{,uuu\lambda}(u_0, \lambda_0; \dot{u}_0, \dot{u}_0, \hat{v}) + 2\E_{,uu\lambda\lambda}(u_0, \lambda_0; \dot{u}_0, \hat{v}) + 2\E_{,uu\lambda}(u_0, \lambda_0; \ddot{u}_0, \hat{v}) &\\
%     %
%     + \epsilon_{,\lambda\lambda\lambda} \, \EE_{,u\epsilon}(u_0, \lambda_0, 0; \hat{v})
%     %
%     + \E_{,uuu}(u_0, \lambda_0; \dot{u}_0, \ddot{u}_0, \hat{v}) + \E_{,uu\lambda}(u_0, \lambda_0; \ddot{u}_0, \hat{v}) &\\
%      %
%     + \E_{,uu\lambda\lambda}(u_0, \lambda_0; \dot{u}_0, \hat{v}) + \E_{,u\lambda\lambda\lambda}(u_0, \lambda_0; \hat{v}) &= 0
%   \end{aligned}
% \end{equation}
\begin{equation}
  \begin{aligned}[b]
    \E_{,uuuu}(u_0, \lambda_0; \dot{u}_0, \dot{u}_0, \dot{u}_0, \hat{v})
    + 3\E_{,uuu\lambda}(u_0, \lambda_0; \dot{u}_0, \dot{u}_0, \hat{v})
    + 3\E_{,uu\lambda\lambda}(u_0, \lambda_0; \dot{u}_0, \hat{v}) &\\
    + \E_{,u\lambda\lambda\lambda}(u_0, \lambda_0; \hat{v})
    + 3\E_{,uuu}(u_0, \lambda_0; \ddot{u}_0, \dot{u}_0, \hat{v})
    + 3\E_{,uu\lambda}(u_0, \lambda_0; \ddot{u}_0, \hat{v}) &\\
    + \epsilon_{,\lambda\lambda\lambda} \, \EE_{,u\epsilon}(u_0, \lambda_0, 0; \hat{v}) &= 0
  \end{aligned}
\end{equation}
which reduces
\begin{equation}
  \dddot{\E}_1(\lambda; \hat{v}) + \epsilon_{,\lambda\lambda\lambda} \, \EE_{,u\epsilon}(u_0, \lambda_0, 0; \hat{v}) = 0,
\end{equation}
and \(\epsilon_{,\lambda\lambda\lambda}(0, \ldots, 0, \lambda_0) = 0\) since the first term vanishes.

\paragraph{Summary}

It has been found that
\begin{equation}
  \epsilon_{,i}(0, \ldots 0, \lambda_0) = 0, \quad  \epsilon_{,\lambda}(0, \ldots 0, \lambda_0) = 0, \quad \epsilon_{,\lambda\lambda}(0, \ldots 0, \lambda_0) = 0, \quad \epsilon_{,\lambda\lambda\lambda}(0, \ldots 0, \lambda_0) = 0,
\end{equation}
and, for all \(\hat{v} \in V\)
\begin{equation}
  \label{eq:20221116151527}
  \E_{,uuu}(u_0, \lambda_0; v_i, v_j, \hat{v}) + \epsilon_{,ij} \, \EE_{,u\epsilon}(u_0, \lambda_0, 0; \hat{v}) = 0,\\
\end{equation}
\begin{equation}
  \label{eq:20221116151638}
  \dot{\E}_2(\lambda_0; v_i, \hat{v}) + \epsilon_{,i\lambda} \, \EE_{,u\epsilon}(u_0, \lambda_0, 0; \hat{v}) = 0,\\
\end{equation}
\begin{equation}
  \label{eq:20221116152151}
  \begin{aligned}[b]
    \E_{,uuuu}(u_0, \lambda_0; v_i, v_j, v_k, \hat{v}) + \E_{,uuu}(u_0, \lambda_0; w_{ij} + \epsilon_{,ij} \, w_{,\epsilon},  v_k, \hat{v})&\\
    + \E_{,uuu}(u_0, \lambda_0; w_{ik} + \epsilon_{,ik} \,  w_{,\epsilon}, v_j, \hat{v}) + \E_{,uuu}(u_0, \lambda_0; v_i, w_{jk} + \epsilon_{,jk} \,  w_{,\epsilon}, \hat{v})&\\
    + \epsilon_{,ijk} \, \EE_{,u\epsilon}(u_0, \lambda_0, 0; \hat{v}) + \epsilon_{,ij} \, \EE_{,uu\epsilon}(u_0, \lambda_0, 0; v_k, \hat{v}) &\\
    + \epsilon_{,ik} \, \EE_{,uu\epsilon}(u_0, \lambda_0, 0; v_j, \hat{v}) + \epsilon_{,jk} \, \EE_{,uu\epsilon}(u_0, \lambda_0, 0; v_i, \hat{v}) &= 0,
  \end{aligned}
\end{equation}
\begin{equation}
  \label{eq:20221116152457}
  \begin{aligned}[b]
    \dot{\E}_3(\lambda_0; v_i, v_j, \hat{v}) + \dot{\E}_2(\lambda_0; w_{ij} + \epsilon_{,ij} \, w_\epsilon, \hat{v}) &\\
    + \E_{,uuu}(u_0, \lambda_0; w_i + \epsilon_{,i\lambda} \,  w_\epsilon, v_j, \hat{v}) + \E_{,uuu}(u_0, \lambda_0; v_i, w_j + \epsilon_{,j\lambda} \,  w_\epsilon, \hat{v}) &\\
    + \epsilon_{,ij\lambda} \, \EE_{,u\epsilon}(u_0, \lambda_0, 0; \hat{v}) + \epsilon_{,ij} \, \EE_{,uu\epsilon}(u_0, \lambda_0, 0; \dot{u}_0, \hat{v}) + \epsilon_{,ij} \, \EE_{,u\lambda\epsilon}(u_0, \lambda_0, 0; \hat{v}) &\\
    + \epsilon_{,i\lambda} \, \EE_{,uu\epsilon}(u_0, \lambda_0, 0; v_j, \hat{v}) + \epsilon_{,j\lambda} \, \EE_{,uu\epsilon}(u_0, \lambda_0, 0; v_i, \hat{v}) &= 0,
  \end{aligned}
\end{equation}
% \begin{equation}
%   \begin{aligned}[b]
%     \E_{,uuuu}(u_0, \lambda_0; v_i, \dot{u}_0, \dot{u}_0, \hat{v}) + \E_{,uuu\lambda}(u_0, \lambda_0; v_i, \dot{u}_0, \hat{v}) &\\
%     + \E_{,uuu}(u_0, \lambda_0; w_i + \epsilon_{,i\lambda} \, w_{\epsilon}, \dot{u}_0, \hat{v}) + \E_{,uuu}(u_0, \lambda_0; v_i, \ddot{u}_0 + \epsilon_{,\lambda\lambda} \, w_{\epsilon}, \hat{v}) &\\
%     %
%     + \E_{,uuu\lambda}(u_0, \lambda_0; \dot{u}_0, v_i, \hat{v})
%     + \E_{,uu\lambda\lambda}(u_0, \lambda_0; v_i, \hat{v}) &\\
%     + \E_{,uu\lambda}(u_0, \lambda_0; v_i + w_{i} + \epsilon_{,i\lambda} \, w_{\epsilon}, \hat{v})
%     %
%     + \epsilon_{,\lambda\lambda} \, \EE_{,uu\epsilon}(u_0, \lambda_0, 0; v_i, \hat{v}) &\\
%     %
%     + \epsilon_{,i\lambda\lambda} \, \EE_{,u\epsilon}(u_0, \lambda_0, 0; \hat{v})
%     + 2\epsilon_{,i\lambda} \, \EE_{,uu\epsilon}(u_0, \lambda_0, 0; \dot{u}_0, \hat{v})
%     + 2\epsilon_{,i\lambda} \, \EE_{,u\lambda\epsilon}(u_0, \lambda_0, 0; \hat{v}) &= 0.
%   \end{aligned}
% \end{equation}
\begin{equation}
  \label{eq:20221116152647}
  \begin{aligned}[b]
    \E_{,uuuu}(u_0, \lambda_0; v_i, \dot{u}_0, \dot{u}_0, \hat{v}) + 2\E_{,uuu\lambda}(u_0, \lambda_0; v_i, \dot{u}_0, \hat{v}) &\\
    + \E_{,uu\lambda\lambda}(u_0, \lambda_0; v_i, \hat{v}) + \E_{,uuu}(u_0, \lambda_0; w_i + \epsilon_{,i\lambda} \, w_{\epsilon}, \dot{u}_0, \hat{v}) &\\
    + \E_{,uuu}(u_0, \lambda_0; v_i, \ddot{u}_0, \hat{v}) + \E_{,uu\lambda}(u_0, \lambda_0; v_i + w_{i} + \epsilon_{,i\lambda} \, w_{\epsilon}, \hat{v}) &\\
    + \epsilon_{,i\lambda\lambda} \, \EE_{,u\epsilon}(u_0, \lambda_0, 0; \hat{v}) + 2\epsilon_{,i\lambda} \bigl[ \EE_{,uu\epsilon}(u_0, \lambda_0, 0; \dot{u}_0, \hat{v}) + \EE_{,u\lambda\epsilon}(u_0, \lambda_0, 0; \hat{v}) \bigr] &= 0.
  \end{aligned}
\end{equation}

\begin{equation*}
  \ddot{\E}_1(\lambda; \bullet) = \E_{,uuu}(u^\ast, \lambda; \dot{u}^\ast, \dot{u}^\ast, \bullet) + 2\E_{,uu\lambda}(u^\ast, \lambda; \dot{u}^\ast, \bullet) + \E_{,uu}(u^\ast, \lambda; \ddot{u}^\ast, \bullet) + \E_{,u\lambda\lambda}(u^\ast, \lambda; \bullet)
\end{equation*}


Introducing
\begin{equation}
  E'_i = \EE_{u\epsilon}(u_0, \lambda_0, 0; v_i)
\end{equation}
and testing Eq.~\eqref{eq:20221116151527} with \(v_k\) and Eq.~\eqref{eq:20221116151638} with \(v_j\), leads to the
following equations
\begin{equation}
  \label{eq:20221116153503}
  E_k' \, \epsilon_{,ij} + E_{ijk} = 0 \quad \text{and} \quad   E_j' \, \epsilon_{,i\lambda} + F_{ij} = 0.
\end{equation}

Testing Eq.~\eqref{eq:20221116152151} with \(v_l\) leads to
\begin{equation}
  \begin{aligned}[b]
     E_l' \, \epsilon_{,ijk} + E_{ijkl} + \bigl[ \E_3(\lambda_0; w_\epsilon, v_k, v_l) + E_{kl}' \bigr] \epsilon_{,ij} &\\
    + \bigl[ \E_3(\lambda_0;  w_\epsilon, v_j, v_l) + E_{jl}' \bigr] \epsilon_{,ik} + \bigl[ \E_3(\lambda_0; w_\epsilon, v_i, v_l) + E_{il}' \bigr] \epsilon_{,jk} &= 0\\
  \end{aligned}
\end{equation}

Testing Eq.~\eqref{eq:20221116152457} with \(v_k\) leads to
\begin{equation}
  \begin{aligned}[b]
    \epsilon_{,ij\lambda} \, \EE_{,u\epsilon}(u_0, \lambda_0, 0; v_k) + \dot{\E}_3(\lambda_0; v_i, v_j, v_k) + \dot{\E}_2(\lambda_0; w_{ij}, v_k) &\\
    + \E_3(\lambda_0; w_i, v_j, v_k) + \E_3(\lambda_0; v_i, w_j, v_k) &\\
    + \epsilon_{,ij} \bigl[ \dot{\E}_2(\lambda_0; w_\epsilon, v_k) + \EE_{,uu\epsilon}(u_0, \lambda_0, 0; \dot{u}_0, v_k) + \EE_{,u\lambda\epsilon}(u_0, \lambda_0, 0; v_k)\bigr] &\\
    + \epsilon_{,i\lambda} \bigl[ \E_3(\lambda_0; w_\epsilon, v_j, v_k) + \EE_{,uu\epsilon}(u_0, \lambda_0, 0; v_j, v_k) \bigr] &\\
    + \epsilon_{,j\lambda} \bigl[ \E_3(\lambda_0; w_\epsilon, v_i, v_k) + \EE_{,uu\epsilon}(u_0, \lambda_0, 0; v_i, v_k) \bigr] &= 0,
  \end{aligned}
\end{equation}


If \(m \neq 1\), Eq.~\eqref{eq:20221116153503}\textsubscript{2} can in general not be satisfied. We therefore consider
below the case of a single mode, \(m=1\). The following expressions are found
\begin{equation}
  \label{eq:20221116153503}
   \epsilon_{,11}(0, \lambda_0) = -\frac{E_{111}}{E_1'} \quad \text{and} \quad \epsilon_{,1\lambda}(0, \lambda_0) = -\frac{F_{11}}{E_1'}.
\end{equation}

For an asymmetric bifurcation, \(E_{111} \neq 0\) and the following expansion of \(\epsilon\) as a function of \(\xi_1\) and \(\lambda\) is obtained
\begin{equation}
  \epsilon(\xi_1, \lambda) = -\frac{E_{111}}{2E_1'}\xi_1^2-\frac{F_{11}}{E_1'}\xi_1\bigl( \lambda - \lambda_0 \bigr) + o\bigl[ \xi_1^2 + \bigl( \lambda - \lambda_0 \bigr)^2 \bigr].
\end{equation}

For a symmetric bifurcation, \(E_{111} = 0\) and \(\epsilon_{,11}(0, \lambda_0) = 0\). Equation~\eqref{eq:20221116152151} then reads
\begin{equation}
  \begin{aligned}[b]
    \E_{,uuuu}(u_0, \lambda_0; v_i, v_j, v_k, \hat{v}) + \E_{,uuu}(u_0, \lambda_0; w_{ij}, v_k, \hat{v}) + \E_{,uuu}(u_0, \lambda_0; w_{ik}, v_j, \hat{v}) &\\
    + \E_{,uuu}(u_0, \lambda_0; v_i, w_{jk}, \hat{v}) + \epsilon_{,ijk} \, \EE_{,u\epsilon}(u_0, \lambda_0, 0; \hat{v}) &= 0
  \end{aligned}
\end{equation}
which, upon testing with \(v_l\), reduces to [see Eq.~\eqref{eq:20221116155507}]
\begin{equation}
  E_{ijkl} + \epsilon_{,ijk} \, F_l' = 0
\end{equation}
and
\begin{equation}
  \epsilon_{,111} = - \frac{E_{1111}}{F_1'}.
\end{equation}



\begin{equation*}
  \dot{\E}_1(\lambda; \bullet) = \E_{,uu}(u^\ast, \lambda; \dot{u}^\ast, \bullet) + \E_{,u\lambda}(u^\ast, \lambda; \bullet)
\end{equation*}
\begin{equation*}
  \ddot{\E}_1(\lambda; \bullet) = \E_{,uuu}(u^\ast, \lambda; \dot{u}^\ast, \dot{u}^\ast, \bullet) + 2\E_{,uu\lambda}(u^\ast, \lambda; \dot{u}^\ast, \bullet) + \E_{,uu}(u^\ast, \lambda; \ddot{u}^\ast, \bullet) + \E_{,u\lambda\lambda}(u^\ast, \lambda; \bullet)
\end{equation*}
\begin{equation*}
  \begin{aligned}[b]
    \dddot{\E}_1(\lambda; \bullet) ={} & \E_{,uuuu}(u^\ast, \lambda; \dot{u}^\ast, \dot{u}^\ast, \dot{u}^\ast, \bullet) + 3\E_{,uuu\lambda}(u^\ast, \lambda; \dot{u}^\ast, \dot{u}^\ast, \bullet) + 3\E_{,uu\lambda\lambda}(u^\ast, \lambda; \dot{u}, \bullet) + \E_{,u\lambda\lambda\lambda}(u^\ast, \lambda; \bullet)\\
                           &+ 3\E_{,uuu}(u^\ast, \lambda; \dot{u}^\ast, \ddot{u}^\ast, \bullet) + 3\E_{,uu\lambda}(u^\ast, \lambda; \ddot{u}^\ast, \bullet) + \E_{,uu}(u^\ast, \lambda; \dddot{u}^\ast, \bullet)
  \end{aligned}
\end{equation*}

\appendix

\section{Some useful results from multilinear algebra}

\color{gray}

\subsection{Kernel of a bilinear, symmetric, positive form}

In this section, $\mathcal{B}$ denotes a bilinear, symmetric, positive form over the vector space $U$. Its kernel
$\ker \mathcal{B}$ is defined as follows
\begin{equation}
 \ker \mathcal{B}= \bigl\{ u \in U, \mathcal{B}(u, u) = 0 \bigr\} .
\end{equation}

\begin{theorem}
  The kernel of a bilinear, symmetric, positive form is a vector subspace.
\end{theorem}
\begin{proof}
  We must show that, for all $u, v \in\ker \mathcal{B}$ and $\alpha \in \reals$,
  $w = u + \alpha v \in \ker \mathcal{B}$, in other words, it must be shown that $\mathcal{B}(w, w) = 0$. From the
  bilinearity and symmetry of $\mathcal{B}$
 \begin{equation*}
   \mathcal{B}(w, w) = \mathcal{B}(u + \alpha v, u + \alpha v)
   = \mathcal{B}(u, u) + 2 \alpha \mathcal{B}(u, v) + \alpha^2 \mathcal{B}(v, v),
 \end{equation*}

 Since $u, v \in \ker\mathcal{B}$, the first and the last term vanish, and the above identity reduces to
 \begin{equation*}
   \mathcal{B}(w, w) = 2\alpha \mathcal{B}(u, v)
 \end{equation*}

 The bilinear form $\mathcal{B}$ is positive, therefore the left-hand side is positive, \emph{for all values of
   $\alpha \in \reals$}. The quantity $\mathcal{B}(u, v) = 0$ is necessarily null, and $\mathcal{B}(w, w) = 0$, which
 completes the proof.
\end{proof}

\begin{theorem}
  Let $\mathcal{B}$ be a bilinear, symmetric, positive form over the vector space $U$ and $u \in U$. Then
 \begin{equation*}
  u \in \ker\mathcal{B} \quad \text{iff} \quad \text{pour tout } v \in U, \mathcal{B}(u, v) = 0.
 \end{equation*}
\end{theorem}

\begin{proof}
  If for all $v \in U$, $\mathcal{B}(u, v) = 0$, then in particular $\mathcal{B}(u, u) = 0$ and
  $u \in \ker \mathcal{B}$.

  Converely, let $u \in \ker \mathcal{B}$, $v \in U$ et $\alpha \in \reals$. Similarly to the previous proof, we write
  that $\mathcal{B}(w, w) \geq 0$, with $w = \alpha u + v$
  \begin{equation*}
    \mathcal{B}(w, w) = \mathcal{B}(u, u) + 2 \alpha \mathcal{B}(u, v) +\mathcal{B}(v, v) = 2 \alpha \mathcal{B}(u, v) +\mathcal{B}(v, v) \geq 0,
  \end{equation*}
  ($\mathcal{B}(u, u) = 0$ since $u \in \ker \mathcal{B}$). The above expression is of degree 1 in $\alpha$, with a
  constant sign. Therefore the linear term in $\alpha$ must vanish: $\mathcal{B}(u, v) = 0$.
\end{proof}

\color{black}

\subsection{On trilinear, symmetric forms}

In this section, $\mathcal T$ denotes a trilinear, symmetric form over the vector space $U$.

\begin{theorem}
  \label{thr:20220802112835}
  Let $\mathcal T$ be a trilinear, symmetric form, such that
  \begin{equation}
    \label{eq:20220802111745}
    \mathcal T(u, u, u) = 0 \quad \text{for all} \quad u \in U.
  \end{equation}
  Then
  \begin{equation}
    \mathcal T(u, v, w) = 0 \quad \text{for all} \quad u, v, w \in U.
  \end{equation}
\end{theorem}
\begin{proof}
  The form $\mathcal T$ being trilinear and symmetric, we have, for all $u, v, w \in U$ and $\alpha, \beta \in \reals$
  \begin{multline*}
    \mathcal T(u + \alpha v + \beta w, u + \alpha v + \beta w, u + \alpha v + \beta w) = \mathcal T(u, u, u) + 3\alpha \mathcal T(u, u, v)\\
    + 3\beta \mathcal T(u, u, w) + 3\alpha^2 \mathcal T(u, u, v) + 6 \alpha \beta \mathcal T(u, v, w) + 3 \beta^2 \mathcal T(u, u, w)\\
    + \alpha^3 \mathcal T(v, v, v) + 3 \alpha^2 \beta \mathcal T(v, v, w) + 3 \alpha \beta^2 \mathcal T(v, w, w) + \beta^3 \mathcal T(w, w, w)
  \end{multline*}
  and, upon simplification using Eq.~\eqref{eq:20220802111745}
  \begin{multline}
    \label{eq:20220802112309}
    3\alpha \mathcal T(u, u, v) + 3\beta \mathcal T(u, u, w) + 3\alpha^2 \mathcal T(u, v, v) + 6 \alpha \beta \mathcal T(u, v, w)\\
    + 3 \beta^2 \mathcal T(u, w, w) + 3 \alpha^2 \beta \mathcal T(v, v, w) + 3 \alpha \beta^2 \mathcal T(v, w, w) = 0.
  \end{multline}
  In particular taking successively $\alpha = ±1$, $\beta = 0$ and $w = 0$ delivers
  \begin{equation*}
    ±3 \mathcal T(u, u, v) + 3 \mathcal T(u, u, v) = 0 \quad \text{for all} \quad u, v \in U,
  \end{equation*}
  from which it results that $\mathcal T(u, u, v) = 0$ for all $u, v \in U$. Plugging into Eq.~\eqref{eq:20220802112309}
  with $\alpha = \beta = 1$ results in: $\mathcal T(u, v, w) = 0$ for all $u, v, w \in U$.
\end{proof}

\section{Asymptotic expansions along a bifurcated branch}
\label{sec:20220905060440}

In this section, the asymptotic expansions along the bifurcated branch of the energy, its residual and its hessian are
derived.

\subsection{Principle of the derivation}
\label{sec:20220107121442}
% 02/06/2022 --- 099042106e938251657847daca64c8fcbaa833c3
%
% Validation des calculs de ce paragraphe

Introducing $\Lambda$ and $U$, which are functions of $\eta$ only,
\begin{align}
  \label{eq:20211112155446}
  \Lambda(\eta) & = \lambda(\eta) - \lambda_0 = \eta \order[1]\lambda + \tfrac{1}{2} \eta^2 \order[2]\lambda + \tfrac{1}{6} \eta^3 \order[3]\lambda + \cdots,\\
  \label{eq:20211112113028}
  U(\eta) & = u(\eta) - u^{\ast}[\lambda(\eta)] = \eta \order[1]u + \tfrac{1}{2} \eta^2 \order[2]u + \tfrac{1}{6} \eta^3 \order[3]u + \cdots,
\end{align}
the functional $\mathcal{F}(u, \lambda)$ is evaluated along the bifurcated branch, thus defining the function $f(\eta)$
\begin{equation*}
  f(\eta) = F\{ u^{\ast} [\lambda_0 + \Lambda(\eta)] + U(\eta), \lambda_0 + \Lambda(\eta) \}.
\end{equation*}

We seek the Taylor expansion of $f$ at $\eta = 0$, which requires the successive derivatives of $f$. To this end, it is
convenient to introduce the function $F$ defined as follows
\begin{equation*}
  F(\eta, \lambda) =\mathcal{F}[u^{\ast}(\lambda) + U(\eta), \lambda],
\end{equation*}
where $\lambda$ and $\eta$ are temporarily seen as independent variables. Since
$f(\eta) = F[\eta, \lambda_0 + \Lambda(\eta)]$, the following identities hold
\begin{gather*}
  f'(\eta) = \partial_{\eta} F + \Lambda' \partial_{\lambda} F,\\
  f''(\eta) = \partial_{\eta\eta}^2 F + 2\Lambda' \partial_{\eta\lambda}^2F + \Lambda'^2 \partial_{\lambda\lambda}^2 F + \Lambda'' \partial_{\lambda} F,\\
  \begin{aligned}[b]
    f'''(\eta) ={}
    & \partial_{\eta\eta\eta}^3 F + 3\Lambda' \partial_{\eta\eta\lambda}^3F + 3\Lambda'^2 \partial_{\eta\lambda\lambda}^3F + \lambda'^3 \partial_{\lambda\lambda\lambda}^3 F\\
    & + 3\Lambda'' \partial_{\eta\lambda}^2 F + 3\Lambda' \Lambda'' \partial_{\lambda \lambda}^2 F + \Lambda''' \partial_{\lambda} F,
  \end{aligned}\\
  \begin{aligned}[b]
    f''''(\eta) ={}
    & \partial_{\eta\eta\eta\eta}^4 F + 4\Lambda' \partial_{\eta\eta\eta\lambda}^4F + 6\Lambda'^2 \partial_{\eta\eta\lambda\lambda}^4F + 4\Lambda'^3 \partial_{\eta\lambda\lambda\lambda}^4F + \Lambda'^4 \partial_{\lambda\lambda\lambda\lambda}^4 F\\
    & + 6\Lambda'' \partial_{\eta\eta\lambda}^3 F + 12\Lambda' \Lambda'' \partial_{\eta\lambda\lambda}^3F + 6\Lambda'^2 \Lambda'' \partial_{\lambda\lambda\lambda}^3 F\\
    & + 4 \Lambda''' \partial_{\eta\lambda}^2 F + \bigl( 3\Lambda''^2 + 4 \Lambda' \Lambda''' \bigr) \partial_{\lambda\lambda}^2 F + \lambda'''' \partial_{\lambda}F,
  \end{aligned}
\end{gather*}
where $\Lambda$ and its derivatives are evaluated at $\eta$, whilie $F$ and its partial derivatives are evaluated at
$[\eta, \lambda_0 + \Lambda(\eta)]$. At $\eta = 0$, the above relations read
\begin{gather}
  \label{eq:20220107060454}
  f'(0) = \partial_{\eta} F + \order[1]\lambda \partial_{\lambda} F,\\
  \label{eq:20220107124311}
  f''(0) = \partial_{\eta\eta}^2 F + 2 \order[1]\lambda \partial_{\eta\lambda}^2 F + \bigl( \order[1]\lambda \bigr)^2 \partial_{\lambda\lambda}^2 F + \order[2]\lambda \partial_{\lambda} F,\\
  \label{eq:20220107060500}
  \begin{aligned}[b]
    f'''(0) ={}
    & \partial_{\eta\eta\eta}^3 F + 3 \order[1]\lambda \partial_{\eta\eta\lambda}^3 F + 3 \bigl( \order[1]\lambda \bigr)^2 \partial_{\eta\lambda\lambda}^3 F + \bigl( \order[1]\lambda \bigr)^3 \partial_{\lambda\lambda\lambda}^3 F\\
    & + 3 \order[2]\lambda \partial_{\eta\lambda}^2 F + 3 \order[1]\lambda \order[2]\lambda \partial_{\lambda\lambda}^2 F + \order[3]\lambda \partial_{\lambda} F,
  \end{aligned}\\
  \label{eq:20220602185935}
  \begin{aligned}[b]
    f''''(0) ={}
    & \partial_{\eta\eta\eta\eta}^4F + 4 \order[1]\lambda \partial_{\eta\eta\eta\lambda}^4 F + 6 \bigl( \order[1]\lambda \bigr)^2 \partial_{\eta\eta\lambda\lambda}^4 F + 4 \bigl( \order[1]\lambda \bigr)^3 \partial_{\eta\lambda\lambda\lambda}^4 F + \bigl( \order[1]\lambda \bigr)^4 \partial_{\lambda\lambda\lambda\lambda}^4 F\\
    & + 6 \order[2]\lambda \partial_{\eta\eta\lambda}^3 F + 12 \order[1]\lambda \order[2]\lambda \partial_{\eta\lambda\lambda}^3 F + 6 \bigl( \order[1]\lambda \bigr)^2 \order[2]\lambda \partial_{\lambda\lambda\lambda}^3 F\\
    & + 4 \order[3]\lambda \partial_{\eta\lambda}^2 F + \bigl(3 \bigl( \order[2]\lambda \bigr)^2 + 4 \order[1]\lambda \order[3]\lambda\bigr) \partial_{\lambda\lambda}^2 F + \lambda_4 \partial_{\lambda} F,
  \end{aligned}
\end{gather}
where $F$ and its partial derivatives are now evaluated at $(\eta = 0, \lambda = \lambda_0)$. The values of $f'(0)$,
$f''(0)$, \dots thus found are used in the remainder of Sec.~\ref{sec:20220905060440} for various choices of the
functional $\mathcal F$.

\subsection{Application to the residual}
\label{sec:20211112182000}
% 03/06/2022 --- b028b234970605720c9022c16c7fc3012997ced7
%
% Validation des calculs de ce paragraphe

In order to derive the Taylor expansion of the residual $\E_{,u}$, the method described in Sec.~\ref{sec:20220107121442}
is applied to
\begin{equation}
  \label{eq:20220107054629}
  f(\eta) = \E_{, u} [u(\eta), \lambda(\eta); \hat{u}]
  \quad \text{and} \quad
  F(\eta, \lambda) = \E_{, u}[u^{\ast}(\lambda) + U(\eta), \lambda; \hat{u}],
\end{equation}
the test function $\hat{u}$ being fixed. It is first observed that
$F(0, \lambda) = \E_{, u} [u^{\ast} (\lambda), \lambda; \hat{u}] = 0$, since $u^{\ast}(\lambda)$ is an equilibrium point
for all $\lambda$ close to $\lambda_0$. Upon derivation with respect to $\lambda$, we get
\begin{equation*}
  \frac{\partial^k F}{\partial \lambda^k}(0, \lambda) = 0 \quad \text{for all} \quad k \geq 0.
\end{equation*}
From the definition~\eqref{eq:20220107054629} of $F$, we find successively
\begin{equation*}
  \partial_{\eta}F(\eta, \lambda) = \E_{, u u}[u^{\ast}(\lambda) + U(\eta), \lambda; U'(\eta), \hat{u}],
\end{equation*}
\begin{equation*}
  \begin{aligned}[b]
    \partial_{\eta \eta}^2 F(\eta, \lambda) ={}
    & \E_{, uuu}[u^{\ast}(\lambda) + U(\eta), \lambda; U'(\eta), U'(\eta), \hat{u}]\\
    & + \E_{, uu} [u^{\ast}(\lambda) + U(\eta), \lambda; U''(\eta), \hat{u}],
  \end{aligned}
\end{equation*}
\begin{equation*}
  \begin{aligned}[b]
    \partial_{\eta\eta\eta}^3 F(\eta, \lambda) ={}
    & \E_{, uuuu}[u^{\ast}(\lambda) + U(\eta), \lambda; U'(\eta), U'(\eta), U'(\eta), \hat{u}]\\
    & + 3\E_{, u u u}[u^{\ast}(\lambda) + U(\eta), \lambda; U'(\eta), U''(\eta), \hat{u}]\\
    & + \E_{, uu}[u^{\ast}(\lambda) + U(\eta), \lambda; U'''(\eta), \hat{u}],
  \end{aligned}
\end{equation*}
and, at $\eta = 0$
\begin{gather*}
  \partial_{\eta}F(0, \lambda) = \E_2(\lambda; \order[1]u, \hat{u}),\\
  \partial_{\eta\eta}^2 F(0, \lambda) = \E_3(\lambda; \order[1]u, \order[1]u, \hat{u}) + \E_2(\lambda; \order[2]u, \hat{u}),\\
  \partial_{\eta\eta\eta}^3 F(0, \lambda) = \E_4(\lambda; \order[1]u, \order[1]u, \order[1]u, \hat{u}) + 3\E_3(\lambda; \order[1]u, \order[2]u, \hat{u}) + \E_2(\lambda; \order[3]u, \hat{u}).
\end{gather*}
Upon derivation with respect to $\lambda$, we find successively
\begin{gather*}
  \partial_{\eta\lambda}^2 F(0, \lambda) = \dot{\E}_2(\lambda; \order[1]u, \hat{u}),\\
  \partial_{\eta\lambda\lambda}^3 F(0, \lambda) = \ddot{\E}_2(\lambda; \order[1]u, \hat{u}),\\
  \partial_{\eta\eta\lambda}^3 F(0, \lambda) = \dot{\E}_3(\lambda; \order[1]u, \order[1]u, \hat{u}) + \dot{\E_2}(\lambda; \order[2]u, \hat{u}).
\end{gather*}

Upon insertion into Eqs.~\eqref{eq:20220107060454}--\eqref{eq:20220602185935}, we get the following expressions of the
sucessive derivatives of $f$ at $\eta=0$
\begin{gather*}
  f'(0) = \E_2(\lambda_0; \order[1]u, \hat{u}),\\
  f''(0) = \E_3(\lambda_0; \order[1]u, \order[1]u, \hat{u}) + \E_2(\lambda_0; \order[2]u, \hat{u}) + 2 \order[1]\lambda \dot{\E}_2(\lambda_0; \order[1]u, \hat{u}),\\
  \begin{aligned}[b]
    f'''(0) ={}
    & \E_4(\lambda_0; \order[1]u, \order[1]u, \order[1]u, \hat{u}) + 3\E_3(\lambda_0; \order[1]u, \order[2]u, \hat{u}) + \E_2(\lambda_0 ; \order[3]u, \hat{u})\\
    & + 3\order[1]\lambda \dot{\E}_3(\lambda_0; \order[1]u, \order[1]u, \hat{u}) + 3\order[1]\lambda \dot{\E}_2(\lambda_0; \order[2]u, \hat{u})\\
    & + 3 \bigl( \order[1]\lambda \bigr)^2 \ddot{\E}_2(\lambda_0; \order[1]u, \hat{u}) + 3 \order[2]\lambda \dot{\E}_2(\lambda_0; \order[1]u, \hat{u}),
  \end{aligned}
\end{gather*}
which finally delivers the following expansion of the residual
\begin{equation}
  \label{eq:20220107080901}
  \begin{gathered}[b]
    \E_{, u}[u(\eta), \lambda(\eta)] ={} \eta \E_2(\lambda_0; \order[1]u, \hat{u}) + \tfrac{1}{2} \eta^2 \bigl[\E_3(\lambda_0; \order[1]u, \order[1]u, \hat{u})  + \E_2(\lambda_0; \order[2]u, \hat{u})\\
    {} + 2 \order[1]\lambda \dot{\E}_2(\lambda_0; \order[1]u, \hat{u})\bigr] + \tfrac{1}{6} \eta^3 \bigl[ \E_4(\lambda_0; \order[1]u, \order[1]u, \order[1]u, \hat{u}) + 3\E_3(\lambda_0; \order[1]u, \order[2]u, \hat{u})\\
    {} + \E_2(\lambda_0; \order[3]u, \hat{u}) + 3\order[1]\lambda \dot{\E}_3(\lambda_0; \order[1]u, \order[1]u, \hat{u}) + 3\order[1]\lambda \dot{\E}_2(\lambda_0; \order[2]u, \hat{u})\\
    {} + 3 \bigl( \order[1]\lambda \bigr)^2 \ddot{\E}_2(\lambda_0; \order[1]u, \hat{u}) + 3 \order[2]\lambda \dot{\E}_2(\lambda_0 ; \order[1]u, \hat{u}) \bigr] + o(\eta^3),
  \end{gathered}
\end{equation}
up to third-order terms.

\subsection{Application to the energy}
\label{sec:20220525053434}
% 07/06/2022 --- dd1a4abf18cd94861d754bf3e19a54b8974bb2e8
%
% Relecture de tous les calculs de ce paragraphe

The method described in Sec.~\ref{sec:20220107121442} is applied to the energy difference between the fundamental and
bifurcated branches
\begin{equation}
  F(\eta, \lambda) = \E[u^{\ast}(\lambda) + U(\eta), \lambda] - \E[u^{\ast}(\lambda), \lambda]
  \quad \text{et} \quad
  f(\eta) = F [\eta, \lambda_0 + \Lambda(\eta)].
\end{equation}
Observing that $F(0, \lambda) = 0$ for all $\lambda$, we first get
\begin{equation*}
  \frac{\partial^k F}{\partial \lambda^k}(0, \lambda) = 0 \quad \text{for all} \quad k \geq 0,
\end{equation*}
while the partial derivatives of $F$ with respect to $\eta$ read
\begin{gather*}
  \partial_{\eta} F(\eta, \lambda) = \E_{, u}(U'),\\
  \partial_{\eta\eta}^2 F(\eta, \lambda) = \E_{, uu} (U', U') + \E_{, u} (U''),\\
  \partial_{\eta\eta\eta}^3 F(\eta, \lambda) = \E_{, uuu}(U', U', U') + 3\E_{, uu}(U', U'') + \E_{, u}(U'''),\\
  \begin{aligned}[b]
    \partial_{\eta\eta\eta\eta}^4 F ={}
    & \E_{, uuuu}(U', U', U', U') + 6\E_{,uuu}(U', U', U'')\\
    & + 3\E_{, uu}(U'', U'') + 4\E_{, uu}(U', U''') + \E_{, u}(U''''),
  \end{aligned}
\end{gather*}
where the partial derivatives of $\E$ are evaluated at $[u^{\ast}(\lambda) + U(\eta), \lambda]$, while the derivatives
of $U$ are evaluated at $\eta$. For $\eta = 0$, observing that $\E_{, u}[u^{\ast}(\lambda), \lambda] = 0$
\begin{gather*}
  \partial_{\eta} F(0, \lambda) = 0,\\
  \partial_{\eta\eta}^2 F(0, \lambda) =\E_2(\lambda ; \order[1]u, \order[1]u),\\
  \partial_{\eta\eta\eta}^3 F(0, \lambda) = \E_3(\lambda; \order[1]u, \order[1]u, \order[1]u) + 3\E_2(\lambda; \order[1]u, \order[2]u),\\
  \begin{aligned}[b]
    \partial_{\eta\eta\eta\eta}^4 F(\eta, \lambda) ={} & \E_4(\lambda; \order[1]u, \order[1]u, \order[1]u, \order[1]u) + 6\E_3(\lambda; \order[1]u, \order[1]u, \order[2]u)\\
    & + 3\E_2(\lambda; \order[2]u, \order[2]u) + 4\E_2(\lambda; \order[1]u, \order[3]u),
  \end{aligned}
\end{gather*}
and, upon derivation with respect to $\lambda$
\begin{equation*}
  \begin{gathered}
    \partial_{\eta\lambda}^2 F(0, \lambda) = 0,\\
    \partial_{\eta\eta\lambda}^3 F(0, \lambda) = \dot{\E}_2(\lambda; \order[1]u, \order[1]u),\\
    \partial_{\eta\lambda\lambda}^3 F(0, \lambda) = 0,\\
  \end{gathered}
  \qquad
  \begin{gathered}
    \partial_{\eta\eta\eta\lambda}^4 F(0, \lambda) = \dot{\E}_3(\lambda; \order[1]u, \order[1]u, \order[1]u) + 3\dot{\E}_2(\lambda; \order[1]u, \order[2]u),\\
    \partial_{\eta\eta\lambda\lambda}^4 F(0, \lambda) = \ddot{\E}_2(\lambda; \order[1]u, \order[1]u),\\
    \partial_{\eta\lambda\lambda\lambda}^4 F(0, \lambda) = 0
  \end{gathered}
\end{equation*}
and finally
\begin{gather}
  f'(0) = 0,\\
  f''(0) = \E_2(\lambda_0; \order[1]u, \order[1]u),\\
  f'''(0) =\E_3(\lambda_0; \order[1]u, \order[1]u, \order[1]u) + 3\E_2(\lambda_0; \order[1]u, \order[2]u) + 3\order[1]\lambda \dot{\E}_2(\lambda_0; \order[1]u, \order[1]u),\\
  \label{eq:20220905063614}
  \begin{aligned}[b]
    f''''(0) ={}
    & \E_4(\lambda_0; \order[1]u, \order[1]u, \order[1]u, \order[1]u) + 6\E_3(\lambda_0; \order[1]u, \order[1]u, \order[2]u) + 3\E_2(\lambda_0; \order[2]u, \order[2]u)\\
    & + 4\E_2(\lambda_0; \order[1]u, \order[3]u) + 4 \order[1]\lambda \dot{\E}_3(\lambda_0; \order[1]u, \order[1]u, \order[1]u) + 12 \order[1]\lambda \dot{\E}_2(\lambda_0; \order[1]u, \order[2]u)\\
    & + 6( \order[1]\lambda )^2 \ddot{\E}_2(\lambda_0; \order[1]u, \order[1]u) + 6\order[2]\lambda \dot{\E}_2(\lambda_0; \order[1]u, \order[1]u).
  \end{aligned}
\end{gather}

Since $\order[1]u \in V$, we have $\E_2(\lambda_0; \order[1]u, \order[k]u) = 0$ for $k = 1, 2, 3$. Therefore $f''(0)=0$
and, using Eq.~\eqref{eq:20220524133816}
\begin{equation}
  \label{eq:20220601055448}
  f'''(0) = \order[1]\lambda F_{ij} \order[1]{\xi_i} \order[1]{\xi_j},
\end{equation}

Turning now to $f''''(0)$, we plug the decompositions \eqref{eq:20220524133944} and \eqref{eq:20220524134613} of
$\order[1]u$ and $\order[2]u$ successively into each term of Eq.~\eqref{eq:20220905063614}.
\begin{equation*}
  \begin{aligned}[b]
    \E_3(\lambda_0; \order[1]u, \order[1]u, \order[2]u)
    ={} & \E_3(v_i, v_j, v_k) \order[1]{\xi_i} \order[1]{\xi_j} \order[2]{\xi_k} + \E_3(v_i, v_j, w_{k l}) \order[1]{\xi_i} \order[1]{\xi_j} \order[1]{\xi_k} \order[1]{\xi_l}\\
    & + 2\order[1]\lambda \E_3(v_i, v_j, w_k) \order[1]{\xi_i} \order[1]{\xi_j} \order[1]{\xi_k} \\
    ={} & \E_3(v_i, v_j, v_k) \order[1]{\xi_i} \order[1]{\xi_j} \order[2]{\xi_k} + \E_3(v_i, v_j, w_{k l}) \order[1]{\xi_i} \order[1]{\xi_j} \order[1]{\xi_k} \order[1]{\xi_l}\\
    & - 2\order[1]\lambda \E_2(w_{ij}, w_k) \order[1]{\xi_i} \order[1]{\xi_j} \order[1]{\xi_k}, \qquad \text{[using Eq.~\eqref{eq:20220519164523}]}\\
    ={} & \E_3(v_i, v_j, v_k) \order[1]{\xi_i} \order[1]{\xi_j} \order[2]{\xi_k} + \E_3(v_i, v_j, w_{kl}) \order[1]{\xi_i} \order[1]{\xi_j} \order[1]{\xi_k} \order[1]{\xi_l}\\
    & - 2\order[1]\lambda \E_2(w_{i}, w_{jk}) \order[1]{\xi_i} \order[1]{\xi_j} \order[1]{\xi_k}\\
    ={} & \E_3(v_i, v_j, v_k) \order[1]{\xi_i} \order[1]{\xi_j} \order[2]{\xi_k} + \E_3(v_i, v_j, w_{kl}) \order[1]{\xi_i} \order[1]{\xi_j} \order[1]{\xi_k} \order[1]{\xi_l}\\
    & + 2 \order[1]\lambda \dot{\E}_2(v_{i}, w_{jk}) \order[1]{\xi_i} \order[1]{\xi_j} \order[1]{\xi_k}. \qquad \text{[using Eq.~\eqref{eq:20220524134525}]}
  \end{aligned}
\end{equation*}
Then
\begin{equation*}
  \begin{aligned}[b]
    \E_2(\order[2]u, \order[2]u)
    ={} & \E_2(\order[2]{\xi_i} v_i + \order[1]{\xi_i} \order[1]{\xi_j} w_{i j} + 2\order[1]\lambda \order[1]{\xi_i} w_i, \order[2]{\xi_k} v_k + \order[1]{\xi_k} \order[1]{\xi_l} w_{k l} + 2\order[1]\lambda \order[1]{\xi_k} w_k)\\
    ={} & \E_2(\order[1]{\xi_i} \order[1]{\xi_j} w_{i j} + 2 \order[1]\lambda \order[1]{\xi_i} w_i, \order[1]{\xi_k} \order[1]{\xi_l} w_{k l} + 2 \order[1]\lambda \order[1]{\xi_k} w_k)\\
    ={} & \E_2(w_{i j}, w_{k l}) \order[1]{\xi_i} \order[1]{\xi_j} \order[1]{\xi_k} \order[1]{\xi_l} + 4 \order[1]\lambda \E_2(w_i, w_{j k}) \order[1]{\xi_i} \order[1]{\xi_j} \order[1]{\xi_k}\\
    & + 4 ( \order[1]\lambda )^2 \E_2(w_i, w_j) \order[1]{\xi_i} \order[1]{\xi_j}\\
    ={} & \E_2(w_{i j}, w_{k l}) \order[1]{\xi_i} \order[1]{\xi_j} \order[1]{\xi_k} \order[1]{\xi_l} + 4 \order[1]\lambda \E_2(w_i, w_{j k}) \order[1]{\xi_i} \order[1]{\xi_j} \order[1]{\xi_k}\\
    &+ 2 ( \order[1]\lambda )^2 \bigl[\E_2(w_i, w_j) + \E_2(w_j, w_i)\bigr] \order[1]{\xi_i} \order[1]{\xi_j}\\
    ={} & -\E_3(v_i, v_j, w_{k l}) \order[1]{\xi_i} \order[1]{\xi_j} \order[1]{\xi_k} \order[1]{\xi_l} - 4 \order[1]\lambda \dot{\E}_2 (v_i, w_{j k}) \order[1]{\xi_i} \order[1]{\xi_j} \order[1]{\xi_k}\\
    & - 2 ( \order[1]\lambda )^2 \bigl[\dot{\E}_2(v_i, w_j) + \dot{\E}_2(v_j, w_i)\bigr] \order[1]{\xi_i} \order[1]{\xi_j}
  \end{aligned}
\end{equation*}
finally
\begin{equation*}
  \begin{aligned}[b]
    \dot{\E}_2(\order[1]u, \order[2]u)
    ={} & \dot{\E}_2 (v_i, v_j) \order[1]{\xi_i} \order[2]{\xi_j} + \dot{\E}_2(v_i, w_{j k}) \order[1]{\xi_i} \order[1]{\xi_j} \order[1]{\xi_k} + 2\order[1]\lambda \dot{\E}_2(v_i, w_j) \order[1]{\xi_i} \order[1]{\xi_j}\\
    ={} & \dot{\E}_2(v_i, v_j) \order[1]{\xi_i} \order[2]{\xi_j} + \dot{\E}_2(v_i, w_{j k}) \order[1]{\xi_i} \order[1]{\xi_j} \order[1]{\xi_k} + \order[1]\lambda [\dot{\E}_2(v_i, w_j) + \dot{\E}_2(v_j, w_i)] \order[1]{\xi_i} \order[1]{\xi_j}.
  \end{aligned}
\end{equation*}
Gathering the above results
% \begin{equation*}
%   \begin{aligned}[b]
%     f''''(0) ={}
%     & \E_4(v_i, v_j, v_k, v_l) \order[1]{\xi_i} \order[1]{\xi_j} \order[1]{\xi_k} \order[1]{\xi_l}\\
%     & + 6\bigl[\E_3(v_i, v_j, v_k) \order[1]{\xi_i} \order[1]{\xi_j} \order[2]{\xi_k}
%       + \E_3(v_i, v_j, w_{kl}) \order[1]{\xi_i} \order[1]{\xi_j} \order[1]{\xi_k} \order[1]{\xi_l}
%       + 2 \order[1]\lambda \dot{\E}_2(v_{i}, w_{jk}) \order[1]{\xi_i} \order[1]{\xi_j} \order[1]{\xi_k}\bigr]\\
%     & -3\bigl\{ \E_3(v_i, v_j, w_{k l}) \order[1]{\xi_i} \order[1]{\xi_j} \order[1]{\xi_k} \order[1]{\xi_l} + 4 \order[1]\lambda \dot{\E}_2 (v_i, w_{j k}) \order[1]{\xi_i} \order[1]{\xi_j} \order[1]{\xi_k}\\
%     & + 2 ( \order[1]\lambda )^2 \bigl[\dot{\E}_2(v_i, w_j) + \dot{\E}_2(v_j, w_i)\bigr] \order[1]{\xi_i} \order[1]{\xi_j} \bigr\}\\
%     & + 4 \order[1]\lambda \dot{\E}_3(v_i, v_j, v_k) \order[1]{\xi_i} \order[1]{\xi_j} \order[1]{\xi_k}\\
%     & + 12 \order[1]\lambda \bigl\{ \dot{\E}_2(v_i, v_j) \order[1]{\xi_i} \order[2]{\xi_j} + \dot{\E}_2(v_i, w_{j k}) \order[1]{\xi_i} \order[1]{\xi_j} \order[1]{\xi_k} + \order[1]\lambda \bigl[\dot{\E}_2(v_i, w_j) + \dot{\E}_2(v_j, w_i)\bigr] \order[1]{\xi_i} \order[1]{\xi_j} \bigr\}\\
%     & + 6( \order[1]\lambda )^2 \ddot{\E}_2(v_i, v_j) \order[1]{\xi_i} \order[1]{\xi_j}\\
%     & + 6\order[2]\lambda \dot{\E}_2(v_i, v_j) \order[1]{\xi_i} \order[1]{\xi_j}.
%   \end{aligned}
% \end{equation*}
% \begin{equation*}
%   \begin{aligned}[b]
%     f''''(0) ={}
%     & \E_4(v_i, v_j, v_k, v_l) \order[1]{\xi_i} \order[1]{\xi_j} \order[1]{\xi_k} \order[1]{\xi_l}
%       + 6\E_3(v_i, v_j, v_k) \order[1]{\xi_i} \order[1]{\xi_j} \order[2]{\xi_k}
%       + 6\E_3(v_i, v_j, w_{kl}) \order[1]{\xi_i} \order[1]{\xi_j} \order[1]{\xi_k} \order[1]{\xi_l}\\
%     & + 12 \order[1]\lambda \dot{\E}_2(v_{i}, w_{jk}) \order[1]{\xi_i} \order[1]{\xi_j} \order[1]{\xi_k} - 3 \E_3(v_i, v_j, w_{k l}) \order[1]{\xi_i} \order[1]{\xi_j} \order[1]{\xi_k} \order[1]{\xi_l} -12 \order[1]\lambda \dot{\E}_2 (v_i, w_{j k}) \order[1]{\xi_i} \order[1]{\xi_j} \order[1]{\xi_k}\\
%     & - 6 ( \order[1]\lambda )^2 \bigl[\dot{\E}_2(v_i, w_j) + \dot{\E}_2(v_j, w_i)\bigr] \order[1]{\xi_i} \order[1]{\xi_j}
%       + 4 \order[1]\lambda \dot{\E}_3(v_i, v_j, v_k) \order[1]{\xi_i} \order[1]{\xi_j} \order[1]{\xi_k}\\
%     & + 12 \order[1]\lambda \dot{\E}_2(v_i, v_j) \order[1]{\xi_i} \order[2]{\xi_j} + 12 \order[1]\lambda \dot{\E}_2(v_i, w_{j k}) \order[1]{\xi_i} \order[1]{\xi_j} \order[1]{\xi_k}\\
%     & + 12 ( \order[1]\lambda )^2 \bigl[\dot{\E}_2(v_i, w_j)
%       + \dot{\E}_2(v_j, w_i)\bigr] \order[1]{\xi_i} \order[1]{\xi_j}\\
%     & + 6( \order[1]\lambda )^2 \ddot{\E}_2(v_i, v_j) \order[1]{\xi_i} \order[1]{\xi_j}
%       + 6\order[2]\lambda \dot{\E}_2(v_i, v_j) \order[1]{\xi_i} \order[1]{\xi_j}.
%   \end{aligned}
% \end{equation*}
% \begin{equation*}
%   \begin{aligned}[b]
%     f''''(0) ={}
%     & \bigl[ \E_4(v_i, v_j, v_k, v_l) + 3\E_3(v_i, v_j, w_{kl}) \bigr] \order[1]{\xi_i} \order[1]{\xi_j} \order[1]{\xi_k} \order[1]{\xi_l}\\
%     & + \bigl[ 4 \order[1]\lambda \dot{\E}_3(v_i, v_j, v_k) + 12 \order[1]\lambda  \dot{\E}_2(v_i, w_{j k}) \bigr] \order[1]{\xi_i} \order[1]{\xi_j} \order[1]{\xi_k}\\
%     & + \bigl\{ 6( \order[1]\lambda )^2 \ddot{\E}_2(v_i, v_j) + 6( \order[1]\lambda )^2 \bigl[\dot{\E}_2(v_i, w_j) + \dot{\E}_2(v_j, w_i)\bigr] + 6\order[2]\lambda \dot{\E}_2(v_i, v_j) \bigr\} \order[1]{\xi_i} \order[1]{\xi_j}\\
%     & + \bigl[ 6\E_3(v_i, v_j, v_k) \order[1]{\xi_i} \order[1]{\xi_j} + 12 \order[1]\lambda \dot{\E}_2(v_j, v_k) \order[1]{\xi_j} \bigr] \order[2]{\xi_k}
%   \end{aligned}
% \end{equation*}
\begin{equation*}
  \begin{aligned}[b]
    f''''(0) ={}
    & \bigl[ \E_4(v_i, v_j, v_k , v_l) + 3\E_3(v_i, v_j, w_{k l}) \bigr] \order[1]{\xi_i} \order[1]{\xi_j} \order[1]{\xi_k} \order[1]{\xi_l}\\
    & + 4 \order[1]\lambda \bigl[\dot{\E}_3(v_i, v_j, v_k) + 3 \dot{\E}_2(v_i, w_{j k})\bigr] \order[1]{\xi_i} \order[1]{\xi_j} \order[1]{\xi_k}\\
    & + \bigl\{6 ( \order[1]\lambda )^2 \bigl[ \ddot{\E}_2 (v_i, v_j) + \dot{\E}_2(v_i, w_j) + \dot{\E}_2(v_j, w_i) \bigr] + 6\order[2]\lambda \dot{\E}_2(v_i, v_j) \bigr\} \order[1]{\xi_i} \order[1]{\xi_j}\\
    & + 6\bigl[ \underbrace{\E_3(v_i, v_j, v_k) \order[1]{\xi_j} \order[1]{\xi_k} + 2 \order[1]\lambda \dot{\E}_2(v_i, v_j) \order[1]{\xi_j}}_{=0\text{ from Eq.~\eqref{eq:20220524135036}}} \bigr] \order[2]{\xi_i},
  \end{aligned}
\end{equation*}
Upon introduction of the tensors $E_{ijkl}$, $\mathring{E}_{ijk}$, $F_{ij}$ and $\mathring{F}_{ij}$
\begin{equation}
  \label{eq:20220601055512}
  f''''(0) = E_{i j k l} \order[1]{\xi_i} \order[1]{\xi_j} \order[1]{\xi_k} \order[1]{\xi_l} + 4 \order[1]\lambda \mathring{E}_{i j k} \order[1]{\xi_i} \order[1]{\xi_j} \order[1]{\xi_k} + 6 \bigl[ ( \order[1]\lambda )^2 \mathring{F}_{i j} + \order[2]\lambda F_{i j}\bigr] \order[1]{\xi_i} \order[1]{\xi_j},
\end{equation}
which finally leads to the Taylor expansion~\eqref{eq:20220525053600}.

\subsection{Application to the hessian of the energy}
\label{sec:20220616055207}
% 08/06/2022 --- aea0da72c80440d74d38d8ace59f381061f71c3e
%
% Relecture de tous les calculs de ce paragraphe

The method described in Sec.~\ref{sec:20220107121442} is now applied to $f(\eta) = F [\eta, \lambda_0 + \Lambda(\eta)]$,
with
\begin{equation*}
  F(\eta, \lambda) = \E_{, u u} [u^{\ast}(\lambda) + U(\eta), \lambda; \hat{u}, \hat{v}].
\end{equation*}
where $\hat{u}, \hat{v} \in U$ are fixed. This will deliver a Taylor expansion of the hessian of the energy,
$\E_{,uu}$. It is first observed that $F(0, \lambda) = \E_2(\lambda; \hat{u}, \hat{v})$ and, upon derivation with
respect to $\lambda$
\begin{equation*}
  \partial_{\lambda} F(0, \lambda) = \dot{\E}_2(\lambda; \hat{u}, \hat{v})
  \quad \text{and} \quad
  \partial_{\lambda\lambda}^2 F(0, \lambda) = \ddot{\E}_2(\lambda; \hat{u}, \hat{v}).
\end{equation*}

Successive differentiation of the definition of $F$ with respect to $\eta$ also leads to
\begin{gather*}
  \partial_{\eta} F(\eta, \lambda) = \E_{, uuu}(U', \hat{u}, \hat{v}),\\
  \partial_{\eta\eta}^2 F(\eta, \lambda) = \E_{, uuuu}(U', U', \hat{u}, \hat{v}) + \E_{, uuu}(U'', \hat{u}, \hat{v}),
\end{gather*}
where the differentials of $\E$ are evaluated at $[u^{\ast}(\lambda) + U(\eta), \lambda]$, while the dérivatives of $U$
are evaluated at $y$. At $\eta = 0$, the above relations read
\begin{gather*}
  \partial_{\eta} F(0, \lambda) = \E_3(\lambda; \order[1]u, \hat{u}, \hat{v}),\\
  \partial_{\eta\eta}^2 F(0, \lambda) = \E_4(\lambda ; \order[1]u, \order[1]u, \hat{u}, \hat{v}) + \E_3(\lambda; \order[2]u, \hat{u}, \hat{v}),
\end{gather*}
and, upon differentiation with respect to $\lambda$
\begin{equation*}
  \partial_{\eta \lambda}^2 F(0, \lambda) = \dot{\E}_3(\lambda; \order[1]u, \hat{u}, \hat{v}).
\end{equation*}

The Taylor expansion~\eqref{eq:20220531054247} of the hessian is finally retrieved by plugging the above results into
expressions~\eqref{eq:20220107060454} and \eqref{eq:20220107124311} of the derivatives of $f$
\begin{gather*}
  f'(0) = \E_3(\lambda_0; \order[1]u, \hat{u}, \hat{v}) + \order[1]\lambda \dot{\E}_2(\lambda_0; \hat{u}, \hat{v}),\\
  \begin{aligned}[b]
    f''(0) = {} & \E_4(\lambda_0; \order[1]u, \order[1]u, \hat{u}, \hat{v}) + \E_3(\lambda_0; \order[2]u, \hat{u}, \hat{v}) + 2\order[1]\lambda \dot{\E}_3(\lambda_0; \order[1]u, \hat{u}, \hat{v})\\
                & + ( \order[1]\lambda )^2 \ddot{\E}_2(\lambda_0; \hat{u}, \hat{v}) + \order[2]\lambda \dot{\E}_2(\lambda_0; \hat{u}, \hat{v}).
  \end{aligned}
\end{gather*}

\subsection{Asymptotic expansions of the eigenvalues and eigenvectors of the Hessian}
\label{sec:20220616074108}

In this appendix, Eqs.~\eqref{eq:20220609133608}, \eqref{eq:20220609133629} and \eqref{eq:20220616082923} are
derived. The postulated expansions~\eqref{eq:20220617064633} are plugged into the asymptotic expansion
\eqref{eq:20220531054247} of the Hessian on the one hand
\begin{equation*}
  \begin{aligned}[b]
    \E_{, uu} [u(\eta), \lambda(\eta); x, \hat{u}] ={}
    & \E_2(\order[0]x, \hat{u}) + \eta \bigl[ \E_2(\order[1]x, \hat{u}) + \E_3(\order[1]u, \order[0]x, \hat{u}) + \order[1]\lambda \dot{\E}_2(\order[0]x, \hat{u})\bigr]\\
    & + \tfrac{1}{2} \eta^2 \bigl[\E_2(\order[2]x, \hat{u}) + 2\E_3(\order[1]u, \order[1]x, \hat{u}) + 2 \order[1]\lambda \dot{\E}_2(\order[1]x, \hat{u})\\
    & + \E_4(\order[1]u, \order[1]u, \order[0]x, \hat{u}) + \E_3(\order[2]u, \order[0]x, \hat{u}) + 2\order[1]\lambda \dot{\E}_3(\order[1]u, \order[0]x, \hat{u})\\
    & + ( \order[1]\lambda )^2 \ddot{\E}_2(\order[0]x, \hat{u}) + \order[2]\lambda \dot{\E}_2(\order[0]x, \hat{u}) \bigr] + o(\eta^2)
  \end{aligned}
\end{equation*}
(where the $\E_k$ and $\dot{\E}_k$ are all evaluated at $\lambda=\lambda_0$) and into the scalar product
$\alpha \langle x, \hat{u} \rangle$ on the other hand
\begin{equation*}
  \begin{aligned}[b]
    \alpha \langle x, \hat{u} \rangle ={}
    & \order[0]\alpha \langle \order[0]x, \hat{u} \rangle + \eta \bigl(\order[1]\alpha \langle \order[0]x, \hat{u} \rangle + \order[0]\alpha \langle \order[1]x, \hat{u} \rangle\bigr)\\
    & + \tfrac{1}{2} \eta^2 \bigl(\order[0]\alpha \langle \order[2]x, \hat{u} \rangle + 2 \order[1]\alpha \langle \order[1]x, \hat{u} \rangle + \order[2]\alpha \langle \order[0]x, \hat{u} \rangle\bigr) + o(\eta^2).
  \end{aligned}
\end{equation*}

Equating both expressions for all $\hat{u} \in U$ [see Eq.~\eqref{eq:20220617074949}] leads to three variational
problems (for the $\eta^0$, $\eta^1$ and $\eta^2$ terms) that are discussed below.

\paragraph{Variational problem of order 0} Find $\order[0]x \in U$ and $\order[0]\alpha\in\reals$ such
that, for all $\hat{u} \in U$
\begin{equation*}
  \E_2(\lambda_0; \order[0]x, \hat{u}) = \order[0]\alpha \langle \order[0]x, \hat{u} \rangle.
\end{equation*}

The above equation shows that $(\order[0]\alpha, \order[0]x)$ is an eigenpair of $\E_2(\lambda_0)$. As discussed in
Sec.~\ref{sec:20220617075558}, only the case $\order[0]\alpha = 0$ is relevant. Then $\order[0]x \in V$, which is
expressed by the expansion~\eqref{eq:20220904160057} of $\order[0]x$.

\paragraph{Variational problem of order 1} Find $\order[1]x \in U$ and $\order[1]\alpha\in\reals$ such
that, for all $\hat{u} \in U$
\begin{equation}
  \label{eq:20220609131953}
  \E_2(\lambda_0; \order[1]x, \hat{u}) + \E_3(\lambda_0; \order[1]u, \order[0]x, \hat{u}) + \order[1]\lambda \dot{\E}_2(\lambda_0; \order[0]x, \hat{u})
  = \order[1]\alpha \langle \order[0]x, \hat{u} \rangle,
\end{equation}
or, equivalently, plugging the expansions~\eqref{eq:20220524133944} and \eqref{eq:20220609133608} of $\order[1]u$ and
$\order[0]x$ in the $v_i$ basis
\begin{equation}
  \label{eq:20220617080547}
  \E_2(\lambda_0; \order[1]x, \hat{u}) + \E_3(\lambda_0; v_j, v_k, \hat{u}) \order[0]{\chi_j} \order[1]{\xi_k} + \order[1]\lambda \dot{\E}_2(\lambda_0; v_j, \hat{u}) \order[0]{\chi_j}
  = \order[1]\alpha \order[0]{\chi_j} \langle v_j, \hat{u} \rangle.
\end{equation}

For $\hat{u} = v_i$, observing that $\langle v_i, v_j \rangle = \delta_{ij}$ since $(v_i)$ is orthonormal, the above
equation reads
\begin{equation}
  \bigl[\E_3(\lambda_0; v_i, v_j, v_k) \order[1]{\xi_k} + \order[1]\lambda \dot{\E}_2(\lambda_0; v_i, v_j)\bigr] \order[0]{\chi_j} = \order[1]\alpha \order[0]{\chi_i},
\end{equation}
which reduces to Eq.~\eqref{eq:20220609133608}.

The test function is now picked in $W = V^\perp$, and $\order[1]x$ is decomposed as the sum of its projections onto $V$
and $W$: $\order[1]x = \order[1]{\chi_i} v_i + \order[1]{y}$, where $\order[1]y \in W$. Eq.~\eqref{eq:20220617080547}
then delivers the following variational problem: find $\order[1]y \in W$ such that, for all $\hat{w} \in W$,
\begin{equation}
  \E_2(\order[1]y, \hat{w}) + \E_3(v_i, v_j, \hat{w}) \order[0]{\chi_i} \order[1]{\xi_j} + \order[1]\lambda \dot{\E}_2(v_i, \hat{w}) \order[0]{\chi_i} = 0,
\end{equation}
(observe that $\langle v_j, \hat{w} \rangle = 0$ since $V$ and $W$ are orthogonal subspaces). The solution to the above
problem is expressed as a linear combination of the $w_i$ and $w_{ij}$ defined by the variational problems
\eqref{eq:20220524134525} and \eqref{eq:20220519164523}, respectively:
$\order[1]y = \order[0]{\chi_i} \order[1]{\xi_j} w_{i j} + \order[1]\lambda \order[0]{\chi_i} w_i$, and the
decomposition~\eqref{eq:20220609133629} is retrieved.

\paragraph{Variational problem of order 2} For all $\hat{u} \in U$,
\begin{multline*}
    \E_2(\lambda_0; \order[2]x, \hat{u})
    + 2\E_3(\lambda_0; \order[1]u, \order[1]x, \hat{u})
    + 2 \order[1]\lambda \dot{\E}_2(\lambda_0; \order[1]x, \hat{u})\\
    + \E_4(\lambda_0; \order[1]u, \order[1]u, \order[0]x, \hat{u})
    + \E_3(\lambda_0; \order[2]u, \order[0]x, \hat{u})
    + 2\order[1]\lambda \dot{\E}_3(\lambda_0; \order[1]u, \order[0]x, \hat{u})\\
    + ( \order[1]\lambda )^2 \ddot{\E}_2(\lambda_0; \order[0]x, \hat{u})
    + \order[2]\lambda \dot{\E}_2(\lambda_0; \order[0]x, \hat{u})
    = 2 \order[1]\alpha \langle \order[1]x, \hat{u} \rangle
    + \order[2]\alpha \langle \order[0]x, \hat{u} \rangle.
\end{multline*}

For $\hat{u} = \hat{v}_i$, plugging the decompositions \eqref{eq:20220524133944}, \eqref{eq:20220524134613},
\eqref{eq:20220609133608} and \eqref{eq:20220609133629} of $\order[1]u$, $\order[2]u$, $\order[0]x $ et $\order[1]x$
delivers
% \begin{multline*}
%   2\E_3(v_i, \order[1]x, \order[1]u)
%   + 2 \order[1]\lambda \dot{\E}_2(v_i, \order[1]x)
%   + \E_4(v_i, \order[0]x, \order[1]u, \order[1]u)\\
%   + \E_3(v_i, \order[0]x, \order[2]u)
%   + 2\order[1]\lambda \dot{\E}_3(v_i, \order[0]x, \order[1]u)
%   + ( \order[1]\lambda )^2 \ddot{\E}_2(v_i, \order[0]x)\\
%   + \order[2]\lambda \dot{\E}_2(v_i, \order[0]x)
%   = 2\order[1]\alpha \langle v_i, \order[1]x \rangle
%   + \order[2]\alpha \langle v_i, \order[0]x \rangle,
% \end{multline*}
% \begin{multline*}
%   2\E_3(v_i, \order[1]{\chi_i}v_j + \order[0]{\chi_j}\order[1]{\xi_k}w_{jk}+ \order[1]\lambda \order[0]{\chi_j} w_j, \order[1]{\xi_l} v_l)
%   + 2 \order[1]\lambda \dot{\E}_2(v_i, \order[1]{\chi_i}v_j + \order[0]{\chi_j}\order[1]{\xi_k}w_{jk} + \order[1]\lambda \order[0]{\chi_j} w_j)\\
%   + \E_4(v_i, \order[0]{\chi_j} v_j, \order[1]{\xi_k} v_k, \order[1]{\xi_l} v_l)
%   + \E_3(v_i, \order[0]{\chi_j} v_j, \order[2]{\xi_k} v_k + \order[1]{\xi_k} \order[1]{\xi_l} w_{kl} + 2\order[1]\lambda \order[1]{\xi_k} w_k)\\
%   + 2\order[1]\lambda \dot{\E}_3(v_i, \order[0]{\chi_j} v_j, \order[1]{\xi_k} v_k)
%   + ( \order[1]\lambda )^2 \ddot{\E}_2(v_i, \order[0]{\chi_j} v_j) + \order[2]\lambda \dot{\E}_2(v_i, \order[0]{\chi_j} v_j)\\
%   = 2\order[1]\alpha \langle v_i,  \order[1]{\chi_i}v_j + \order[0]{\chi_j}\order[1]{\xi_k}w_{jk} + \order[1]\lambda \order[0]{\chi_j} w_j \rangle
%   + \order[2]\alpha \langle v_i, \order[0]{\chi_j} v_j\rangle,
% \end{multline*}
% \begin{multline*}
%   2\E_3(v_i, v_j,  v_k) \order[1]{\chi_i} \order[1]{\xi_k}
%   + 2\E_3(v_i, w_{jk}, v_l) \order[0]{\chi_j} \order[1]{\xi_k} \order[1]{\xi_l}
%   + 2 \order[1]\lambda \E_3(v_i,  w_j, v_k) \order[0]{\chi_j} \order[1]{\xi_k}\\
%   + 2 \order[1]\lambda \dot{\E}_2(v_i, v_j) \order[1]{\chi_i}
%   + 2 \order[1]\lambda \dot{\E}_2(v_i, w_{jk}) \order[0]{\chi_j} \order[1]{\xi_k}
%   + ( \order[1]\lambda )^2 \dot{\E}_2(v_i, w_j) \order[0]{\chi_j}\\
%   + \E_4(v_i, v_j,  v_k, v_l) \order[0]{\chi_j} \order[1]{\xi_k} \order[1]{\xi_l}
%   + \E_3(v_i, v_j, v_k) \order[0]{\chi_j} \order[2]{\xi_k}
%   + \E_3(v_i, v_j, w_{kl}) \order[0]{\chi_j} \order[1]{\xi_k} \order[1]{\xi_l}\\
%   + 2\order[1]\lambda \E_3(v_i, v_j, w_k) \order[0]{\chi_j} \order[1]{\xi_k}
%   + 2\order[1]\lambda \dot{\E}_3(v_i, v_j,  v_k) \order[0]{\chi_j} \order[1]{\xi_k}
%   + ( \order[1]\lambda )^2 \ddot{\E}_2(v_i, v_j) \order[0]{\chi_j}\\
%   + \order[2]\lambda \dot{\E}_2(v_i, v_j) \order[0]{\chi_j} = 2\order[1]\alpha\order[1]{\chi_i} + \order[2]\alpha \order[0]{\chi_i}.
% \end{multline*}
\begin{multline*}
  \bigl[ \E_4(v_i, v_j,  v_k, v_l) + 2\E_3(v_i, w_{jk}, v_l) + \E_3(v_i, v_j, w_{kl})\bigr] \order[0]{\chi_j} \order[1]{\xi_k} \order[1]{\xi_l}\\
  + 2 \order[1]\lambda \bigl[ \E_3(v_i,  w_j, v_k) + \dot{\E}_2(v_i, w_{jk}) + \E_3(v_i, v_j, w_k) + \dot{\E}_3(v_i, v_j,  v_k) \bigr] \order[0]{\chi_j} \order[1]{\xi_k}\\
  + ( \order[1]\lambda )^2 \bigl[\dot{\E}_2(v_i, w_j) + \ddot{\E}_2(v_i, v_j)\bigr] \order[0]{\chi_j} + \bigl[\E_3(v_i, v_j, v_k) \order[2]{\xi_k} + \order[2]\lambda \dot{\E}_2(v_i, v_j)\bigr] \order[0]{\chi_j} \\
  +2\bigl[\E_3(v_i, v_j,  v_k)  \order[1]{\xi_k} + \order[1]\lambda \dot{\E}_2(v_i, v_j)\bigr] \order[1]{\chi_i} = 2\order[1]\alpha\order[1]{\chi_i} + \order[2]\alpha \order[0]{\chi_i}.
\end{multline*}

The $\order[0]{\chi_j} \order[1]{\xi_k}$ term is transformed with Eqs.~\eqref{eq:20220524134525} and
\eqref{eq:20220519164523}
\begin{multline*}
  \bigl[ \E_4(v_i, v_j,  v_k, v_l)
  + \E_3(v_i, w_{jk}, v_l)
  + \E_3(v_i, w_{jl}, v_k)
  + \E_3(v_i, v_j, w_{kl})\bigr] \order[0]{\chi_j} \order[1]{\xi_k} \order[1]{\xi_l}\\
  + 2\order[1]\lambda \bigl[ -\E_2(w_{ik},  w_j) - \E_2(w_i, w_{jk}) - \E_2(w_{ij}, w_k) + \dot{\E}_3(v_i, v_j,  v_k) \bigr] \order[0]{\chi_j} \order[1]{\xi_k}\\
  + ( \order[1]\lambda )^2 \bigl[\dot{\E}_2(v_i, w_j) + \ddot{\E}_2(v_i, v_j)\bigr] \order[0]{\chi_j}
  + \bigl[\E_3(v_i, v_j, v_k) \order[2]{\xi_k} + \order[2]\lambda \dot{\E}_2(v_i, v_j)\bigr] \order[0]{\chi_j} \\
  +2\bigl[\E_3(v_i, v_j,  v_k)  \order[1]{\xi_k}
  + \order[1]\lambda \dot{\E}_2(v_i, v_j)\bigr] \order[1]{\chi_i}
  = 2\order[1]\alpha\order[1]{\chi_i}
  + \order[2]\alpha \order[0]{\chi_i},
\end{multline*}
and Eq.~\eqref{eq:20220616082923} results from the application of Eqs.~\eqref{eq:20220617084433} and
\eqref{eq:20220617085256}.

\end{document}

%%% Local Variables:
%%% coding: utf-8
%%% fill-column: 120
%%% mode: latex
%%% TeX-engine: xetex
%%% TeX-master: t
%%% End:
