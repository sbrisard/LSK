\newcommand{\sbtitle}{Notes relatives à la méthode asymptotique de Lyapunov–Schmidt–Koiter}
\newcommand{\sbauthor}{Sébastien Brisard}
\newcommand{\sbemail}{sebastien.brisard@univ-eiffel.fr}
\newcommand{\sbaddress}{Univ Gustave Eiffel, Ecole des Ponts, IFSTTAR, CNRS, Navier, F-77454 Marne-la-Vall\'ee, France}
\newcommand{\sbsubject}{Note bibliographique}

\documentclass[12pt, final]{scrartcl}
%\setkomafont{disposition}{\rmfamily}

\usepackage{polyglossia}
\setdefaultlanguage{english}

\usepackage{amsfonts}
\usepackage{amsmath}
\usepackage{amssymb}

\usepackage{amsthm}
\theoremstyle{definition}
\renewcommand{\qedsymbol}{}
\newtheorem{remark}{Remark}
\newtheorem{theorem}{Theorem}

\usepackage[backend=biber,bibencoding=utf8,doi=false,giveninits=true,isbn=false,maxnames=10,minnames=5,sortcites=true,style=authoryear,texencoding=utf8,url=false]{biblatex}
\addbibresource{LSK-notes.bib}

\usepackage[breaklinks=true, colorlinks=true, pdftitle={\sbtitle}, pdfauthor={\sbauthor}, pdfsubject={\sbsubject}, urlcolor=blue]{hyperref}

\usepackage[color={1 1 0}]{pdfcomment}

\usepackage{unicode-math}
% \setmainfont{XITS}
% \setmathfont{XITS Math}
\setmainfont{Asana Math}
\setmathfont{Asana Math}

\newcommand{\E}{\mathcal E}
\newcommand{\D}{\mathrm{d}}
\newcommand{\order}[2][1]{#2^{(#1)}}
\newcommand{\reals}{\mathbb{R}}

\begin{document}
\title{\sbtitle}
\author{\sbauthor\thanks{\sbaddress~--- \sbemail}}
\maketitle

\section{Notations}

L'espace des champs cinématiquement admissibles est noté $U$. On suppose qu'il
a la structure d'espace vectoriel. L'énergie du système est notée $\E(u, \lambda)$,
où $\lambda$ désigne un paramètre de chargement. Soit $u^{\ast}(\lambda)$ la branche
fondamentale. Par définition
\begin{equation}
  \E_{,u}[u^{\ast}(\lambda), \lambda; \hat{u}]=0 \quad \text{pour tout} \quad \hat{u}\in U,
\end{equation}
and, deriving twice with respect to $\lambda$, we find successively, for all $\hat{u} \in U$
\begin{equation}
  \label{eq:20220901143843}
  \E_{,uu}[u^\ast(\lambda), \lambda; \dot{u}^\ast(\lambda), \hat{u}] + \E_{,u\lambda}[u^\ast(\lambda), \lambda; \hat{u}] = 0
\end{equation}
and
\begin{multline}
  \label{eq:20220901143902}
  \E_{,uuu}[u^\ast(\lambda), \lambda; \dot{u}^\ast(\lambda), \dot{u}^\ast(\lambda), \hat{u}] + 2\E_{,uu\lambda}[u^\ast(\lambda), \lambda; \dot{u}^\ast(\lambda), \hat{u}]\\
  + \E_{,u\lambda\lambda}[u^\ast(\lambda), \lambda; \hat{u}] + \E_{,uu}[u^\ast(\lambda), \lambda; \ddot{u}^\ast(\lambda), \hat{u}] = 0
\end{multline}





Il sera commode d'introduire les notations suivantes
\begin{equation}
  \E₂(\lambda) = \E_{,uu}[u^{\ast}(\lambda), \lambda], \quad \E₃(\lambda) = \E_{,uuu}[u^{\ast}(\lambda), \lambda], \quad \E₄(\lambda) = \E_{,uuuu} [u^{\ast}(\lambda), \lambda].
\end{equation}
Noter que $\E₂$, $\E₃$ et $\E₄$ sont des formes bi-, tri- et
quadri-linéaires, respectivement. L'application de ces formes à des éléments de
$U$ sera notée $\E₂(\lambda; u, v)$, $\E₃(\lambda; u, v, w)$, etc. La dérivée de ces
formes par rapport à $\lambda$ sera notée à l'aide d'un point supérieur
($\dot{\E}_2$, $\dot{\E}_3$, \dots).

On suppose que l'équilibre est stable pour des valeurs suffisamment petites de
$\lambda$. Plus précisément, on suppose que $\E₂(\lambda)$ est définie positive pour tout
$\lambda < \lambda₀$. Pour $\lambda = \lambda₀$, la forme quadratique $\E₂(\lambda₀)$ n'est plus que
positive. On note $u₀ = u^{\ast}(\lambda₀)$, $\dot{u}₀ = \dot{u}^\ast(\lambda₀)$ et
$\ddot{u}₀ = \ddot{u}^\ast(\lambda₀)$ de sorte que les
Éqs.~\eqref{eq:20220901143843} et \eqref{eq:20220901143902} s'écrivent, en
$\lambda = \lambda₀$
\begin{gather}
  \label{eq:20220901144331}
  \E_{,uu}(u₀, \lambda₀; \dot{u}₀, \bullet) + \E_{,u\lambda}(u₀, \lambda₀; \bullet) = 0\\
  \label{eq:20220901144335}
  \E_{,uuu}(u₀, \lambda₀; \dot{u}₀, \dot{u}₀, \bullet) + 2\E_{,uu\lambda}(u₀, \lambda₀; \dot{u}₀, \bullet) + \E_{,u\lambda\lambda}(u₀, \lambda₀; \bullet) + \E_{,uu}(u₀, \lambda₀; \ddot{u}₀, \bullet) = 0
\end{gather}

On s'intéresse dans ce qui suit à toutes les courbes d'équilibre qui passent par
le point $(u₀, \lambda₀)$.

\pdfmarkupcomment{Noter que dans ce qui suit}{Est-ce encore d'actualité ?}, on
convient que les formes $\E₂$, $\E₃$ et $\E₄$ sont implicitement évaluées en
$\lambda₀$ lorsque $\lambda$ n'est pas rappelé : ainsi, on notera $\E₂(•, •)$ plutôt
que $\E₂(\lambda₀ ; •, •)$.

Par hypothèse, $\E₂(\lambda₀)$ est positive, sans être définie
positive~; soit $V$ son noyau, qui forme un sous-espace vectoriel de $U$. On
suppose que $V$ est de dimension finie $m = \dim V$. Soit
$(v₁, \ldots, v_m)$ une base orthonormée de ce noyau pour le produit scalaire
$〈 •, • 〉$(qui n'est pas précisé pour le moment). On introduit le
sous-espace supplémentaire orthogonal $W$ de $V$ dans $U$
\begin{equation}
  U = V \overset{\perp}{\otimes} W.
\end{equation}

\begin{remark}
  \label{rem:20220902095055}
  The bilinear form $\E₂$ being elliptic over $W$, variational problems of the
  type: find $w \in W$ such that
  \begin{equation}
    \E₂(w, \hat{w})+\ell(\hat{w}) = 0 \quad \text{for all} \quad \hat{w} \in W
  \end{equation}
  are well-posed for any linear form $\ell$ over $W$. In particular, for
  $\ell=0$, the unique solution to the variational problem
  \begin{equation}
    \E₂(w, \hat{w}) = 0 \quad \text{for all} \quad \hat{w} \in W
  \end{equation}
  is $w = 0$.
\end{remark}

For $1 ≤ i, j ≤ m$, we introduce the solutions $w_i, w_{ij} \in W$ to the
following variational problems
\begin{gather}
  \label{eq:20220524134525}
  \E₂(\lambda₀; w_i, \hat{w}) + \dot{\E}₂(\lambda₀; v_i, \hat{w}) = 0,\\
  \label{eq:20220519164523}
  \E₂(\lambda₀; w_{i j}, \hat{w})+\E₃(\lambda₀; v_i, v_j, \hat{w}) = 0,
\end{gather}
for all $\hat{w} \in W$. Since $w_{i}$ and $w_{ij}$ belong to $W$, we have
$〈 w_{i}, v 〉 = 〈 w_{ij}, v 〉 = 0$ for all $v \in V$. Since $\E₂(\lambda₀; •,
•)$ is symmetric, it can be verified that $w_{ij}=w_{ji}$. We also introduce
the following tensors, defined in $V$
\begin{gather}
  E_{ijk} = \E₃(\lambda₀; v_i, v_j, v_k) + \E₂(\lambda₀; v_i, w_{jk}) + \E₂(\lambda₀; v_j, w_{ki}) + \E₂(\lambda₀; v_k, w_{ij}),\\
  E_{ijkl} = \E₄(\lambda₀ ; v_i, v_j, v_k, v_l) + \E₃(\lambda₀ ; v_i, v_j, w_{kl}) + \E₃(\lambda₀ ; v_i, v_k, w_{lj}) + \E₃(\lambda₀ ; v_i, v_l, w_{jk}),\\
  F_{ij} = \dot{\E}₂(\lambda₀; v_i, v_j) + \E₂(\lambda₀; v_i, w_j) + \E₂(\lambda₀; v_j, w_i),
\end{gather}
as well as the derivatives
\begin{gather}
  \label{eq:20220615063626}
  \mathring{E}_{ijk} = \dot{\E}₃(\lambda₀; v_i, v_j, v_k) + \dot{\E₂}(\lambda₀; v_i, w_{jk}) + \dot{\E}₂(\lambda₀; v_j, w_{ki}) + \dot{\E}₂(\lambda₀; v_k, w_{ij}),\\
  \label{eq:20220615063633}
  \mathring{F}_{ij} = \ddot{\E}₂(\lambda₀; v_i, v_j) + \dot{\E}₂(\lambda₀; v_i, w_j) + \dot{\E}₂(\lambda₀; v_j, w_i).
\end{gather}

Note that, since $\E₂(\lambda₀; v_i, •) = 0$, the above expressions simplify as follows
\begin{gather}
  \label{eq:20220524135619}
  E_{ijk} = \E₃(\lambda₀; v_i, v_j, v_k),\\
  \label{eq:20220524135553}
  E_{ijkl} = \E₄(\lambda₀ ; v_i, v_j, v_k, v_l) + \E₃(\lambda₀ ; v_i, v_j, w_{kl}) + \E₃(\lambda₀ ; v_i, v_k, w_{jl}) + \E₃(\lambda₀ ; v_i, v_l, w_{jk}),\\
  \label{eq:20220524135643}
  F_{ij} = \dot{\E}₂(\lambda₀; v_i, v_j).
\end{gather}

The tensors $E_{ijk}$, $F_{ij}$, $\mathring{E}_{ijk}$ and
$\mathring{F}_{ij}$ are fully symmetric. Furthermore, the following expression
of $E_{ijkl}$ result from Eq.~\eqref{eq:20220519164523}
\begin{equation}
  \label{eq:20220802081116}
  E_{ijkl} = \E₄(\lambda₀ ; v_i, v_j, v_k, v_l) - \E₂(\lambda₀ ; w_{ij}, w_{kl}) - \E₂(\lambda₀ ; w_{ik}, w_{jl}) - \E₂(\lambda₀ ; w_{il}, w_{jk}),
\end{equation}
which shows that $E_{ijkl}$ is also fully symmetric. We close this section,
with two useful identities
\begin{equation}
  \label{eq:20220617084433}
  \begin{aligned}[b]
    \mathring{F}_{ij} ={} & \ddot{\E}₂(\lambda₀; v_i, v_j) + \dot{\E}₂(\lambda₀; v_i, w_j) + \dot{\E}₂(\lambda₀; v_j, w_i)\\
    ={} & \ddot{\E}₂(\lambda₀; v_i, v_j) + \dot{\E}₂(\lambda₀; v_i, w_j) - \E₂(\lambda₀; w_j, w_i) & \text{Eq.~\eqref{eq:20220524134525}, with $v_i = v_j$ and $\hat{w} = w_i$}\\
    ={} & \ddot{\E}₂(\lambda₀; v_i, v_j)  + 2\dot{\E}₂(\lambda₀; v_i, w_j) & \text{Eq.~\eqref{eq:20220524134525}, with $\hat{w} = w_j$}\\
    ={} & \ddot{\E}₂(\lambda₀; v_i, v_j) + 2\dot{\E}₂(\lambda₀; v_j, w_i), & \text{symmetry w.r.t. $i$ and $j$}
  \end{aligned}
\end{equation}
and, from Eq.~\eqref{eq:20220519164523}
\begin{equation}
  \label{eq:20220617085256}
  \begin{aligned}[b]
  \mathring{E}_{ijk} ={}& \dot{\E}₃(\lambda₀; v_i, v_j, v_k) + \dot{\E}₂(\lambda₀; v_i, w_{jk}) + \dot{\E}₂(\lambda₀; v_j, w_{ik}) + \dot{\E}₂(\lambda₀; v_k, w_{ij})\\
  ={}& \dot{\E}₃(\lambda₀; v_i, v_j, v_k) - \bigl[\E₂(\lambda₀; w_i, w_{jk}) + \E₂(\lambda₀; w_j, w_{ik}) + \E₂(\lambda₀; w_k, w_{ij})\bigr].
  \end{aligned}
\end{equation}

\section{Analysis of the critical point}
\label{sec:20220802061621}

In this section, we discuss the stability of the critical point $(u₀, \lambda₀)$. To
this end, we evaluate the potential energy in a neighboring state
$u₀ + u$, where $u \in U$ is ``small''. We have, to the fourth order
\begin{equation}
  \begin{aligned}[b]
    \E(u₀ + u, \lambda₀) - \E(u₀, \lambda₀) ={}
    &\tfrac{1}{2} \E₂(\lambda₀; u, u) + \tfrac{1}{6} \E₃(\lambda₀; u, u, u)\\
    &+ \tfrac{1}{24} \E₄(\lambda₀; u, u, u, u) + o(〈 u , u 〉²),
  \end{aligned}
\end{equation}
where the linear term has been omitted, $u₀$ being a critical point of the
energy. Since $v \in V$, we have $\E₂(\lambda₀; v, •) = 0$. We now expand $u$ as
$u = ξ v + η w$, with $ξ, η \in \reals$ and $v \in V$ and $w \in W$ are fixed,
orthogonal directions. Owing to the multi-linearity and symmetry of the
successive differential of $\E$, the above expression expands as
follows
\begin{equation}
  \begin{aligned}[b]
    \E(u₀ + u, \lambda₀) - \E(u₀, \lambda₀) ={}
    & \tfrac{1}{2} η² \E₂(\lambda₀; w, w) + \tfrac{1}{6} ξ³ \E₃(\lambda₀; v, v, v)\\
    & + \tfrac{1}{2} ξ² η \E₃(\lambda₀; v, v, w) + \tfrac{1}{2} ξ η² \E₃(\lambda₀; v, w, w)\\
    & + \tfrac{1}{6} η³ \E₃(\lambda₀; w, w, w) + \tfrac{1}{24} ξ⁴ \E₄(\lambda₀; v, v, v, v)\\
    & + \tfrac{1}{6} ξ³ η \E₄(\lambda₀; v, v, v, w) + \tfrac{1}{4} ξ² η² \E₄(\lambda₀; v, v, w, w)\\
    & + \tfrac{1}{6} ξ η³ \E₄(\lambda₀; v, w, w, w) + \tfrac{1}{24} η⁴ \E₄(\lambda₀; w, w, w, w)\\
    & + o\bigl[\bigl(ξ² + η²\bigr)²\bigr].
  \end{aligned}
\end{equation}

For the equilibrium to be stable, the above expression must be $≥ 0$ for all
$ξ$ et $η$ small enough. Taking first $η = 0$, we get the following necessary conditions
\begin{equation}
  \label{eq:20211108164416}
  \E₃(\lambda₀; v, v, v) = 0 \quad \text{and} \quad \E₄(\lambda₀; v, v, v, v) \geq 0 \quad \text{for all} \quad v \in V.
\end{equation}

\begin{remark}
  Note that, from Theorem~\ref{thr:20220802112835}, the first of these two
  conditions is equivalent to $E_{ijk}=0$, for all $i, j, k = 1, \ldots m$.
\end{remark}

In other words, if there exists $v \in V$ such that $\E₃(\lambda₀; v, v, v) \neq 0$
or $\E₄(v, v, v, v) < 0$, then the equilibrium is \emph{unstable} at the
critical point. The above conditions are not sufficient. Indeed, assuming
conditions~\eqref{eq:20211108164416} to hold, we now take $η = ξ²$
\begin{equation}
  \begin{aligned}[b]
    \E(u₀ + u, \lambda₀) - \E(u₀, \lambda₀) ={} & \tfrac{1}{2} ξ⁴ \bigl[ \E₂(\lambda₀; w, w) + \E₃(\lambda₀; v, v, w)\\
    & + \tfrac{1}{12} \E₄(\lambda₀; v, v, v, v) \bigr] + o(ξ⁴)
  \end{aligned}
\end{equation}
and we get the further necessary condition
\begin{equation}
  \label{eq:20211109145356}
  \E₂(w, w) + \E₃(v, v, w) + \tfrac{1}{12} \E₄(v, v, v, v) \geq 0 \quad \text{for all} \quad v \in V \quad \text{and} \quad w \in W.
\end{equation}

The direction $v \in V$ being fixed, the above expression is minimal when $w$
satisfies the following variational problem
\begin{equation}
  \label{eq:20211109145224}
  2\E₂(w, \hat{w}) +\E₃(v, v, \hat{w}) = 0 \quad \text{for all} \quad \hat{w} \in W.
\end{equation}

Expanding $v \in V$ in the $(v_i)$ basis as follows: $v = ξ_i v_i$, it is
observed that the solution to the above variational problem is
$w = \tfrac{1}{2} ξ_i ξ_j w_{ij}$, where $w_{ij}$ is the solution to the
elementary variational problem \eqref{eq:20220519164523}. For this value of
$w$, condition~\eqref{eq:20211109145356} reads
\begin{equation}
  \bigl[\E₄(v_i, v_j, v_k, v_l) - 3\E₂(w_{ij}, w_{kl})\bigr] ξ_i ξ_j ξ_k ξ_l \geq 0 \quad \text{for all} \quad ξ_1, \ldots, ξ_m \in \reals,
\end{equation}
which, in view of definition~\eqref{eq:20220802081116} of $E_{ijkl}$, is equivalent to
\begin{equation}
  E_{ijkl} ξ_i ξ_j ξ_k ξ_l \geq 0 \quad \text{for all} \quad ξ_m, \ldots, ξ_m \in \reals.
\end{equation}

Note that Eq.~\eqref{eq:20211109145224} implies $\E₄(\lambda₀; v, v, v, v) ≥ 0$,
which becomes a redundant necessary condition. Indeed, plugging
$w= ξ_i ξ_j w_{ij}$ into Eq.~\eqref{eq:20211109145224} cancels the first two
terms. To sum up, we have the following \emph{necessary} conditions for
stability
\begin{equation}
  E_{ijk} ξ_i ξ_j ξ_k = 0 \quad \text{and} \quad E_{ijkl} ξ_i ξ_j ξ_k ξ_l ≥ 0 \quad \text{for all} \quad ξ_m, \ldots, ξ_m \in \reals.
\end{equation}

\pdfmarkupcomment{Conversely, the following condition is \emph{sufficient} to ensure stability of the critical point}{\`A d\'emontrer}
\begin{equation}
  E_{ijk} ξ_i ξ_j ξ_k = 0 \quad \text{and} \quad E_{ijkl} ξ_i ξ_j ξ_k ξ_l > 0 \quad \text{for all} \quad ξ_m, \ldots, ξ_m \in \reals.
\end{equation}

\section{Analysis of bifurcated branches}
\label{sec:20220617075558}

In this section, we show that, besides the fundamental branch $u^\ast(\lambda)$,
other (bifurcated) equilibrium branches may pass through the critical point
$(u₀, \lambda₀)$. The starting point is the characterization of an equilibrium by
the stationarity of the energy, which defines all equilibrium branches as
implicit functions, which can be expanded with respect to some perturbation
parameter.

The first approach (see Sec.~\ref{sec:20220902091527}) relies on the
Lyapunov–Schmidt decomposition of the equilibrium branch over $V$ and
$W$. However, this approach leads to tedious derivations. This approach has
historical and pedagogical value: in particular, it provides a meaning to
$w_i$ and $w_{ij}$ defined by Eqs.~\eqref{eq:20220524134525} and
\eqref{eq:20220519164523}. In Sec.~\ref{sec:20220902092109}, a more systematic
approach is developed, that I found in the paper by \textcite[][Appendix
A]{chak2018}.

\subsection{The Lyapunov–Schmidt decomposition}
\label{sec:20220902091527}

The following decomposition of the equilibrium state $u$ at the load-level
$\lambda$is postulated
\begin{equation}
  \label{eq:20220902174235}
  u = u^\ast(\lambda) + ξ_i v_i + w, \quad \text{with} \quad w \in W.
\end{equation}

It follows from the orthogonality of $V$ and $W$ that $〈v_i, w〉 = 0$ for
all $i=1, \ldots, m$. Stationarity of the energy is expressed as follows
\begin{equation}
  \E_{,u}[u^\ast(\lambda) + ξ_i v_i + w, \lambda; \hat{u}] = 0, \quad \text{for all} \quad \hat{u} \in U
\end{equation}
or, equivalently
\begin{equation}
  \label{eq:20220901120544}
  \E_{,u}[u^\ast(\lambda) + ξ_i v_i + w, \lambda; \hat{v}] = 0, \quad \text{for all} \quad \hat{v} \in V
\end{equation}
and
\begin{equation}
  \label{eq:20220825143616}
  \E_{,u}[u^\ast(\lambda) + ξ_i v_i + w, \lambda; \hat{w}] = 0, \quad \text{for all} \quad \hat{w} \in W.
\end{equation}
The method proceeds in three steps. In \textbf{Step 1},
Eq.~\eqref{eq:20220825143616} is used to define $w$ as an implicit function of
$ξ₁$, \dots, $ξ_m$ and $\lambda$. Then, in \textbf{Step 2},
Eq.~\eqref{eq:20220825143616} is used to define $\lambda$ as an implicit function of
$ξ₁$, \dots, $ξ_m$. Finally, a parametrization $η$ of $ξ₁$, \dots
$ξ_m$ is introduced in \textbf{Step 3} and the Taylor expansion of $u$ and
$\lambda$ with respect to $η$ is derived. These steps are presented below.

\paragraph{Step 1: $w$ as a function of $ξ_i$ and $\lambda$} In this paragraph,
$\hat{w}$ denotes an arbitrary test function in $W$. From the implicit
function theorem, Eq.~\eqref{eq:20220825143616} defines a function
$(ξ_1, \ldots, ξ_m, \lambda) \mapsto w(ξ_1, \ldots, ξ_m, \lambda)$ in the neighborhood of
$(ξ₁, \ldots, ξ_m, \lambda) = (0, \ldots, 0, \lambda₀)$. Why the theorem applies will be
clarified below. Eq.~\eqref{eq:20220825143616} is first differentiated with respect to
$ξ_i$
\begin{equation}
  \label{eq:20220826140926}
  \E_{,uu}(u^\ast + ξ_k v_k + w, \lambda; v_i + w_{,i}, \hat{w}) = 0.
\end{equation}

Substituting $ξ_1 = \cdots = ξ_m = 0, \lambda = \lambda₀$ in the above equations and
observing that $\E₂(\lambda₀; v_i, W) = 0$ since $v_i \in V$, we get
\begin{equation}
\label{eq:20220825150219}
  \E₂(\lambda₀; v_i + w_{,i}, \hat{w}) = \E₂(\lambda₀; w_{,i}, \hat{w}) = 0.
\end{equation}

Since $w \in W$ for all $ξ^i$ and $\lambda$, we have $w_{,i} \in W$ and,
Remark~\ref{rem:20220902095055} leads to $w_{,i} = 0$ at the point
$ξ₁ = 0, \ldots, ξ_m = 0$ and $\lambda = \lambda₀$. Eq.~\eqref{eq:20220825143616} is
then differentiated with respect to  $\lambda$
\begin{equation}
  \label{eq:20220830145945}
  \E_{,uu}(u^\ast + ξ_i v_i + w, \lambda; \dot{u}^\ast + w_{,\lambda}, \hat{w}) + \E_{,u\lambda}(u^\ast + ξ_i v_i + w, \lambda; \hat{w}) = 0
\end{equation}
and, at $ξ₁ = \ldots = ξ_m = 0$
\begin{equation}
  \label{eq:20220830151513}
  \E_{,uu}(u^\ast, \lambda; w_{,\lambda}, \hat{w})
  + \underbrace{\E_{,uu}(u^\ast, \lambda; \dot{u}^\ast, \hat{w}) + \E_{,u\lambda}(u^\ast, \lambda; \hat{w})}_{=0 \quad \text{see Eq.~\eqref{eq:20220901143843}}}
  = \E₂(\lambda; w_{,\lambda}, \hat{w}) = 0,
\end{equation}
which proves similarly that the derivative of $w$ with respect to $\lambda$
vanishes at the critical point. We have found so far that
\begin{equation}
  \frac{∂w}{∂ξ_1} \biggr\rvert_{ξ_1 = \cdots = ξ_m = 0, \lambda = \lambda₀}
  = \ldots =
  \frac{∂w}{∂ξ_m} \biggr\rvert_{ξ_1 = \cdots = ξ_m = 0, \lambda = \lambda₀}
  = \frac{∂w}{∂\lambda} \biggr\rvert_{ξ_1 = \cdots = ξ_m = 0, \lambda = \lambda₀}= 0.
\end{equation}

To express the second-order derivatives of $w$, Eq.~\eqref{eq:20220826140926}
is differentiated first with respect to $ξ_j$, then with respect to
$\lambda$. This delivers
\begin{equation}
  \E_{,uuu}(u^\ast + ξ_k v_k + w, \lambda; v_i + w_{,i}, v_j + w_{,j}, \hat{w}) + \E_{,uu}(u^\ast + ξ_k v_k + w, \lambda; w_{,ij}, \hat{w}) = 0
\end{equation}
and
\begin{equation}
  \begin{aligned}[b]
    \E_{,uuu}(u^\ast + ξ_k v_k + w, \lambda; v_i + w_{,i}, \dot{u}^\ast + w_{,\lambda}, \hat{w}) &\\
    + \E_{,uu\lambda}(u^\ast + ξ_k v_k + w, \lambda; v_i + w_{,i}, \hat{w}) + \E_{,uu}(u^\ast + ξ_k v_k + w, \lambda; w_{,i\lambda}, \hat{w}) &= 0
  \end{aligned}
\end{equation}
and, at $ξ_1 = \cdots = ξ_m = 0, \lambda = \lambda₀$ (recalling that, at this point,
$w_{,1} = \cdots = w_{, m} = w_{,\lambda} = 0$)
\begin{equation}
  \E_3(\lambda₀; v_i, v_j, \hat{w}) + \E₂(\lambda₀; w_{,ij}, \hat{w}) = 0
  \quad \text{and} \quad
  \dot{\E}₂(\lambda₀; v_i, \hat{w}) + \E₂(\lambda₀; w_{,i\lambda}, \hat{w}) = 0.
\end{equation}
The variational problems~\eqref{eq:20220524134525} and \eqref{eq:20220519164523}
are recognized, leading to
\begin{equation}
  \frac{∂²w}{∂ξ_i ∂ξ_j}\biggr\rvert_{ξ_1 = \cdots = ξ_m = 0, \lambda = \lambda₀} = w_{ij}
  \quad\text{and}\quad
  \frac{∂²w}{∂\lambda ∂ξ_i}\biggr\rvert_{ξ_1 = \cdots = ξ_m = 0, \lambda = \lambda₀} = w_{i}.
\end{equation}

The $w_i$ and $w_{ij}$ defined by the variational
problems~\eqref{eq:20220524134525} and \eqref{eq:20220519164523} therefore
appear as the second-order derivatives of $w$ at $ξ_k = 0$ and $\lambda = \lambda_0$,
with respect to $\lambda$, $ξ_i$ and $ξ_i$, $ξ_j$.

Finally, differentiating Eq.~\eqref{eq:20220830151513} with respect to $\lambda$ leads to
\begin{equation}
  \dot{\E}₂(\lambda; w_{,\lambda}, \hat{w}) + \E₂(\lambda; w_{,\lambda\lambda}, \hat{w}) = 0
\end{equation}
and, at $\lambda = \lambda₀$
\begin{equation}
  \frac{∂²w}{∂\lambda²}\biggr\rvert_{ξ_1 = \cdots = ξ_m = 0, \lambda = \lambda₀} = 0.
\end{equation}

We have obtained the following Taylor expansion of the component $w$ of the
LSK expansion of $u$
\begin{equation}
  w(ξ_1, \ldots, ξ_m, \lambda) = \tfrac{1}{2} ξ_i ξ_j w_{ij} + \bigl( \lambda - \lambda₀ \bigr) ξ_i w_i + o\Bigl(ξ₁² + \cdots + ξ_m² + \bigl(\lambda - \lambda₀\bigr)^2\Bigr).
\end{equation}

\paragraph{Step 2: $\lambda$ as a function of $ξ_i$} We now turn to
Eq.~\eqref{eq:20220901120544}. Since $w$ is a function of $\lambda$ and $ξ_k$
($k = 1, \ldots, m$) this equation implicitly defines $\lambda$ as a function of
$ξ_k$, the derivatives of which can be evaluated at $ξ₁ = \cdots = ξ_m =
0$. In this paragraph, $\hat{v}$ denotes an arbitrary element of
$V$. Besides, unless otherwise mentioned, the differentials of the energy
$\E_{,uu}$, $\E_{,u\lambda}$, $\E_{,\lambda\lambda}$, $\E_{,uuu}$ \dots{} are evaluated at
$u = u^\ast(\lambda) + ξ_k v_k + w(ξ_k, \lambda)$. Differentiating first
Eq.~\eqref{eq:20220901120544} with respect to $ξ_i$
\begin{equation}
  \label{eq:20220901121940}
  \E_{,uu}[v_i + w_{,i} + \lambda_{,i} (\dot{u}^\ast + w_{,\lambda}), \hat{v}] + \lambda_{, i} \E_{,u\lambda}(\hat{v}) = 0,
\end{equation}
then with respect to $ξ_j$
\begin{equation}
  \label{eq:20220901125230}
  \begin{gathered}[b]
    \E_{,uuu}[v_i + w_{,i} + \lambda_{,i} (\dot{u}^\ast + w_{,\lambda}), v_j + w_{,j} + \lambda_{,j} (\dot{u}^\ast + w_{,\lambda}), \hat{v}]\\
    + \lambda_{,j}\E_{,uu\lambda}[v_i + w_{,i} + \lambda_{,i} (\dot{u}^\ast + w_{,\lambda}), \hat{v}]\\
    + \E_{,uu}[w_{,ij} + \lambda_{,ij} (\dot{u}^\ast + w_{,\lambda}) + \lambda_{,i}\lambda_{,j} (\ddot{u}^\ast + w_{,\lambda\lambda}), \hat{v}]\\
    + \lambda_{, ij} \E_{,u\lambda}(\hat{v}) + \lambda_{, i} \E_{,uu\lambda}[v_j + w_{,j} + \lambda_{,j} (\dot{u}^\ast + w_{,\lambda}), \hat{v}] + \lambda_{,i} \lambda_{,j} \E_{,u\lambda\lambda}(\hat{v})= 0,
  \end{gathered}
\end{equation}

Eqs.~\eqref{eq:20220901121940} and \eqref{eq:20220901125230} are then evaluated
at $ξ₁ = \cdots = ξ_m = 0$, delivering
\begin{equation}
  \label{eq:20220901152056}
  \underbrace{\E_{,uu}(u₀, \lambda₀; v_i, \hat{v})}_{=0 \text{ since } \hat{v} \in V}
  + \lambda_{, i} \bigl[ \underbrace{\E_{,uu}(u₀, \lambda₀; \dot{u}₀, \hat{v}) +  \E_{,u\lambda}(u₀, \lambda₀; \hat{v})}_{ = 0 \text{ from Eq.~\eqref{eq:20220901143843}}} \bigr] = 0,
\end{equation}
and
% \begin{equation}
%   \begin{gathered}[b]
%     \E_{,uuu}(u₀, \lambda₀; v_i + \lambda_{,i}\dot{u}₀, v_j + \lambda_{,j} \dot{u}₀, \hat{v}) + \lambda_{,j}\E_{,uu\lambda}(u₀, \lambda₀; v_i + \lambda_{,i} \dot{u}₀, \hat{v})\\
%     + \E_{,uu}(u₀, \lambda₀; w_{ij} + \lambda_{,ij} \dot{u}₀ + w_{,\lambda} + \lambda_{,i}\lambda_{,j} \ddot{u}₀, \hat{v})\\
%     + \lambda_{, ij} \E_{,u\lambda}(u₀, \lambda₀; \hat{v}) + \lambda_{, i} \E_{,uu\lambda}(u₀, \lambda₀; v_j + \lambda_{,j} \dot{u}₀, \hat{v}) + \lambda_{,i} \lambda_{,j} \E_{,u\lambda\lambda}(u₀, \lambda₀; \hat{v}) = 0
%   \end{gathered}
% \end{equation}
\begin{equation}
  \label{eq:20220901152145}
  \begin{gathered}[b]
    \E_{,uuu}(u₀, \lambda₀; v_i , v_j, \hat{v}) + \underbrace{\E_{,uu}(u₀, \lambda₀; w_{ij}, \hat{v})}_{=0 \text{ since } \hat{v} \in V}\\
    +\lambda_{,i} \bigl[\E_{,uuu}(u₀, \lambda₀; v_j , \dot{u}₀, \hat{v}) + \E_{,uu\lambda}[u₀, \lambda₀; v_j, \hat{v}]\bigr]\\
    +\lambda_{,j} \bigl[\E_{,uuu}(u₀, \lambda₀; v_i , \dot{u}₀, \hat{v}) + \E_{,uu\lambda}(u₀, \lambda₀; v_i, \hat{v})\bigr]\\
    +\lambda_{,ij} \bigl[ \underbrace{\E_{,uu}(u₀, \lambda₀;  \dot{u}₀, \hat{v}) + \E_{,u\lambda}(u₀, \lambda₀; \hat{v})}_{ = 0 \text{ from Eq.~\eqref{eq:20220901143843}}} \bigr]\\
    +\lambda_{,i} \lambda_{,j}\bigl[ \underbrace{\E_{,uuu}(u₀, \lambda₀; \dot{u}₀ , \dot{u}₀, \hat{v}) + 2\E_{,uu\lambda}(u₀, \lambda₀; \dot{u}₀, \hat{v}) + \E_{,u\lambda\lambda}(u₀, \lambda₀; \hat{v}) + \E_{,uu}(u₀, \lambda₀; \ddot{u}₀, \hat{v})}_{ = 0 \text{ from Eq.~\eqref{eq:20220901143902}}} \bigr] = 0
  \end{gathered}
\end{equation}

Eq.~\eqref{eq:20220901152056} is non-informative (identically
satisfied), while Eq.~\eqref{eq:20220901152145} simplifies as follows
\begin{equation}
  \begin{aligned}[b]
    \E_{,uuu}(u₀, \lambda₀; v_i , v_j, \hat{v}) + \lambda_{,i} \bigl[ \underbrace{\E_{,uuu}(u₀, \lambda₀; v_j , \dot{u}₀, \hat{v}) + \E_{,uu\lambda}(u₀, \lambda₀; v_j, \hat{v})}_{=\dot{\E}₂(\lambda₀; v_j, \hat{v})} \bigr]&\\
    +\lambda_{,j} \bigl[ \underbrace{\E_{,uuu}(u₀, \lambda₀; v_i , \dot{u}₀, \hat{v}) + \E_{,uu\lambda}(u₀, \lambda₀; v_i, \hat{v})}_{\lambda_{,j} \dot{\E}₂(\lambda₀; v_i, \hat{v})} \bigr] &= 0
  \end{aligned}
\end{equation}
and, recognizing derivatives of $\E₂$ with respect to $\lambda$, we finally get
\begin{equation}
    \E₃(\lambda₀; v_i , v_j, \hat{v}) + \lambda_{,i} \dot{\E}₂(\lambda₀; v_j, \hat{v}) + \lambda_{,j} \dot{\E}₂(\lambda₀; v_i, \hat{v}) = 0.
\end{equation}
Testing with $v_k \in V$, the above equation reads
\begin{equation}
  \E₃(\lambda₀; v_i , v_j, v_k) + \lambda_{,i} \dot{\E}₂(\lambda₀; v_j, v_k) + \lambda_{,j} \dot{\E}₂(\lambda₀; v_i, v_k) = 0,
\end{equation}
or, with Eqs.~\eqref{eq:20220524135619} and \eqref{eq:20220524135643}
\begin{equation}
  \label{eq:20220902125031}
  E_{ijk} +  F_{jk} \frac{∂\lambda}{∂ξ_i} \biggr\rvert_{ξ_1 = \cdots = ξ_m = 0} + F_{ik} \frac{∂\lambda}{∂ξ_j} \biggr\rvert_{ξ_1 = \cdots = ξ_m = 0} = 0.
\end{equation}

In order to evaluate the second order partial derivatives of $\lambda$,
Eq.~\eqref{eq:20220901125230} should be further differentiated with respect to
$ξ_k$. This leads to extremely tedious derivations, and we will adopt an
alternative approach in Sec.~\ref{sec:20220902092109}.

\paragraph{Step 3: parametrization of the bifurcated branch} The bifurcated
branch is a curve $(u, \lambda) \in ℝ ^ {m + 1}$, which is parametrized by $η$:
$[u(η), \lambda(η)]$, with $u(0) = u₀$ and $\lambda(0) = \lambda₀$; primed quantities
denoting derivatives with respect to $η$, we introduce
\begin{equation}
  \order[1]{ξ_i} = ξ_i'(0), \quad
  \order[2]{ξ_i} = ξ_i''(0), \quad \ldots, \quad
  \order[1]{\lambda} = \lambda'(0), \quad \ldots
\end{equation}
and first observe that
\begin{equation}
  \order[1]{\lambda} = \order[1]{ξ_i} \frac{∂\lambda}{∂ξ_i} \biggr\rvert_{ξ_1 = \cdots = ξ_m = 0}
\end{equation}

Multiplying both sides of Eq.~\eqref{eq:20220902125031} by
$\order[1]{ξ_i} \order{1}{ξ_j}$ therefore results in the following identity
\begin{equation}
  \begin{aligned}[b]
    0 &= E_{ijk} \order[1]{ξ_i} \order[1]{ξ_j} +  F_{jk} \order[1]{ξ_i} \order[1]{ξ_j} \frac{∂\lambda}{∂ξ_i} \biggr\rvert_{ξ_1 = \cdots = ξ_m = 0} + F_{ik} \order[1]{ξ_i} \order[1]{ξ_j} \frac{∂\lambda}{∂ξ_j} \biggr\rvert_{ξ_1 = \cdots = ξ_m = 0}\\
    &= E_{ijk} \order[1]{ξ_i} \order[1]{ξ_j} +  F_{jk} \order[1]{\lambda} \order[1]{ξ_j} + F_{ik} \order[1]{ξ_i} \order[1]{\lambda}
  \end{aligned}
\end{equation}
and, rearranging
\begin{equation}
  E_{ijk} \order[1]{ξ_j} \order[1]{ξ_k} +  2 \order[1]{\lambda} F_{ij}  \order[1]{ξ_j} = 0,
\end{equation}
to be compared with Eq.~\eqref{eq:20220524135036}. We now turn to $w$
\begin{equation}
  w'(η) = w_{,i} ξ_i' + w_{,\lambda} \lambda'
  \quad \text{and} \quad
  w''(η) = w_{,ij} ξ_i' ξ_j' + 2 w_{,i\lambda} ξ_i' \lambda' + w_{,i} ξ_i'' + w_{,\lambda\lambda} \lambda^{'2} + w_{,\lambda} \lambda''
\end{equation}
and, at $η = 0$
\begin{equation}
  w'(0) = 0 \quad \text{and} \quad w''(0) = \order[1]{ξ_i} \order[1]{ξ_j} w_{ij}  + 2 \order[1]{\lambda} \order[1]{ξ_i} w_i
\end{equation}
and we get the Taylor expansion of the bifurcated branch as $η → 0$
\begin{equation}
  u(η) = u^\ast[\lambda(η)] + \order[1]{ξ_i} v_i + \tfrac{1}{2} \bigl( \order[2]{ξ_i} v_i + \order[1]{ξ_i} \order[1]{ξ_j} w_{ij}  + 2\order[1]{\lambda} \order[1]{ξ_i} w_i\bigr) + o(η²),
\end{equation}
to be compared with Eq.~\eqref{eq:20220524134613}.

\subsection{Alternative route to the asymptotic expansions}
\label{sec:20220902092109}

Following the Appendix A of Ref.~\parencite{chak2018}, we introduce the
following parametrization of the bifurcated branch
\begin{align}
  \label{eq:20211115075817}
  \lambda &=  \lambda₀ + η \order[1]{\lambda} + \tfrac{1}{2} η² \order[2]{\lambda} + \tfrac{1}{6} η³ \order[3]{\lambda} + \cdots,\\
  \label{eq:20211115075835}
  u &= u^{\ast}(\lambda) + η \order[1]{u} + \tfrac{1}{2} η² \order[2]{u} + \tfrac{1}{6} η³ \order[3]{u} + \cdots,
\end{align}
where the parameter $η$ is not specified, but for the fact that $η = 0$
corresponds to the critical point $(u₀, \lambda₀)$. Note that, in
Eq.~\eqref{eq:20211115075835}, $u^\ast$ is evaluated at $\lambda$ rather than
$\lambda_0$.

Expressing that the energy is stationary along the bifurcated equilibrium path
leads to the identification of the coefficients $\order[k]\lambda$ and
$\order[k]u$ of the expansions~\eqref{eq:20211115075817} and
\eqref{eq:20211115075835}. In other words, the residual $\E_{, u} [u(η), \lambda(η)]$
vanishes for all $η$ close to $0$. The residual is expanded with respect to
the powers of $η$ in Appendix~\ref{sec:20211112182000} [see
Eq.~\eqref{eq:20220107080901}]. Since all the terms of this expansion must
vanish, we get successively, for all $\hat{u} \in U$
\begin{equation}
  \label{eq:20211112182917}
  \E₂(\lambda₀; \order[1]u, \hat{u}) = 0,
\end{equation}
\begin{equation}
  \label{eq:20220524133447}
  \E₃(\lambda₀; \order[1]u, \order[1]u, \hat{u}) + 2\order[1]\lambda\dot{\E}₂(\lambda₀; \order[1]u, \hat{u}) + \E₂(\lambda₀; \order[2]u, \hat{u}) = 0,
\end{equation}
\begin{equation}
  \label{eq:20220708060436}
  \begin{aligned}[b]
    \E₄(\lambda₀; \order[1]u, \order[1]u, \order[1]u, \hat{u}) + 3\E₃(\lambda₀; \order[1]u, \order[2]u, \hat{u}) + \E₂(\lambda₀; \order[3]u, \hat{u})&\\
    + 3\order[1]\lambda\dot{\E}₃(\lambda₀; \order[1]u, \order[1]u, \hat{u}) + 3\order[1]\lambda\dot{\E}₂(\lambda₀;  \order[2]u, \hat{u})&\\
    + 3(\order[1]\lambda)^2\ddot{\E}₂(\lambda₀; \order[1]u, \hat{u}) + 3\order[2]\lambda\dot{\E}₂(\lambda₀; \order[1]u, \hat{u}) & = 0.
  \end{aligned}
\end{equation}
It results from Eq.~\eqref{eq:20211112182917} that $\order[1]u \in V$. Testing
with $\hat{v} \in V$ (rather than $\hat{u} \in U$),
Eq.~\eqref{eq:20220524133447} shows that $\order[1]u$ est solves the following
variational problem: find $\order[1]u \in V$ such that
\begin{equation}
  \label{eq:20220524133816}
  \tfrac{1}{2} \E₃(\lambda₀; \order[1]u, \order[1]u, \hat{v}) + \order[1]\lambda\dot{\E}₂(\lambda₀; \order[1]u, \hat{v}) = 0,
\end{equation}
pour tout $\hat{v} \in V$. The above problem can be transformed into a system of
scalar equations. Indeed, expanding the $\order[1]u \in V$ in the basis
$(v_i)_{1 ≤ i ≤ m}$ as follows
\begin{equation}
  \label{eq:20220524133944}
  \order[1]u = \order[1]{ξ_i} v_i
\end{equation}
and plugging the definitions~\eqref{eq:20220524135619} and
\eqref{eq:20220524135643} of $E_{ijk}$ and $F_{ij}$ into
Eq.~\eqref{eq:20220524133816}
\begin{equation}
  \label{eq:20220524135036}
  \tfrac{1}{2} E_{ijk} \order[1]{ξ_j} \order[1]{ξ_k} + \order[1]\lambda F_{ij} \order[1]{ξ_j} = 0.
\end{equation}

In order to find the higher-order terms, namely $\order[2]\lambda$ et
$\order[2]u$, we postulate the following decomposition
\begin{equation}
  \order[2]u = \order[2]{ξ_i} v_i + \order[2]w,
\end{equation}
where $\order[2]w \in W$ is the orthogonal projection of $\order[2]u$ onto
$W$. Then $\E₂(\order[2]u, \hat{u}) = \E₂(\order[2]{w}, \hat{u})$ and
Eq.~\eqref{eq:20220524133447} reads
\begin{equation}
 \E₃(\lambda₀; \order[1]u, \order[1]u, \hat{u}) + 2\order[1]\lambda \dot{\E}₂(\lambda₀; \order[1]u, \hat{u}) + \E₂(\lambda₀; \order[2]w, \hat{u}) = 0,
\end{equation}
for all $\hat{u} \in U$. Testing now with $\hat{w} \in W$ (rather than
$\hat{u} \in U$), we get the following variational problem: find
$\order[2]w \in W$ such that
\begin{equation}
  \label{eq:20211210131623}
  \E₂(\lambda₀; \order[2]w, \hat{w}) + \order[1]{ξ_i} \order[1]{ξ_j} \E₃(\lambda₀; v_i, v_j, \hat{w}) + 2\order[1]\lambda \order[1]{ξ_i} \dot{\E}₂(\lambda₀; v_i, \hat{w}) = 0,
\end{equation}
for all $\hat{w} \in W$. The solution to the variational
problem~\eqref{eq:20211210131623} is expressed as a linear combination of the
$w_i$ and $w_{ij}$ [defined by the variational
problems~\eqref{eq:20220524134525} and \eqref{eq:20220519164523}]:
$\order[2]w = \order[1]{ξ_i} \order[1]{ξ_j} w_{ij} + 2\order[1]\lambda \order[1]{ξ_i}
w_i$ and
\begin{equation}
  \label{eq:20220524134613}
  \order[2]u = \order[2]{ξ_i} v_i + \order[1]{ξ_i} \order[1]{ξ_j} w_{ij} + 2\order[1]\lambda \order[1]{ξ_i} w_i.
\end{equation}

Plugging expressions~\eqref{eq:20220524133944} and \eqref{eq:20220524134613}
into Eq.~\eqref{eq:20220708060436} and taking further $\hat{u} = v_i$
[remember that $\E₂(\lambda₀; v_i, •) = 0$], we then get
% \begin{equation*}
%   \begin{aligned}[b]
%     \E₄(\lambda₀; v_i, \order[1]{ξ_j} v_j, \order[1]{ξ_k} v_k, \order[1]{ξ_l} v_l)
%     + 3\E₃(\lambda₀; v_i, \order[1]{ξ_j} v_j, \order[2]{ξ_k} v_k + \order[1]{ξ_k} \order[1]{ξ_l} w_{kl}
%     + 2\order[1]\lambda \order[1]{ξ_k} w_k)&\\
%   + 3\order[1]\lambda \dot{\E}₃(\lambda₀; v_i, \order[1]{ξ_j} v_j, \order[1]{ξ_k} v_k)
%     + 3\order[1]\lambda \dot{\E}₂(\lambda₀; v_i, \order[2]{ξ_j} v_j + \order[1]{ξ_j} \order[1]{ξ_k} w_{jk} + 2\order[1]\lambda \order[1]{ξ_j} w_j)&\\
%     + 3( \order[1]\lambda )^2 \ddot{\E}₂(\lambda₀; v_i, \order[1]{ξ_j} v_j) + 3\order[2]\lambda \dot{\E}₂(\lambda₀; v_i, \order[1]{ξ_j} v_j) &= 0
%   \end{aligned}
% \end{equation*}
% \begin{equation*}
%   \begin{aligned}[b]
%     \E₄(\lambda₀; v_i, v_j, v_k, v_l) \order[1]{ξ_j} \order[1]{ξ_k} \order[1]{ξ_l}
%     + 3\E₃(\lambda₀; v_i, v_j, v_k) \order[1]{ξ_j} \order[2]{ξ_k}
%     + 3\E₃(\lambda₀; v_i, v_j, w_{kl}) \order[1]{ξ_j} \order[1]{ξ_k} \order[1]{ξ_l}&\\
%     + 6\order[1]\lambda \E₃(\lambda₀; v_i, v_j, w_k) \order[1]{ξ_j} \order[1]{ξ_k}
%     + 3\order[1]\lambda \dot{\E}₃(\lambda₀; v_i, v_j, v_k) \order[1]{ξ_j} \order[1]{ξ_k}
%     + 3\order[1]\lambda \dot{\E}₂(\lambda₀; v_i, v_j) \order[2]{ξ_j}&\\
%     + 3\order[1]\lambda \dot{\E}₂(\lambda₀; v_i, w_{jk}) \order[1]{ξ_j} \order[1]{ξ_k}
%     + 6( \order[1]\lambda )^2 \dot{\E}₂(\lambda₀; v_i, w_j) \order[1]{ξ_j}
%     + 3( \order[1]\lambda )^2 \ddot{\E}₂(\lambda₀; v_i, v_j) \order[1]{ξ_j}&\\
%     + 3\order[2]\lambda \dot{\E}₂(\lambda₀; v_i, v_j) \order[1]{ξ_j} &= 0
%   \end{aligned}
% \end{equation*}
\begin{equation*}
  \begin{aligned}[b]
    \bigl[\E₄(\lambda₀; v_i, v_j, v_k, v_l) + 3\E₃(\lambda₀; v_i, v_j, w_{kl})\bigr] \order[1]{ξ_j} \order[1]{ξ_k} \order[1]{ξ_l}&\\
    + 3\order[1]\lambda \bigl[2\E₃(\lambda₀; v_i, v_j, w_k) + \dot{\E}₃(\lambda₀; v_i, v_j, v_k) + \dot{\E}₂(\lambda₀; v_i, w_{jk}) \bigr] \order[1]{ξ_j} \order[1]{ξ_k}&\\
    + 3 \bigl\{ ( \order[1]\lambda )^2 \bigl[ 2 \dot{\E}₂(\lambda₀; v_i, w_j) + \ddot{\E}₂(\lambda₀; v_i, v_j) \bigr] + \order[2]\lambda \dot{\E}₂(\lambda₀; v_i, v_j) \bigr\}\order[1]{ξ_j}&\\
    + 3\bigl[\E₃(\lambda₀; v_i, v_j, v_k) \order[1]{ξ_k} + \order[1]\lambda \dot{\E}₂(\lambda₀; v_i, v_j)\bigr] \order[2]{ξ_j} &= 0
  \end{aligned}
\end{equation*}
It results from the variational problems \eqref{eq:20220524134525} and
\eqref{eq:20220519164523} that
\begin{equation*}
  \E₃(\lambda₀; v_i, v_j, w_k) = -\E₂(\lambda₀ ; w_{ij}, w_k) = \dot{\E}₂(\lambda₀; v_k, w_{ij}),
\end{equation*}
therefore
\begin{equation*}
  \begin{aligned}[b]
    \E₃(\lambda₀; v_i, v_j, w_k) \order[1]{ξ_j} \order[1]{ξ_k} &= \tfrac{1}{2} \bigl[ \E₃(\lambda₀; v_i, v_j, w_k) + \E₃(\lambda₀; v_i, v_k, w_j)\bigr] \order[1]{ξ_j} \order[1]{ξ_k}\\
                                    &= \tfrac{1}{2} \bigl[ \dot{\E}₂(\lambda₀; v_k, w_{ij}) + \dot{\E}₂(\lambda₀; v_j, w_{ik}) \bigr] \order[1]{ξ_j} \order[1]{ξ_k}.
  \end{aligned}
\end{equation*}
Similarly,
\begin{equation*}
  \begin{aligned}[b]
    \dot{\E}₂(\lambda₀; v_i, w_j) &= -\E₂(\lambda₀; w_i, w_j) = -\E₂(\lambda₀; w_j, w_i) = \dot{\E}₂(\lambda₀; v_j, w_i)\\
                           &= \tfrac{1}{2} \bigl[ \dot{\E}₂(\lambda₀; v_i, w_j) + \dot{\E}₂(\lambda₀; v_j, w_i) \bigr].
  \end{aligned}
\end{equation*}
% \begin{equation*}
%   \begin{aligned}[b]
%     \bigl[\E₄(\lambda₀; v_i, v_j, v_k, v_l) + \E₃(\lambda₀; v_i, v_j, w_{kl}) + \E₃(\lambda₀; v_i, v_k, w_{jl}) + \E₃(\lambda₀; v_i, v_l, w_{jk}) \bigr] \order[1]{ξ_j} \order[1]{ξ_k} \order[1]{ξ_l}&\\
%   + 3\order[1]\lambda \bigl[\dot{\E}₃(\lambda₀; v_i, v_j, v_k) + \dot{\E}₂(\lambda₀; v_i, w_{jk}) + \dot{\E}₂(\lambda₀; v_j, w_{ik}) + \dot{\E}₂(\lambda₀; v_k, w_{ij}) \bigr] \order[1]{ξ_j} \order[1]{ξ_k}&\\
%   + 3( \order[1]\lambda )^2 \bigl[ \ddot{\E}₂(\lambda₀; v_i, v_j) + \dot{\E}₂(\lambda₀; v_i, w_j) + \dot{\E}₂(\lambda₀; v_j, w_i) \bigr] \order[1]{ξ_j}&\\
%   + 3\bigl[\E₃(\lambda₀; v_i, v_j, v_k) \order[1]{ξ_k} + \order[1]\lambda \dot{\E}₂(\lambda₀; v_i, v_j)\bigr] \order[2]{ξ_j} + 3\order[2]\lambda \dot{\E}₂(\lambda₀; v_i, v_j) \order[1]{ξ_j} &= 0
%   \end{aligned}
% \end{equation*}

Finally, the definitions \eqref{eq:20220615063626}, \eqref{eq:20220615063633},
\eqref{eq:20220524135619}, \eqref{eq:20220524135553} and
\eqref{eq:20220524135643} of $E_{ijk}$, $E_{ijkl}$, $F_{ij}$,
$\mathring{E}_{ijk}$ and $\mathring{F}_{ij}$ lead to the following compact
bifurcation equation
\begin{equation}
  \label{eq:20220601070917}
  \tfrac{1}{3} E_{ijkl} \order[1]{ξ_j} \order[1]{ξ_k} \order[1]{ξ_l} + \order[1]\lambda \bigl( \mathring{E}_{ijk} \order[1]{ξ_k} + \order[1]\lambda \mathring{F}_{ij} \bigr)\order[1]{ξ_j} + \bigl(E_{ijk} \order[1]{ξ_k} + \order[1]\lambda F_{ij}\bigr) \order[2]{ξ_j} + \order[2]\lambda F_{ij} \order[1]{ξ_j} = 0.
\end{equation}

In order to analyse the stability of the bifurcated branches thus found, one
must look at the Hessian of the energy. It is first observed that, on the
fundamental branch
\begin{equation}
 \E₂(\lambda; \hat{u}, \hat{v}) = \E₂(\lambda₀; \hat{u}, \hat{v}) + \bigl(\lambda - \lambda₀\bigr) \dot{\E}₂(\lambda₀; \hat{u}, \hat{v}) + o(\lambda - \lambda₀).
\end{equation}

In what follows, it will be assumed that $\dot{\E}₂(\lambda₀)≠0$ and that $\E₂(\lambda)$
(which is positive definite over $V$ for $\lambda<\lambda₀$ and null for $\lambda=\lambda₀$) is
negative definite for $\lambda>\lambda₀$ sufficiently small (the fundamental branch is
strictly unstable beyond the critical load). From the above expansion, it
results that $\dot{\E}₂(\lambda₀)$ is negative definite over $V$. In other words,
$-F_{ij}$ is a positive definite tensor. The asymptotic expansion of the
Hessian of the energy along the bifurcated branch is derived in
appendix~\ref{sec:20220616055207}. For all $\hat{u}, \hat{v} \in U$
\begin{equation}
  \label{eq:20220531054247}
  \begin{aligned}[b]
    \E_{, uu}[u(η), \lambda(η); \hat{u}, \hat{v}] ={}
    & \E₂(\lambda₀ ; \hat{u}, \hat{v}) + η \bigl[\E₃(\lambda₀ ; \order[1]u, \hat{u}, \hat{v})  + \order[1]\lambda \dot{\E}₂(\lambda₀; \hat{u}, \hat{v})\bigr]\\
    &+ \tfrac{1}{2} η² \bigl[\E₄(\lambda₀; \order[1]u, \order[1]u, \hat{u}, \hat{v}) + \E₃(\lambda₀; \order[2]u, \hat{u}, \hat{v})\\
    & + 2\order[1]\lambda \dot{\E}₃(\lambda₀; \order[1]u, \hat{u}, \hat{v}) + ( \order[1]\lambda )² \ddot{\E}₂(\lambda₀; \hat{u}, \hat{v})\\
    & + \order[2]\lambda \dot{\E}₂(\lambda₀; \hat{u}, \hat{v}) \bigr] + o(η²).
  \end{aligned}
\end{equation}

Stability analysis is performed by means of the eigenvalues $α \in \reals$ and
eigenvectors $x \in U$ of the Hessian
\begin{equation}
  \label{eq:20220617074949}
  \E_{, u u} [u(η), \lambda(η); x, \hat{u}] = α 〈 x, \hat{u} 〉 \quad \text{for all} \quad \hat{u} \in V,
\end{equation}
where $α$ and $x$ are expanded to second order in $η$
\begin{equation}
  \label{eq:20220617064633}
  α = \order[0]α + η \order[1]α + \tfrac{1}{2} η² \order[2]α + o(η²)
  \quad \text{and} \quad
  x = \order[0]x + η \order[1]x + \tfrac{1}{2} η² \order[2]x + o(η²).
\end{equation}

The following results are proved in Appendix~\ref{sec:20220616074108}: first,
$(\order[0]α, x_0)$ is necessarily an eigenpair of $\E₂(\lambda₀)$. Since $\E₂ (\lambda₀)$ is
positive, $\order[0]α ≥ 0$. If $\order[0]α>0$, then $α>0$ in the neighborhood of
$\lambda₀$. Potentially unstable modes are therefore such that $\order[0]α=0$. In other
words, $\order[0]x \in V$ and
\begin{equation}
  \label{eq:20220904160057}
  \order[0]x = \order[0]{χ_i} v_i
\end{equation}
furthermore, $(\order[1]α, \order[0]{χ_i})$ is an eigenpair of the symmetric
tensor $(E_{ijk} \order[1]{ξ_k} + \order[1]\lambda F_{ij})$
\begin{equation}
  \label{eq:20220609133608}
  \bigl(E_{ijk} \order[1]{ξ_k} + \order[1]\lambda F_{ij} \bigr) \order[0]{χ_j} = \order[1]α \order[0]{χ_i}.
\end{equation}
As for the higher order terms, it is also found that
\begin{equation}
  \label{eq:20220609133629}
  \order[1]x = \order[1]{χ_i} v_i +  \order[0]{χ_i} \order[1]{ξ_j} w_{i j} + \order[1]\lambda \order[0]{χ_i} w_i
\end{equation}
and
\begin{equation}
  \label{eq:20220616082923}
  \begin{aligned}[b]
    \bigl[E_{ijkl} \order[1]{ξ_k} \order[1]{ξ_l} + \order[1]\lambda\bigl(2 \mathring{E}_{ijk} \order[1]{ξ_k} + \order[1]\lambda \mathring{F}_{ij}\bigr) + E_{ijk} \order[2]{ξ_k}
    + \order[2]\lambda F_{ij} \bigr] \order[0]{χ_j} &\\
    + 2\bigl(E_{ijk}  \order[1]{ξ_k} + \order[1]\lambda F_{ij} \bigr) \order[1]{χ_j}
    & = 2\order[1]α\order[1]{χ_i} + \order[2]α \order[0]{χ_i}.
  \end{aligned}
\end{equation}

Finally, to close this analysis of the bifurcated branches, the following
asymptotic expansion of the energy is derived in
Appendix~\ref{sec:20220525053434}
\begin{equation}
  \label{eq:20220525053600}
  \begin{aligned}[b]
    \E[u(η), \lambda(η)] ={} & \E\{u^{\ast}[\lambda(η)], \lambda(η)\} + \tfrac{1}{6} \order[1]\lambda η³ F_{i j} \order[1]{ξ_i} \order[1]{ξ_j} + \tfrac{1}{24} η⁴ \bigl\{E_{ijkl} \order[1]{ξ_i} \order[1]{ξ_j} \order[1]{ξ_k} \order[1]{ξ_l}\\
    & + 4\order[1]\lambda \mathring{E}_{ijk} \order[1]{ξ_i} \order[1]{ξ_j} \order[1]{ξ_k} + 6 \bigl[( \order[1]\lambda )^2 \mathring{F}_{ij} + \order[2]\lambda F_{ij}\bigr] \order[1]{ξ_i} \order[1]{ξ_j}\bigr\} + o(η⁴).
  \end{aligned}
\end{equation}

\section{Discussion}

In this section, we discuss the two main cases of bifurcations, namely
\emph{asymmetric} and \emph{symmetric}. In each case, we analyse the stability
of the bifurcated branch.

\begin{remark}
  The boundary case is unclear to me. I think that whether a bifurcation is
  symmetric or asymmetric should depend on the value of $\order[1]\lambda$ only. If
  $\order[1]\lambda ≠ 0$, the bifurcated branch is \emph{asymmetric}. Conversely, if
  $\order[1]\lambda = 0$ and $\order[2]\lambda ≠ 0$, then the bifurcated branch is \emph{symmetric}.

  In the literature, the discussion is placed on $E_{ijk}$. If $\order[1]\lambda ≠ 0$,
  surely one of the $E_{ijk}$ is non-zero also. However, I believe it is
  \emph{not} a sufficient condition: one of the bifurcated branches could be
  symmetric $(\order[1]\lambda = 0)$, even if all $E_{ijk}$ are not null. It is true
  however that \emph{all} bifurcated branches are symmetric if, and only if,
  $E_{ijk}=0$ for all $i, j, k = 1, \ldots, m$. Therefore, the two cases
  that will be discussed below are: (1) one of the bifurcated branches is
  asymmetric and (2) all bifurcated branches are symmetric. The mixed case ``one
  of the bifurcated branches is symmetric'' will \emph{not} be discussed.
\end{remark}

\subsection{Asymmetric bifurcated branch}

We first consider the situation where $\order[1]\lambda ≠ 0$ on the bifurcated
branch. The bifurcation equation~\eqref{eq:20220524135036} shows that
necessarily, $E_{ijk}$ is not identically nul. This equation has at most
$(2^m - 1)$ pairs of real solutions $(\order[1]\lambda, \order[1]u)$ et
$(- \order[1]\lambda, - \order[1]u)$; furthermore, multiplication by
$\order[1]{ξ_i}$ shows that
\begin{equation}
  \label{eq:20220801085236}
  \order[1]\lambda = -\frac{E_{ijk} \order[1]{ξ_i} \order[1]{ξ_j} \order[1]{ξ_k}}{2 F_{ij} \order[1]{ξ_i} \order[1]{ξ_j}}.
\end{equation}

\begin{remark}
  I can't prove that the bifurcation equation~\eqref{eq:20220524135036} has at
  most $(2^m - 1)$ pairs of real solutions.
\end{remark}

Along the bifurcated branch, we have $\lambda = \lambda₀ + η \order[1]\lambda + o(η)$, and $η$ can be
eliminated. In other words, $η=\lambda$ ($\order[1]\lambda=1$ and $\order[2]\lambda = \order[3]\lambda = \cdots = 0$) can
be selected as a parameter. It is therefore possible to express the bifurcated
branch as a function of $\lambda$: $u(\lambda)$. For example, combining
Eqs.~\eqref{eq:20220524133816} and \eqref{eq:20220531054247}, we find that
\begin{equation}
  \begin{aligned}[b]
    \E_{, uu}[u(η), \lambda(η); \order[1]u, \order[1]u]
    &= η \bigl[\E₃(\lambda₀ ; \order[1]u, \order[1]u, \order[1]u)  + \order[1]\lambda \dot{\E}₂(\lambda₀; \order[1]u, \order[1]u)\bigr] + o(η)\\
    &= - η \order[1]\lambda \dot{\E}₂(\lambda₀; \order[1]u, \order[1]u) + o(η),
  \end{aligned}
\end{equation}
or
\begin{equation}
  \label{eq:20220819160235}
  \E_{, uu}[u(\lambda), \lambda; \order[1]u, \order[1]u] = -\bigl( \lambda - \lambda₀ \bigr) \dot{\E}₂(\lambda₀; \order[1]u, \order[1]u) + o(\lambda - \lambda₀).
\end{equation}

For $\lambda < \lambda₀$, the above quantity is \emph{negative} (since $\dot{\E}₂$ is
negative definite). In other words

\begin{center}
  \framebox{For asymmetric bifurcations, below the critical load, the bifurcated
    branch is unstable}
\end{center}

To investigate the stability above the critical load, we need to analyse the
sign of the eigenvalues $α$ of the Hessian. At first order,
$α = η \order[1]α + o(η)$, where $\order[1]α$ is an eigenvalue of
$(E_{ijk} \order[1]{ξ_k} + \order[1]\lambda F_{ij})$. Let $α_{\min}$ and $α_{\max}$ be the minimum
and maximum eigenvalues of this second-order tensor. Three cases must be
discussed
\begin{enumerate}
\item If $α_{\min} α_{\max} > 0$, then $(E_{ijk} \order[1]{ξ_k} + \order[1]\lambda F_{ij})$ is
  positive or negative definite: all eigenvalues have the same sign,
  $\epsilon \in \{-1, +1\}$. Then the sign of the eigenvalues $α$ of the
  Hessian is $\epsilon η$ and there is a stability switch at the critical
  load. Since the bifurcated branch is unstable \emph{below} the critical load,
  this means that it is \emph{stable} above the critical load.
\item If $α_{\min} α_{\max} < 0$, then the extremal eigenvalues of the Hessian
  are $η α_{\min}$ and $η α_{\max}$, the product of which is
  $η² α_{\min} α_{\max} < 0$. The bifurcated branch is \emph{unstable} for all
  values of $\lambda$.
\item If $α_{\min} α_{\max} = 0$, the analysis is inconclusive.
\end{enumerate}

To close this section, it is observed that the dominant term of the
expansion~\eqref{eq:20220525053600} of the potential energy along the bifurcated
branch is of the third order in $η$
\begin{equation}
  \E[u(η), \lambda(η)] = \E\{u^{\ast}[\lambda(η)], \lambda(η)\} + \tfrac{1}{6} \order[1]\lambda η³ F_{i j} \order[1]{ξ_i} \order[1]{ξ_j} + o(η³).
\end{equation}

Eliminating $\lambda$ and plugging expression~\eqref{eq:20220801085236} of $\order[1]\lambda$
delivers the expression of the potential energy, where $\lambda$ is the parameter
\begin{equation}
  \begin{aligned}[b]
    \E[u(\lambda), \lambda] &= \E[u^{\ast}(\lambda), \lambda] + \frac{\bigl(\lambda - \lambda₀\bigr)³}{6\bigl( \order[1]\lambda \bigr)^2} F_{i j} \order[1]{ξ_i} \order[1]{ξ_j} + o(\lambda³)\\
    &= \E[u^{\ast}(\lambda), \lambda] + \frac{2 \bigl( F_{i j} \order[1]{ξ_i} \order[1]{ξ_j} \bigr)³}{3 \bigl( E_{ijk} \order[1]{ξ_i} \order[1]{ξ_j} \order[1]{ξ_k} \bigr)²} \bigl(\lambda - \lambda₀\bigr)³ + o(\lambda³).
  \end{aligned}
\end{equation}

Recalling that $F_{i j} \order[1]{ξ_i} \order[1]{ξ_j} < 0$, it is found that, above the critical
load, the potential energy is \emph{smaller} along the bifurcated branch than
along the fundamental branch.

\begin{remark}
  As expected, the above expression does not depend on the scaling of $\order[1]u$ (of the $\order[1]{ξ_i}$).
\end{remark}
\begin{remark}
  It has been shown in Sec.~\ref{sec:20220802061621} that, when $E_{ijk}$ is
  not identically null, the bifurcation point is \emph{unstable}.
\end{remark}

\subsection{A particular case of symmetric bifurcation}

We now consider the case $E_{ijk}=0$ for all $i, j, k = 1, \ldots, m$. Then
[see Eq.~\eqref{eq:20220524135036}] $\order[1]\lambda = 0$ on \emph{all} bifurcated
branches. It is assumed that, on the bifurcated branch under consideration, the
next term of the expansion of $\lambda$ is non-zero: $\order[2]\lambda ≠ 0$. The bifurcation is
\emph{symmetric}, and the bifurcation equation~\eqref{eq:20220601070917} reduces
to
\begin{equation}
  \label{eq:20220801092222}
  \tfrac{1}{3} E_{ijkl} \order[1]{ξ_j} \order[1]{ξ_k} \order[1]{ξ_l}  + \order[2]\lambda F_{ij} \order[1]{ξ_j} = 0,
\end{equation}
which has at most $(3^m - 1) / 2$ pairs of real solutions $(\order[2]\lambda, \order[1]u)$ and
$(- \order[2]\lambda, - \order[1]u)$. Upon multiplication by $\order[1]{ξ_i}$, the above equation delivers
the following expression of $\order[2]\lambda$
\begin{equation}
  \label{eq:20220801093236}
  \order[2]\lambda = -\frac{E_{ijkl} \order[1]{ξ_i} \order[1]{ξ_j} \order[1]{ξ_k} \order[1]{ξ_l}}{3 F_{ij} \order[1]{ξ_i} \order[1]{ξ_j}}.
\end{equation}

Since $F_{ij} \order[1]{ξ_i} \order[1]{ξ_j} < 0$, $\order[2]\lambda$ has the same sign as
$E_{ijkl}\order[1]{ξ_i} \order[1]{ξ_j} \order[1]{ξ_k} \order[1]{ξ_l}$. In other words, if
$E_{ijkl}\order[1]{ξ_i} \order[1]{ξ_j} \order[1]{ξ_k} \order[1]{ξ_l} > 0$, (resp. $<0$) then the bifurcated branch
exists above (resp. below) the critical load $\lambda₀$ only.

\begin{remark}
  I can't prove that the bifurcation equation~\eqref{eq:20220801092222} has at
  most $(3^m - 1) / 2$ pairs of real solutions.
\end{remark}

Turning now to the eigenpairs of the Hessian of the energy along
the bifurcated branch, Eq.~\eqref{eq:20220609133608} shows that $\order[1]α = 0$. Then
$α = \order[2]α η² / 2 + o(η²)$ and, from Eq.~\eqref{eq:20220616082923}
\begin{equation}
  \bigl(E_{ijkl} \order[1]{ξ_k} \order[1]{ξ_l} + \order[2]\lambda F_{ij} \bigr) \order[0]{χ_j} = \order[2]α \order[0]{χ_i}.
\end{equation}

If $(E_{ijkl} \order[1]{ξ_k} \order[1]{ξ_l} + \order[2]\lambda F_{ij} )$ is positive definite, then the
bifurcated branch is stable (note that, in that case, the bifurcated branch
exists above the critical load only). If one of the eigenvalues of this tensor
is $<0$, then the bifurcated branch is unstable. The stability is undecided
when all eigenvalues are $≥ 0$.

\begin{remark}
  Note that, from Eq.~\eqref{eq:20220801092222},
  \begin{equation}
    E_{ijkl} \order[1]{ξ_i} \order[1]{ξ_j} \order[1]{ξ_k} \order[1]{ξ_l} + \order[2]\lambda F_{ij} \order[1]{ξ_i} \order[1]{ξ_j} = \tfrac{2}{3} E_{ijkl} \order[1]{ξ_i} \order[1]{ξ_j} \order[1]{ξ_k} \order[1]{ξ_l}
  \end{equation}

\end{remark}

To conclude this section, it is observed that, when $\order[1]\lambda = 0$, the dominant
term of the potential energy along the bifurcated branch is of the fourth order
[see Eq.~\eqref{eq:20220525053600}]. Combining with Eq.~\eqref{eq:20220801093236},
\begin{equation}
  \label{eq:20220801094437}
  \begin{aligned}[b]
    \E[u(η), \lambda(η)]
    &= \E\{u^{\ast}[\lambda(η)], \lambda(η)\} + \tfrac{1}{24} η⁴ \bigl(E_{ijkl} \order[1]{ξ_i} \order[1]{ξ_j} \order[1]{ξ_k} \order[1]{ξ_l}  + 6  \order[2]\lambda F_{ij} \order[1]{ξ_i} \order[1]{ξ_j}\bigr) + o(η⁴)\\
    &= \E\{u^{\ast}[\lambda(η)], \lambda(η)\} - \tfrac{1}{24} η⁴ E_{ijkl} \order[1]{ξ_i} \order[1]{ξ_j} \order[1]{ξ_k} \order[1]{ξ_l} + o(η⁴).
  \end{aligned}
\end{equation}

The expansion $\lambda = \lambda₀ + \order[2]\lambda η² / 2 + o(η²)$ can be inverted as follows
\begin{equation}
  η⁴ = \frac{4 \bigl(\lambda - \lambda₀\bigr)²}{\bigl( \order[2]\lambda \bigr)^2} + o(\lambda²) = \frac{36 \bigl( F_{ij} \order[1]{ξ_i} \order[1]{ξ_j} \bigr)²}{\bigl( E_{ijkl} \order[1]{ξ_i} \order[1]{ξ_j} \order[1]{ξ_k} \order[1]{ξ_l} \bigr)²} \bigl( \lambda - \lambda₀ \bigr)²
\end{equation}
and expression~\eqref{eq:20220801094437} reads
\begin{equation}
  \E[u(η), \lambda(η)] = \E\{u^{\ast}[\lambda(η)], \lambda(η)\} - \frac{3 \bigl( F_{ij} \order[1]{ξ_i} \order[1]{ξ_j} \bigr)²}{2 E_{ijkl} \order[1]{ξ_i} \order[1]{ξ_j} \order[1]{ξ_k} \order[1]{ξ_l} } \bigl( \lambda - \lambda₀ \bigr)²  + o(\lambda²).
\end{equation}

Again, the above expression does not depend on the scaling of $\order[1]u$ (of the
$\order[1]{ξ_i}$). Note that, if $E_{ijkl} \order[1]{ξ_i} \order[1]{ξ_j} \order[1]{ξ_k} \order[1]{ξ_l} > 0$, then only loads
that are greater than the critical load can be reached on the bifurcated branch,
where the energy is lower than the fundamental branch.

\medskip

The above discussion simplifies considerably when there is only one buckling
mode ($m = 1$). This is addressed in the next section.

\section{The case of a single mode}

In this section, we discuss the case $m = 1$; all tensors considered above
($F_{ij}$, $E_{ijk}$, $E_{ijkl}$) then reduce to simple scalars. To avoid
ambiguity, indices are kept: $F_{11}$, $E_{11}$, $E_{11}$. Since
$\dot{\E}₂(\lambda₀)$ is negative definite over $V$, we have $F_{11} < 0$.

It is first observed that the following conditions are \emph{necessary} to
ensure stability of the critical point
\begin{equation}
  E_{111} = 0 \quad \text{and} \quad E_{1111} ≥ 0,
\end{equation}
which shows that \emph{asymmetric} bifurcation points are always
\emph{unstable}.

\subsection{Asymmetric bifurcations}

We first consider the case $E_{111} ≠ 0$. Owing to the discussion above, the
bifurcation point is unstable. Setting $\order[1]\lambda = 1$,
Eq.~\eqref{eq:20220524135036} delivers
\begin{equation}
  E_{111} \order[1]ξ_1 + 2F_{11} = 0 \quad \text{and} \quad u(\lambda) = u^\ast(\lambda) - \frac{2F_{11}}{E_{111}} \bigl( \lambda - \lambda_0 \bigr) v_1 + o(\lambda - \lambda_0).
\end{equation}

Furthermore, the hessian of the energy along the bifurcated branch is retrieved
from Eq.~\eqref{eq:20220819160235}
\begin{equation}
  \begin{aligned}[b]
    \E_{, uu}[u(η), \lambda(η), v_1, v_1] &= η \bigl(E_{111} \order[1]{ξ_1} + \order[1]\lambda F_{11}\bigr) + o(η) = -2 η F_{11} + o(η)\\
    &= -2 F_{11} \bigl( \lambda - \lambda₀ \bigr) + o(\lambda - \lambda₀).
  \end{aligned}
\end{equation}

Asymmetric bifurcations branches are \emph{unstable} for $\lambda ≤ \lambda_0$ and
\emph{stable} for $\lambda > \lambda₀$ (stability switch).

\subsection{Symmetric bifurcations}

We now consider the case $E_{111}=0$. From the general discussion of
Sec.~\ref{sec:20220802061621}, the bifurcation point is \emph{stable} if
$E_{1111} > 0$ and \emph{unstable} if $E_{1111} < 0$. The bifurcation
equation~\eqref{eq:20220801092222} reduces to
\begin{equation}
  E_{1111} \bigl( \order[1]{ξ_1} \bigr)^2 + 3\order[2]\lambda F_{11} = 0,
\end{equation}
which in particular shows that $\order[2]\lambda$ has the same sign as
$E_{1111}$. Since the expansion of $\lambda$ reads:
$\lambda = \lambda_0 + \order[2]\lambda η^2 / 2$, the bifurcation branch exists only for loads
\emph{above} the critical load ($\lambda ≥ \lambda_0$) if $E_{1111} > 0$ and only for
loads \emph{below} the critical load ($\lambda ≤ \lambda_0$) if $E_{1111} < 0$.

From Eq.~\eqref{eq:20220531054247}, the hessian of the energy along the
bifurcated branch reads
\begin{equation*}
  \E_{, uu}[u(η), \lambda(η); v_1, v_1] = \tfrac{1}{2} η² \bigl[ E_{1111}\bigl(ξ_1^1\bigr) ^2 + \order[2]\lambda F_{11} \bigr] + o(η²) = - η² \order[2]\lambda F_{11} + o(η²),
\end{equation*}
which has the sign of $\order[2]\lambda$. Therefore the Hessian is positive
(resp. negative) definite if $E_{1111} > 0$ (resp $< 0$).

To sum up, if $E_{1111} > 0$, then the bifurcation branch (including the
critical point) is \emph{stable} and exists only for loads greater than the
critical load. Conversely, if $E_{1111} < 0$, then the bifurcation branch
(including the critical point) is \emph{unstable} and exists only for loads
lower than the critical load.

\appendix

\section{Some useful results from multilinear algebra}

\subsection{Kernel of a bilinear, symmetric, positive form}

In this section, $\mathcal{B}$ denotes a bilinear, symmetric, positive form
over the vector space $U$. Its kernel $\ker \mathcal{B}$ is defined as
follows
\begin{equation}
 \ker \mathcal{B}= \bigl\{ u \in U, \mathcal{B}(u, u) = 0 \bigr\} .
\end{equation}

\begin{theorem}
  The kernel of a bilinear, symmetric, positive form is a vector subspace.
\end{theorem}
\begin{proof}
  We must show that, for all $u, v \in\ker \mathcal{B}$ and $α \in \reals$,
  $w = u + α v \in \ker \mathcal{B}$, in other words, it must be shown that
  $\mathcal{B}(w, w) = 0$. From the bilinearity and symmetry of
  $\mathcal{B}$
 \begin{equation*}
   \mathcal{B}(w, w) = \mathcal{B}(u + α v, u + α v)
   = \mathcal{B}(u, u) + 2 α \mathcal{B}(u, v) + α² \mathcal{B}(v, v),
 \end{equation*}

 Since $u, v \in \ker\mathcal{B}$, the first and the last term vanish, and the above identity reduces to
 \begin{equation*}
   \mathcal{B}(w, w) = 2α \mathcal{B}(u, v)
 \end{equation*}

 The bilinear form $\mathcal{B}$ is positive, therefore the left-hand side is
 positive, \emph{for all values of $α \in \reals$}. The quantity
 $\mathcal{B}(u, v) = 0$ is necessarily null, and $\mathcal{B}(w, w) = 0$,
 which completes the proof.
\end{proof}

\begin{theorem}
  Let $\mathcal{B}$ be a bilinear, symmetric, positive form over the vector space $U$ and $u \in U$. Then
 \begin{equation*}
  u \in \ker\mathcal{B} \quad \text{iff} \quad \text{pour tout } v \in U, \mathcal{B}(u, v) = 0.
 \end{equation*}
\end{theorem}

\begin{proof}
  If for all $v \in U$, $\mathcal{B}(u, v) = 0$, then in particular
  $\mathcal{B}(u, u) = 0$ and $u \in \ker \mathcal{B}$.

  Converely, let $u \in \ker \mathcal{B}$, $v \in U$ et $α \in
  \reals$. Similarly to the previous proof, we write that
  $\mathcal{B}(w, w) ≥ 0$, with $w = α u + v$
  \begin{equation*}
    \mathcal{B}(w, w) = \mathcal{B}(u, u) + 2 α \mathcal{B}(u, v) +\mathcal{B}(v, v) = 2 α \mathcal{B}(u, v) +\mathcal{B}(v, v) ≥ 0,
  \end{equation*}
  ($\mathcal{B}(u, u) = 0$ since $u \in \ker \mathcal{B}$). The above
  expression is of degree 1 in $α$, with a constant sign. Therefore the linear
  term in $α$ must vanish: $\mathcal{B}(u, v) = 0$.
\end{proof}

\subsection{On trilinear, symmetric forms}

In this section, $𝒯$ denotes a trilinear, symmetric form over the vector space
$U$.

\begin{theorem}
  \label{thr:20220802112835}
  Let $𝒯$ be a trilinear, symmetric form, such that
  \begin{equation}
    \label{eq:20220802111745}
    𝒯(u, u, u) = 0 \quad \text{for all} \quad u \in U.
  \end{equation}
  Then
  \begin{equation}
    𝒯(u, v, w) = 0 \quad \text{for all} \quad u, v, w \in U.
  \end{equation}
\end{theorem}
\begin{proof}
  The form $𝒯$ being trilinear and symmetric, we have, for all $u, v, w \in U$
  and $α, β \in ℝ$
  \begin{multline*}
    𝒯(u + αv + βw, u + αv + βw, u + αv + βw) = 𝒯(u, u, u) + 3α 𝒯(u, u, v)\\
    + 3β 𝒯(u, u, w) + 3α² 𝒯(u, u, v) + 6 α β 𝒯(u, v, w) + 3 β² 𝒯(u, u, w)\\
    + α³ 𝒯(v, v, v) + 3 α² β 𝒯(v, v, w) + 3 α β² 𝒯(v, w, w) + β³ 𝒯(w, w, w)
  \end{multline*}
  and, upon simplification using Eq.~\eqref{eq:20220802111745}
  \begin{multline}
    \label{eq:20220802112309}
    3α 𝒯(u, u, v) + 3β 𝒯(u, u, w) + 3α² 𝒯(u, v, v) + 6 α β 𝒯(u, v, w)\\
    + 3 β² 𝒯(u, w, w) + 3 α² β 𝒯(v, v, w) + 3 α β² 𝒯(v, w, w) = 0.
  \end{multline}
  In particular taking successively $α = ±1$, $β = 0$ and $w = 0$ delivers
  \begin{equation*}
    ±3 𝒯(u, u, v) + 3 𝒯(u, u, v) = 0 \quad \text{for all} \quad u, v \in U,
  \end{equation*}
  from which it results that $𝒯(u, u, v) = 0$ for all $u, v \in U$. Plugging
  into Eq.~\eqref{eq:20220802112309} with $α = β = 1$ results in:
  $𝒯(u, v, w) = 0$ for all $u, v, w \in U$.
\end{proof}

\section{Asymptotic expansions along a bifurcated branch}
\label{sec:20220905060440}

In this section, the asymptotic expansions along the bifurcated branch of the
energy, its residual and its hessian are derived.

\subsection{Principle of the derivation}
\label{sec:20220107121442}
% 02/06/2022 — 099042106e938251657847daca64c8fcbaa833c3
%
% Validation des calculs de ce paragraphe

Introducing $\Lambda$ and $U$, which are functions of $η$ only,
\begin{align}
  \label{eq:20211112155446}
  \Lambda(η) & = \lambda(η) - \lambda₀ = η \order[1]\lambda + \tfrac{1}{2} η² \order[2]\lambda + \tfrac{1}{6} η³ \order[3]\lambda + \cdots,\\
  \label{eq:20211112113028}
  U(η) & = u(η) - u^{\ast}[\lambda(η)] = η \order[1]u + \tfrac{1}{2} η² \order[2]u + \tfrac{1}{6} η³ \order[3]u + \cdots,
\end{align}
the functional $\mathcal{F}(u, \lambda)$ is evaluated along the bifurcated branch,
thus defining the function $f(η)$
\begin{equation*}
  f(η) = F\{ u^{\ast} [\lambda₀ + \Lambda(η)] + U(η), \lambda₀ + \Lambda(η) \}.
\end{equation*}

We seek the Taylor expansion of $f$ at $η = 0$, which requires the
successive derivatives of $f$. To this end, it is convenient to introduce the
function $F$ defined as follows
\begin{equation*}
  F(η, \lambda) =\mathcal{F}[u^{\ast}(\lambda) + U(η), \lambda],
\end{equation*}
where $\lambda$ and $η$ are temporarily seen as independent variables. Since
$f(η) = F[η, \lambda₀ + \Lambda(η)]$, the following identities hold
\begin{gather*}
  f'(η) = ∂_{η} F + \Lambda' ∂_{\lambda} F,\\
  f''(η) = ∂_{ηη}² F + 2\Lambda' ∂_{η\lambda}²F + \Lambda'^2 ∂_{\lambda\lambda}² F + \Lambda'' ∂_{\lambda} F,\\
  \begin{aligned}[b]
    f'''(η) ={}
    & ∂_{ηηη}³ F + 3\Lambda' ∂_{ηη\lambda}³F + 3\Lambda'^2 ∂_{η\lambda\lambda}³F + \lambda'^3 ∂_{\lambda\lambda\lambda}³ F\\
    & + 3\Lambda'' ∂_{η\lambda}² F + 3\Lambda' \Lambda'' ∂_{\lambda \lambda}² F + \Lambda''' ∂_{\lambda} F,
  \end{aligned}\\
  \begin{aligned}[b]
    f''''(η) ={}
    & ∂_{ηηηη}⁴ F + 4\Lambda' ∂_{ηηη\lambda}⁴F + 6\Lambda'^2 ∂_{ηη\lambda\lambda}⁴F + 4\Lambda'^3 ∂_{η\lambda\lambda\lambda}⁴F + \Lambda'^4 ∂_{\lambda\lambda\lambda\lambda}⁴ F\\
    & + 6\Lambda'' ∂_{ηη\lambda}³ F + 12\Lambda' \Lambda'' ∂_{η\lambda\lambda}³F + 6\Lambda'^2 \Lambda'' ∂_{\lambda\lambda\lambda}³ F\\
    & + 4 \Lambda''' ∂_{η\lambda}² F + \bigl( 3\Lambda''^2 + 4 \Lambda' \Lambda''' \bigr) ∂_{\lambda\lambda}² F + \lambda'''' ∂_{\lambda}F,
  \end{aligned}
\end{gather*}
where $\Lambda$ and its derivatives are evaluated at $η$, whilie $F$ and its
partial derivatives are evaluated at $[η, \lambda₀ + \Lambda(η)]$. At $η = 0$, the above
relations read
\begin{gather}
  \label{eq:20220107060454}
  f'(0) = ∂_{η} F + \order[1]\lambda ∂_{\lambda} F,\\
  \label{eq:20220107124311}
  f''(0) = ∂_{ηη}² F + 2 \order[1]\lambda ∂_{η\lambda}² F + \bigl( \order[1]\lambda \bigr)^2 ∂_{\lambda\lambda}² F + \order[2]\lambda ∂_{\lambda} F,\\
  \label{eq:20220107060500}
  \begin{aligned}[b]
    f'''(0) ={}
    & ∂_{ηηη}³ F + 3 \order[1]\lambda ∂_{ηη\lambda}³ F + 3 \bigl( \order[1]\lambda \bigr)^2 ∂_{η\lambda\lambda}³ F + \bigl( \order[1]\lambda \bigr)^3 ∂_{\lambda\lambda\lambda}³ F\\
    & + 3 \order[2]\lambda ∂_{η\lambda}² F + 3 \order[1]\lambda \order[2]\lambda ∂_{\lambda\lambda}² F + \order[3]\lambda ∂_{\lambda} F,
  \end{aligned}\\
  \label{eq:20220602185935}
  \begin{aligned}[b]
    f''''(0) ={}
    & ∂_{ηηηη}⁴F + 4 \order[1]\lambda ∂_{ηηη\lambda}⁴ F + 6 \bigl( \order[1]\lambda \bigr)^2 ∂_{ηη\lambda\lambda}⁴ F + 4 \bigl( \order[1]\lambda \bigr)^3 ∂_{η\lambda\lambda\lambda}⁴ F + \bigl( \order[1]\lambda \bigr)^4 ∂_{\lambda\lambda\lambda\lambda}⁴ F\\
    & + 6 \order[2]\lambda ∂_{ηη\lambda}³ F + 12 \order[1]\lambda \order[2]\lambda ∂_{η\lambda\lambda}³ F + 6 \bigl( \order[1]\lambda \bigr)^2 \order[2]\lambda ∂_{\lambda\lambda\lambda}³ F\\
    & + 4 \order[3]\lambda ∂_{η\lambda}² F + \bigl(3 \bigl( \order[2]\lambda \bigr)^2 + 4 \order[1]\lambda \order[3]\lambda\bigr) ∂_{\lambda\lambda}² F + \lambda₄ ∂_{\lambda} F,
  \end{aligned}
\end{gather}
where $F$ and its partial derivatives are now evaluated at
$(η = 0, \lambda = \lambda₀)$. The values of $f'(0)$, $f''(0)$, \dots thus found are
used in the remainder of Sec.~\ref{sec:20220905060440} for various choices of
the functional $\mathcal F$.

\subsection{Application to the residual}
\label{sec:20211112182000}
% 03/06/2022 — b028b234970605720c9022c16c7fc3012997ced7
%
% Validation des calculs de ce paragraphe

In order to derive the Taylor expansion of the residual $\E_{,u}$, the method
described in Sec.~\ref{sec:20220107121442} is applied to
\begin{equation}
  \label{eq:20220107054629}
  f(η) = \E_{, u} [u(η), \lambda(η); \hat{u}]
  \quad \text{and} \quad
  F(η, \lambda) = \E_{, u}[u^{\ast}(\lambda) + U(η), \lambda; \hat{u}],
\end{equation}
the test function $\hat{u}$ being fixed. It is first observed that
$F(0, \lambda) = \E_{, u} [u^{\ast} (\lambda), \lambda; \hat{u}] = 0$, since $u^{\ast}(\lambda)$ is
an equilibrium point for all $\lambda$ close to $\lambda₀$. Upon derivation with respect
to $\lambda$, we get
\begin{equation*}
  \frac{∂^k F}{∂ \lambda^k}(0, \lambda) = 0 \quad \text{for all} \quad k ≥ 0.
\end{equation*}
From the definition~\eqref{eq:20220107054629} of $F$, we find successively
\begin{equation*}
  ∂_{η}F(η, \lambda) = \E_{, u u}[u^{\ast}(\lambda) + U(η), \lambda; U'(η), \hat{u}],
\end{equation*}
\begin{equation*}
  \begin{aligned}[b]
    ∂_{η η}² F(η, \lambda) ={}
    & \E_{, uuu}[u^{\ast}(\lambda) + U(η), \lambda; U'(η), U'(η), \hat{u}]\\
    & + \E_{, uu} [u^{\ast}(\lambda) + U(η), \lambda; U''(η), \hat{u}],
  \end{aligned}
\end{equation*}
\begin{equation*}
  \begin{aligned}[b]
    ∂_{ηηη}³ F(η, \lambda) ={}
    & \E_{, uuuu}[u^{\ast}(\lambda) + U(η), \lambda; U'(η), U'(η), U'(η), \hat{u}]\\
    & + 3\E_{, u u u}[u^{\ast}(\lambda) + U(η), \lambda; U'(η), U''(η), \hat{u}]\\
    & + \E_{, uu}[u^{\ast}(\lambda) + U(η), \lambda; U'''(η), \hat{u}],
  \end{aligned}
\end{equation*}
and, at $η = 0$
\begin{gather*}
  ∂_{η}F(0, \lambda) = \E₂(\lambda; \order[1]u, \hat{u}),\\
  ∂_{ηη}² F(0, \lambda) = \E₃(\lambda; \order[1]u, \order[1]u, \hat{u}) + \E₂(\lambda; \order[2]u, \hat{u}),\\
  ∂_{ηηη}³ F(0, \lambda) = \E₄(\lambda; \order[1]u, \order[1]u, \order[1]u, \hat{u}) + 3\E₃(\lambda; \order[1]u, \order[2]u, \hat{u}) + \E₂(\lambda; \order[3]u, \hat{u}).
\end{gather*}
Upon derivation with respect to $\lambda$, we find successively
\begin{gather*}
  ∂_{η\lambda}² F(0, \lambda) = \dot{\E}₂(\lambda; \order[1]u, \hat{u}),\\
  ∂_{η\lambda\lambda}³ F(0, \lambda) = \ddot{\E}₂(\lambda; \order[1]u, \hat{u}),\\
  ∂_{ηη\lambda}³ F(0, \lambda) = \dot{\E}₃(\lambda; \order[1]u, \order[1]u, \hat{u}) + \dot{\E₂}(\lambda; \order[2]u, \hat{u}).
\end{gather*}

Upon insertion into Eqs.~\eqref{eq:20220107060454}--\eqref{eq:20220602185935},
we get the following expressions of the sucessive derivatives of $f$ at
$η=0$
\begin{gather*}
  f'(0) = \E₂(\lambda₀; \order[1]u, \hat{u}),\\
  f''(0) = \E₃(\lambda₀; \order[1]u, \order[1]u, \hat{u}) + \E₂(\lambda₀; \order[2]u, \hat{u}) + 2 \order[1]\lambda \dot{\E}₂(\lambda₀; \order[1]u, \hat{u}),\\
  \begin{aligned}[b]
    f'''(0) ={}
    & \E₄(\lambda₀; \order[1]u, \order[1]u, \order[1]u, \hat{u}) + 3\E₃(\lambda₀; \order[1]u, \order[2]u, \hat{u}) + \E₂(\lambda₀ ; \order[3]u, \hat{u})\\
    & + 3\order[1]\lambda \dot{\E}₃(\lambda₀; \order[1]u, \order[1]u, \hat{u}) + 3\order[1]\lambda \dot{\E}₂(\lambda₀; \order[2]u, \hat{u})\\
    & + 3 \bigl( \order[1]\lambda \bigr)^2 \ddot{\E}₂(\lambda₀; \order[1]u, \hat{u}) + 3 \order[2]\lambda \dot{\E}₂(\lambda₀; \order[1]u, \hat{u}),
  \end{aligned}
\end{gather*}
which finally delivers the following expansion of the residual
\begin{equation}
  \label{eq:20220107080901}
  \begin{gathered}[b]
    \E_{, u}[u(η), \lambda(η)] ={} η \E₂(\lambda₀; \order[1]u, \hat{u}) + \tfrac{1}{2} η² \bigl[\E₃(\lambda₀; \order[1]u, \order[1]u, \hat{u})  + \E₂(\lambda₀; \order[2]u, \hat{u})\\
    {} + 2 \order[1]\lambda \dot{\E}₂(\lambda₀; \order[1]u, \hat{u})\bigr] + \tfrac{1}{6} η³ \bigl[ \E₄(\lambda₀; \order[1]u, \order[1]u, \order[1]u, \hat{u}) + 3\E₃(\lambda₀; \order[1]u, \order[2]u, \hat{u})\\
    {} + \E₂(\lambda₀; \order[3]u, \hat{u}) + 3\order[1]\lambda \dot{\E}₃(\lambda₀; \order[1]u, \order[1]u, \hat{u}) + 3\order[1]\lambda \dot{\E}₂(\lambda₀; \order[2]u, \hat{u})\\
    {} + 3 \bigl( \order[1]\lambda \bigr)^2 \ddot{\E}₂(\lambda₀; \order[1]u, \hat{u}) + 3 \order[2]\lambda \dot{\E}₂(\lambda₀ ; \order[1]u, \hat{u}) \bigr] + o(η³),
  \end{gathered}
\end{equation}
up to third-order terms.

\subsection{Application to the energy}
\label{sec:20220525053434}
% 07/06/2022 — dd1a4abf18cd94861d754bf3e19a54b8974bb2e8
%
% Relecture de tous les calculs de ce paragraphe

The method described in Sec.~\ref{sec:20220107121442} is applied to the energy
difference between the fundamental and bifurcated branches
\begin{equation}
  F(η, \lambda) = \E[u^{\ast}(\lambda) + U(η), \lambda] - \E[u^{\ast}(\lambda), \lambda]
  \quad \text{et} \quad
  f(η) = F [η, \lambda₀ + \Lambda(η)].
\end{equation}
Observing that $F(0, \lambda) = 0$ for all $\lambda$, we first get
\begin{equation*}
  \frac{∂^k F}{∂ \lambda^k}(0, \lambda) = 0 \quad \text{for all} \quad k ≥ 0,
\end{equation*}
while the partial derivatives of $F$ with respect to $η$ read
\begin{gather*}
  ∂_{η} F(η, \lambda) = \E_{, u}(U'),\\
  ∂_{ηη}² F(η, \lambda) = \E_{, uu} (U', U') + \E_{, u} (U''),\\
  ∂_{ηηη}³ F(η, \lambda) = \E_{, uuu}(U', U', U') + 3\E_{, uu}(U', U'') + \E_{, u}(U'''),\\
  \begin{aligned}[b]
    ∂_{ηηηη}⁴ F ={}
    & \E_{, uuuu}(U', U', U', U') + 6\E_{,uuu}(U', U', U'')\\
    & + 3\E_{, uu}(U'', U'') + 4\E_{, uu}(U', U''') + \E_{, u}(U''''),
  \end{aligned}
\end{gather*}
where the partial derivatives of $\E$ are evaluated at
$[u^{\ast}(\lambda) + U(η), \lambda]$, while the derivatives of $U$ are evaluated at
$η$. For $η = 0$, observing that $\E_{, u}[u^{\ast}(\lambda), \lambda] = 0$
\begin{gather*}
  ∂_{η} F(0, \lambda) = 0,\\
  ∂_{ηη}² F(0, \lambda) =\E₂(\lambda ; \order[1]u, \order[1]u),\\
  ∂_{ηηη}³ F(0, \lambda) = \E₃(\lambda; \order[1]u, \order[1]u, \order[1]u) + 3\E₂(\lambda; \order[1]u, \order[2]u),\\
  \begin{aligned}[b]
    ∂_{ηηηη}⁴ F(η, \lambda) ={} & \E₄(\lambda; \order[1]u, \order[1]u, \order[1]u, \order[1]u) + 6\E₃(\lambda; \order[1]u, \order[1]u, \order[2]u)\\
    & + 3\E₂(\lambda; \order[2]u, \order[2]u) + 4\E₂(\lambda; \order[1]u, \order[3]u),
  \end{aligned}
\end{gather*}
and, upon derivation with respect to $\lambda$
\begin{equation*}
  \begin{gathered}
    ∂_{η\lambda}² F(0, \lambda) = 0,\\
    ∂_{ηη\lambda}³ F(0, \lambda) = \dot{\E}₂(\lambda; \order[1]u, \order[1]u),\\
    ∂_{η\lambda\lambda}³ F(0, \lambda) = 0,\\
  \end{gathered}
  \qquad
  \begin{gathered}
    ∂_{ηηη\lambda}⁴ F(0, \lambda) = \dot{\E}₃(\lambda; \order[1]u, \order[1]u, \order[1]u) + 3\dot{\E}₂(\lambda; \order[1]u, \order[2]u),\\
    ∂_{ηη\lambda\lambda}⁴ F(0, \lambda) = \ddot{\E}₂(\lambda; \order[1]u, \order[1]u),\\
    ∂_{η\lambda\lambda\lambda}⁴ F(0, \lambda) = 0
  \end{gathered}
\end{equation*}
and finally
\begin{gather}
  f'(0) = 0,\\
  f''(0) = \E₂(\lambda₀; \order[1]u, \order[1]u),\\
  f'''(0) =\E₃(\lambda₀; \order[1]u, \order[1]u, \order[1]u) + 3\E₂(\lambda₀; \order[1]u, \order[2]u) + 3\order[1]\lambda \dot{\E}₂(\lambda₀; \order[1]u, \order[1]u),\\
  \label{eq:20220905063614}
  \begin{aligned}[b]
    f''''(0) ={}
    & \E₄(\lambda₀; \order[1]u, \order[1]u, \order[1]u, \order[1]u) + 6\E₃(\lambda₀; \order[1]u, \order[1]u, \order[2]u) + 3\E₂(\lambda₀; \order[2]u, \order[2]u)\\
    & + 4\E₂(\lambda₀; \order[1]u, \order[3]u) + 4 \order[1]\lambda \dot{\E}₃(\lambda₀; \order[1]u, \order[1]u, \order[1]u) + 12 \order[1]\lambda \dot{\E}₂(\lambda₀; \order[1]u, \order[2]u)\\
    & + 6( \order[1]\lambda )^2 \ddot{\E}₂(\lambda₀; \order[1]u, \order[1]u) + 6\order[2]\lambda \dot{\E}₂(\lambda₀; \order[1]u, \order[1]u).
  \end{aligned}
\end{gather}

Since $\order[1]u \in V$, we have $\E₂(\lambda₀; \order[1]u, \order[k]u) = 0$ for
$k = 1, 2, 3$. Therefore $f''(0)=0$ and, using
Eq.~\eqref{eq:20220524133816}
\begin{equation}
  \label{eq:20220601055448}
  f'''(0) = \order[1]\lambda F_{ij} \order[1]{ξ_i} \order[1]{ξ_j},
\end{equation}

Turning now to $f''''(0)$, we plug the decompositions
\eqref{eq:20220524133944} and \eqref{eq:20220524134613} of $\order[1]u$ and
$\order[2]u$ successively into each term of Eq.~\eqref{eq:20220905063614}.
\begin{equation*}
  \begin{aligned}[b]
    \E₃(\lambda₀; \order[1]u, \order[1]u, \order[2]u)
    ={} & \E₃(v_i, v_j, v_k) \order[1]{ξ_i} \order[1]{ξ_j} \order[2]{ξ_k} + \E₃(v_i, v_j, w_{k l}) \order[1]{ξ_i} \order[1]{ξ_j} \order[1]{ξ_k} \order[1]{ξ_l}\\
    & + 2\order[1]\lambda \E₃(v_i, v_j, w_k) \order[1]{ξ_i} \order[1]{ξ_j} \order[1]{ξ_k} \\
    ={} & \E₃(v_i, v_j, v_k) \order[1]{ξ_i} \order[1]{ξ_j} \order[2]{ξ_k} + \E₃(v_i, v_j, w_{k l}) \order[1]{ξ_i} \order[1]{ξ_j} \order[1]{ξ_k} \order[1]{ξ_l}\\
    & - 2\order[1]\lambda \E₂(w_{ij}, w_k) \order[1]{ξ_i} \order[1]{ξ_j} \order[1]{ξ_k}, \qquad \text{[using Eq.~\eqref{eq:20220519164523}]}\\
    ={} & \E₃(v_i, v_j, v_k) \order[1]{ξ_i} \order[1]{ξ_j} \order[2]{ξ_k} + \E₃(v_i, v_j, w_{kl}) \order[1]{ξ_i} \order[1]{ξ_j} \order[1]{ξ_k} \order[1]{ξ_l}\\
    & - 2\order[1]\lambda \E₂(w_{i}, w_{jk}) \order[1]{ξ_i} \order[1]{ξ_j} \order[1]{ξ_k}\\
    ={} & \E₃(v_i, v_j, v_k) \order[1]{ξ_i} \order[1]{ξ_j} \order[2]{ξ_k} + \E₃(v_i, v_j, w_{kl}) \order[1]{ξ_i} \order[1]{ξ_j} \order[1]{ξ_k} \order[1]{ξ_l}\\
    & + 2 \order[1]\lambda \dot{\E}₂(v_{i}, w_{jk}) \order[1]{ξ_i} \order[1]{ξ_j} \order[1]{ξ_k}. \qquad \text{[using Eq.~\eqref{eq:20220524134525}]}
  \end{aligned}
\end{equation*}
Then
\begin{equation*}
  \begin{aligned}[b]
    \E₂(\order[2]u, \order[2]u)
    ={} & \E₂(\order[2]{ξ_i} v_i + \order[1]{ξ_i} \order[1]{ξ_j} w_{i j} + 2\order[1]\lambda \order[1]{ξ_i} w_i, \order[2]{ξ_k} v_k + \order[1]{ξ_k} \order[1]{ξ_l} w_{k l} + 2\order[1]\lambda \order[1]{ξ_k} w_k)\\
    ={} & \E₂(\order[1]{ξ_i} \order[1]{ξ_j} w_{i j} + 2 \order[1]\lambda \order[1]{ξ_i} w_i, \order[1]{ξ_k} \order[1]{ξ_l} w_{k l} + 2 \order[1]\lambda \order[1]{ξ_k} w_k)\\
    ={} & \E₂(w_{i j}, w_{k l}) \order[1]{ξ_i} \order[1]{ξ_j} \order[1]{ξ_k} \order[1]{ξ_l} + 4 \order[1]\lambda \E₂(w_i, w_{j k}) \order[1]{ξ_i} \order[1]{ξ_j} \order[1]{ξ_k}\\
    & + 4 ( \order[1]\lambda )^2 \E₂(w_i, w_j) \order[1]{ξ_i} \order[1]{ξ_j}\\
    ={} & \E₂(w_{i j}, w_{k l}) \order[1]{ξ_i} \order[1]{ξ_j} \order[1]{ξ_k} \order[1]{ξ_l} + 4 \order[1]\lambda \E₂(w_i, w_{j k}) \order[1]{ξ_i} \order[1]{ξ_j} \order[1]{ξ_k}\\
    &+ 2 ( \order[1]\lambda )^2 \bigl[\E₂(w_i, w_j) + \E₂(w_j, w_i)\bigr] \order[1]{ξ_i} \order[1]{ξ_j}\\
    ={} & -\E₃(v_i, v_j, w_{k l}) \order[1]{ξ_i} \order[1]{ξ_j} \order[1]{ξ_k} \order[1]{ξ_l} - 4 \order[1]\lambda \dot{\E}₂ (v_i, w_{j k}) \order[1]{ξ_i} \order[1]{ξ_j} \order[1]{ξ_k}\\
    & - 2 ( \order[1]\lambda )^2 \bigl[\dot{\E}₂(v_i, w_j) + \dot{\E}₂(v_j, w_i)\bigr] \order[1]{ξ_i} \order[1]{ξ_j}
  \end{aligned}
\end{equation*}
finally
\begin{equation*}
  \begin{aligned}[b]
    \dot{\E}₂(\order[1]u, \order[2]u)
    ={} & \dot{\E}₂ (v_i, v_j) \order[1]{ξ_i} \order[2]{ξ_j} + \dot{\E}₂(v_i, w_{j k}) \order[1]{ξ_i} \order[1]{ξ_j} \order[1]{ξ_k} + 2\order[1]\lambda \dot{\E}₂(v_i, w_j) \order[1]{ξ_i} \order[1]{ξ_j}\\
    ={} & \dot{\E}₂(v_i, v_j) \order[1]{ξ_i} \order[2]{ξ_j} + \dot{\E}₂(v_i, w_{j k}) \order[1]{ξ_i} \order[1]{ξ_j} \order[1]{ξ_k} + \order[1]\lambda [\dot{\E}₂(v_i, w_j) + \dot{\E}₂(v_j, w_i)] \order[1]{ξ_i} \order[1]{ξ_j}.
  \end{aligned}
\end{equation*}
Gathering the above results
% \begin{equation*}
%   \begin{aligned}[b]
%     f''''(0) ={}
%     & \E₄(v_i, v_j, v_k, v_l) \order[1]{ξ_i} \order[1]{ξ_j} \order[1]{ξ_k} \order[1]{ξ_l}\\
%     & + 6\bigl[\E₃(v_i, v_j, v_k) \order[1]{ξ_i} \order[1]{ξ_j} \order[2]{ξ_k}
%       + \E₃(v_i, v_j, w_{kl}) \order[1]{ξ_i} \order[1]{ξ_j} \order[1]{ξ_k} \order[1]{ξ_l}
%       + 2 \order[1]\lambda \dot{\E}₂(v_{i}, w_{jk}) \order[1]{ξ_i} \order[1]{ξ_j} \order[1]{ξ_k}\bigr]\\
%     & -3\bigl\{ \E₃(v_i, v_j, w_{k l}) \order[1]{ξ_i} \order[1]{ξ_j} \order[1]{ξ_k} \order[1]{ξ_l} + 4 \order[1]\lambda \dot{\E}₂ (v_i, w_{j k}) \order[1]{ξ_i} \order[1]{ξ_j} \order[1]{ξ_k}\\
%     & + 2 ( \order[1]\lambda )^2 \bigl[\dot{\E}₂(v_i, w_j) + \dot{\E}₂(v_j, w_i)\bigr] \order[1]{ξ_i} \order[1]{ξ_j} \bigr\}\\
%     & + 4 \order[1]\lambda \dot{\E}₃(v_i, v_j, v_k) \order[1]{ξ_i} \order[1]{ξ_j} \order[1]{ξ_k}\\
%     & + 12 \order[1]\lambda \bigl\{ \dot{\E}₂(v_i, v_j) \order[1]{ξ_i} \order[2]{ξ_j} + \dot{\E}₂(v_i, w_{j k}) \order[1]{ξ_i} \order[1]{ξ_j} \order[1]{ξ_k} + \order[1]\lambda \bigl[\dot{\E}₂(v_i, w_j) + \dot{\E}₂(v_j, w_i)\bigr] \order[1]{ξ_i} \order[1]{ξ_j} \bigr\}\\
%     & + 6( \order[1]\lambda )^2 \ddot{\E}₂(v_i, v_j) \order[1]{ξ_i} \order[1]{ξ_j}\\
%     & + 6\order[2]\lambda \dot{\E}₂(v_i, v_j) \order[1]{ξ_i} \order[1]{ξ_j}.
%   \end{aligned}
% \end{equation*}
% \begin{equation*}
%   \begin{aligned}[b]
%     f''''(0) ={}
%     & \E₄(v_i, v_j, v_k, v_l) \order[1]{ξ_i} \order[1]{ξ_j} \order[1]{ξ_k} \order[1]{ξ_l}
%       + 6\E₃(v_i, v_j, v_k) \order[1]{ξ_i} \order[1]{ξ_j} \order[2]{ξ_k}
%       + 6\E₃(v_i, v_j, w_{kl}) \order[1]{ξ_i} \order[1]{ξ_j} \order[1]{ξ_k} \order[1]{ξ_l}\\
%     & + 12 \order[1]\lambda \dot{\E}₂(v_{i}, w_{jk}) \order[1]{ξ_i} \order[1]{ξ_j} \order[1]{ξ_k} - 3 \E₃(v_i, v_j, w_{k l}) \order[1]{ξ_i} \order[1]{ξ_j} \order[1]{ξ_k} \order[1]{ξ_l} -12 \order[1]\lambda \dot{\E}₂ (v_i, w_{j k}) \order[1]{ξ_i} \order[1]{ξ_j} \order[1]{ξ_k}\\
%     & - 6 ( \order[1]\lambda )^2 \bigl[\dot{\E}₂(v_i, w_j) + \dot{\E}₂(v_j, w_i)\bigr] \order[1]{ξ_i} \order[1]{ξ_j}
%       + 4 \order[1]\lambda \dot{\E}₃(v_i, v_j, v_k) \order[1]{ξ_i} \order[1]{ξ_j} \order[1]{ξ_k}\\
%     & + 12 \order[1]\lambda \dot{\E}₂(v_i, v_j) \order[1]{ξ_i} \order[2]{ξ_j} + 12 \order[1]\lambda \dot{\E}₂(v_i, w_{j k}) \order[1]{ξ_i} \order[1]{ξ_j} \order[1]{ξ_k}\\
%     & + 12 ( \order[1]\lambda )^2 \bigl[\dot{\E}₂(v_i, w_j)
%       + \dot{\E}₂(v_j, w_i)\bigr] \order[1]{ξ_i} \order[1]{ξ_j}\\
%     & + 6( \order[1]\lambda )^2 \ddot{\E}₂(v_i, v_j) \order[1]{ξ_i} \order[1]{ξ_j}
%       + 6\order[2]\lambda \dot{\E}₂(v_i, v_j) \order[1]{ξ_i} \order[1]{ξ_j}.
%   \end{aligned}
% \end{equation*}
% \begin{equation*}
%   \begin{aligned}[b]
%     f''''(0) ={}
%     & \bigl[ \E₄(v_i, v_j, v_k, v_l) + 3\E₃(v_i, v_j, w_{kl}) \bigr] \order[1]{ξ_i} \order[1]{ξ_j} \order[1]{ξ_k} \order[1]{ξ_l}\\
%     & + \bigl[ 4 \order[1]\lambda \dot{\E}₃(v_i, v_j, v_k) + 12 \order[1]\lambda  \dot{\E}₂(v_i, w_{j k}) \bigr] \order[1]{ξ_i} \order[1]{ξ_j} \order[1]{ξ_k}\\
%     & + \bigl\{ 6( \order[1]\lambda )^2 \ddot{\E}₂(v_i, v_j) + 6( \order[1]\lambda )^2 \bigl[\dot{\E}₂(v_i, w_j) + \dot{\E}₂(v_j, w_i)\bigr] + 6\order[2]\lambda \dot{\E}₂(v_i, v_j) \bigr\} \order[1]{ξ_i} \order[1]{ξ_j}\\
%     & + \bigl[ 6\E₃(v_i, v_j, v_k) \order[1]{ξ_i} \order[1]{ξ_j} + 12 \order[1]\lambda \dot{\E}₂(v_j, v_k) \order[1]{ξ_j} \bigr] \order[2]{ξ_k}
%   \end{aligned}
% \end{equation*}
\begin{equation*}
  \begin{aligned}[b]
    f''''(0) ={}
    & \bigl[ \E₄(v_i, v_j, v_k , v_l) + 3\E₃(v_i, v_j, w_{k l}) \bigr] \order[1]{ξ_i} \order[1]{ξ_j} \order[1]{ξ_k} \order[1]{ξ_l}\\
    & + 4 \order[1]\lambda \bigl[\dot{\E}₃(v_i, v_j, v_k) + 3 \dot{\E}₂(v_i, w_{j k})\bigr] \order[1]{ξ_i} \order[1]{ξ_j} \order[1]{ξ_k}\\
    & + \bigl\{6 ( \order[1]\lambda )^2 \bigl[ \ddot{\E}₂ (v_i, v_j) + \dot{\E}₂(v_i, w_j) + \dot{\E}₂(v_j, w_i) \bigr] + 6\order[2]\lambda \dot{\E}₂(v_i, v_j) \bigr\} \order[1]{ξ_i} \order[1]{ξ_j}\\
    & + 6\bigl[ \underbrace{\E₃(v_i, v_j, v_k) \order[1]{ξ_j} \order[1]{ξ_k} + 2 \order[1]\lambda \dot{\E}₂(v_i, v_j) \order[1]{ξ_j}}_{=0\text{ from Eq.~\eqref{eq:20220524135036}}} \bigr] \order[2]{ξ_i},
  \end{aligned}
\end{equation*}
Upon introduction of the tensors $E_{ijkl}$, $\mathring{E}_{ijk}$,
$F_{ij}$ and $\mathring{F}_{ij}$
\begin{equation}
  \label{eq:20220601055512}
  f''''(0) = E_{i j k l} \order[1]{ξ_i} \order[1]{ξ_j} \order[1]{ξ_k} \order[1]{ξ_l} + 4 \order[1]\lambda \mathring{E}_{i j k} \order[1]{ξ_i} \order[1]{ξ_j} \order[1]{ξ_k} + 6 \bigl[ ( \order[1]\lambda )^2 \mathring{F}_{i j} + \order[2]\lambda F_{i j}\bigr] \order[1]{ξ_i} \order[1]{ξ_j},
\end{equation}
which finally leads to the Taylor expansion~\eqref{eq:20220525053600}.

\subsection{Application to the hessian of the energy}
\label{sec:20220616055207}
% 08/06/2022 — aea0da72c80440d74d38d8ace59f381061f71c3e
%
% Relecture de tous les calculs de ce paragraphe

The method described in Sec.~\ref{sec:20220107121442} is now applied to
$f(η) = F [η, \lambda₀ + \Lambda(η)]$, with
\begin{equation*}
  F(η, \lambda) = \E_{, u u} [u^{\ast}(\lambda) + U(η), \lambda; \hat{u}, \hat{v}].
\end{equation*}
where $\hat{u}, \hat{v} \in U$ are fixed. This will deliver a Taylor expansion
of the hessian of the energy, $\E_{,uu}$. It is first observed that
$F(0, \lambda) = \E₂(\lambda; \hat{u}, \hat{v})$ and, upon derivation with respect to $\lambda$
\begin{equation*}
  ∂_{\lambda} F(0, \lambda) = \dot{\E}₂(\lambda; \hat{u}, \hat{v})
  \quad \text{and} \quad
  ∂_{\lambda\lambda}² F(0, \lambda) = \ddot{\E}₂(\lambda; \hat{u}, \hat{v}).
\end{equation*}

Successive differentiation of the definition of $F$ with respect to $η$ also
leads to
\begin{gather*}
  ∂_{η} F(η, \lambda) = \E_{, uuu}(U', \hat{u}, \hat{v}),\\
  ∂_{ηη}² F(η, \lambda) = \E_{, uuuu}(U', U', \hat{u}, \hat{v}) + \E_{, uuu}(U'', \hat{u}, \hat{v}),
\end{gather*}
where the differentials of $\E$ are evaluated at $[u^{\ast}(\lambda) + U(η), \lambda]$,
while the dérivatives of $U$ are evaluated at $y$. At $η = 0$, the above
relations read
\begin{gather*}
  ∂_{η} F(0, \lambda) = \E₃(\lambda; \order[1]u, \hat{u}, \hat{v}),\\
  ∂_{ηη}² F(0, \lambda) = \E₄(\lambda ; \order[1]u, \order[1]u, \hat{u}, \hat{v}) + \E₃(\lambda; \order[2]u, \hat{u}, \hat{v}),
\end{gather*}
and, upon differentiation with respect to $\lambda$
\begin{equation*}
  ∂_{η \lambda}² F(0, \lambda) = \dot{\E}₃(\lambda; \order[1]u, \hat{u}, \hat{v}).
\end{equation*}

The Taylor expansion~\eqref{eq:20220531054247} of the hessian is finally
retrieved by plugging the above results into
expressions~\eqref{eq:20220107060454} and \eqref{eq:20220107124311} of the
derivatives of $f$
\begin{gather*}
  f'(0) = \E₃(\lambda₀; \order[1]u, \hat{u}, \hat{v}) + \order[1]\lambda \dot{\E}₂(\lambda₀; \hat{u}, \hat{v}),\\
  \begin{aligned}[b]
    f''(0) = {} & \E₄(\lambda₀; \order[1]u, \order[1]u, \hat{u}, \hat{v}) + \E₃(\lambda₀; \order[2]u, \hat{u}, \hat{v}) + 2\order[1]\lambda \dot{\E}₃(\lambda₀; \order[1]u, \hat{u}, \hat{v})\\
                & + ( \order[1]\lambda )^2 \ddot{\E}₂(\lambda₀; \hat{u}, \hat{v}) + \order[2]\lambda \dot{\E}₂(\lambda₀; \hat{u}, \hat{v}).
  \end{aligned}
\end{gather*}

\subsection{Asymptotic expansions of the eigenvalues and eigenvectors of the Hessian}
\label{sec:20220616074108}

In this appendix, Eqs.~\eqref{eq:20220609133608}, \eqref{eq:20220609133629} and
\eqref{eq:20220616082923} are derived. The postulated
expansions~\eqref{eq:20220617064633} are plugged into the asymptotic expansion
\eqref{eq:20220531054247} of the Hessian on the one hand
\begin{equation*}
  \begin{aligned}[b]
    \E_{, uu} [u(η), \lambda(η); x, \hat{u}] ={}
    & \E₂(\order[0]x, \hat{u}) + η \bigl[ \E₂(\order[1]x, \hat{u}) + \E₃(\order[1]u, \order[0]x, \hat{u}) + \order[1]\lambda \dot{\E}₂(\order[0]x, \hat{u})\bigr]\\
    & + \tfrac{1}{2} η² \bigl[\E₂(\order[2]x, \hat{u}) + 2\E₃(\order[1]u, \order[1]x, \hat{u}) + 2 \order[1]\lambda \dot{\E}₂(\order[1]x, \hat{u})\\
    & + \E₄(\order[1]u, \order[1]u, \order[0]x, \hat{u}) + \E₃(\order[2]u, \order[0]x, \hat{u}) + 2\order[1]\lambda \dot{\E}₃(\order[1]u, \order[0]x, \hat{u})\\
    & + ( \order[1]\lambda )^2 \ddot{\E}₂(\order[0]x, \hat{u}) + \order[2]\lambda \dot{\E}₂(\order[0]x, \hat{u}) \bigr] + o(η²)
  \end{aligned}
\end{equation*}
(where the $\E_k$ and $\dot{\E}_k$ are all evaluated at $\lambda=\lambda₀$) and into the
scalar product $α 〈 x, \hat{u} 〉$ on the other hand
\begin{equation*}
  \begin{aligned}[b]
    α 〈 x, \hat{u} 〉 ={}
    & \order[0]α 〈 \order[0]x, \hat{u} 〉 + η \bigl(\order[1]α 〈 \order[0]x, \hat{u} 〉 + \order[0]α 〈 \order[1]x, \hat{u} 〉\bigr)\\
    & + \tfrac{1}{2} η² \bigl(\order[0]α 〈 \order[2]x, \hat{u} 〉 + 2 \order[1]α 〈 \order[1]x, \hat{u} 〉 + \order[2]α 〈 \order[0]x, \hat{u} 〉\bigr) + o(η²).
  \end{aligned}
\end{equation*}

Equating both expressions for all $\hat{u} \in U$ [see
Eq.~\eqref{eq:20220617074949}] leads to three variational problems (for the
$η⁰$, $η¹$ and $η²$ terms) that are discussed below.

\paragraph{Variational problem of order 0} Find $\order[0]x \in U$ and $\order[0]α\in\reals$ such
that, for all $\hat{u} \in U$
\begin{equation*}
  \E₂(\lambda₀; \order[0]x, \hat{u}) = \order[0]α 〈 \order[0]x, \hat{u} 〉.
\end{equation*}

The above equation shows that $(\order[0]α, \order[0]x)$ is an eigenpair of $\E₂(\lambda₀)$. As
discussed in Sec.~\ref{sec:20220617075558}, only the case $\order[0]α = 0$ is
relevant. Then $\order[0]x \in V$, which is expressed by the
expansion~\eqref{eq:20220904160057} of $\order[0]x$.

\paragraph{Variational problem of order 1} Find $\order[1]x \in U$ and $\order[1]α\in\reals$ such
that, for all $\hat{u} \in U$
\begin{equation}
  \label{eq:20220609131953}
  \E₂(\lambda₀; \order[1]x, \hat{u}) + \E₃(\lambda₀; \order[1]u, \order[0]x, \hat{u}) + \order[1]\lambda \dot{\E}₂(\lambda₀; \order[0]x, \hat{u})
  = \order[1]α 〈 \order[0]x, \hat{u} 〉,
\end{equation}
or, equivalently, plugging the expansions~\eqref{eq:20220524133944} and
\eqref{eq:20220609133608} of $\order[1]u$ and $\order[0]x$ in the $v_i$ basis
\begin{equation}
  \label{eq:20220617080547}
  \E₂(\lambda₀; \order[1]x, \hat{u}) + \E₃(\lambda₀; v_j, v_k, \hat{u}) \order[0]{χ_j} \order[1]{ξ_k} + \order[1]\lambda \dot{\E}₂(\lambda₀; v_j, \hat{u}) \order[0]{χ_j}
  = \order[1]α \order[0]{χ_j} 〈 v_j, \hat{u} 〉.
\end{equation}

For $\hat{u} = v_i$, observing that $〈 v_i, v_j 〉 = δ_{ij}$ since
$(v_i)$ is orthonormal, the above equation reads
\begin{equation}
  \bigl[\E₃(\lambda₀; v_i, v_j, v_k) \order[1]{ξ_k} + \order[1]\lambda \dot{\E}₂(\lambda₀; v_i, v_j)\bigr] \order[0]{χ_j} = \order[1]α \order[0]{χ_i},
\end{equation}
which reduces to Eq.~\eqref{eq:20220609133608}.

The test function is now picked in $W = V^\perp$, and $\order[1]x$ is
decomposed as the sum of its projections onto $V$ and $W$:
$\order[1]x = \order[1]{χ_i} v_i + \order[1]{y}$, where $\order[1]y \in
W$. Eq.~\eqref{eq:20220617080547} then delivers the following variational
problem: find $\order[1]y \in W$ such that, for all $\hat{w} \in W$,
\begin{equation}
  \E₂(\order[1]y, \hat{w}) + \E₃(v_i, v_j, \hat{w}) \order[0]{χ_i} \order[1]{ξ_j} + \order[1]\lambda \dot{\E}₂(v_i, \hat{w}) \order[0]{χ_i} = 0,
\end{equation}
(observe that $〈 v_j, \hat{w} 〉 = 0$ since $V$ and $W$ are orthogonal
subspaces). The solution to the above problem is expressed as a linear
combination of the $w_i$ and $w_{ij}$ defined by the variational problems
\eqref{eq:20220524134525} and \eqref{eq:20220519164523}, respectively:
$\order[1]y = \order[0]{χ_i} \order[1]{ξ_j} w_{i j} + \order[1]\lambda \order[0]{χ_i}
w_i$, and the decomposition~\eqref{eq:20220609133629} is retrieved.

\paragraph{Variational problem of order 2} For all $\hat{u} \in U$,
\begin{multline*}
    \E₂(\lambda₀; \order[2]x, \hat{u})
    + 2\E₃(\lambda₀; \order[1]u, \order[1]x, \hat{u})
    + 2 \order[1]\lambda \dot{\E}₂(\lambda₀; \order[1]x, \hat{u})\\
    + \E₄(\lambda₀; \order[1]u, \order[1]u, \order[0]x, \hat{u})
    + \E₃(\lambda₀; \order[2]u, \order[0]x, \hat{u})
    + 2\order[1]\lambda \dot{\E}₃(\lambda₀; \order[1]u, \order[0]x, \hat{u})\\
    + ( \order[1]\lambda )^2 \ddot{\E}₂(\lambda₀; \order[0]x, \hat{u})
    + \order[2]\lambda \dot{\E}₂(\lambda₀; \order[0]x, \hat{u})
    = 2 \order[1]α 〈 \order[1]x, \hat{u} 〉
    + \order[2]α 〈 \order[0]x, \hat{u} 〉.
\end{multline*}

For $\hat{u} = \hat{v}_i$, plugging the decompositions
\eqref{eq:20220524133944}, \eqref{eq:20220524134613}, \eqref{eq:20220609133608}
and \eqref{eq:20220609133629} of $\order[1]u$, $\order[2]u$, $\order[0]x $ et $\order[1]x$ delivers
% \begin{multline*}
%   2\E₃(v_i, \order[1]x, \order[1]u)
%   + 2 \order[1]\lambda \dot{\E}₂(v_i, \order[1]x)
%   + \E₄(v_i, \order[0]x, \order[1]u, \order[1]u)\\
%   + \E₃(v_i, \order[0]x, \order[2]u)
%   + 2\order[1]\lambda \dot{\E}₃(v_i, \order[0]x, \order[1]u)
%   + ( \order[1]\lambda )^2 \ddot{\E}₂(v_i, \order[0]x)\\
%   + \order[2]\lambda \dot{\E}₂(v_i, \order[0]x)
%   = 2\order[1]α 〈 v_i, \order[1]x 〉
%   + \order[2]α 〈 v_i, \order[0]x 〉,
% \end{multline*}
% \begin{multline*}
%   2\E₃(v_i, \order[1]{χ_i}v_j + \order[0]{χ_j}\order[1]{ξ_k}w_{jk}+ \order[1]\lambda \order[0]{χ_j} w_j, \order[1]{ξ_l} v_l)
%   + 2 \order[1]\lambda \dot{\E}₂(v_i, \order[1]{χ_i}v_j + \order[0]{χ_j}\order[1]{ξ_k}w_{jk} + \order[1]\lambda \order[0]{χ_j} w_j)\\
%   + \E₄(v_i, \order[0]{χ_j} v_j, \order[1]{ξ_k} v_k, \order[1]{ξ_l} v_l)
%   + \E₃(v_i, \order[0]{χ_j} v_j, \order[2]{ξ_k} v_k + \order[1]{ξ_k} \order[1]{ξ_l} w_{kl} + 2\order[1]\lambda \order[1]{ξ_k} w_k)\\
%   + 2\order[1]\lambda \dot{\E}₃(v_i, \order[0]{χ_j} v_j, \order[1]{ξ_k} v_k)
%   + ( \order[1]\lambda )^2 \ddot{\E}₂(v_i, \order[0]{χ_j} v_j) + \order[2]\lambda \dot{\E}₂(v_i, \order[0]{χ_j} v_j)\\
%   = 2\order[1]α 〈 v_i,  \order[1]{χ_i}v_j + \order[0]{χ_j}\order[1]{ξ_k}w_{jk} + \order[1]\lambda \order[0]{χ_j} w_j 〉
%   + \order[2]α 〈 v_i, \order[0]{χ_j} v_j〉,
% \end{multline*}
% \begin{multline*}
%   2\E₃(v_i, v_j,  v_k) \order[1]{χ_i} \order[1]{ξ_k}
%   + 2\E₃(v_i, w_{jk}, v_l) \order[0]{χ_j} \order[1]{ξ_k} \order[1]{ξ_l}
%   + 2 \order[1]\lambda \E₃(v_i,  w_j, v_k) \order[0]{χ_j} \order[1]{ξ_k}\\
%   + 2 \order[1]\lambda \dot{\E}₂(v_i, v_j) \order[1]{χ_i}
%   + 2 \order[1]\lambda \dot{\E}₂(v_i, w_{jk}) \order[0]{χ_j} \order[1]{ξ_k}
%   + ( \order[1]\lambda )^2 \dot{\E}₂(v_i, w_j) \order[0]{χ_j}\\
%   + \E₄(v_i, v_j,  v_k, v_l) \order[0]{χ_j} \order[1]{ξ_k} \order[1]{ξ_l}
%   + \E₃(v_i, v_j, v_k) \order[0]{χ_j} \order[2]{ξ_k}
%   + \E₃(v_i, v_j, w_{kl}) \order[0]{χ_j} \order[1]{ξ_k} \order[1]{ξ_l}\\
%   + 2\order[1]\lambda \E₃(v_i, v_j, w_k) \order[0]{χ_j} \order[1]{ξ_k}
%   + 2\order[1]\lambda \dot{\E}₃(v_i, v_j,  v_k) \order[0]{χ_j} \order[1]{ξ_k}
%   + ( \order[1]\lambda )^2 \ddot{\E}₂(v_i, v_j) \order[0]{χ_j}\\
%   + \order[2]\lambda \dot{\E}₂(v_i, v_j) \order[0]{χ_j} = 2\order[1]α\order[1]{χ_i} + \order[2]α \order[0]{χ_i}.
% \end{multline*}
\begin{multline*}
  \bigl[ \E₄(v_i, v_j,  v_k, v_l) + 2\E₃(v_i, w_{jk}, v_l) + \E₃(v_i, v_j, w_{kl})\bigr] \order[0]{χ_j} \order[1]{ξ_k} \order[1]{ξ_l}\\
  + 2 \order[1]\lambda \bigl[ \E₃(v_i,  w_j, v_k) + \dot{\E}₂(v_i, w_{jk}) + \E₃(v_i, v_j, w_k) + \dot{\E}₃(v_i, v_j,  v_k) \bigr] \order[0]{χ_j} \order[1]{ξ_k}\\
  + ( \order[1]\lambda )^2 \bigl[\dot{\E}₂(v_i, w_j) + \ddot{\E}₂(v_i, v_j)\bigr] \order[0]{χ_j} + \bigl[\E₃(v_i, v_j, v_k) \order[2]{ξ_k} + \order[2]\lambda \dot{\E}₂(v_i, v_j)\bigr] \order[0]{χ_j} \\
  +2\bigl[\E₃(v_i, v_j,  v_k)  \order[1]{ξ_k} + \order[1]\lambda \dot{\E}₂(v_i, v_j)\bigr] \order[1]{χ_i} = 2\order[1]α\order[1]{χ_i} + \order[2]α \order[0]{χ_i}.
\end{multline*}

The $\order[0]{χ_j} \order[1]{ξ_k}$ term is transformed with Eqs.~\eqref{eq:20220524134525} and
\eqref{eq:20220519164523}
\begin{multline*}
  \bigl[ \E₄(v_i, v_j,  v_k, v_l)
  + \E₃(v_i, w_{jk}, v_l)
  + \E₃(v_i, w_{jl}, v_k)
  + \E₃(v_i, v_j, w_{kl})\bigr] \order[0]{χ_j} \order[1]{ξ_k} \order[1]{ξ_l}\\
  + 2\order[1]\lambda \bigl[ -\E₂(w_{ik},  w_j) - \E₂(w_i, w_{jk}) - \E₂(w_{ij}, w_k) + \dot{\E}₃(v_i, v_j,  v_k) \bigr] \order[0]{χ_j} \order[1]{ξ_k}\\
  + ( \order[1]\lambda )^2 \bigl[\dot{\E}₂(v_i, w_j) + \ddot{\E}₂(v_i, v_j)\bigr] \order[0]{χ_j}
  + \bigl[\E₃(v_i, v_j, v_k) \order[2]{ξ_k} + \order[2]\lambda \dot{\E}₂(v_i, v_j)\bigr] \order[0]{χ_j} \\
  +2\bigl[\E₃(v_i, v_j,  v_k)  \order[1]{ξ_k}
  + \order[1]\lambda \dot{\E}₂(v_i, v_j)\bigr] \order[1]{χ_i}
  = 2\order[1]α\order[1]{χ_i}
  + \order[2]α \order[0]{χ_i},
\end{multline*}
and Eq.~\eqref{eq:20220616082923} results from the application of
Eqs.~\eqref{eq:20220617084433} and \eqref{eq:20220617085256}.

\end{document}

%%% Local Variables:
%%% coding: utf-8
%%% fill-column: 80
%%% mode: latex
%%% TeX-engine: xetex
%%% TeX-master: t
%%% End:
