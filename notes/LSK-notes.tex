\documentclass{article}
\usepackage[french]{babel}
\usepackage{amsmath,amssymb,latexsym,theorem}
\usepackage[tikz]{mdframed}

%%%%%%%%%% Start TeXmacs macros
\newcommand{\mathlambda}{\lambda}
\newcommand{\nobracket}{}
\newcommand{\nocomma}{}
\newcommand{\nosymbol}{}
\newcommand{\textdots}{...}
\newcommand{\tmaffiliation}[1]{\\ #1}
\newcommand{\tmem}[1]{{\em #1\/}}
\newcommand{\tmemail}[1]{\\ \textit{Email:} \texttt{#1}}
\newenvironment{proof}{\noindent\textbf{Proof\ }}{\hspace*{\fill}$\Box$\medskip}
\mdfsetup{linecolor=black,linewidth=0.5pt,skipabove=0.5em,skipbelow=0.5em,hidealllines=true,innerleftmargin=0pt,innerrightmargin=0pt,innertopmargin=0pt,innerbottommargin=0pt}
{\theorembodyfont{\rmfamily}\newtheorem{remark}{Remark}}
\newtheorem{theorem}{Theorem}
\newmdenv[hidealllines=false,innertopmargin=1ex,innerbottommargin=1ex,innerleftmargin=1ex,innerrightmargin=1ex]{tmframed}
%%%%%%%%%% End TeXmacs macros

%


\begin{document}

\title{Notes relatives {\`a} la m{\'e}thode asymptotique de
Lyapunov--Schmidt--Koiter}

\author{
  S{\'e}bastien Brisard
  \tmaffiliation{Univ Gustave Eiffel, Ecole des Ponts, IFSTTAR, CNRS, Navier,
  F-77454 Marne-la-Vall{\'e}e, France}
  \tmemail{sebastien.brisard@univ-eiffel.fr}
}

\maketitle

\begin{abstract}
  blabla
\end{abstract}

\section{Notations}

L'espace des champs cin{\'e}matiquement admissibles est not{\'e} $U$. On
suppose qu'il a la structure d'espace vectoriel. L'{\'e}nergie du syst{\`e}me
est not{\'e}e $\mathcal{E} (u, \lambda)$, o{\`u} $\lambda$ d{\'e}signe un
param{\`e}tre de chargement. Soit $u^{\ast} (\lambda)$ la branche
fondamentale. Par d{\'e}finition
\begin{equation}
  \mathcal{E}_{, u} [u^{\ast} (\lambda), \lambda ; \hat{u}] = 0 \quad
  \text{pour tout} \quad \hat{u} \in U.
\end{equation}
Il sera commode d'introduire les notations suivantes
\begin{equation}
  \mathcal{E}_2 (\lambda) =\mathcal{E}_{, u \nocomma u}  [u^{\ast} (\lambda),
  \lambda], \quad \mathcal{E}_3 (\lambda) =\mathcal{E}_{, u \nocomma u
  \nocomma u} [u^{\ast} (\lambda), \lambda], \quad \mathcal{E}_4 (\lambda)
  =\mathcal{E}_{, u \nocomma u \nocomma u \nocomma u} [u^{\ast} (\lambda),
  \lambda] .
\end{equation}
Noter que $\mathcal{E}_2$, $\mathcal{E}_3$ et $\mathcal{E}_4$ sont des formes
bi-, tri- et quadri-lin{\'e}aires, respectivement. L'application de ces formes
{\`a} des {\'e}l{\'e}ments de $U$ sera not{\'e}e $\mathcal{E}_2 (\lambda ; u,
v)$, $\mathcal{E}_3 (\lambda ; u, v, w)$, etc.... La d{\'e}riv{\'e}e de ces
formes par rapport {\`a} $\lambda$ sera not{\'e}e {\`a} l'aide d'un point
sup{\'e}rieur ($\dot{\mathcal{E}_2}$, $\dot{\mathcal{E}_3}$, ...).

On suppose que l'{\'e}quilibre est stable pour des valeurs suffisamment
petites de $\lambda$. Plus pr{\'e}cis{\'e}ment, on suppose que $\mathcal{E}_2
(\lambda)$ est d{\'e}finie positive pour tout $\lambda < \lambda_0$. Pour
$\lambda = \lambda_0$, la forme quadratique $\mathcal{E}_2 (\lambda_0)$ n'est
plus que positive. En notant $u_0 = u^{\ast} (\lambda_0)$ la position
d'{\'e}quilibre obtenue pour la valeur critique $\lambda_0$ du param{\`e}tre
de chargement $\lambda$, on s'int{\'e}resse {\`a} toutes les courbes
d'{\'e}quilibre qui passent par le point $(u_0, \lambda_0)$.

Noter que dans ce qui suit, on convient que les formes $\mathcal{E}_2$,
$\mathcal{E}_3$ et $\mathcal{E}_4$ sont implicitement {\'e}valu{\'e}es en
$\lambda_0$ lorsque $\lambda$ n'est pas rappel{\'e} : ainsi, on notera
$\mathcal{E}_2 (\bullet, \bullet)$ plut{\^o}t que $\mathcal{E}_2 (\lambda_0 ;
\bullet, \bullet)$.

\section{Analyse de la branche fondamentale}

On s'int{\'e}resse dans ce paragraphe {\`a} la stabilit{\'e} du point critique
$(u_0, \lambda_0) .$ Par hypoth{\`e}se, $\mathcal{E}_2 (\lambda_0)$ est
positive, sans {\^e}tre d{\'e}finie positive~; soit $V$ son noyau, qui forme
un sous-espace vectoriel de $U$. On suppose que $V$ est de dimension finie $m
= \dim V$. Soit $(v_1, \ldots, v_m)$ une base orthonorm{\'e}e de ce noyau pour
le produit scalaire $\langle \bullet, \bullet \rangle$ (qui n'est pas
pr{\'e}cis{\'e} pour le moment). On introduit le sous-espace
suppl{\'e}mentaire orthogonal $W$ de $V$ dans $U$
\begin{equation}
  U = V \overset{\perp}{\otimes} W.
\end{equation}
Pour {\'e}tudier la stabilit{\'e} de l'{\'e}quilibre, on calcule l'{\'e}nergie
dans un {\'e}tat $u_0 + \xi v + \eta w$ voisin du point d'{\'e}quilibre $u_0$,
avec $\xi, \eta \in \mathbb{R}$ {\guillemotleft} petits {\guillemotright}, $v
\in V$ and $w \in W$. On obtient alors, {\`a} l'ordre 4 en $\xi$ et $\eta$
\begin{eqnarray}
  \Delta \mathcal{E} & = & \mathcal{E} (u_0 + \xi v + \eta w, \lambda_0)
  -\mathcal{E} (u_0, \lambda_0) \nonumber\\
  & = & \tfrac{1}{2} \mathcal{E}_2 (\xi v + \eta w, \xi v + \eta w) +
  \tfrac{1}{6} \mathcal{E}_3 (\xi v + \eta w, \xi v + \eta w, \xi v + \eta w)
  \nonumber\\
  &  & \nosymbol + \tfrac{1}{24} \mathcal{E}_4 (\xi v + \eta w, \xi v + \eta
  w, \xi v + \eta w, \xi v + \eta w) +\mathcal{O} [(\xi^2 + \eta^2)^2], 
\end{eqnarray}
o{\`u} le terme lin{\'e}aire a {\'e}t{\'e} omis puisque $u_0$ est un point
critique de l'{\'e}nergie. En tenant compte de la multilin{\'e}arit{\'e} et de
la sym{\'e}trie des diff{\'e}rentielles successives de l'{\'e}nergie
$\mathcal{E}$, ainsi que du fait que $\mathcal{E}_2 (v, \bullet) = 0$ (puisque
$v \in V$), l'expression pr{\'e}c{\'e}dente s'{\'e}crit
\begin{eqnarray}
  \Delta \mathcal{E} & = & \tfrac{1}{2} \eta^2 \mathcal{E}_2 (w, w) +
  \tfrac{1}{6} \xi^3 \mathcal{E}_3 (v, v, v) + \tfrac{1}{2} \xi^2 \eta
  \mathcal{E}_3 (v, v, w) \nonumber\\
  &  & + \tfrac{1}{2} \xi \eta^2 \mathcal{E}_3 (v, w, w) + \tfrac{1}{6}
  \eta^3 \mathcal{E}_3 (w, w, w) \nonumber\\
  &  & + \tfrac{1}{24} \xi^4 \mathcal{E}_4 (v, v, v, v) + \tfrac{1}{6} \xi^3
  \eta \mathcal{E}_4 (v, v, v, w) \nonumber\\
  &  & + \tfrac{1}{4} \xi^2 \eta^2 \mathcal{E}_4 (v, v, w, w) + \tfrac{1}{6}
  \xi \eta^3 \mathcal{E}_4 (v, w, w, w) \nonumber\\
  &  & + \tfrac{1}{24} \eta^4 \mathcal{E}_4 (w, w, w, w) +\mathcal{O} [(\xi^2
  + \eta^2)^2], 
\end{eqnarray}
o{\`u} l'on convient que toutes les diff{\'e}rentielles de $\mathcal{E}$ sont
{\'e}valu{\'e}es au point d'{\'e}quilibre $u_0$.

Pour que l'{\'e}quilibre soit stable, il faut que cette expression soit
positive ou nulle pour tous $\xi$ et $\eta$ suffisamment petits. En prenant
tout d'abord $\eta = 0$, on obtient les conditions n{\'e}cessaires
\begin{equation}
  \label{eq20211108164416} \mathcal{E}_3 (v, v, v) = 0 \quad \text{et} \quad
  \mathcal{E}_4 (v, v, v, v) \geq 0 \quad \text{pour tout} \quad v \in V.
\end{equation}
En d'autres termes, s'il existe $v \in V$ tel que $\mathcal{E}_3 (v, v, v)
\neq 0$ ou $\mathcal{E}_4 (v, v, v, v) < 0$, alors l'{\'e}quilibre est
{\tmem{instable}}. Les conditions pr{\'e}c{\'e}dentes ne sont pas suffisantes
pour assurer la stabilit{\'e}. En effet, supposant ces conditions remplies, on
prend maintenant $\eta = \xi^2$
\begin{equation}
  \Delta \mathcal{E}= \tfrac{1}{2} \xi^4  \left[ \mathcal{E}_2 (w, w)
  +\mathcal{E}_3 (v, v, w) + \tfrac{1}{12} \mathcal{E}_4 (v, v, v, v) \right]
  + o (\xi^4)
\end{equation}
et on obtient la condition n{\'e}cessaire suppl{\'e}mentaire
\begin{equation}
  \label{eq20211109145356} \mathcal{E}_2 (w, w) +\mathcal{E}_3 (v, v, w) +
  \tfrac{1}{12} \mathcal{E}_4 (v, v, v, v) \geq 0,
\end{equation}
pour tous $v \in V$ et $w \in W$. Pour $v \in V$ fix{\'e}, l'expression
pr{\'e}c{\'e}dente est minimale lorsque $w$ satisfait le probl{\`e}me
variationnel
\begin{equation}
  \label{eq20211109145224} 2\mathcal{E}_2 (w, \hat{w}) +\mathcal{E}_3 (v, v,
  \hat{w}) = 0 \quad \text{pour tout} \quad \hat{w} \in W.
\end{equation}
Soit $w_{i \nocomma j} \in W$ l'unique solution du probl{\`e}me variationnel
suivant
\begin{equation}
  \label{eq:pbvar wij} \mathcal{E}_2 (w_{i \nocomma j}, \hat{w})
  +\mathcal{E}_3 (v_i, v_j, \hat{w}) = 0 \quad \text{pour tout} \quad \hat{w}
  \in W.
\end{equation}
Alors, pour $v = \xi^i v_i$, la solution du probl{\`e}me
variationnel~\eqref{eq20211109145224} est $w = = \tfrac{1}{2} \xi^i \xi^j w_{i
\nocomma j}$. Pour cette valeur de $v$, la condition~\eqref{eq20211109145356}
s'{\'e}crit
\begin{equation}
  [\mathcal{E}_4 (v_i, v_j, v_k, v_l) - 3\mathcal{E}_2 (w_{i \nocomma j}, w_{k
  \nocomma l})] \xi^i \xi^j \xi^k \xi^l \geq 0,
\end{equation}
pour tous $\xi_i, \xi_j, \xi_k, \xi_l \in \mathbb{R}$. On peut montrer que
l'in{\'e}galit{\'e} stricte est une condition {\tmem{suffisante}} de
stabilit{\'e}.

\section{Bifurcations}

On {\'e}crit toute courbe d'{\'e}quilibre passant par le point $(u_0,
\lambda_0)$ sous la forme param{\'e}trique suivante
\begin{eqnarray}
  \lambda & = & \lambda_0 + \eta \lambda_1 + \tfrac{1}{2} \eta^2 \lambda_2 +
  \tfrac{1}{6} \eta^3 \lambda_3 + \cdots,  \label{eq20211115075817}\\
  u & = & u^{\ast} (\lambda) + \eta u_1 + \tfrac{1}{2} \eta^2 u_2 +
  \tfrac{1}{6} \eta^3 u_3 + \cdots,  \label{eq20211115075835}
\end{eqnarray}
o{\`u} $\eta$ est un param{\`e}tre, non pr{\'e}cis{\'e} pour le moment. Noter
que, dans la repr{\'e}sentation param{\'e}trique de $u$, $u^{\ast}$ est
{\'e}valu{\'e} en $\lambda$ et pas en $\lambda_0$.

Les coefficients $\lambda_k$ et $u_k$ des
d{\'e}veloppements~\eqref{eq20211115075817} et \eqref{eq20211115075835} sont
identifi{\'e}s en {\'e}crivant que l'{\'e}nergie est stationnaire le long de
la courbe d'{\'e}quilibre, c'est-{\`a}-dire que le r{\'e}sidu $\mathcal{E}_{,
u}  [u (\eta), \lambda (\eta)]$ est nul. Le d{\'e}veloppement limit{\'e} du
r{\'e}sidu est {\'e}tabli au voisinage de $\eta = 0$ dans
l'annexe~\ref{sec20211112182000} [voir {\'E}q.~\eqref{eq20220107080901}]. En
{\'e}crivant que tous ses termes s'annulent, on trouve successivement, pour
tout $\hat{u} \in U$
\begin{equation}
  \label{eq20211112182917} \mathcal{E}_2 (\lambda_0 ; u_1, \hat{u}) = 0,
\end{equation}
\begin{equation}
  \label{eq:res2} \mathcal{E}_3 (\lambda_0 ; u_1, u_1, \hat{u}) + 2 \lambda_1 
  \dot{\mathcal{E}_2} (\lambda_0 ; u_1, \hat{u}) +\mathcal{E}_2 (\lambda_0 ;
  u_2, \hat{u}) = 0,
\end{equation}
\begin{eqnarray}
  \mathcal{E}_4 (\lambda_0 ; u_1, u_1, u_1, \hat{u}) + 3\mathcal{E}_3
  (\lambda_0 ; u_1, u_2, \hat{u}) +\mathcal{E}_2 (\lambda_0 ; u_3, \hat{u}) & 
  &  \nonumber\\
  + 3 \lambda_1  \dot{\mathcal{E}_3} (\lambda_0 ; u_1, u_1, \hat{u}) + 3
  \lambda_1  \dot{\mathcal{E}_2} (\lambda_0 ; u_2, \hat{u}) &  &  \nonumber\\
  + 3 \lambda_1^2  \ddot{\mathcal{E}_2} (\lambda_0 ; u_1, \hat{u}) + 3
  \lambda_2  \dot{\mathcal{E}_2} (\lambda_0 ; u_1, \hat{u}) & = & 0. 
  \label{eq:res3}
\end{eqnarray}
On d{\'e}duit de l'{\'e}quation~\eqref{eq20211112182917} que $u_1 \in V$. En
prenant la fonction test {\'e}galement dans $V$, on d{\'e}duit de
l'{\'e}quation~\eqref{eq:res2} que $u_1$ est solution du probl{\`e}me suivant
: trouver $u_1 \in V$ tel que
\begin{equation}
  \label{eq:bifurcation 1a} \tfrac{1}{2} \mathcal{E}_3 (\lambda_0 ; u_1, u_1,
  \hat{v}) + \lambda_1  \dot{\mathcal{E}_2} (\lambda_0 ; u_1, \hat{v}) = 0,
\end{equation}
pour tout $\hat{v} \in V$. On remarque d'ores et d{\'e}j{\`a} que si \ l'est
{\'e}galement. Il est commode de transformer l'{\'e}quation de bifurcation
\eqref{eq:bifurcation 1a}, intrins{\`e}que, en un syst{\`e}me d'{\'e}quations
scalaires. {\`A} cet effet, on d{\'e}compose $u_1 \in V$ dans la base
$(v_i)_{1 \leqslant i \leqslant m}$
\begin{equation}
  \label{eq:decomposition u1} u_1 = \xi_1^i v_i .
\end{equation}
En prenant $\hat{v} = v_i$, l'{\'e}quation~\eqref{eq:bifurcation 1a}
s'{\'e}crit
\begin{equation}
  \label{eq:bifurcation 1b} \tfrac{1}{2} \mathcal{E}_3 (\lambda_0 ; v_i, v_j,
  v_k)  \hspace{0.17em} \xi_1^j \xi_1^k + \lambda_1  \dot{\mathcal{E}}_2
  (\lambda_0 ; v_i, v_j)  \hspace{0.17em} \xi_1^j = 0.
\end{equation}
On obtient ainsi un syst{\`e}me de $m$ {\'e}quations quadratiques {\`a} $(m +
1)$ inconnues, qui permet en g{\'e}n{\'e}ral de d{\'e}terminer les valeurs de
$\lambda_1$ et $u_1$ (voir discussion ci-apr{\`e}s ***TODO -- Compl{\'e}ter
r{\'e}f{\'e}rence***).

Afin de d{\'e}terminer les termes suivants du d{\'e}veloppement asymptotique
de la branche bifurqu{\'e}e, soit $\lambda_2$ et $u_2$, on introduit la
d{\'e}composition
\begin{equation}
  u_2 = \xi_2^i v_i + \tilde{u}_2,
\end{equation}
o{\`u} $\tilde{u}_2 \in W$ est la projection orthogonale de $u_2$ sur
$W$(notation provisoire). On a alors $\mathcal{E}_2 (u_2, \hat{u})
=\mathcal{E}_2 (\tilde{u}_2, \hat{u})$ et l'{\'e}quation~\eqref{eq:res2}
s'{\'e}crit
\begin{equation}
  \mathcal{E}_3 (\lambda_0 ; u_1, u_1, \hat{u}) + 2 \lambda_1 
  \dot{\mathcal{E}_2} (\lambda_0 ; u_1, \hat{u}) +\mathcal{E}_2 (\lambda_0 ;
  \tilde{u}_2, \hat{u}) = 0,
\end{equation}
pour tout $\hat{u} \in U$. En prenant cette fois-ci la fonction test dans
l'espace $W$, on obtient le probl{\`e}me variationnel suivant~: trouver
$\tilde{u}_2 \in W$ tel que
\begin{equation}
  \label{eq20211210131623} \mathcal{E}_2 (\lambda_0 ; {\tilde{u}_2} , \hat{w})
  + \xi_1^i \xi_1^j \mathcal{E}_3 (\lambda_0 ; v_i, v_j, \hat{w}) + 2
  \lambda_1 \xi_1^i  \dot{\mathcal{E}_2} (\lambda_0 ; v_i, \hat{w}) = 0,
\end{equation}
pour tout $\hat{w} \in W$. Soient $w_i \in W$ les solutions des probl{\`e}mes
variationnels suivants
\begin{equation}
  \label{eq:pbvar wi} \mathcal{E}_2 (\lambda_0 ; w_i, \hat{w}) + 2
  \dot{\mathcal{E}_2} (\lambda_0 ; v_i, \hat{w}) = 0,
\end{equation}
pour tout $\hat{w} \in W$. La solution du
probl{\`e}me~\eqref{eq20211210131623} s'obtient par simple combinaison
lin{\'e}aire des $w_i$ et $w_{ij}$ [on rappelle que ces derniers sont
d{\'e}finis par le probl{\`e}me variationnel~\eqref{eq:pbvar wij}]
\begin{equation}
  \tilde{u}_2 = \xi_1^i \xi_1^j w_{i \nocomma j} + \lambda_1 \xi_1^i w_i,
\end{equation}
de sorte que
\begin{equation}
  \label{eq:decomposition u2} u_2 = \xi_2^i v_i + \xi_1^i \xi_1^j w_{i
  \nocomma j} + \lambda_1 \xi_1^i w_i .
\end{equation}
En introduisant les expressions~\eqref{eq:decomposition u1} et
\eqref{eq:decomposition u2} dans l'{\'e}quation~\eqref{eq:res3} et en prenant
de plus $\hat{u} = v_i$, on obtient alors les {\'e}quations suivantes
\begin{eqnarray}
  3 [\mathcal{E}_3 (\lambda_0 ; v_i, v_j, v_k) \xi_1^k + \lambda_1 
  \dot{\mathcal{E}}_2 (\lambda_0 ; v_i, v_j)] \xi_2^j + 3 \lambda_2 
  \dot{\mathcal{E}}_2 (\lambda_0 ; v_i, v_j) \xi_1^j &  &  \nonumber\\
  + [\mathcal{E}_4 (\lambda_0 ; v_i, v_j, v_k, v_l) + 3\mathcal{E}_3
  (\lambda_0 ; v_i, v_j, w_{k \nocomma l})] \xi_1^j \xi_1^k \xi_1^l &  & 
  \nonumber\\
  + 3 \lambda_1  [\dot{\mathcal{E}}_3 (\lambda_0 ; v_i, v_j, v_k)
  +\mathcal{E}_3 (\lambda_0 ; v_i, v_j, w_k) + \dot{\mathcal{E}_2} (\lambda_0
  ; v_i, w_{j \nocomma k})] \xi_1^j \xi_1^k &  &  \nonumber\\
  + 3 \lambda_1^2  [\ddot{\mathcal{E}}_2 (\lambda_0 ; v_i, v_j) +
  \dot{\mathcal{E}_2} (\lambda_0 ; v_i, w_j)] \xi_1^j & = & 0, 
  \label{eq:bifurcation 2a}
\end{eqnarray}
qui permet en principe de d{\'e}terminer $\lambda_2$ ainsi que les $\xi_2^i$.
On montre dans le paragraphe \ref{sec:Simplification des �quations de
bifurcation} que les {\'e}quations \eqref{eq:bifurcation 1b} et
\eqref{eq:bifurcation 2a} peuvent s'{\'e}crire sous la forme suivante
\begin{equation}
  \label{eq:bifurcation 1c} \tfrac{1}{2} E_{i \nocomma j \nocomma k}
  (\lambda_0) \xi_1^j \xi_1^k + \lambda_1 F_{i \nocomma j} (\lambda_0) \xi_1^j
  = 0,
\end{equation}
\begin{equation}
  \label{eq:bifurcation 2b} \tfrac{1}{3} E_{i \nocomma j \nocomma k \nocomma
  l} (\lambda_0)  \hspace{0.17em} \xi_1^j \xi_1^k \xi_1^l + \lambda_2 F_{i
  \nocomma j} (\lambda_0) \xi_1^j + A_{i \nocomma j} (\lambda_0) \xi_2^j +
  \lambda_1  \dot{A}_{i \nocomma j} (\lambda_0) \xi_1^j = 0,
\end{equation}
o{\`u} les tenseurs $E_{i \nocomma j \nocomma k}$, $E_{i \nocomma j \nocomma k
\nocomma l}$, $F_{i \nocomma j}$ et $A_{i \nocomma j}$ sont d{\'e}finis comme
suit ***je ne suis pas s{\^u}r du terme faisant intervenir $\dot{A}_{i
\nocomma j} (\lambda_0)$***
\begin{equation}
  \label{eq:def Eijk} E_{i \nocomma j \nocomma k} (\lambda) =\mathcal{E}_3
  (\lambda ; v_i, v_j, v_k) +\mathcal{E}_2 (\lambda  ; v_i, w_{j \nocomma k})
  +\mathcal{E}_2 (\lambda ; v_j, w_{i \nocomma k}) +\mathcal{E}_2 (\lambda ;
  v_k, w_{i \nocomma j}),
\end{equation}
\begin{equation}
  \label{eq:def Eijkl} E_{i \nocomma j \nocomma k \nocomma l} (\lambda)
  =\mathcal{E}_4 (\lambda  ; v_i, v_j, v_k, v_l) +\mathcal{E}_3 (\lambda ;
  v_i, v_j, w_{k \nocomma l}) +\mathcal{E}_3 (\lambda ; v_i, v_k, w_{l
  \nocomma j}) +\mathcal{E}_3 (\lambda ; v_i, v_l, w_{j \nocomma k}),
\end{equation}
\begin{equation}
  \label{eq:def Fij} F_{i \nocomma j} (\lambda) = \dot{\mathcal{E}}_2 (\lambda
  ; v_i, v_j) + \tfrac{1}{2}  [\mathcal{E}_2 (\lambda  ; v_i, w_j)
  +\mathcal{E}_2 (\lambda  ; v_j, w_i)],
\end{equation}
\begin{equation}
  \label{eq:def Aij} A_{i \nocomma j} (\lambda) = E_{i \nocomma j \nocomma k}
  (\lambda) \xi_1^k + \lambda_1 F_{i \nocomma j} (\lambda) .
\end{equation}
Noter que tous ces tenseurs sont {\tmem{sym{\'e}triques}}. On remarque que,
puisque $\mathcal{E}_2 (\lambda_0 ; v_i, \bullet) = 0$, on a les
simplifications suivantes en $\lambda = \lambda_0$ : $E_{i \nocomma j \nocomma
k} (\lambda_0) =\mathcal{E}_3 (\lambda_0 ; v_i, v_j, v_k)$ et $F_{i \nocomma
j} (\lambda_0) = \dot{\mathcal{E}}_2 (\lambda_0 ; v_i, v_j)$.

\paragraph{Si la forme $\mathcal{E}_3 (\lambda_0)$ n'est pas nulle sur
$V$}L'{\'e}quation \eqref{eq:bifurcation 1c} admet au plus $(2^m - 1)$ paires
de solutions r{\'e}elles $(\lambda_1, u_1)$ et $(- \lambda_1, - u_1)$.

\begin{remark}
  Je ne sais pas d{\'e}montrer ce r{\'e}sultat sur le nombre de solutions
  r{\'e}elles.
\end{remark}

\paragraph{Si la forme $\mathcal{E}_3 (\lambda_0)$ est nulle sur
$V$}L'{\'e}quation \eqref{eq:bifurcation 1a} conduit n{\'e}cessairement {\`a}
$\lambda_1 = 0$, puisque $\dot{\mathcal{E}}_2 (\lambda_0)$ est d{\'e}finie
n{\'e}gative. D{\`e}s lors, l'{\'e}quation \eqref{eq:bifurcation 2b}
s'{\'e}crit\marginpar{Expliquer pourquoi la forme quadratique
$\dot{\mathcal{E}}_2 (\lambda_0)$ est bien d{\'e}finie n{\'e}gative}
\begin{equation}
  \tfrac{1}{3} E_{i \nocomma j \nocomma k \nocomma l} (\lambda_0) 
  \hspace{0.17em} \xi_1^j \xi_1^k \xi_1^l + \lambda_2 F_{i \nocomma j}
  (\lambda_0) \xi_1^j = 0.
\end{equation}
Cette {\'e}quation admet cette fois au plus $\frac{3^m - 1}{2}$ paires de
solutions r{\'e}elles $(\lambda_2, u_1)$ et $(- \lambda_2, - u_1)$.

\begin{remark}
  Je ne sais pas non plus d{\'e}montrer ce r{\'e}sultat sur le nombre de
  solutions r{\'e}elles.
\end{remark}

\begin{tmframed}
  \paragraph{Note du 29/04/2022}J'ai relu tous les calculs pr{\'e}c{\'e}dents.
  Il reste {\`a} reprendre les calculs des d{\'e}veloppements asymptotiques de
  l'{\'e}nergie et de sa hessienne, pour tenir compte en particulier des
  factorielles introduites maintenant dans les d{\'e}veloppements
  asymptotiques. Il faudrait {\'e}galement introduire les tenseurs
  pr{\'e}c{\'e}dents dans les expressions de l'{\'e}nergie et de sa hessienne.
\end{tmframed}

Le d{\'e}veloppement limit{\'e} suivant de l'{\'e}nergie le long de la branche
bifurqu{\'e}e est {\'e}tabli dans l'annexe~\ref{sec:DL energie}
\begin{eqnarray}
  \mathcal{E} [u (\eta), \lambda (\eta)] & = & \mathcal{E} [u^{\ast} [\lambda
  (\eta)], \lambda (\eta)] + \tfrac{1}{6} \lambda_1 \eta^3 F_{i \nocomma j}
  (\lambda_0) \xi_1^i \xi_1^j \nonumber\\
  &  & \tfrac{1}{24} \eta^4  \{ E_{i \nocomma j \nocomma k \nocomma l}
  (\lambda_0) \xi_1^i \xi_1^j \xi_1^k \xi_1^l + 4 \mathlambda_1  \dot{E}_{i
  \nocomma j \nocomma k} (\lambda_0) \xi_1^i \xi_1^j \xi_1^k \nobracket
  \nonumber\\
  &  & + \nobracket 6 [\mathlambda_1^2  \dot{F}_{i \nocomma j}
  (\mathlambda_0) + \lambda_2 F_{i \nocomma j} (\lambda_0)] \xi_1^i \xi_1^j \}
  + o (\eta^4) .  \label{eq:DL energie}
\end{eqnarray}
Si $\lambda_1 \neq 0$, le premier terme non-nul du d{\'e}veloppement
limit{\'e} pr{\'e}c{\'e}dent est d'ordre 3
\begin{equation}
  \mathcal{E} [u (\eta), \lambda (\eta)] =\mathcal{E} (u^{\ast} [\lambda
  (\eta)], \lambda (\eta)) + \tfrac{1}{6} \lambda_1 \eta^3 F_{i \nocomma j}
  (\lambda_0) \xi_1^i \xi_1^j + o (\eta^3),
\end{equation}
tandis que si $\lambda_1 = 0$, le premier terme est d'ordre 4
\begin{equation}
  \mathcal{E} [u (\eta), \lambda (\eta)] =\mathcal{E} (u^{\ast} [\lambda
  (\eta)], \lambda (\eta)) + \tfrac{1}{4} \lambda_2 \eta^4 F_{i \nocomma j}
  (\lambda_0) \xi_1^i \xi_1^j + o (\eta^4) .
\end{equation}
\begin{center}
  ***
\end{center}

Pour analyser la stabilit{\'e} de la branche bifurqu{\'e}e ainsi trouv{\'e}e,
il faut d{\'e}terminer le signe de la hessienne de l'{\'e}nergie. On peut
d'ores et d{\'e}j{\`a} remarquer que, sur la branche fondamentale ($u_1 = u_2
= 0$), en prenant $\eta = \lambda - \lambda_0$ ($\lambda_1 = 1$)
\begin{equation}
  \mathcal{E}_2 (\lambda ; \hat{u}, \hat{v}) =\mathcal{E}_2 (\lambda_0 ;
  \hat{u}, \hat{v}) + (\lambda - \lambda_0)  \dot{\mathcal{E}}_2 (\lambda_0 ;
  \hat{u}, \hat{v}) + o (\lambda - \lambda_0) .
\end{equation}
Dans ce qui suit, on supposera que $\dot{\mathcal{E}}_2 (\lambda_0) \neq 0$.
Pour $\hat{v} \in V$, l'{\'e}galit{\'e} pr{\'e}c{\'e}dente s'{\'e}crit
\begin{equation}
  \mathcal{E}_2 (\lambda_0 ; \hat{v}, \hat{v}) = (\lambda - \lambda_0) 
  \dot{\mathcal{E}}_2 (\hat{v}, \hat{v}) + o (\lambda - \lambda_0) .
\end{equation}
Comme la branche fondamentale est stable pour $\lambda < \lambda_0$, on doit
avoir $\dot{\mathcal{E}}_2 (\lambda_0 ; \hat{v}, \hat{v}) < 0$. La forme
quadratique $\dot{\mathcal{E}}_2 (\lambda_0)$ est donc d{\'e}finie
n{\'e}gative sur $V$. Le d{\'e}veloppement limit{\'e} de la hessienne de
l'{\'e}nergie le long de la branche bifurqu{\'e}e est {\'e}tabli dans
l'annexe~\ref{sec:DL hessienne}. Pour tous $\hat{u}, \hat{v} \in U$, on trouve
\begin{eqnarray}
  \mathcal{E}_{, u \nocomma u} [u (\eta), \lambda (\eta) ; \hat{u}, \hat{v}] &
  = & \mathcal{E}_2 (\lambda_0 ; \hat{u}, \hat{v}) + \eta [\mathcal{E}_3
  (\lambda_0 ; u_1, \hat{u}, \hat{v}) \nobracket \nobracket + \lambda_1 
  \dot{\mathcal{E}_2} (\lambda_0 ; \hat{u}, \hat{v})] \nonumber\\
  &  & \nosymbol + \tfrac{1}{2} \eta^2  [\mathcal{E}_4 (\lambda_0 ; u_1, u_1,
  \hat{u}, \hat{v}) \nobracket +\mathcal{E}_3 (\lambda_0 ; u_2, \hat{u},
  \hat{v}) + \lambda_2  \dot{\mathcal{E}_2} (\lambda_0 ; \hat{u}, \hat{v})
  \nonumber\\
  &  & \nosymbol + 2 \lambda_1  \dot{\mathcal{E}_3} (\lambda_0 ; u_1,
  \hat{u}, \hat{v}) + \lambda_1^2  \ddot{\mathcal{E}_2} (\lambda_0 ; \hat{u},
  \hat{v}) \nobracket] + o (\eta^2) .  \label{eq:DL hessienne}
\end{eqnarray}
Pour une analyse de stabilit{\'e}, on doit prendre $\hat{u} = \hat{v}$, soit
\begin{eqnarray}
  \mathcal{E}_{, u \nocomma u} [u (\eta), \lambda (\eta) ; \hat{u}, \hat{u}] &
  = & \mathcal{E}_2 (\lambda_0 ; \hat{u}, \hat{u}) + \eta [\mathcal{E}_3
  (\lambda_0 ; u_1, \hat{u}, \hat{u}) + \lambda_1  \dot{\mathcal{E}}_2
  (\lambda_0 ; \hat{u}, \hat{u})] \nonumber\\
  &  & + \tfrac{1}{2} \eta^2  [\mathcal{E}_4 (\lambda_0 ; u_1, u_1, \hat{u},
  \hat{u}) +\mathcal{E}_3 (\lambda_0 ; u_2, \hat{u}, \hat{u}) + \lambda_2 
  \dot{\mathcal{E}}_2 (\lambda_0 ; \hat{u}, \hat{u}) \nobracket \nonumber\\
  &  & + \nobracket 2 \lambda_1  \dot{\mathcal{E}}_3 (\lambda_0 ; u_1,
  \hat{u}, \hat{u}) + \lambda_1^2  \ddot{\mathcal{E}}_2 (\lambda_0 ; \hat{u},
  \hat{u})] + o (\eta^2) .  \label{eq:DL hessienne diag}
\end{eqnarray}
On peut d{\'e}composer le vecteur $\hat{u} \in U$ de fa{\c c}on unique sous la
forme $\hat{u} = \hat{v} + \hat{w}$, avec $\hat{v} \in V$ et $\hat{w} \in W$.
Le terme constant du d{\'e}veloppement pr{\'e}c{\'e}dent vaut alors
$\mathcal{E}_2 (\lambda_0 ; \hat{w}, \hat{w})$. Si $\hat{w} \neq 0$, alors ce
terme constant est strictement positif, puisque la hessienne est d{\'e}finie
positive sur $W$ en $\lambda = \lambda_0$. Au voisinage du point de
bifurcation, la hessienne sur la branche bifurqu{\'e}e est donc positive pour
tout $\hat{u} \in U$ ayant une composante dans $W$. Il suffit donc
d'{\'e}tudier le signe de la hessienne sur la branche bifurqu{\'e}e pour
$\hat{u} \in V$, soit $\hat{u} = \hat{\xi}^i v_i$. L'expression~\eqref{eq:DL
hessienne diag} se simplifie alors sous la forme suivante

Compte-tenu de l'expression~\eqref{eq:decomposition u2}
\begin{eqnarray}
  \mathcal{E}_{, u \nocomma u} [u (\eta), \lambda (\eta) ; \hat{u}, \hat{u}] &
  = & \eta [\mathcal{E}_3 (\lambda_0 ; v_i, v_j, v_k) \xi_1^k + \lambda_1 
  \dot{\mathcal{E}}_2 (\lambda_0 ; v_i, v_j)]  \hat{\xi}^i  \hat{\xi}^j
  \nonumber\\
  &  & + \tfrac{1}{2} \eta^2  [\mathcal{E}_4 (\lambda_0 ; v_i, v_j, v_k, v_l)
  \xi_1^k \xi_1^l +\mathcal{E}_3 (\lambda_0 ; v_i, v_j, v_k) \xi_2^k
  \nobracket \nonumber\\
  &  & +\mathcal{E}_3 (\lambda_0 ; v_i, v_j, w_{k \nocomma l}) \xi_1^k
  \xi_1^l + \lambda_1 \mathcal{E}_3 (\lambda_0 ; v_i, v_j, w_k) \xi_1^k +
  \lambda_2  \dot{\mathcal{E}}_2 (\lambda_0 ; v_i, v_j) \nonumber\\
  &  & \nobracket + 2 \lambda_1  \dot{\mathcal{E}}_3 (\lambda_0 ; v_i, v_j,
  v_k) \xi_1^k + \lambda_1^2  \ddot{\mathcal{E}}_2 (\lambda_0 ; v_i, v_j)] 
  \hat{\xi}^i  \hat{\xi}^j + o (\eta^2) \nonumber
\end{eqnarray}
Si $\lambda_1 \neq 0$, il suffit d'{\'e}tudier le signe de la forme
quadratique $[E_{i \nocomma j \nocomma k} (\lambda_0) \xi_1^k + \lambda_1 F_{i
\nocomma j} (\lambda_0)] .$ Si $\lambda_1 = 0$ et que $\mathcal{E}_3
(\lambda_0) = 0$ sur $V$, alors le d{\'e}veloppement limit{\'e}
pr{\'e}c{\'e}dent s'{\'e}crit
\begin{eqnarray}
  \mathcal{E}_{, u \nocomma u} [u (\eta), \lambda (\eta) ; \hat{u}, \hat{u}] &
  = & \tfrac{1}{2} \eta^2  \{ [\mathcal{E}_4 (\lambda_0 ; v_i, v_j, v_k, v_l)
  \nobracket +\mathcal{E}_3 (\lambda_0 ; v_i, v_j, w_{k \nocomma l})] \xi_1^k
  \xi_1^l \nonumber\\
  &  & + \nobracket \lambda_2  \dot{\mathcal{E}}_2 (\lambda_0 ; v_i, v_j) \} 
  \hat{\xi}^i  \hat{\xi}^j + o (\eta^2) \nonumber
\end{eqnarray}
\begin{tmframed}
  12/05/2022 Relecture jusqu'{\`a} l'{\'e}galit{\'e} pr{\'e}c{\'e}dente. Je
  suis un peu surpris, car je m'attendais {\`a} un terme en $3\mathcal{E}_3
  (\lambda_0 ; v_i, v_j, w_{k \nocomma l})${\textdots}
\end{tmframed}

Compte-tenu de la relation~\eqref{eq20211112183220}, on trouve pour $\hat{v} =
u_1$ ($\hat{\xi}^i = \xi_1^i$)
\begin{equation}
  \mathcal{E}_{, u \nocomma u} [u (\eta), \lambda (\eta) ; u_1, u_1] = -
  \lambda_1 \eta \dot{\mathcal{E}}_2 (\lambda_0 ; u_1, u_1) + o (\eta) .
\end{equation}
Si $\lambda_1 \neq 0$, l'expression pr{\'e}c{\'e}dente peut {\'e}galement
s'{\'e}crire
\begin{equation}
  \mathcal{E}_{, u \nocomma u} [u (\eta), \lambda (\eta) ; u_1, u_1] = -
  (\lambda - \lambda_0)  \dot{\mathcal{E}}_2 (\lambda_0 ; u_1, u_1) + o
  (\lambda - \lambda_0),
\end{equation}
qui est n{\'e}gative pour $\lambda < \lambda_0$: la branche bifurqu{\'e}e est
instable sous la charge critique. Il reste alors {\`a} {\'e}tudier le signe de
la hessienne de la branche bifurqu{\'e}e au-del{\`a} de la charge critique
($\lambda > \lambda_0$).

\section{Cas d'un mode de flambement simple ($m = 1$)}

Lorsque $m = \dim V = 1$, la base $v_1, \ldots, v_m$ est r{\'e}duite au seul
vecteur $v_1$ et $u_1$ est parall{\`e}le {\`a} ce vecteur. Comme $\lVert u_1
\rVert = 1$, on a donc n{\'e}cessairement $u_1 = v_1$ (quitte {\`a} changer
$\eta$ en $- \eta$). L'{\'e}quation de bifurcation~\eqref{eq20220216140121}
s'{\'e}crit alors
\begin{equation}
  \label{eq20220203144712} \mathcal{E}_{1 \nocomma 1 \nocomma 1} (\lambda_0) +
  2 \lambda_1  \dot{\mathcal{E}}_{1 \nocomma 1} (\lambda_0) = 0, \quad
  \text{soit} \quad \lambda_1 = - \frac{\mathcal{E}_{1 \nocomma 1 \nocomma 1}
  (\lambda_0)}{2 \dot{\mathcal{E}}_{1 \nocomma 1} (\lambda_0)},
\end{equation}
o{\`u} on remarque que le quotient a un sens, puisque $\dot{\mathcal{E}_2}
(\lambda_0)$ est d{\'e}finie n{\'e}gative sur $V$. On trouve donc les
d{\'e}veloppements limit{\'e}s
\begin{equation}
  \lambda = \lambda_0 + \lambda_1 \eta + o (\eta)  \quad \text{et} \quad u =
  u^{\ast} (\lambda) + \eta v_1 + o (\eta),
\end{equation}
soit finalement, en {\'e}liminant $\eta$
\begin{equation}
  \lambda = \lambda_0 - \frac{\xi \mathcal{E}_{1 \nocomma 1 \nocomma 1}
  (\lambda_0)}{2 \dot{\mathcal{E}}_{1 \nocomma 1} (\lambda_0)} + o (\xi),
  \quad \text{avec} \quad \xi = \langle u (\lambda) - u^{\ast} (\lambda), v_1
  \rangle .
\end{equation}
Pour d{\'e}terminer la stabilit{\'e} de la branche bifurqu{\'e}e, on calcule
la hessienne en $(v_1, v_1)$. L'{\'e}quation~\eqref{eq20220203144500}
s'{\'e}crit
\begin{equation}
  \mathcal{E}_{, u \nocomma u} [u (\eta), \lambda (\eta) ; v_1, v_1] = \eta
  [\mathcal{E}_{1 \nocomma 1 \nocomma 1} (\lambda_0) + \lambda_1 
  \dot{\mathcal{E}}_{1 \nocomma 1} (\lambda_0)] + o (\eta),
\end{equation}
soit, en substituant l'{\'e}quation~\eqref{eq20220203144712}
\begin{equation}
  \mathcal{E}_{, u \nocomma u} [u (\eta), \lambda (\eta) ; v_1, v_1] = -
  \lambda_1 \eta \dot{\mathcal{E}}_{1 \nocomma 1} (\lambda_0) + o (\eta) .
\end{equation}
Ce d{\'e}veloppement ne permet de conclure que si le terme lin{\'e}aire est
non-nul, soit $\mathcal{E}_{1 \nocomma 1 \nocomma 1} (\lambda_0) \neq 0$ [voir
{\'E}q.~\eqref{eq20220203144712}]. Dans ce cas, le d{\'e}veloppement
asymptotique pr{\'e}c{\'e}dent s'{\'e}crit {\'e}galement
\begin{equation}
  \mathcal{E}_{, u \nocomma u} [u (\eta), \lambda (\eta) ; v_1, v_1] = -
  (\lambda - \lambda_0)  \dot{\mathcal{E}}_{1 \nocomma 1} (\lambda_0) + o
  (\lambda - \lambda_0) .
\end{equation}
Comme $\dot{\mathcal{E}}_2 (\lambda_0)$ est d{\'e}finie n{\'e}gative, la
branche bifurqu{\'e}e est {\tmem{instable}} pour $\lambda < \lambda_0$ et
{\tmem{stable}} pour $\lambda > \lambda_0$ lorsque $\mathcal{E}_{1 \nocomma 1
\nocomma 1} (\lambda_0) \neq 0$.

Supposons maintenant que $\mathcal{E}_{1 \nocomma 1 \nocomma 1} (\lambda_0) =
0$~; alors $\lambda_1 = 0$ et il faut calculer au moins un terme
suppl{\'e}mentaire dans le d{\'e}veloppement limit{\'e} de la Hessienne.
L'{\'e}quation de bifurcation~\eqref{eq20220216141706} s'{\'e}crit
\begin{equation}
  \label{eq20220217164528} \mathcal{E}_{1 \nocomma 1 \nocomma 1 \nocomma 1}
  (\lambda_0) + 6\mathcal{E}_3 (\lambda_0 ; v_1, v_1, u_2) + 6 \lambda_2 
  \dot{\mathcal{E}}_{1 \nocomma 1} (\lambda_0) = 0.
\end{equation}
En introduisant le d{\'e}veloppement~\eqref{eq20220124135324} de $u_2$ et en
utilisant le probl{\`e}me variationnel~\eqref{eq20211221155859}
\begin{equation}
  u_2 = \xi_2 v_1 + w_{1 \nocomma 1} + \lambda_1 w_1,
\end{equation}
donc
\begin{equation}
  \mathcal{E}_3 (\lambda_0 ; v_1, v_1, u_2) =\mathcal{E}_3 (\lambda_0 ; v_1,
  v_1, w_{1 \nocomma 1}) = - 2\mathcal{E}_2 (\lambda_0 ; w_{11}, w_{11})
\end{equation}
soit finalement
\[ \lambda_2 = - \frac{\mathcal{E}_{1 \nocomma 1 \nocomma 1 \nocomma 1}
   (\lambda_0) - 12\mathcal{E}_2 (\lambda_0 ; w_{11}, w_{11})}{6
   \dot{\mathcal{E}}_{1 \nocomma 1} (\lambda_0)}, \]
le quotient ayant une nouvelle fois un sens. Le d{\'e}veloppement
asymptotique~\eqref{eq20211115082025} de la Hessienne s'{\'e}crit alors, en
tenant compte de l'{\'E}q.~\eqref{eq20220217164528}
\begin{eqnarray}
  \mathcal{E}_{, u \nocomma u} [u (\eta), \lambda (\eta) ; v_1, v_1] & = &
  \tfrac{1}{2} \eta^2  [\mathcal{E}_{1 \nocomma 1 \nocomma 1 \nocomma 1}
  (\lambda_0) + 2\mathcal{E}_3 (\lambda_0 ; v_1, v_1, u_2) + 2 \lambda_2 
  \dot{\mathcal{E}}_{1 \nocomma 1} (\lambda_0)] + o (\eta^2) \nonumber\\
  & = & \tfrac{5}{12} \eta^2 \mathcal{E}_{1 \nocomma 1 \nocomma 1 \nocomma 1}
  (\lambda_0) + o (\eta^2) . 
\end{eqnarray}

\section{Propri{\'e}t{\'e}s des formes bilin{\'e}aires sym{\'e}triques,
positives}

Dans ce qui suit, $\mathcal{B}$ d{\'e}signe une forme bilin{\'e}aire
sym{\'e}trique et positive sur l'espace vectoriel $U$. On d{\'e}finit son
noyau $\ker \mathcal{B}$ de la fa{\c c}on suivante
\begin{equation}
  \ker \mathcal{B}= \{u \in U, \mathcal{B}(u, u) = 0\} .
\end{equation}
\begin{theorem}
  Le noyau d'une forme bilin{\'e}aire, sym{\'e}trique et positive est un
  sous-espace vectoriel.
\end{theorem}

\begin{proof}
  Soient $u, v \in \ker \mathcal{B}$, $\alpha \in \mathbb{R}$ et $w = u +
  \alpha v$. Montrons que $w \in \ker \mathcal{B}$. Il suffit d'{\'e}valuer
  $\mathcal{B} (w, w)$
  \begin{equation}
    \mathcal{B} (w, w) =\mathcal{B} (u + \alpha v, u + \alpha v) =\mathcal{B}
    (u, u) + 2 \alpha \mathcal{B} (u, v) + \alpha^2 \mathcal{B} (v, v),
  \end{equation}
  o{\`u} l'on a tenu compte de la sym{\'e}trie de $\mathcal{B}$ pour
  {\'e}crire que $\mathcal{B} (u, v) =\mathcal{B} (v, u)$. Comme $u, v \in
  \ker \mathcal{B}$, le premier et le dernier terme sont nuls, soit
  $\mathcal{B} (w, w) = 2 \alpha \mathcal{B} (u, v)$. La forme bilin{\'e}aire
  {\'e}tant positive, cette grandeur est positive, {\tmem{quelle que soit la
  valeur de $\alpha \in \mathbb{R}$}}. On en d{\'e}duit donc que $\mathcal{B}
  (u, v) = 0$, puis que $\mathcal{B} (w, w) = 0$ et donc que $w \in \ker
  \mathcal{B}.$
\end{proof}

\begin{theorem}
  Soit $u \in V$. Alors
  \begin{equation}
    u \in \ker \mathcal{B} \quad \text{ssi} \quad \text{pour tout } v \in V,
    \mathcal{B} (u, v) = 0.
  \end{equation}
\end{theorem}

\begin{proof}
  Soient $u \in \ker \mathcal{B}$, $v \in V$ et $\alpha \in \mathbb{R}$. Comme
  pr{\'e}c{\'e}demment, on {\'e}crit que $\mathcal{B} (w, w) \geq 0$, avec $w
  = \alpha u + v$
  \begin{equation}
    \mathcal{B} (w, w) = 2 \alpha \mathcal{B} (u, v) +\mathcal{B} (v, v) \geq
    0,
  \end{equation}
  o{\`u} l'on a tenu compte de ce que $\mathcal{B} (u, u) = 0$. L'expression
  pr{\'e}c{\'e}dente, affine en $\alpha$, a un signe constant. Le terme
  lin{\'e}aire en $\alpha$ est donc nul, soit $\mathcal{B} (u, v) = 0$.
  R{\'e}ciproquement, si $\mathcal{B} (u, v) = 0$ pour tout $v \in V$, alors
  $\mathcal{B} (u, u) = 0$ (en prenant $v = u$).
\end{proof}

\section{D{\'e}veloppements limit{\'e}s le long d'une branche bifurqu{\'e}e du
diagramme d'{\'e}quilibre}

\subsection{Principe du calcul}\label{sec20220107121442}

On pose dans ce qui suit
\begin{eqnarray}
  \Lambda (\eta) & = & \lambda (\eta) - \lambda_0 = \eta \lambda_1 +
  \tfrac{1}{2} \eta^2 \lambda_2 + \tfrac{1}{6} \eta^3 \lambda_3 + \cdots, 
  \label{eq20211112155446}\\
  U (\eta) & = & u (\eta) - u^{\ast} [\lambda (\eta)] = \eta u_1 +
  \tfrac{1}{2} \eta^2 u_2 + \tfrac{1}{6} \eta^3 u_3 + \cdots . 
  \label{eq20211112113028}
\end{eqnarray}
On consid{\`e}re une fonctionnelle $\mathcal{F}$ de $u$ et $\lambda$~:
$\mathcal{F} (u, \lambda)$. Cette fonctionnelle est {\'e}valu{\'e}e le long de
la branche bifurqu{\'e}e. En d'autres termes, on consid{\`e}re
\begin{equation}
  f (\eta) = F \{ u^{\ast} [\lambda_0 + \Lambda (\eta)] + U (\eta), \lambda_0
  + \Lambda (\eta) \} .
\end{equation}
On souhaite {\'e}tablir un d{\'e}veloppement limit{\'e} de $f$ au voisinage de
$\eta = 0$, ce qui conduit {\`a} calculer les d{\'e}riv{\'e}es successives de
$f$ en $\eta = 0$, puisque
\begin{equation}
  f (\eta) = f (0) + \eta f' (0) + \tfrac{1}{2} \eta^2 f'' (0) + \cdots
\end{equation}
Pour calculer ces d{\'e}riv{\'e}es, il sera commode d'introduire la fonction
auxiliaire $F$
\begin{equation}
  F (\eta, \lambda) =\mathcal{F} [u^{\ast} (\lambda) + U (\eta), \lambda],
\end{equation}
dans laquelle les variables $\lambda$ et $\eta$ sont provisoirement
consid{\'e}r{\'e}es comme ind{\'e}pendantes. On a
\begin{equation}
  f (\eta) = F [\eta, \lambda_0 + \Lambda (\eta)],
\end{equation}
d'o{\`u} l'on d{\'e}duit successivement que
\begin{equation}
  \label{eq20211112162417} f' (\eta) = \partial_{\eta} F + \Lambda'
  \partial_{\lambda} F,
\end{equation}
\begin{equation}
  \label{eq20211112165810} f'' (\eta) = \partial_{\eta \nocomma \eta}^2 F + 2
  \Lambda' \partial_{\eta \nocomma \lambda}^2 {F + \Lambda'}^2
  \partial_{\lambda \nocomma \lambda}^2 F + \Lambda'' \partial_{\lambda} F,
\end{equation}
\begin{eqnarray}
  \label{eq20211112173223} f''' (\eta) & = & \partial_{\eta \nocomma \eta
  \nocomma \eta}^3 F + 3 \Lambda' \partial_{\eta \nocomma \eta \nocomma
  \lambda}^3 {F + 3 \Lambda'}^2 \partial_{\eta \nocomma \lambda \nocomma
  \lambda}^3 {F + \Lambda'}^3 \partial_{\lambda \nocomma \lambda \nocomma
  \lambda}^3 F + 3 \Lambda'' \partial_{\eta \nocomma \lambda}^2 F + 3 \Lambda'
  \Lambda'' \partial_{\lambda \nocomma \lambda}^2 F \nonumber\\
  &  & \nosymbol + \Lambda''' \partial_{\lambda} F 
\end{eqnarray}
\begin{eqnarray}
  f'''' (\eta) & = & \partial_{\eta \nocomma \eta \nocomma \eta \nocomma
  \eta}^4 F + 4 \Lambda' \partial_{\eta \nocomma \eta \nocomma \eta \nocomma
  \lambda}^4 {F + 6 \Lambda'}^2 \partial_{\eta \nocomma \eta \nocomma \lambda
  \nocomma \lambda}^4 {F + 4 \Lambda'}^3 \partial_{\eta \nocomma \lambda
  \nocomma \lambda \nocomma \lambda}^4 {F + \Lambda'}^4 \partial_{\lambda
  \nocomma \lambda \nocomma \lambda \nocomma \lambda}^4 F + 6 \Lambda''
  \partial_{\eta \nocomma \eta \nocomma \lambda}^3 F \nonumber\\
  &  & + 12 \Lambda' \Lambda'' \partial_{\eta \nocomma \lambda \nocomma
  \lambda}^3 {F + 6 \Lambda'}^2 \Lambda'' \partial_{\lambda \nocomma \lambda
  \nocomma \lambda}^3 F + 4 \Lambda''' \partial_{\eta \nocomma \lambda}^2 F +
  \left( {3 \Lambda''}^2 + 4 \Lambda' \Lambda''' \right) \partial_{\lambda
  \nocomma \lambda}^2 F \\
  &  & + \Lambda'''' \partial_{\lambda} F 
\end{eqnarray}
o{\`u} $\Lambda$ et ses d{\'e}riv{\'e}es sont {\'e}valu{\'e}es en $\eta$,
tandis que $F$ et ses d{\'e}riv{\'e}es partielles sont {\'e}valu{\'e}es en
$[\eta, \lambda_0 + \Lambda (\eta)]$. En $\eta = 0$, les relations
pr{\'e}c{\'e}dentes s'{\'e}crivent
\begin{equation}
  \label{eq20220107060454} f' (0) = \partial_{\eta} F + \lambda_1
  \partial_{\lambda} F,
\end{equation}
\begin{equation}
  \label{eq20220107124311} f'' (0) = \partial_{\eta \nocomma \eta}^2 F + 2
  \lambda_1 \partial_{\eta \nocomma \lambda}^2 F + \lambda_2
  \partial_{\lambda} F + \lambda_1^2 \partial_{\lambda \nocomma \lambda}^2 F,
\end{equation}
\begin{eqnarray}
  f''' (0) & = & \partial_{\eta \nocomma \eta \nocomma \eta}^3 F + 3 \lambda_1
  \partial_{\eta \nocomma \eta \nocomma \lambda}^3 F + 3 \lambda_1^2
  \partial_{\eta \nocomma \lambda \nocomma \lambda}^3 F + \lambda_1^3
  \partial_{\lambda \nocomma \lambda \nocomma \lambda}^3 F + 3 \lambda_2
  \partial_{\eta \nocomma \lambda}^2 F + 3 \lambda_1 \lambda_2
  \partial_{\lambda \nocomma \lambda}^2 F \nonumber\\
  &  & \nosymbol + \lambda_3 \partial_{\lambda} F,  \label{eq20220107060500}
\end{eqnarray}
\begin{eqnarray}
  f'''' (0) & = & \partial_{\eta \nocomma \eta \nocomma \eta \nocomma \eta}^4
  F + 4 \lambda_1 \partial_{\eta \nocomma \eta \nocomma \eta \nocomma
  \lambda}^4 F + 6 \lambda_1^2 \partial_{\eta \nocomma \eta \nocomma \lambda
  \nocomma \lambda}^4 F + 4 \lambda_1^3 \partial_{\eta \nocomma \lambda
  \nocomma \lambda \nocomma \lambda}^4 F + \lambda_1^4 \partial_{\lambda
  \nocomma \lambda \nocomma \lambda \nocomma \lambda}^4 F + 6 \lambda_2
  \partial_{\eta \nocomma \eta \nocomma \lambda}^3 F \nonumber\\
  &  & + 12 \lambda_1 \lambda_2 \partial_{\eta \nocomma \lambda \nocomma
  \lambda}^3 F + 6 \lambda_1^2 \lambda_2 \partial_{\lambda \nocomma \lambda
  \nocomma \lambda}^3 F + 4 \lambda_3 \partial_{\eta \nocomma \lambda}^2 F +
  (3 \lambda_2^2 + 4 \lambda_1 \lambda_3) \partial_{\lambda \nocomma
  \lambda}^2 F \nonumber\\
  &  & + \lambda_4 \partial_{\lambda} F, 
\end{eqnarray}
o{\`u} $F$ et ses d{\'e}riv{\'e}es sont {\'e}valu{\'e}es en $(0, \lambda_0)$.

\subsection{D{\'e}veloppement limit{\'e} du
r{\'e}sidu}\label{sec20211112182000}

On cherche un d{\'e}veloppement limit{\'e} du r{\'e}sidu (c'est-{\`a}-dire de
la premi{\`e}re variation de l'{\'e}nergie). La fonction test $\hat{u} \in U$
{\'e}tant fix{\'e}e, la m{\'e}thode pr{\'e}c{\'e}dente est donc appliqu{\'e}e
avec
\begin{equation}
  \label{eq20220107054629} f (\eta) =\mathcal{E}_{, u} [u (\eta), \lambda
  (\eta) ; \hat{u}] \quad \text{et} \quad F (\eta, \lambda) =\mathcal{E}_{, u}
  [u^{\ast} (\lambda) + U (\eta), \lambda ; \hat{u}] .
\end{equation}
On remarque tout d'abord que $F (0, \lambda) =\mathcal{E}_{, u} [u^{\ast}
(\lambda), \lambda ; \hat{u}] = 0$, puisque $u^{\ast} (\lambda)$ est un point
d'{\'e}quilibre. En d{\'e}rivant par rapport {\`a} $\lambda$, on obtient
\begin{equation}
  \label{eq20211112164240} \frac{\partial^k F}{\partial \lambda^k} (0,
  \lambda) = 0.
\end{equation}
En d{\'e}rivant par rapport {\`a} $\eta$ l'expression~\eqref{eq20220107054629}
de $F$, on obtient successivement
\begin{equation}
  \partial_{\eta} F (\eta, \lambda) =\mathcal{E}_{, u \nocomma u} [u^{\ast}
  (\lambda) + U (\eta), \lambda ; U' (\eta), \hat{u}],
\end{equation}
\begin{eqnarray}
  \partial_{\eta \nocomma \eta}^2 F (\eta, \lambda) & = & \mathcal{E}_{, u
  \nocomma u \nocomma u} [u^{\ast} (\lambda) + U (\eta), \lambda ; U' (\eta),
  U' (\eta), \hat{u}] \nonumber\\
  &  & \nosymbol +\mathcal{E}_{, u \nocomma u} [u^{\ast} (\lambda) + U
  (\eta), \lambda ; U'' (\eta), \hat{u}], 
\end{eqnarray}
\begin{eqnarray}
  \partial_{\eta \nocomma \eta \nocomma \eta}^3 F (\eta, \lambda) & = &
  \mathcal{E}_{, u \nocomma u \nocomma u \nocomma u} [u^{\ast} (\lambda) + U
  (\eta), \lambda ; U' (\eta), U' (\eta), U' (\eta), \hat{u}] \nonumber\\
  &  & \nosymbol + 3\mathcal{E}_{, u \nocomma u \nocomma u} [u^{\ast}
  (\lambda) + U (\eta), \lambda ; U' (\eta), U'' (\eta), \hat{u}] \nonumber\\
  &  & \nosymbol +\mathcal{E}_{, u \nocomma u} [u^{\ast} (\lambda) + U
  (\eta), \lambda ; U''' (\eta), \hat{u}], 
\end{eqnarray}
soit, en $\eta = 0$
\begin{equation}
  \partial_{\eta} F (0, \lambda) =\mathcal{E}_2 (\lambda ; u_1, \hat{u}),
\end{equation}
\begin{equation}
  \partial_{\eta \nocomma \eta}^2 F (0, \lambda) =\mathcal{E}_3 (\lambda ;
  u_1, u_1, \hat{u}) +\mathcal{E}_2 (\lambda ; u_2, \hat{u}),
\end{equation}
\begin{equation}
  \partial_{\eta \nocomma \eta \nocomma \eta}^3 F (0, \lambda) =\mathcal{E}_4
  (\lambda ; u_1, u_1, u_1, \hat{u}) + 3\mathcal{E}_3 (\lambda ; u_1, u_2,
  \hat{u}) +\mathcal{E}_2 (\lambda ; u_3, \hat{u}) .
\end{equation}
Les d{\'e}riv{\'e}es crois{\'e}es de $F$ en $(0, \lambda)$ s'obtiennent par
simple d{\'e}rivation des relations pr{\'e}c{\'e}dentes par rapport {\`a}
$\lambda$
\begin{equation}
  \partial_{\eta \nocomma \lambda}^2 F (0, \lambda) = \dot{\mathcal{E}_2}
  (\lambda ; u_1, \hat{u}),
\end{equation}
\begin{equation}
  \partial_{\eta \nocomma \eta \nocomma \lambda}^3 F (0, \lambda) =
  \dot{\mathcal{E}_3} (\lambda ; u_1, u_1, \hat{u}) + \dot{\mathcal{E}_2}
  (\lambda ; u_2, \hat{u}),
\end{equation}
\begin{equation}
  \partial_{\eta \nocomma \lambda \nocomma \lambda}^3 F (0, \lambda) =
  \ddot{\mathcal{E}_2} (\lambda ; u_1, \hat{u}) .
\end{equation}
En ins{\'e}rant les r{\'e}sultats pr{\'e}c{\'e}dentes dans les relations
g{\'e}n{\'e}rales~\eqref{eq20220107060454}--\eqref{eq20220107060500}, on
trouve alors les expressions suivantes des d{\'e}riv{\'e}es successives de $f$
en $\eta = 0$
\begin{equation}
  f' (0) =\mathcal{E}_2 (\lambda_0 ; u_1, \hat{u}),
\end{equation}
\begin{equation}
  f'' (0) =\mathcal{E}_3 (\lambda_0 ; u_1, u_1, \hat{u}) + 2 \lambda_1 
  \dot{\mathcal{E}_2} (\lambda_0 ; u_1, \hat{u}) +\mathcal{E}_2 (\lambda_0 ;
  u_2, \hat{u}),
\end{equation}
\begin{eqnarray}
  f''' (0) & = & \mathcal{E}_4 (\lambda_0 ; u_1, u_1, u_1, \hat{u}) +
  3\mathcal{E}_3 (\lambda_0 ; u_1, u_2, \hat{u}) +\mathcal{E}_2 (\lambda_0 ;
  u_3, \hat{u}) \nonumber\\
  &  & \nosymbol + 3 \lambda_1  [\dot{\mathcal{E}_3} (\lambda_0 ; u_1, u_1,
  \hat{u}) + \dot{\mathcal{E}_2} (\lambda_0 ; u_2, \hat{u})] \nonumber\\
  &  & \nosymbol + 3 \lambda_1^2  \ddot{\mathcal{E}_2} (\lambda_0 ; u_1,
  \hat{u}) + 3 \lambda_2  \dot{\mathcal{E}_2} (\lambda_0 ; u_1, \hat{u}) . 
\end{eqnarray}
On en d{\'e}duit finalement le d{\'e}veloppement limit{\'e} {\`a} l'ordre 3 en
$\eta$ du r{\'e}sidu
\begin{eqnarray}
  \mathcal{E}_{, u} [u (\eta), \lambda (\eta)] & = & \eta \mathcal{E}_2
  (\lambda_0 ; u_1, \hat{u}) \nonumber\\
  &  & \nosymbol + \tfrac{1}{2} \eta^2  [\mathcal{E}_3 (\lambda_0 ; u_1, u_1,
  \hat{u}) + 2 \lambda_1  \dot{\mathcal{E}_2} (\lambda_0 ; u_1, \hat{u})
  +\mathcal{E}_2 (\lambda_0 ; u_2, \hat{u})] \nonumber\\
  &  & \nosymbol + \tfrac{1}{6} \eta^3  \{ \mathcal{E}_4 (\lambda_0 ; u_1,
  u_1, u_1, \hat{u}) + 3\mathcal{E}_3 (\lambda_0 ; u_1, u_2, \hat{u})
  \nobracket +\mathcal{E}_2 (\lambda_0 ; u_3, \hat{u}) \nonumber\\
  &  & \nosymbol + 3 \lambda_1  [\dot{\mathcal{E}_3} (\lambda_0 ; u_1, u_1,
  \hat{u}) + \dot{\mathcal{E}_2} (\lambda_0 ; u_2, \hat{u})] + 3 \lambda_1^2 
  \ddot{\mathcal{E}_2} (\lambda_0 ; u_1, \hat{u}) \nonumber\\
  &  & \nobracket \nosymbol + 3 \lambda_2  \dot{\mathcal{E}_2} (\lambda_0 ;
  u_1, \hat{u}) \} + o (\eta^3) .  \label{eq20220107080901}
\end{eqnarray}
\subsection{D{\'e}veloppement limit{\'e} de l'{\'e}nergie}\label{sec:DL
energie}

On s'int{\'e}resse ici {\`a} l'{\'e}cart d'{\'e}nergie, pour un chargement
$\lambda$ donn{\'e}, entre la branche bifurqu{\'e}e et la branche
fondamentale, soit
\[ F (\eta, \lambda) =\mathcal{E} [u^{\ast} (\lambda) + U (\eta), \lambda]
   -\mathcal{E} [u^{\ast} (\lambda), \lambda] \]
et
\[ f (\eta) = F [\eta, \lambda_0 + \Lambda (\eta)] . \]
On observe tout d'abord que $F (0, \lambda) = 0$ pour tout $\lambda$, donc
\[ \frac{\partial^k F}{\partial \lambda^k} (0, \lambda) = 0 \quad (k \geq 0),
\]
tandis que les d{\'e}riv{\'e}es de $F$ par rapport {\`a} $\eta$ s'{\'e}crivent
\[ \partial_{\eta} F (\eta, \lambda) =\mathcal{E}_{, u} [u^{\ast} (\lambda) +
   U (\eta), \lambda ; U' (\eta)], \]
\[ \partial_{\eta \nocomma \eta}^2 F (\eta, \lambda) =\mathcal{E}_{, u
   \nocomma u} [u^{\ast} (\lambda) + U (\eta), \lambda ; U' (\eta), U' (\eta)]
   +\mathcal{E}_{, u} [u^{\ast} (\lambda) + U (\eta), \lambda ; U'' (\eta)],
\]
\begin{eqnarray*}
  \partial_{\eta \nocomma \eta \nocomma \eta}^3 F (\eta, \lambda) & = &
  \mathcal{E}_{, u \nocomma u \nocomma u} [u^{\ast} (\lambda) + U (\eta),
  \lambda ; U' (\eta), U' (\eta), U' (\eta)]\\
  &  & \nosymbol \nosymbol \nosymbol + 3\mathcal{E}_{, u \nocomma u}
  [u^{\ast} (\lambda) + U (\eta), \lambda ; U' (\eta), U'' (\eta)]\\
  &  & \nosymbol \nosymbol \nosymbol +\mathcal{E}_{, u} [u^{\ast} (\lambda) +
  U (\eta), \lambda ; U''' (\eta)],
\end{eqnarray*}
\begin{eqnarray}
  \partial_{\eta \nocomma \eta \nocomma \eta \nocomma \eta}^4 F (\eta,
  \lambda) & = & \mathcal{E}_{, u \nocomma u \nocomma u \nocomma u} [u^{\ast}
  (\lambda) + U (\eta), \lambda ; U' (\eta), U' (\eta), U' (\eta), U' (\eta)]
  \nonumber\\
  &  & \nosymbol + 6\mathcal{E}_{, u \nocomma u \nocomma u} [u^{\ast}
  (\lambda) + U (\eta), \lambda ; U' (\eta), U' (\eta), U'' (\eta)]
  \nonumber\\
  &  & \nosymbol + 3\mathcal{E}_{, u \nocomma u} [u^{\ast} (\lambda) + U
  (\eta), \lambda ; U'' (\eta), U'' (\eta)] \nonumber\\
  &  & \nosymbol + 3\mathcal{E}_{, u \nocomma u} [u^{\ast} (\lambda) + U
  (\eta), \lambda ; U' (\eta), U''' (\eta)] \nonumber\\
  &  & \nosymbol +\mathcal{E}_{, u} [u^{\ast} (\lambda) + U (\eta), \lambda ;
  U'''' (\eta)], \nonumber
\end{eqnarray}
soit, en $\eta = 0$, en observant que $\mathcal{E}_{, u} [u^{\ast} (\lambda),
\lambda] = 0$
\[ \partial_{\eta} F (0, \lambda) = 0, \]
\[ \partial_{\eta \nocomma \eta}^2 F (0, \lambda) =\mathcal{E}_2 (\lambda ;
   u_1, u_1), \]
\[ \partial_{\eta \nocomma \eta \nocomma \eta}^3 F (0, \lambda) =\mathcal{E}_3
   (\lambda ; u_1, u_1, u_1) + 3\mathcal{E}_2 (\lambda ; u_1, u_2), \]
\begin{eqnarray}
  \partial_{\eta \nocomma \eta \nocomma \eta \nocomma \eta}^4 F (\eta,
  \lambda) & = & \mathcal{E}_4 (\lambda ; u_1, u_1, u_1, u_1) + 6\mathcal{E}_3
  (\lambda ; u_1, u_1, u_2) + 3\mathcal{E}_2 (\lambda ; u_2, u_2) \nonumber\\
  &  & \nosymbol + 3\mathcal{E}_2 (\lambda ; u_1, u_3) . \nonumber
\end{eqnarray}
On en d{\'e}duit que
\[ \partial_{\eta \nocomma \lambda}^2 F (0, \lambda) = 0, \]
\[ \partial_{\eta \nocomma \eta \nocomma \lambda}^3 F (0, \lambda) =
   \dot{\mathcal{E}}_2 (\lambda ; u_1, u_1), \]
\[ \partial_{\eta \nocomma \lambda \nocomma \lambda}^3 F (0, \lambda) = 0, \]
\[ \partial_{\eta \nocomma \eta \nocomma \eta \nocomma \lambda}^4 F (0,
   \lambda) = \dot{\mathcal{E}}_3 (\lambda ; u_1, u_1, u_1) + 3
   \dot{\mathcal{E}}_2 (\lambda ; u_1, u_2), \text{} \text{} \]
\[ \partial_{\eta \nocomma \eta \nocomma \lambda \nocomma \lambda}^4 F (0,
   \lambda) = \ddot{\mathcal{E}}_2 (\lambda ; u_1, u_1), \]
\[ \partial_{\eta \nocomma \lambda \nocomma \lambda \nocomma \lambda}^4 F (0,
   \lambda) = 0 \]
et finalement
\[ f' (0) = 0, \]
\[ f'' (0) =\mathcal{E}_2 (\lambda_0 ; u_1, u_1), \]
\[ f''' (0) =\mathcal{E}_3 (\lambda_0 ; u_1, u_1, u_1) + 3\mathcal{E}_2
   (\lambda_0 ; u_1, u_2) + 3 \lambda_1  \dot{\mathcal{E}}_2 (\lambda_0 ; u_1,
   u_1), \]
\begin{eqnarray}
  f'''' (0) & = & \mathcal{E}_4 (\lambda_0 ; u_1, u_1, u_1, u_1) +
  6\mathcal{E}_3 (\lambda_0 ; u_1, u_1, u_2) \nonumber\\
  &  & \nosymbol + 3\mathcal{E}_2 (\lambda_0 ; u_2, u_2) + 3\mathcal{E}_2
  (\lambda_0 ; u_1, u_3) \nonumber\\
  &  & \nosymbol + 4 \lambda_1  \dot{\mathcal{E}}_3 (\lambda_0 ; u_1, u_1,
  u_1) + 12 \lambda_1  \dot{\mathcal{E}}_2 (\lambda_0 ; u_1, u_2) \nonumber\\
  &  & \nosymbol + 6 \lambda_1^2  \ddot{\mathcal{E}}_2 (\lambda_0 ; u_1, u_1)
  + 6 \lambda_2  \dot{\mathcal{E}}_2 (\lambda_0 ; u_1, u_1) . \nonumber
\end{eqnarray}
Les relations pr{\'e}c{\'e}dentes se simplifient notamment en tenant compte de
ce que $u_1 \in V$ : $\mathcal{E}_2 (\lambda_0 ; u_1, u_i) = 0$ pour $i = 1,
2, 3$. On trouve ainsi, pour $f'' (0)$ et $f''' (0)$
\begin{equation}
  \label{eq:DL energie derivee 2nde} f'' (0) = 0
\end{equation}
et
\begin{eqnarray}
  f''' (0) & = & \mathcal{E}_3 (\lambda_0 ; u_1, u_1, u_1) + 3 \lambda_1 
  \dot{\mathcal{E}}_2 (\lambda_0 ; u_1, u_1) \nonumber\\
  & = & - 2 \lambda_1  \dot{\mathcal{E}_2} (\lambda_0 ; u_1, \hat{v}) + 3
  \lambda_1  \dot{\mathcal{E}}_2 (\lambda_0 ; u_1, u_1) \nonumber\\
  & = & \lambda_1 F_{i \nocomma j} (\lambda_0) \xi_1^i \xi_1^j,  \label{eq:DL
  energie derivee 3ieme}
\end{eqnarray}
en utilisant l'{\'e}quation de bifurcation~\eqref{eq:bifurcation 1a} dans la
deuxi{\`e}me ligne. En introduisant les d{\'e}compositions
\eqref{eq:decomposition u1} et \eqref{eq:decomposition u2} de $u_1$ et $u_2$,
on trouve tout d'abord, pour $\mathcal{E}_3 (\lambda_0 ; u_1, u_1, u_2)$
\begin{eqnarray*}
  \mathcal{E}_3 (\lambda_0 ; u_1, u_1, u_2) & = & \mathcal{E}_3 (\lambda_0 ;
  v_i, v_j, v_k) \xi_1^i \xi_1^j \xi_2^k +\mathcal{E}_3 (\lambda_0 ; v_i, v_j,
  w_{k \nocomma l}) \xi_1^i \xi_1^j \xi_1^k \xi_1^l\\
  &  & \nosymbol + \lambda_1 \mathcal{E}_3 (\lambda_0 ; v_i, v_j, w_k)
  \xi_1^i \xi_1^j \xi_1^k\\
  & = & \mathcal{E}_3 (\lambda_0 ; v_i, v_j, v_k) \xi_1^i \xi_1^j \xi_2^k
  +\mathcal{E}_3 (\lambda_0 ; v_i, v_j, w_{k \nocomma l}) \xi_1^i \xi_1^j
  \xi_1^k \xi_1^l\\
  &  & \nosymbol - \lambda_1 \mathcal{E}_2 (\lambda_0 ; w_{i \nocomma j},
  w_k) \xi_1^i \xi_1^j \xi_1^k,
\end{eqnarray*}
en tenant compte de la d{\'e}finition~\eqref{eq:pbvar wij} des $w_{i \nocomma
j}$. Dans le dernier terme de l'expression pr{\'e}c{\'e}dente, les indices
$i$, $j$ et $k$ sont muets, donc
\begin{eqnarray*}
  \mathcal{E}_3 (\lambda_0 ; u_1, u_1, u_2) & = & \mathcal{E}_3 (\lambda_0 ;
  v_i, v_j, v_k) \xi_1^i \xi_1^j \xi_2^k +\mathcal{E}_3 (\lambda_0 ; v_i, v_j,
  w_{k \nocomma l}) \xi_1^i \xi_1^j \xi_1^k \xi_1^l\\
  &  & \nosymbol - \lambda_1 \mathcal{E}_2 (\lambda_0 ; w_{i \nocomma}, w_{j
  \nocomma k}) \xi_1^i \xi_1^j \xi_1^k\\
  & = & \mathcal{E}_3 (\lambda_0 ; v_i, v_j, v_k) \xi_1^i \xi_1^j \xi_2^k
  +\mathcal{E}_3 (\lambda_0 ; v_i, v_j, w_{k \nocomma l}) \xi_1^i \xi_1^j
  \xi_1^k \xi_1^l\\
  &  & \nosymbol + 2 \lambda_1  \dot{\mathcal{E}}_2 (\lambda_0 ; v_{i
  \nocomma}, w_{j \nocomma k}) \xi_1^i \xi_1^j \xi_1^k,
\end{eqnarray*}
en introduisant cette fois-ci la d{\'e}finition~\eqref{eq:pbvar wi} de $w_i .$
On proc{\`e}de de m{\^e}me pour le terme suivant, soit $\mathcal{E}_2
(\lambda_0 ; u_2, u_2)$
\begin{eqnarray*}
  \mathcal{E}_2 (\lambda_0 ; u_2, u_2) & = & \mathcal{E}_2 (\lambda_0 ;
  \xi_1^i v_i + \xi_1^i \xi_1^j w_{i \nocomma j} + \lambda_1 \xi_1^i w_i,
  \xi_1^i v_i + \xi_1^k \xi_1^l w_{k \nocomma l} + \lambda_1 \xi_1^k w_k)\\
  & = & \mathcal{E}_2 (\lambda_0 ; \xi_1^i \xi_1^j w_{i \nocomma j} +
  \lambda_1 \xi_1^i w_i, \xi_1^k \xi_1^l w_{k \nocomma l} + \lambda_1 \xi_1^k
  w_k)\\
  & = & \mathcal{E}_2 (\lambda_0 ; w_{i \nocomma j}, w_{k \nocomma l})
  \xi_1^i \xi_1^j \xi_1^k \xi_1^l + 2 \lambda_1 \mathcal{E}_2 (\lambda_0 ;
  w_{i \nocomma j}, w_k) \xi_1^i \xi_1^j \xi_1^k\\
  &  & \nosymbol + \lambda_1^2 \mathcal{E}_2 (\lambda_0 ; w_i, w_j) \xi_1^i
  \xi_1^j\\
  & = & \mathcal{E}_2 (\lambda_0 ; w_{i \nocomma j}, w_{k \nocomma l})
  \xi_1^i \xi_1^j \xi_1^k \xi_1^l + 2 \lambda_1 \mathcal{E}_2 (\lambda_0 ;
  w_i, w_{j \nocomma k}) \xi_1^i \xi_1^j \xi_1^k\\
  &  & \nosymbol + \tfrac{1}{2} \lambda_1^2  [\mathcal{E}_2 (\lambda_0 ; w_i,
  w_j) +\mathcal{E}_2 (\lambda_0 ; w_j, w_i)] \xi_1^i \xi_1^j\\
  & = & \mathcal{E}_2 (\lambda_0 ; w_{i \nocomma j}, w_{k \nocomma l})
  \xi_1^i \xi_1^j \xi_1^k \xi_1^l - 4 \lambda_1  \dot{\mathcal{E}}_2
  (\lambda_0 ; v_i, w_{j \nocomma k}) \xi_1^i \xi_1^j \xi_1^k\\
  &  & \nosymbol - \lambda_1^2  [\dot{\mathcal{E}}_2 (\lambda_0 ; v_i, w_j) +
  \dot{\mathcal{E}}_2 (\lambda_0 ; v_j, w_i)] \xi_1^i \xi_1^j\\
  & = & \mathcal{E}_3 (\lambda_0 ; v_i, v_j, w_{k \nocomma l}) \xi_1^i
  \xi_1^j \xi_1^k \xi_1^l - 4 \lambda_1  \dot{\mathcal{E}}_2 (\lambda_0 ; v_i,
  w_{j \nocomma k}) \xi_1^i \xi_1^j \xi_1^k\\
  &  & \nosymbol - \lambda_1^2  [\dot{\mathcal{E}}_2 (\lambda_0 ; v_i, w_j) +
  \dot{\mathcal{E}}_2 (\lambda_0 ; v_j, w_i)] \xi_1^i \xi_1^j
\end{eqnarray*}
et enfin
\begin{eqnarray*}
  \dot{\mathcal{E}}_2 (\lambda_0 ; u_1, u_2) & = & \dot{\mathcal{E}}_2
  (\lambda_0 ; v_i, v_j) \xi_1^i \xi_2^j + \dot{\mathcal{E}}_2 (\lambda_0 ;
  v_i, w_{j \nocomma k}) \xi_1^i \xi_1^j \xi_1^k + \lambda_1 
  \dot{\mathcal{E}}_2 (\lambda_0 ; v_i, w_j) \xi_1^i \xi_1^j\\
  & = & \dot{\mathcal{E}}_2 (\lambda_0 ; v_i, v_j) \xi_1^i \xi_2^j +
  \dot{\mathcal{E}}_2 (\lambda_0 ; v_i, w_{j \nocomma k}) \xi_1^i \xi_1^j
  \xi_1^k\\
  &  & \nosymbol + \tfrac{1}{2} \lambda_1  [\dot{\mathcal{E}}_2 (\lambda_0 ;
  v_i, w_j) + \dot{\mathcal{E}}_2 (\lambda_0 ; v_j, w_i)] \xi_1^i \xi_1^j .
\end{eqnarray*}
En rassemblant les r{\'e}sultats pr{\'e}c{\'e}dents, on trouve pour $f''''
(0)$
\begin{eqnarray*}
  f'''' (0) & = & \mathcal{E}_4 (\lambda_0 ; v_i, v_j, v_k {, v_l} ) \xi_1^i
  \xi_1^j \xi_1^k \xi_1^l + 6\mathcal{E}_3 (\lambda_0 ; v_i, v_j, v_k) \xi_1^i
  \xi_1^j \xi_2^k\\
  &  & \nosymbol + 6\mathcal{E}_3 (\lambda_0 ; v_i, v_j, w_{k \nocomma l})
  \xi_1^i \xi_1^j \xi_1^k \xi_1^l + 12 \lambda_1  \dot{\mathcal{E}}_2
  (\lambda_0 ; v_{i \nocomma}, w_{j \nocomma k}) \xi_1^i \xi_1^j \xi_1^k\\
  &  & \nosymbol - 3\mathcal{E}_3 (\lambda_0 ; v_i, v_j, w_{k \nocomma l})
  \xi_1^i \xi_1^j \xi_1^k \xi_1^l - 12 \lambda_1  \dot{\mathcal{E}}_2
  (\lambda_0 ; v_i, w_{j \nocomma k}) \xi_1^i \xi_1^j \xi_1^k\\
  &  & \nosymbol - 3 \lambda_1^2  [\dot{\mathcal{E}}_2 (\lambda_0 ; v_i, w_j)
  + \dot{\mathcal{E}}_2 (\lambda_0 ; v_j, w_i)] \xi_1^i \xi_1^j + 4 \lambda_1 
  \dot{\mathcal{E}}_3 (\lambda_0 ; v_i, v_j, v_k) \xi_1^i \xi_1^j \xi_1^k\\
  &  & \nosymbol + 12 \mathlambda_1  \dot{\mathcal{E}}_2 (\lambda_0 ; v_i,
  v_j) \xi_1^i \xi_2^j + 12 \mathlambda_1  \dot{\mathcal{E}}_2 (\lambda_0 ;
  v_i, w_{j \nocomma k}) \xi_1^i \xi_1^j \xi_1^k\\
  &  & \nosymbol + 6 \mathlambda_1^2  [\dot{\mathcal{E}}_2 (\lambda_0 ; v_i,
  w_j) + \dot{\mathcal{E}}_2 (\lambda_0 ; v_j, w_i)] \xi_1^i \xi_1^j + 6
  \lambda_1^2  \ddot{\mathcal{E}}_2 (\lambda_0 ; v_i, v_j) \xi_1^i \xi_1^j\\
  &  & \nosymbol + 6 \lambda_2  \dot{\mathcal{E}}_2 (\lambda_0 ; v_i, v_j)
  \xi_1^i \xi_1^j\\
  & = & \left[ \mathcal{E}_4 (\lambda_0 ; v_i, v_j, v_k {, v_l} ) +
  3\mathcal{E}_3 (\lambda_0 ; v_i, v_j, w_{k \nocomma l}) \right] \xi_1^i
  \xi_1^j \xi_1^k \xi_1^l\\
  &  & \nosymbol + 4 \lambda_1  [\dot{\mathcal{E}}_3 (\lambda_0 ; v_i, v_j,
  v_k) + 3 \dot{\mathcal{E}}_2 (\lambda_0 ; v_i, w_{j \nocomma k})] \xi_1^i
  \xi_1^j \xi_1^k\\
  &  & \nosymbol + \{ 3 \mathlambda_1^2  [\dot{2 \ddot{\mathcal{E}}_2
  (\lambda_0 ; v_i, v_j) + \dot{\mathcal{E}}}_2 (\lambda_0 ; v_i, w_j) +
  \dot{\mathcal{E}}_2 (\lambda_0 ; v_j, w_i)] + 6 \lambda_2 
  \dot{\mathcal{E}}_2 (\lambda_0 ; v_i, v_j) \} \xi_1^i \xi_1^j\\
  &  & \nosymbol + 6\mathcal{E}_3 (\lambda_0 ; v_i, v_j, v_k) \xi_1^i \xi_1^j
  \xi_2^k + 12 \mathlambda_1  \dot{\mathcal{E}}_2 (\lambda_0 ; v_i, v_j)
  \xi_1^i \xi_2^j\\
  & = & E_{i \nocomma j \nocomma k \nocomma l} (\lambda_0) \xi_1^i \xi_1^j
  \xi_1^k \xi_1^l + 4 \mathlambda_1  \dot{E}_{i \nocomma j \nocomma k}
  (\lambda_0) \xi_1^i \xi_1^j \xi_1^k + 6 [\mathlambda_1^2  \dot{F}_{i
  \nocomma j} (\mathlambda_0) + \lambda_2 F_{i \nocomma j} (\lambda_0)]
  \xi_1^i \xi_1^j\\
  &  & \nosymbol + 6 [E_{i \nocomma j \nocomma k} (\lambda_0) \xi_1^k + 2
  \mathlambda_1 F_{i \nocomma j} (\lambda_0)] \xi_1^i \xi_2^j,
\end{eqnarray*}
et on observe que le dernier terme (en $\xi_1^i \xi_2^j$) est nul, du fait de
l'{\'e}quation de bifurcation~\eqref{eq:bifurcation 1c}. On obtient donc
\begin{equation}
  \label{eq:DL energie derivee 4ieme} f'''' (0) = E_{i \nocomma j \nocomma k
  \nocomma l} (\lambda_0) \xi_1^i \xi_1^j \xi_1^k \xi_1^l + 4 \mathlambda_1 
  \dot{E}_{i \nocomma j \nocomma k} (\lambda_0) \xi_1^i \xi_1^j \xi_1^k + 6
  [\mathlambda_1^2  \dot{F}_{i \nocomma j} (\mathlambda_0) + \lambda_2 F_{i
  \nocomma j} (\lambda_0)] \xi_1^i \xi_1^j .
\end{equation}
Le d{\'e}veloppement limit{\'e}~\eqref{eq:DL energie} est alors obtenu en
rassemblant les r{\'e}sultats~\eqref{eq:DL energie derivee 2nde}, \eqref{eq:DL
energie derivee 3ieme} et \eqref{eq:DL energie derivee 4ieme}.

\begin{remark}
  On peut r{\'e}{\'e}crire $f'''' (0)$ en tenant compte de l'{\'e}quation de
  bifurcation~\eqref{eq:bifurcation 2b}. En multipliant celle-ci par
  $\xi_1^i$, on trouve en effet
  \begin{eqnarray*}
    E_{i \nocomma j \nocomma k \nocomma l} (\lambda_0) \xi_1^i 
    \hspace{0.17em} \xi_1^j \xi_1^k \xi_1^l & = & - 3 \lambda_2 F_{i \nocomma
    j} (\lambda_0) \xi_1^i \xi_1^j - 3 A_{i \nocomma j} (\mathlambda_0)
    \xi_1^i \xi_2^j - 3 \lambda_1  [\dot{E}_{i \nocomma j \nocomma k}
    (\mathlambda_0) \xi_1^k + \lambda_1  \dot{F}_{i \nocomma j} (\lambda_0)]
    \xi_1^i \xi_1^j\\
    & = & - 3 \lambda_1  \dot{E}_{i \nocomma j \nocomma k} (\mathlambda_0)
    \xi_1^i \xi_1^j \xi_1^k - 3 [\lambda_1^2  \dot{F}_{i \nocomma j}
    (\lambda_0) + \lambda_2 F_{i \nocomma j} (\lambda_0)] \xi_1^i \xi_1^j - 3
    A_{i \nocomma j} (\mathlambda_0) \xi_1^i \xi_2^j,
  \end{eqnarray*}
  soit
  \[ f'''' (0) = \mathlambda_1  \dot{E}_{i \nocomma j \nocomma k} (\lambda_0)
     \xi_1^i \xi_1^j \xi_1^k + 3 [\mathlambda_1^2  \dot{F}_{i \nocomma j}
     (\mathlambda_0) + \lambda_2 F_{i \nocomma j} (\lambda_0)] \xi_1^i \xi_1^j
     - 3 A_{i \nocomma j} (\mathlambda_0) \xi_1^i \xi_2^j . \]
\end{remark}

\subsection{D{\'e}veloppement limit{\'e} de la hessienne}\label{sec:DL
hessienne}

On cherche maintenant un d{\'e}veloppement limit{\'e} de la hessienne de
l'{\'e}nergie. Les fonctions test $\hat{u}, \hat{v} \in U$ {\'e}tant
fix{\'e}es, on applique la m{\'e}thode du
{\textsection}\ref{sec20220107121442} {\`a} la fonction $f (\eta) = F [\eta,
\lambda_0 + \Lambda (\eta)]$, avec
\[ F (\eta, \lambda) =\mathcal{E}_{, u \nocomma u} [u^{\ast} (\lambda) + U
   (\eta), \lambda ; \hat{u}, \hat{v}] . \]
On observe tout d'abord que $F (0, \lambda) =\mathcal{E}_2 (\lambda ; \hat{u},
\hat{v})$, soit, en d{\'e}rivant par rapport {\`a} $\lambda$
\[ \partial_{\lambda} F (0, \lambda) = \dot{\mathcal{E}_2} (\lambda ; \hat{u},
   \hat{v}) \quad \text{et} \quad \partial_{\lambda \nocomma \lambda}^2 F (0,
   \lambda) = \ddot{\mathcal{E}_2} (\lambda ; \hat{u}, \hat{v}) . \]
On trouve de m{\^e}me successivement
\[ \partial_{\eta} F (\eta, \lambda) =\mathcal{E}_{, u \nocomma u \nocomma u}
   [u^{\ast} (\lambda) + U (\eta), \lambda ; U' (\eta), \hat{u}, \hat{v}], \]
\begin{eqnarray}
  \partial_{\eta \nocomma \eta}^2 F (\eta, \lambda) & = & \mathcal{E}_{, u
  \nocomma u \nocomma u \nocomma u} [u^{\ast} (\lambda) + U (\eta), \lambda ;
  U' (\eta), U' (\eta), \hat{u}, \hat{v}] \nonumber\\
  &  & \nosymbol +\mathcal{E}_{, u \nocomma u \nocomma u} [u^{\ast} (\lambda)
  + U (\eta), \lambda ; U'' (\eta), \hat{u}, \hat{v}], \nonumber
\end{eqnarray}
soit, en $\eta = 0$
\[ \partial_{\eta} F (0, \lambda) =\mathcal{E}_3 (\lambda ; u_1, \hat{u},
   \hat{v}) \text{} \]
et
\[ \partial_{\eta \nocomma \eta}^2 F (0, \lambda) =\mathcal{E}_4 (\lambda ;
   u_1, u_1, \hat{u}, \hat{v}) +\mathcal{E}_3 (\lambda ; u_2, \hat{u},
   \hat{v}), \]
et en d{\'e}rivant cette fois par rapport {\`a} $\lambda$
\[ \partial_{\eta \nocomma \lambda}^2 F (0, \lambda) = \dot{\mathcal{E}_3}
   (\lambda ; u_1, \hat{u}, \hat{v}) . \]
En ins{\'e}rant les r{\'e}sultats pr{\'e}c{\'e}dents dans les
expressions~\eqref{eq20220107060454} et \eqref{eq20220107124311}, on trouve
\[ f' (0) =\mathcal{E}_3 (\lambda_0 ; u_1, \hat{u}, \hat{v}) + \lambda_1
   \dot{\mathcal{E}_2} (\lambda_0 ; \hat{u}, \hat{v}), \]
\begin{eqnarray}
  f'' (0) & = & \mathcal{E}_4 (\lambda_0 ; u_1, u_1, \hat{u}, \hat{v})
  +\mathcal{E}_3 (\lambda_0 ; u_2, \hat{u}, \hat{v}) + \lambda_2 
  \dot{\mathcal{E}_2} (\lambda_0 ; \hat{u}, \hat{v}) \nonumber\\
  &  & \nosymbol \nosymbol + 2 \lambda_1  \dot{\mathcal{E}_3} (\lambda_0 ;
  u_1, \hat{u}, \hat{v}) + \lambda_1^2  \ddot{\mathcal{E}_2} (\lambda_0 ;
  \hat{u}, \hat{v}) . \nonumber
\end{eqnarray}
qui conduisent finalement au d{\'e}veloppement limit{\'e}~\eqref{eq:DL
hessienne}.

\subsection{D{\'e}veloppement limit{\'e} des valeurs propres et vecteurs
propres de la Hessienne}

On cherche les vecteurs propres $x \in U$ et valeurs propres $\alpha \in
\mathbb{R}$ de la hessienne de l'{\'e}nergie. En d'autre terme, on cherche $x$
et $\alpha$ tels que
\begin{equation}
  \mathcal{E}_{, u \nocomma u} [u (\eta), \lambda (\eta) ; x, \hat{u}] =
  \alpha \langle x, \hat{u} \rangle \quad \text{pour tout} \quad \hat{u} \in
  V.
\end{equation}
On cherche les d{\'e}veloppements limit{\'e}s {\`a} l'ordre 2 en $\eta$ de $x$
et $\alpha$
\begin{eqnarray*}
  x & = & x_0 + \eta x_1 + \tfrac{1}{2} \eta^2 x_2 + o (\eta^2),\\
  \alpha & = & \alpha_0 + \eta \alpha_1 + \tfrac{1}{2} \eta^2 \alpha_2 + o
  (\eta^2) .
\end{eqnarray*}
Ces d{\'e}veloppements limit{\'e}s sont tout d'abord ins{\'e}r{\'e}s dans le
d{\'e}veloppement limit{\'e} \eqref{eq:DL hessienne} de la hessienne de
l'{\'e}nergie
\begin{eqnarray*}
  \mathcal{E}_{, u \nocomma u} [u (\eta), \lambda (\eta) ; x, \hat{u}] & = &
  \mathcal{E}_2 (\lambda_0 ; x_0, \hat{u}) + \eta \mathcal{E}_2 (\lambda_0 ;
  x_1, \hat{u}) + \tfrac{1}{2} \eta^2 \mathcal{E}_2 (\lambda_0 ; x_2,
  \hat{u})\\
  &  & + \eta \mathcal{E}_3 (\lambda_0 ; u_1, x_0, \hat{u}) + \eta^2
  \mathcal{E}_3 (\lambda_0 ; u_1, x_1, \hat{u})\\
  &  & + \eta \lambda_1  \dot{\mathcal{E}_2} (\lambda_0 ; x_0, \hat{u}) +
  \eta^2 \lambda_1  \dot{\mathcal{E}_2} (\lambda_0 ; x_1, \hat{u})\\
  &  & + \tfrac{1}{2} \eta^2  [\mathcal{E}_4 (\lambda_0 ; u_1, u_1, x_0,
  \hat{u}) \nobracket +\mathcal{E}_3 (\lambda_0 ; u_2, x_0, \hat{u})\\
  &  & + \lambda_2  \dot{\mathcal{E}_2} (\lambda_0 ; x_0, \hat{u}) + 2
  \lambda_1  \dot{\mathcal{E}_3} (\lambda_0 ; u_1, x, \hat{u})\\
  &  & + \lambda_1^2  \ddot{\mathcal{E}_2} (\lambda_0 ; x, \hat{u})
  \nobracket] + o (\eta^2)\\
  & = & \mathcal{E}_2 (\lambda_0 ; x_0, \hat{u})\\
  &  & + \eta [\mathcal{E}_3 (\lambda_0 ; u_1, x_0, \hat{u}) +\mathcal{E}_2
  (\lambda_0 ; x_1, \hat{u}) + \lambda_1  \dot{\mathcal{E}_2} (\lambda_0 ;
  x_0, \hat{u})]\\
  &  & + \tfrac{1}{2} \eta^2  [\mathcal{E}_2 (\lambda_0 ; x_2, \hat{u}) +
  2\mathcal{E}_3 (\lambda_0 ; u_1, x_1, \hat{u}) + 2 \lambda_1 
  \dot{\mathcal{E}_2} (\lambda_0 ; x_1, \hat{u})]\\
  &  & + \tfrac{1}{2} \eta^2  [\mathcal{E}_4 (\lambda_0 ; u_1, u_1, x_0,
  \hat{u}) +\mathcal{E}_3 (\lambda_0 ; u_2, x_0, \hat{u}) + \lambda_2 
  \dot{\mathcal{E}_2} (\lambda_0 ; x_0, \hat{u})]\\
  &  & + \tfrac{1}{2} \eta^2  [2 \lambda_1  \dot{\mathcal{E}_3} (\lambda_0 ;
  u_1, x, \hat{u}) + \lambda_1^2  \ddot{\mathcal{E}_2} (\lambda_0 ; x,
  \hat{u})] + o (\eta^2)
\end{eqnarray*}
et
\begin{eqnarray*}
  \alpha \langle x, \hat{u} \rangle & = & \alpha_0  \langle x_0, \hat{u}
  \rangle + \eta (\alpha_1 \langle x_0, \hat{u} \rangle + \alpha_0 \langle
  x_1, \hat{u} \rangle)\\
  &  & + \tfrac{1}{2} \eta^2  (\alpha_2 \langle x_0, \hat{u} \rangle + 2
  \alpha_1 \langle x_1, \hat{u} \rangle + \alpha_0 \langle x_2, \hat{u}
  \rangle) + o (\eta^2) .
\end{eqnarray*}
\paragraph{Probl{\`e}me variationnel d'ordre 0}Trouver $x_0 \in U$ et
$\alpha_0 \in \mathbb{R}$ tels que, pour tout $\hat{u} \in U$
\[ \mathcal{E}_2 (\lambda_0 ; x_0, \hat{u}) = \alpha_0  \langle x_0, \hat{u}
   \rangle . \]
On en d{\'e}duit que $x_0$ est le vecteur propre de $\mathcal{E}_2 
(\lambda_0)$ associ{\'e} {\`a} la valeur propre $\alpha_0$. Si $\alpha_0 \neq
0$, $\mathcal{E}_2  (\lambda_0)$ {\'e}tant positive par hypoth{\`e}se, on a
n{\'e}cessairement $\alpha_0 > 0$, et la valeur propre de la hessienne est
positive. On consid{\`e}re donc dans ce qui suit le cas o{\`u} $\alpha_0 = 0$,
c'est-{\`a}-dire que $x_0 \in V$
\[ x_0 = \chi_0^i v_i \]


\paragraph{Probl{\`e}me variationnel d'ordre 1}Trouver $x_1 \in U$ et
$\alpha_1 \in \mathbb{R}$ tels que, pour tout $\hat{u} \in U$
\[ \mathcal{E}_3 (\lambda_0 ; u_1, x_0, \hat{u}) +\mathcal{E}_2 (\lambda_0 ;
   x_1, \hat{u}) + \lambda_1  \dot{\mathcal{E}_2} (\lambda_0 ; x_0, \hat{u}) =
   \alpha_1  \langle x_0, \hat{u} \rangle, \]
soit, en rempla{\c c}ant $u_1$ et $x_0$ par leurs d{\'e}compositions dans la
base $v_i$
\[ \mathcal{E}_3 (\lambda_0 ; v_j, v_k, \hat{u}) \xi_1^k \chi_0^j
   +\mathcal{E}_2 (\lambda_0 ; x_1, \hat{u}) + \lambda_1  \dot{\mathcal{E}_2}
   (\lambda_0 ; v_j, \hat{u}) \chi_0^j = \alpha_1 \chi_0^j  \langle v_j,
   \hat{u} \rangle . \]
En prenant tout d'abord $\hat{u} = v_i$, on obtient
\[ [\mathcal{E}_3 (\lambda_0 ; v_i, v_j, v_k) \xi_1^k + \lambda_1 
   \dot{\mathcal{E}_2} (\lambda_0 ; v_i, v_j)] \chi_0^j = \alpha_1 \chi_0^i,
\]
soit encore
\[ [E_{i \nocomma j \nocomma k} (\lambda_0) \xi_1^k + \lambda_1 F_{i \nocomma
   j} (\lambda_0)] \chi_0^j = \alpha_1 \chi_0^i . \]
Ainsi, le vecteur $\chi_0^i$ appara{\^i}t comme le vecteur propre de la
matrice sym{\'e}trique $[E_{i \nocomma j \nocomma k} (\lambda_0) \xi_1^k +
\lambda_1 F_{i \nocomma j} (\lambda_0)]$ associ{\'e} {\`a} la valeur propre
$\alpha_1$. On doit alors discuter en fonction du type de bifurcation.

\paragraph{Cas d'une bifurcation asym{\'e}trique}Dans ce cas, la forme
trilin{\'e}aire $E_{i \nocomma j \nocomma k} (\lambda_0)$ n'est pas nulle sur
$V$, et $\alpha_1 \neq 0$. Le terme dominant de $\alpha$ est donc d'ordre 1,
tandis que le terme dominant de $x$ est d'ordre 0.

\paragraph{Cas d'une bifurcation sym{\'e}trique}La forme trilin{\'e}aire $E_{i
\nocomma j \nocomma k} (\lambda_0)$ est identiquement nulle sur $V$ ; de plus,
$\lambda_1 = 0$. On trouve alors que $\alpha_1 = 0$, et on ne peut
d{\'e}terminer les $\chi_0^i$. On prend maintenant $\hat{u} = \hat{w} \in W$
dans le probl{\`e}me variationnel d'ordre 1, et on pose $x_1 = \chi_1^i v_i +
y_1$, avec $y_1 \in W$. On obtient alors le probl{\`e}me variationnel suivant
: trouver $y_1 \in W$ tel que, pour tout $\hat{w} \in W$,
\[ \mathcal{E}_3 (\lambda_0 ; v_j, v_k, \hat{w}) \xi_1^k \chi_0^j
   +\mathcal{E}_2 (\lambda_0 ; y_1, \hat{w}) + \lambda_1  \dot{\mathcal{E}_2}
   (\lambda_0 ; v_j, \hat{w}) \chi_0^j = 0. \]
La solution de ce probl{\`e}me est exprim{\'e}e {\`a} l'aide des $w_{i
\nocomma j}$ et $w_i$ d{\'e}finis respectivement par les probl{\`e}mes
variationnels auxiliaires \eqref{eq:pbvar wij} et \eqref{eq:pbvar wi}
\[ y_1 = \xi_1^i \chi_0^j w_{i \nocomma j} + \lambda_1 \chi_0^i w_i, \]
soit
\[ x_1 = \chi_1^i v_i + \xi_1^i \chi_0^j w_{i \nocomma j} + \lambda_1 \chi_0^i
   w_i . \]
Dans le cas d'une bifurcation sym{\'e}trique, le probl{\`e}me aux valeurs
propres d'ordre 2 s'{\'e}crit quant {\`a} lui
\[ \mathcal{E}_2 (\lambda_0 ; x_2, \hat{u}) + 2\mathcal{E}_3 (\lambda_0 ; u_1,
   x_1, \hat{u}) +\mathcal{E}_4  (\lambda_0 ; u_1, u_1, x_0, \hat{u})
   +\mathcal{E}_3 (\lambda_0 ; u_2, x_0, \hat{u}) + \lambda_2 
   \dot{\mathcal{E}_2} (\lambda_0 ; x_0, \hat{u}) = \alpha_2  \langle x_0,
   \hat{u} \rangle \]
soit, en prenant $\hat{u} = \widehat{v_i} \in V$ et en introduisant les
d{\'e}veloppements de $u_1$, $u_2$, $x_0 $ et $x_1$
\[ \mathcal{E}_4  (\lambda_0 ; v_i, v_j, v_k, v_l) \chi_0^j \xi_{1 }^k \xi_1^l
   + 2\mathcal{E}_3 (\lambda_0 ; u_1, x_1, v_i) +\mathcal{E}_3 (\lambda_0 ;
   u_2, x_0, \hat{u}) + \lambda_2  \dot{\mathcal{E}_2} (\lambda_0 ; x_0,
   \hat{u}) \]
\section{Simplification des {\'e}quations de
bifurcation}\label{sec:Simplification des �quations de bifurcation}

Dans ce paragraphe, on simplifie les {\'e}quations de bifurcation
\eqref{eq:bifurcation 1b} et \eqref{eq:bifurcation 2a} pour obtenir les formes
\eqref{eq:bifurcation 1c} et \eqref{eq:bifurcation 2b}. On commence par
sym{\'e}triser les termes cubique, quadratique et lin{\'e}aire en $\xi_1^i$ de
l'{\'e}quation \eqref{eq:bifurcation 2b}.

\paragraph{Terme cubique en $\xi_1^i$}On observe que
\[ \mathcal{E}_3 (\lambda_0 ; v_i, v_j, w_{k \nocomma l}) \xi_1^j \xi_1^k
   \xi_1^l = \text{} \tfrac{1}{3}  [\mathcal{E}_3 (\lambda_0 ; v_i, v_j, w_{k
   \nocomma l}) \nobracket +\mathcal{E}_3 (\lambda_0 ; v_i, v_k, w_{j \nocomma
   l}) +\mathcal{E}_3 (\lambda_0 ; v_i, v_l, w_{j \nocomma k}] \xi_1^j \xi_1^k
   \xi_1^l . \]
On obtient donc l'expression suivante du terme cubique en $\xi_1^i$ dans
l'{\'e}quation de bifurcation \eqref{eq:bifurcation 2a}
\[ \mathcal{E}_4 (\lambda_0 ; v_i, v_j, v_k, v_l) +\mathcal{E}_3 (\lambda_0 ;
   v_i, v_j, w_{k \nocomma l}) +\mathcal{E}_3 (\lambda_0 ; v_i, v_k, w_{j
   \nocomma l}) +\mathcal{E}_3 (\lambda_0 ; v_i, v_l, w_{j \nocomma k}), \]
qui sugg{\`e}re d'introduire le \ $E_{i \nocomma j \nocomma k \nocomma l}
(\lambda)$ d{\'e}fini par l'{\'e}quation \eqref{eq:def Eijkl}. Le terme
cubique en $\xi_1^i$ dans l'{\'e}quation de bifurcation \eqref{eq:bifurcation
2a} est alors simplement~: $E_{i \nocomma j \nocomma k \nocomma l}
(\lambda_0)$.

\paragraph{Terme quadratique en $\xi_1^i$}On observe de m{\^e}me que
\[ \mathcal{E}_3 (\lambda_0 ; v_i, v_j, w_k) \xi_1^j \xi_1^k = \tfrac{1}{2} 
   [\mathcal{E}_3 (\lambda_0 ; v_i, v_j, w_k) +\mathcal{E}_3 (\lambda_0 ; v_i,
   w_j, v_k)] \xi_1^j \xi_1^k . \]
En prenant tout d'abord $\widehat{w} = w_k$ dans le probl{\`e}me variationnel
\eqref{eq:pbvar wij}, on trouve
\[ \mathcal{E}_3 (\lambda_0 ; v_i, v_j, w_k) = -\mathcal{E}_2 (\lambda_0 ;
   w_{i \nocomma j}, w_k), \]
puis, en prenant cette fois $\hat{w} = w_{i \nocomma j}$ dans le probl{\`e}me
variationnel \eqref{eq:pbvar wi}
\[ \mathcal{E}_2 (\lambda_0 ; w_k, w_{i \nocomma j}) = - 2 \dot{\mathcal{E}_2}
   (\lambda_0 ; v_k, w_{i \nocomma j}), \]
soit finalement
\[ \mathcal{E}_3 (\lambda_0 ; v_i, v_j, w_k) \xi_1^j \xi_1^k =
   [\dot{\mathcal{E}}_2 (\lambda_0 ; v_j, w_{i \nocomma k}) +
   \dot{\mathcal{E}}_2 (\lambda_0 ; v_k, w_{i \nocomma j})] \xi_1^j \xi_1^k .
\]
On obtient donc l'expression suivante du terme quadratique en $\xi_1^i$ dans
l'{\'e}quation de bifurcation \eqref{eq:bifurcation 2a}
\[ 3 \lambda_1  [\dot{\mathcal{E}}_3 (\lambda_0 ; v_i, v_j, v_k) +
   \dot{\mathcal{E}_2} (\lambda_0 ; v_i, w_{j \nocomma k}) +
   \dot{\mathcal{E}}_2 (\lambda_0 ; v_j, w_{i \nocomma k}) +
   \dot{\mathcal{E}}_2 (\lambda_0 ; v_k, w_{i \nocomma j})], \]
qui sugg{\`e}re d'introduire le tenseur $E_{i \nocomma j \nocomma k}
(\lambda)$ d{\'e}fini par l'{\'e}quation \eqref{eq:def Eijk}. Le terme
quadratique en $\xi_1^i$ dans l'{\'e}quation de bifurcation
\eqref{eq:bifurcation 2a} est alors simplement~: $3 \lambda_1  \dot{E}_{i
\nocomma j \nocomma k} (\lambda_0)$.

\paragraph{Terme lin{\'e}aire en $\xi_1^i$}Par des arguments similaires, on
{\'e}tablit {\'e}galement que
\[ \dot{\mathcal{E}_2} (\lambda_0 ; v_i, w_j) = - \tfrac{1}{2} \mathcal{E}_2
   (\lambda_0 ; w_i, w_j) = - \tfrac{1}{2} \mathcal{E}_2 (\lambda_0 ; w_j,
   w_i) = \dot{\mathcal{E}_2} (\lambda_0 ; v_j, w_i) . \]
On obtient donc l'expression suivante du terme lin{\'e}aire en $\xi_1^i$ dans
l'{\'e}quation de bifurcation \eqref{eq:bifurcation 2a}
\[ \ddot{\mathcal{E}}_2 (\lambda_0 ; v_i, v_j) + \tfrac{1}{2} 
   [\dot{\mathcal{E}_2} (\lambda_0 ; v_i, w_j) + \dot{\mathcal{E}_2}
   (\lambda_0 ; v_j, w_i)], \]
qui sugg{\`e}re d'introduire le tenseur $F_{i \nocomma j} (\lambda)$
d{\'e}fini par l'{\'e}quation \eqref{eq:def Fij}. Le terme lin{\'e}aire en
$\xi_1^i$ dans l'{\'e}quation de bifurcation \eqref{eq:bifurcation 2a} est
alors simplement~: $3 \lambda_1^2  \dot{F}_{i \nocomma j} (\lambda_0)$.

\paragraph{Synth{\`e}se~: simplification des {\'e}quations
\eqref{eq:bifurcation 1a} et \eqref{eq:bifurcation 2a}}En rassemblant les
r{\'e}sultats pr{\'e}c{\'e}dents, on obtient tout d'abord pour l'{\'e}quation
\eqref{eq:bifurcation 2a}
\[ 3 [E_{i \nocomma j \nocomma k} (\lambda_0) + \lambda_1 F_{i \nocomma j}
   (\lambda_0)] \xi_2^j + 3 \lambda_2 F_{i \nocomma j} (\lambda_0) \xi_1^j +
   E_{i \nocomma j \nocomma k \nocomma l} (\lambda_0)  \hspace{0.17em} \xi_1^j
   \xi_1^k \xi_1^l + 3 \lambda_1  \dot{E}_{i \nocomma j \nocomma k}
   (\lambda_0)  \hspace{0.17em} \xi_1^j \xi_1^k + 3 \lambda_1^2  \dot{F}_{i
   \nocomma j} (\lambda_0) \xi_1^j = 0, \]
qui sugg{\`e}re d'introduire le tenseur $A_{i \nocomma j} (\lambda)$
d{\'e}fini par l'{\'e}quation \eqref{eq:def Aij}. On obtient alors finalement
l'{\'e}quation de bifurcation \eqref{eq:bifurcation 2b}. Les tenseurs $F_{i
\nocomma j}$ et $E_{i \nocomma j \nocomma k}$ ainsi introduits permettent
{\'e}galement de r{\'e}{\'e}crire l'{\'e}quation de bifurcation
\eqref{eq:bifurcation 1b} sous la forme compacte \eqref{eq:bifurcation 1c}.

\end{document}
