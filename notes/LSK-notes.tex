\newcommand{\mytitle}{Notes relatives à la méthode asymptotique de Lyapunov–Schmidt–Koiter}
\newcommand{\myauthor}{Sébastien Brisard}
\newcommand{\myaddress}{Navier, Ecole des Ponts, Univ Gustave Eiffel, IFSTTAR, CNRS, Marne-la-Vall\'ee, France}
\newcommand{\mysubject}{Note bibliographique}

\documentclass[12pt, final]{amsart}

\usepackage{polyglossia}
\setdefaultlanguage{french}

\usepackage{amsfonts}
\usepackage{amsmath}
\usepackage{amssymb}

\usepackage{amsthm}
\newtheorem{remark}{Remarque}
\newtheorem{theorem}{Théorème}

\usepackage[backend=biber,bibencoding=utf8,doi=false,giveninits=true,isbn=false,maxnames=10,minnames=5,sortcites=true,style=authoryear,texencoding=utf8,url=false]{biblatex}
\addbibresource{stab.bib}

\usepackage[breaklinks=true, colorlinks=true, pdftitle={\mytitle}, pdfauthor={\myauthor}, pdfsubject={\mysubject}, urlcolor=blue]{hyperref}

\usepackage[color={1 1 0}]{pdfcomment}

\usepackage[notref, notcite]{showkeys}

\usepackage{unicode-math}
\setmainfont{XITS}
\setmathfont{XITS Math}

\newcommand{\D}{\mathrm{d}}
\DeclareMathOperator{\vect}{vect}

\AtBeginDocument{
  % See http://tex.stackexchange.com/questions/117990/
  \newcommand{\tens}[1]{\symbfsf{#1}}
  \renewcommand{\vec}[1]{\symbf{#1}}
  \let\div\undefined
  \DeclareMathOperator{\div}{div}
  \DeclareMathOperator{\vdiv}{\symbf{div}}
  \DeclareMathOperator{\grad}{\symbf{grad}}
  \DeclareMathOperator{\tgrad}{\symbfsf{grad}}
  \DeclareMathOperator{\sym}{\symbfsf{sym}}
  \DeclareMathOperator{\tr}{tr}
}


\begin{document}
\title{\mytitle}
\author{\myauthor}
\address{\myaddress}
\email{sebastien.brisard@univ-eiffel}

\begin{abstract}
  blabla
\end{abstract}


\maketitle


\section{Notations}

L'espace des champs cinématiquement admissibles est noté \(U\). On suppose
qu'il a la structure d'espace vectoriel. L'énergie du système est noté
\(ℰ\).

\section{Analyse de la branche fondamentale}

Soit \(u₀∈ U\) un point d'équilibre du système~: l'énergie \(ℰ\)
est stationnaire en \(u₀\). On s'intéresse à la stabilité de cet
équilibre. L'équilibre est stable si \(ℰ\) est minimale en ce
point. On sait que \(ℰ_{,uu}(u₀)≥0\) est une condition
\emph{nécessaire} de stabilité. De plus, \(ℰ_{,uu}(u₀)>0\) est une
condition \emph{suffisante} de stabilité.

On s'intéresse ici au cas où \(ℰ_{,uu}(u₀)\) est positive, sans être
définie positive~; soit \(V\) son noyau, qui forme un sous-espace vectoriel de
\(U\). On suppose que \(V\) est de dimension finie \(m=\dim V\). Soit
\(\bigl(v₁, \ldots, vₘ\bigr)\) une base orthonormée de ce noyau pour le
produit scalaire \(〈•,•〉\) (qui n'est pas précisé pour
le moment). On introduit le sous-espace supplémentaire orthogonal \(W\) de
\(V\) dans \(U\)
\begin{equation}
  U=V\stackrel⟂⊕W.
\end{equation}

Pour étudier la stabilité de l'équilibre, on calcule l'énergie dans un état
\(u₀+ξ v+η w\) voisin du point d'équilibre \(u₀\), avec
\(ξ, η∈ℝ\) \guillemotleft{}petits\guillemotright{}, \(v∈ V\) and
\(w∈ W\). On obtient alors, à l'ordre 4 en \(ξ\) et \(η\)
\begin{equation}
  \begin{aligned}[b]
    Δℰ ={}&
    ℰ(u₀+ξ v+η w)-ℰ(u₀)\\
    ={}&\tfrac12ℰ_{,uu}(u₀;ξ v+η w, ξ v+η w)
    +\tfrac16ℰ_{,uuu}(u₀;ξ v+η w, ξ v+η w, ξ v+η w)\\
    &+\tfrac1{24}ℰ_{,uuuu}(u₀;ξ v+η w, ξ v+η w, ξ v+η w,
    ξ v+η w)+𝒪\bigl(\bigl(ξ²+η²\bigr)²\bigr),
  \end{aligned}
\end{equation}
où le terme linéaire a été omis puisque \(u\) est un point critique de
l'énergie. En tenant compte de la multilinéarité et de la symétrie des
différentielles successives de l'énergie \(ℰ\), ainsi que du fait que
\(ℰ_{,uu}(v, •)=0\), l'expression précédente s'écrit
\begin{equation}
  \begin{aligned}[b]
    Δℰ
    ={}&\tfrac12η²ℰ_{,uu}(w, w)
    +\tfrac16ξ³ℰ_{,uuu}(v, v, v)
    +\tfrac12ξ²ηℰ_{,uuu}(v, v, w)\\
    &+\tfrac12ξη²ℰ_{,uuu}(v, w, w)
    +\tfrac16η³ℰ_{,uuu}(w, w, w)\\
    &+\tfrac1{24}ξ⁴ℰ_{,uuuu}(v, v, v, v)
    +\tfrac16ξ³ℰ_{,uuuu}(v, v, v, w)\\
    &+\tfrac14ξ²η²ℰ_{,uuuu}(v, v, w, w)
    +\tfrac16ξη³ℰ_{,uuuu}(v, w, w, w)\\
    &+\tfrac1{24}η⁴ℰ_{,uuuu}(w, w, w, w)
    +𝒪\bigl(\bigl(ξ²+η²\bigr)²\bigr),
  \end{aligned}
\end{equation}
où l'on convient que toutes les différentielles de \(ℰ\) sont évaluées
au point d'équilibre \(u₀\).

Pour que l'équilibre soit stable, il faut que expression soit positive ou nulle
pour tous \(ξ\) et \(η\) suffisamment petits. En prenant tout d'abord
\(η=0\), on obtient les conditions nécessaires
\begin{equation}
  \label{eq:20211108164416}
  ℰ_{,uuu}(v, v, v)=0
  \quad\text{et}\quad
  ℰ_{,uuuu}(v, v, v, v)≥0
  \quad\text{pour tout}\quad v∈ V.
\end{equation}

En d'autres termes, s'il existe \(v∈ V\) tel que
\(ℰ_{,uuu}(v, v, v)≠0\) ou
\(ℰ_{,uuuu}(v, v, v, v)<0\), alors l'équilibre est
\emph{instable}. Les conditions précédentes ne sont pas suffisantes pour
assurer la stabilité. En effet, supposant ces conditions remplies, on prend
maintenant \(η=ξ²\)
\begin{equation}
  Δℰ
  =\tfrac12ξ⁴\bigl[ℰ_{,uu}(w, w)+ℰ_{,uuu}(v, v, w)
  +\tfrac1{12}ℰ_{,uuuu}(v, v, v, v)\bigr]+o(ξ⁴)
\end{equation}
et on obtient la condition nécessaire supplémentaire
\begin{equation}
  \label{eq:20211109145356}
  ℰ_{,uu}(v, v)+ℰ_{,uuu}(v, v, w)
  +\tfrac1{12}ℰ_{,uuuu}(v, v, v, v)≥0,
\end{equation}
pour tous \(v∈ V\) et \(w∈ W\). Pour \(v∈\kerℋ\) fixé,
l'expression précédente est minimale lorsque \(w\) satisfait le problème
variationnel
\begin{equation}
  \label{eq:20211109145224}
  2ℰ_{,uu}(w, \hat{w})+ℰ_{,uuu}(v, v, \hat{w})=0
  \quad\text{pour tout}\quad
  \hat{w}∈ W.
\end{equation}

Soit \(w_{ij}∈ W\) l'unique solution du problème variationnel suivant
\begin{equation}
  \label{eq:20211221155859}
  2ℰ_{,uu}(w_{ij}, \hat{w})
  +ℰ_{,uuu}(vᵢ, vⱼ, \hat{w})=0
  \quad\text{pour tout}\quad\hat{w}∈ W.
\end{equation}

Alors, pour \(v=ξ^i vᵢ\), la solution du problème
variationnel~\eqref{eq:20211109145224} est \(w=ξ^iξ^jw_{ij}\). Pour cette
valeur de \(v\), la condition~\eqref{eq:20211109145356} s'écrit
\begin{equation}
  \bigl[\tfrac1{12}ℰ_{,uuuu}(vᵢ, vⱼ, vₖ, v_l)
  -ℰ_{,uu}(w_{ij}, w_{kl})\bigr]
  ξ^iξ^jξ^kξ^l≥ 0,
\end{equation}
pour tous \(ξᵢ, ξⱼ, ξₖ, ξ_l∈ℝ\). On peut montrer que
l'inégalité stricte est une condition \emph{suffisante} de stabilité.

\section{Bifurcations}

Soit \(u^\ast(λ)\) la branche fondamentale. Par définition
\begin{equation}
  ℰ_{,u}[u^\ast(λ), λ;\hat{u}]=0
  \quad\text{pour tout}\quad\hat{u}∈ U.
\end{equation}

Il sera commode d'introduire les notations suivantes
\begin{align}
  ℋ(λ)&=ℰ_{,uu}[u^\ast(λ), λ],\\
  𝒯(λ)&=ℰ_{,uuu}[u^\ast(λ), λ],\\
  𝒬(λ)&=ℰ_{,uuuu}[u^\ast(λ), λ].
\end{align}

Noter que \(ℋ\), \(𝒯\) et \(𝒬\) sont des formes bi-,
tri- et quadri-linéaires, respectivement. L'application de ces formes à des
éléments de \(U\) sera notée \(ℋ(λ; u, v)\),
\(𝒯(λ; u, v, w)\), etc\dots. La dérivée de ces formes par
rapport à \(λ\) sera notée à l'aide d'un point supérieur
(\(\dot{ℋ}\), \(\dot{𝒯}\), \dots).

On suppose que l'équilibre est stable pour des valeurs suffisamment petites de
\(λ\). Plus précisément, on suppose que \(ℋ(λ)\) est
définie positive pour tout \(λ<λ₀\). Pour \(λ=λ₀\),
la forme quadratique \(ℋ(λ₀)\) n'est plus que positive. En
notant \(u₀=u^\ast(λ₀)\) la position d'équilibre obtenue pour la valeur
critique \(λ₀\) du paramètre de chargement \(λ\), on s'intéresse à
toutes les courbes d'équilibre qui passent par le point \((u₀, λ₀)\).

Noter que dans ce qui suit, on convient que les formes \(ℋ\),
\(𝒯\) et \(𝒬\) sont implicitement évaluées en \(λ₀\)
lorsque \(λ\) n'est pas rappelé : ainsi, on notera
\(ℋ(•, •)\) plutôt que
\(ℋ(λ₀, •, •)\).

On écrit toute courbe d'équilibre passant par le point \((u₀, λ₀)\)
sous la forme paramétrique suivante
\begin{align}
  \label{eq:20211115075817}
  λ&=λ₀+ηλ₁+η²λ₂+η³λ₃+\cdots,\\
  \label{eq:20211115075835}
  u&=u^\ast(λ)+η u₁+η² u₂+η³u₃+\cdots,
\end{align}
où \(η\) est un paramètre, non précisé pour le moment. Noter que, dans la
représentation paramétrique de \(u\), \(u^\ast\) est évalué en \(λ\) et
pas en \(λ₀\).

Les coefficients \(λ ₖ\) et \(uₖ\) des
développements~\eqref{eq:20211115075817} et \eqref{eq:20211115075835} sont
identifiés en écrivant que l'énergie est stationnaire le long de la courbe
d'équilibre, c'est-à-dire que le résidu
\(ℰ_{,u}[u(η), λ(η)]\) est nul. Le développement limité du
résidu est établi au voisinage de \(η=0\) dans
l'annexe~\ref{sec:20211112182000}. En écrivant que tous ses termes s'annulent,
on trouve successivement, pour tout \(\hat{u}∈ U\)
\begin{gather}
  \label{eq:20211112182917}
  ℋ(u₁, \hat{u})=0,\\
  \label{eq:20211112183220}
  𝒯(u₁, u₁, \hat{u})
  +2λ₁\dot{ℋ}(u₁, \hat{u})
  +2ℋ(u₂, \hat{u})=0,\\
  \begin{aligned}[b]
    𝒬(u₁, u₁, u₁, \hat{u})
    +6𝒯(u₁, u₂, \hat{u})+6ℋ(u₃, \hat{u})&\\
    +3λ₁\bigl[\dot{𝒯}(u₁, u₁, \hat{u})
    +2\dot{ℋ}(u₂, \hat{u})\bigr]
    +3λ₁²\ddot{ℋ}(u₁, \hat{u})&\\
    +6λ₂\dot{ℋ}(u₁, \hat{u})&=0.
  \end{aligned}
\end{gather}

On déduit de l'équation~\eqref{eq:20211112182917} que \(u₁∈ V\). On pose
alors \(u₁=ξ₁^i vᵢ\). En prenant \(\hat{u}=vᵢ\),
l'équation~\eqref{eq:20211112183220} s'écrit
\begin{equation}
  𝒯_{ijk}ξ₁^jξ₁^k+2λ₁\dot{ℋ}_{ij}ξ₁^j=0.
\end{equation}

Pour le terme d'ordre 2, on introduit la décomposition~:
\(u₂=ξ₂^ivᵢ+w₂\), où \(w₂∈ W\). On a alors
\(ℋ(u₂, \hat{u})=ℋ(w₂, \hat{u})\) et
l'équation~\eqref{eq:20211112183220} s'écrit
\begin{equation}
  𝒯(u₁, u₁, \hat{u})+2λ₁\dot{ℋ}(u₁, \hat{u})
  +2ℋ(w₂, \hat{u})=0,
\end{equation}
pour tout \(\hat{u}∈ V\). En prenant la fonction test dans l'espace \(W\), on
obtient le problème variationnel suivant~: trouver \(w₂∈ W\) tel
que\pdfmargincomment{Conflit de notations}
\begin{equation}
  \label{eq:20211210131623}
  ξ₁^iξ₁^j𝒯(vᵢ, vⱼ, \hat{w})
  +2λ₁ξ₁^i\dot{ℋ}(vᵢ, \hat{w})
  +2ℋ(w₂, \hat{w})=0,
\end{equation}
pour tout \(\hat{w}∈ W\). Soient \(wᵢ∈ W\) les solutions des problèmes
variationnels suivants
\begin{equation}
  ℋ(wᵢ, \hat{w})+\dot{ℋ}(vᵢ, \hat{w})=0,
\end{equation}
pour tout \(\hat{w}∈ W\). La solution du problème~\eqref{eq:20211210131623}
s'obtient par simple combinaison linéaire des \(wᵢ\) et \(w_{ij}\) introduits
précédemment par le problème variationnel~\eqref{eq:20211221155859}, de sorte
que
\begin{equation}
  w₂=ξ₁^iξ₁^jw_{ij}+λ₁ξ₁^i wᵢ.
\end{equation}

Le développement limité de la hessienne de l'énergie le long de la
branche bifurquée est établi dans l'annexe~\ref{sec:20211115081016}
\begin{equation}
  \label{eq:20211115082025}
  \begin{aligned}[b]
    ℰ_{,uu}[u(η), λ(η)](\hat{u}, \hat{v})
    ={}&ℋ^\ast(\hat{u}, \hat{v})
    +η\bigl[𝒯^\ast(u₁, \hat{u}, \hat{v})
    +λ₁\dot{ℋ}^\ast(\hat{u}, \hat{v})\bigr]\\
    &+\tfrac12η²\bigl[𝒬^\ast(u₁, u₁, \hat{u}, \hat{v})
    +2λ₁\dot{𝒯}^\ast(u₁, \hat{u}, \hat{v})\\
    &+λ₁²\ddot{ℋ}^\ast(\hat{u}, \hat{v})
    +𝒯^\ast(u₂, \hat{u}, \hat{v})
    +λ₂\dot{ℋ}^\ast(\hat{u}, \hat{v})\bigr]+o(η²).
  \end{aligned}
\end{equation}

On peut d'ores et déjà remarquer que, sur la branche fondamentale
(\(u₁=u₂=0\)), en prenant \(η=λ-λ^\ast\) (\(λ₁=1\))
\begin{equation}
  ℋ(\hat{u}, \hat{v})
  =ℰ_{,uu}[u₀(λ), λ](\hat{u}, \hat{v})
  =ℋ^\ast(\hat{u}, \hat{v})
  +\bigl(λ-λ^\ast\bigr)\dot{ℋ}^\ast(\hat{u}, \hat{v})
  +o(λ-λ^\ast).
\end{equation}

Dans ce qui suit, on supposera que \(\dot{ℋ}^\ast≠ 0\). Pour
\(\hat{u}∈\kerℋ^\ast\), l'égalité précédente s'écrit
\begin{equation}
  ℋ(\hat{u}, \hat{u})
  =\bigl(λ-λ^\ast\bigr)\dot{ℋ}^\ast(\hat{u}, \hat{u})
  +o(λ-λ^\ast).
\end{equation}

Comme la branche fondamentale est stable pour \(λ<λ^\ast\), on doit
avoir \(\dot{ℋ}^\ast(\hat{u}, \hat{u})<0\). La forme quadratique
\(\dot{ℋ}^\ast\) est définie négative sur \(\kerℋ^\ast\).


Il faut alors distinguer deux cas. Si la branche trouvée est telle que
\(λ₁≠0\), alors...

Si \(λ₁=0\), alors on doit calculer \(λ₂\)
\begin{equation}
  𝒬^\ast(u₁, u₁, u₁, \hat{u})
  +3𝒯^\ast(u₁, u₂, \hat{u})+ℋ^\ast(u₃, \hat{u})
  +3λ₂\dot{ℋ}^\ast(u₁, \hat{u})=0.
\end{equation}

\appendix

\section{Propriétés des formes bilinéaires symétriques, positives}

Soit \(\mathcal B\) une forme bilinéaire symétrique, positive sur l'espace vectoriel
\(V\). On définit son noyau \(\ker\mathcal B\) de la façon suivante
\begin{equation}
  \ker\mathcal B=\{u∈ V|\mathcal B(u, u)=0\}.
\end{equation}

\begin{theorem}
  Le noyau \(\ker\mathcal B\) d'une forme bilinéaire, symétrique, positive
  \(\mathcal B\) sur l'espace vectoriel \(V\) est un sous-espace vectoriel de
  \(V\).
\end{theorem}
\begin{proof}
  Soient \(u, v∈\ker\mathcal B\), \(\alpha∈ℝ\) et \(w=u+\alpha
  v\). Montrons que \(w∈\ker\mathcal B\). Il suffit d'évaluer
  \(\mathcal B(w, w)\)
  \begin{equation}
    \mathcal B(w, w)=\mathcal B(u+\alpha v, u+\alpha v)=\mathcal B(u, u)+2\alpha\mathcal B(u, v)+\alpha²\mathcal B(v, v),
  \end{equation}
  où l'on a tenu compte de la symétrie de \(\mathcal B\) pour écrire que
  \(\mathcal B(u, v)=\mathcal B(v, u)\). Comme \(u, v∈\ker\mathcal B\), le
  premier et le dernier terme sont nuls, soit
  \(\mathcal B(w, w)=2\alpha\mathcal B(u, v)\). La forme bilinéaire étant
  positive, cette grandeur est positive, \emph{quelle que soit la valeur de
    \(\alpha∈ℝ\)}. On en déduit donc que \(\mathcal B(u, v)=0\), puis
  que \(\mathcal B(w, w)=0\).
\end{proof}

Soit \(u∈ V\). Alors
\begin{equation}
  u∈\ker\mathcal B\quad\text{ssi}\quad\text{pour tout }v∈ V, \mathcal B(u, v)=0.
\end{equation}
\begin{proof}
  Soient \(u∈\ker\mathcal B\), \(v∈ V\) et \(\alpha∈ℝ\). Comme
  précédemment, on écrit que \(\mathcal B(w, w)≥0\), avec \(w=\alpha u+v\)
  \begin{equation}
    \mathcal B(w, w)=2\alpha\mathcal B(u, v)+\mathcal B(v, v)≥0,
  \end{equation}
  où l'on a tenu compte de ce que \(\mathcal B(u, u)=0\). L'expression
  précédente, affine en \(\alpha\), a un signe constant. Le terme linéaire en
  \(\alpha\) est donc nul, soit \(\mathcal B(u, v)=0\).

  Réciproquement, si \(\mathcal B(u, v)=0\) pour tout \(v∈ V\), alors
  \(\mathcal B(u, u)=0\) (en prenant \(v=u\)).
\end{proof}

\section{Développements limités le long d'une branche bifurquée du diagramme
  d'équilibre}

\subsection{Principe du calcul}

On pose dans ce qui suit \(λ(η)=λ(η)-λ₀\) et
\(U(η)=u(η)-u₀[λ(η)]\), de sorte que\pdfmargincomment{À
  compléter}
\begin{align}
  \label{eq:20211112155446}
  λ(η)&=ηλ₁+η²λ₂+η³λ₃+\cdots,\\
  \label{eq:20211112113028}
  U(η)&=η u₁+η² u₂+η³u₃+\cdots.
\end{align}

On considère une quantité \(\mathcal F\), fonction de \(u\) et \(λ\)~:
\(\mathcal F(u, λ)\). Cette fonctionnelle est évaluée le long de la
branche bifurquée. En d'autres termes, on considère
\begin{equation}
  f(η)=F\bigl(u₀[λ₀+λ(η)]+U(η),
  λ₀+λ(η)\bigr).
\end{equation}

On souhaite établir un développement limité de \(f\) au voisinage de
\(η=0\), ce qui conduit à calculer les dérivées successives de \(f\) en
\(η=0\), puisque
\begin{equation}
  f(η)=f(0)+η f'(0)+\tfrac12η²f''(0)+\cdots
\end{equation}

Pour calculer ces dérivées, il sera commode d'introduire la fonction auxiliaire
\(F\)
\begin{equation}
  F(η, λ)=\mathcal F[u₀(λ)+U(η), λ],
\end{equation}
dans laquelle les variables \(λ\) et \(η\) sont provisoirement
considérées comme indépendantes. On a
\begin{equation}
  f(η)=F[η, λ₀+λ(η)],
\end{equation}
d'où l'on déduit successivement que
\begin{equation}
  \label{eq:20211112162417}
  f'(η)=\partial_η F[η, λ₀+λ(η)]
  +λ'(η)\,\partial_λ F[η, λ₀+λ(η)],
\end{equation}
\begin{equation}
  \begin{aligned}[b]
    \label{eq:20211112165810}
    f''(η)={}&\partial_{ηη}²F[η, λ₀+λ(η)]
    +2λ'(η)\,\partial_{ηλ}²F[η, λ₀+λ(η)]\\
    &+λ''(η)\,\partial_λ F[η, λ₀+λ(η)]
    +λ'(η)²\partial_{λλ}²F[η, λ₀+λ(η)]
  \end{aligned}
\end{equation}
et
\begin{equation}
  \label{eq:20211112173223}
  \begin{aligned}[b]
    f'''(η)={}&\partial_{ηηη}³F[η, λ₀+λ(η)]\\
    &+3λ'(η)\,\partial_{ηηλ}³F[η, λ₀+λ(η)]\\
    &+3λ'(η)²\partial_{ηλλ}³F[η, λ₀+λ(η)]\\
    &+3λ''(η)\,\partial_{ηλ}²F[η, λ₀+λ(η)]\\
    &+λ'''(η)\,\partial_λ F[η, λ₀+λ(η)]\\
    &+3λ'(η)\,λ''(η)\,\partial_{λλ}²F[η, λ₀+λ(η)]\\
    &+λ'(η)³\partial_{λλλ}³F[η, λ₀+λ(η)].
  \end{aligned}
\end{equation}

En \(η=0\), les relations précédentes s'écrivent
\begin{align}
  f'(0)={}&\partial_η F+λ₁\partial_λ F\\
  f''(0)={}&\partial_{ηη}²F+2λ₁\partial_{ηλ}²F
             +2λ₂\partial_λ F+λ₁²\partial_{λλ}²F
\end{align}
où \(F\) et ses dérivées sont évaluées en \((0, λ₀)\).

\subsection{Développement limité du résidu}
\label{sec:20211112182000}

Soit \(f(η)=ℰ_{,u}(u, λ)(\hat{u})\) le \emph{résidu}
(\(\hat{u}∈ V\) est arbitraire). Les équations permettant de déterminer les
coefficients des développements~\eqref{eq:20211112155446} et
\eqref{eq:20211112113028} sont obtenues en écrivant que \(f\) et ses dérivées
sont nulles en \(η=0\). En considérant temporairement \(η\) et
\(λ\) comme des variables indépendantes, on introduit
\begin{equation}
  F(η, λ)=ℰ_{,u}[u₀(λ)+U(η), λ](\hat{u})
\end{equation}
et on doit exprimer la nullité de
\(f(η)=F[η, λ^\ast+λ(η)]\) au voisinage de \(η=0\). En
convenant dans ce qui suit que \(U\), \(λ\) et leurs dérivées sont
implicitement évaluées en \(η\), on trouve tout d'abord
avec
\begin{equation}
  \label{eq:20211112164354}
  \partial_η F(η, λ)
  =ℰ_{,uu}[u₀(λ)+U, λ](U', \hat{u}),
\end{equation}
soit, en \(η=0\)
\begin{equation}
  \label{eq:20211112165323}
  \partial_η F(0, λ)=ℋ(u₁, \hat{u}).
\end{equation}

Par ailleurs, puisque \(λ\mapsto u₀(λ)\) est une courbe d'équilibre, on a
\begin{equation}
  \label{eq:20211112164240}
  F(0, λ)=ℰ_{,u}[u₀(λ), λ]=0
  \quad\text{donc}\quad
  \frac{\partial^kF}{\partialλ^k}(0, λ)=0.
\end{equation}

On déduit de ce qui précède que
\begin{equation}
  \label{eq:20211112182300}
  f'(0)=ℋ(u₁, \hat{u}).
\end{equation}

En dérivant la relation~\eqref{eq:20211112162417}, on obtient maintenant l'expression de la dérivée seconde de \(f\)

La relation~\eqref{eq:20211112164240} montre que les deux derniers termes
s'annulent en \(η=0\). Par ailleurs, en dérivant
l'expression~\eqref{eq:20211112164354} par rapport à \(η\)
\begin{equation}
  \label{eq:20211112172446}
  \begin{aligned}[b]
    \partial_{ηη}²F(η, λ)={}&
    ℰ_{,uuu}[u₀(λ)+U, λ](U', U', \hat{u})\\
    &+ℰ_{,uu}[u₀(λ)+U, λ](U'', \hat{u}),
  \end{aligned}
\end{equation}
soit, en \(η=0\)
\begin{equation}
  \label{eq:20211112165830}
  \partial_{ηη}²F(0, λ)=𝒯(u₁, u₁, \hat{u})
  +2ℋ(u₂, \hat{u}).
\end{equation}

Par ailleurs, \(\partial_{ηλ}²F(0, λ)\) s'obtient par simple
dérivation de la relation~\eqref{eq:20211112165323}
\begin{equation}
  \label{eq:20211112165843}
  \partial_{ηλ}²F(0, λ)=\dot{ℋ}(u₁, \hat{u}).
\end{equation}

En rassemblant les expressions~\eqref{eq:20211112165810},
\eqref{eq:20211112165830} et \eqref{eq:20211112165843}, on trouve l'expression
de la dérivée seconde de \(f\) en 0
\begin{equation}
  \label{eq:20211112182333}
  f''(0)=𝒯(u₁, u₁, \hat{u})
  +2λ₁\dot{ℋ}(u₁, \hat{u})
  +2ℋ(u₂, \hat{u}).
\end{equation}

On obtient finalement l'expression de \(f'''(η)\) en dérivant la
relation~\eqref{eq:20211112165810}

La relation~\eqref{eq:20211112164240} montre une nouvelle fois que les trois
derniers termes de l'expression ci-dessus s'annulent en \(η=0\). Par simple
dérivation par rapport à \(λ\) des expressions~\eqref{eq:20211112165830}
et \eqref{eq:20211112165843}, on trouve par ailleurs
\begin{equation}
  \label{eq:20211112173247}
  \partial_{ηηλ}³F(0, λ)
  =\dot{𝒯}(u₁, u₁, \hat{u})+2\dot{ℋ}(u₂, \hat{u})
  \quad\text{et}\quad
  \partial_{ηλλ}³F(0, λ)
  =\ddot{ℋ}(u₁, \hat{u}).
\end{equation}

Finalement, en dérivant l'expression~\eqref{eq:20211112172446} par rapport à
\(η\), on trouve
\begin{equation}
  \begin{aligned}[b]
    \partial_{ηηη}³F(η, λ)={}&
    ℰ_{,uuuu}[u₀(λ)+U, λ](U', U', U', \hat{u})\\
    &+3ℰ_{,uuu}[u₀(λ)+U, λ](U', U'', \hat{u})\\
    &+ℰ_{,uu}[u₀(λ)+U, λ](U''', \hat{u}),
  \end{aligned}
\end{equation}
soit, en \(η=0\)
\begin{equation}
  \label{eq:20211112173300}
  \partial_{ηηη}³F(0, λ)=
  𝒬(u₁, u₁, u₁, \hat{u})
  +6𝒯(u₁, u₂, \hat{u})
  +6ℋ(u₃, \hat{u}).
\end{equation}

En regroupant les relations~\eqref{eq:20211112173223},
\eqref{eq:20211112173247} et \eqref{eq:20211112173300}, on trouve finalement
l'expression de la dérivée troisième de \(f\) en \(η=0\)
\begin{equation}
\label{eq:20211112182402}
  \begin{aligned}[b]
    f'''(0)={}&𝒬(u₁, u₁, u₁, \hat{u})
    +6𝒯(u₁, u₂, \hat{u})+6ℋ(u₃, \hat{u})\\
    &+3λ₁\bigl[\dot{𝒯}(u₁, u₁, \hat{u})
    +2\dot{ℋ}(u₂, \hat{u})\bigr]
    +3λ₁²\ddot{ℋ}(u₁, \hat{u})\\
    &+6λ₂\dot{ℋ}(u₁, \hat{u}).
  \end{aligned}
\end{equation}

\section{Développement limité de la hessienne}
\label{sec:20211115081016}

Le principe du calcul est le même que dans
l'annexe~\ref{sec:20211112182000}. Pour \(\hat{u}, \hat{v}∈ V\), on introduit
\(h(η)=ℰ_{,uu}(u, λ)(\hat{u}, \hat{v})\), où \(λ\) et
\(u\) sont données en fonction de \(η\) par les
expressions~\eqref{eq:20211115075817} et \eqref{eq:20211115075835}.

La fonction \(h(η)\) est la hessienne de l'énergie le long d'une branche du
diagramme d'équilibre. Elle permet d'étudier la stabilité de l'équilibre. On
établit un développement limité de \(h\) en calculant ses dérivées successives
en \(η=0\). \`A cet effet, considérant une nouvelle fois \(η\) et
\(λ\) comme des variables indépendantes, on introduit la fonction
\begin{equation}
  H(η, λ)=ℰ_{,uu}[u₀(λ)+U(η), λ],
\end{equation}
de sorte que
\begin{equation}
  \label{eq:20211114104326}
  h(η)=H[η, λ^\ast+λ(η)].
\end{equation}

On observe tout d'abord que \(h(0)=ℋ^\ast\). Par ailleurs, en dérivant
l'expression~\eqref{eq:20211114104326} par rapport à \(η\)
\begin{equation}
  \label{eq:20211114110210}
  h'(η)=\partial_η H(η, λ^\ast+λ)
  +λ'\partial_λ H(η, λ^\ast+λ),
\end{equation}
avec
\begin{equation}
  \label{eq:20211114110852}
  \partial_η H(η, λ)=ℰ_{,uuu}(u₀+U, λ)(U', \hat{u}, \hat{v}),
\end{equation}
soit, en \(η=0\)
\begin{equation}
  \label{eq:20211114110230}
  \partial_η H(0, λ)=𝒯(u₁, \hat{u}, \hat{v}).
\end{equation}

Par ailleurs
\begin{equation}
  \label{eq:20211114110245}
  H(0, λ)=ℋ(\hat{u}, \hat{v})
  \quad\text{donc}\quad
  \partial_λ H(0, λ)=\dot{ℋ}(\hat{u}, \hat{v}).
\end{equation}

En regroupant les expressions~\eqref{eq:20211114110210},
\eqref{eq:20211114110230} et \eqref{eq:20211114110245}, on trouve la dérivée de
\(h\) en 0
\begin{equation}
  h'(0)=𝒯^\ast(u₁, \hat{u}, \hat{v})
  +λ₁\dot{ℋ}^\ast(\hat{u}, \hat{v}).
\end{equation}

La dérivation de l'expression~\eqref{eq:20211114110210} par rapport à \(η\)
donne maintenant
\begin{equation}
  \label{eq:20211114111822}
  \begin{aligned}[b]
    h''(η)={}&\partial_{ηη}² H(η, λ^\ast+λ)
    +2λ'\partial_{ηλ}² H(η, λ^\ast+λ)\\
    &+λ''\partial_{λ}H(η, λ^\ast+λ)
    +λ'^2\partial_{λλ}²H(η, λ^\ast+λ).
  \end{aligned}
\end{equation}

L'expression de \(\partial_{ηη}²H\) est obtenue en
dérivant~\eqref{eq:20211114110852}
\begin{equation}
  \partial_{ηη}²H(η, λ)
  =ℰ_{,uuuu}(u₀+U, λ)(U', U', \hat{u}, \hat{v})
  +ℰ_{,uuu}(U'', \hat{u}, \hat{v}),
\end{equation}
soit, en \(η=0\)
\begin{equation}
  \label{eq:20211114111716}
  \partial_{ηη}²H(0, λ)=𝒬(u₁, u₁, \hat{u}, \hat{v})
  +2𝒯(u₂, \hat{u}, \hat{v}).
\end{equation}

Ensuite, en dérivant~\eqref{eq:20211114110230} et \eqref{eq:20211114110245} par
rapport à \(λ\), on obtient
\begin{equation}
  \label{eq:20211114111740}
  \partial_{ηλ}²H(0, λ)=\dot{𝒯}(u₁, \hat{u}, \hat{v})
  \quad\text{et}\quad
  \partial_{λλ}²H(0, λ)=\ddot{ℋ}(\hat{u}, \hat{v}).
\end{equation}

Finalement, en substituant \eqref{eq:20211114110245}, \eqref{eq:20211114111716}
et \eqref{eq:20211114111740} dans \eqref{eq:20211114111822}, on obtient
l'expression de la dérivée seconde de h en \(η=0\)
\begin{equation}
  \begin{aligned}[b]
    h''(0)={}&𝒬^\ast(u₁, u₁, \hat{u}, \hat{v})
    +2λ₁\dot{𝒯}^\ast(u₁, \hat{u}, \hat{v})
    +λ₁²\ddot{ℋ}^\ast(\hat{u}, \hat{v})\\
    &+2𝒯^\ast(u₂, \hat{u}, \hat{v})
    +2λ₂\dot{ℋ}^\ast(\hat{u}, \hat{v}).
  \end{aligned}
\end{equation}

\subsection{Développement limité des valeurs propres et vecteurs propres de la Hessienne}

On cherche les vecteurs propres \(v∈ V\) et valeurs propres
\(\alpha∈ℝ\) de la Hessienne
\begin{equation}
  \label{eq:20211115082122}
  ℰ_{,uu}[u(η), λ(η)](v, \hat{u})
  =\alpha〈 v, \hat{u}〉
  \quad\text{pour tout}\quad\hat{u}∈ V.
\end{equation}

On cherche les développements limités à l'ordre 1 en \(η\) de \(v\) et \(\alpha\)
\begin{equation}
  \label{eq:20211115082037}
  v = v₀+η v₁+o(η)
  \quad\text{et}\quad
  \alpha=\alpha₀+η\alpha₁+o(η)
\end{equation}

Les développements limités~\eqref{eq:20211115082025} et
\eqref{eq:20211115082037} sont insérés dans le
problème~\eqref{eq:20211115082122}
\begin{equation}
  \begin{aligned}[b]
    ℰ_{,uu}[u(η), λ(η)](v, \hat{w})={}
    &ℋ^\ast(v₀, \hat{w})
    +η\bigl[𝒯^\ast(u₁, v₀, \hat{w})
    +λ₁\dot{ℋ}^\ast(v₀, \hat{w})\\
    &+ℋ^\ast(v₁, \hat{w})\bigr]+o(η)
  \end{aligned}
\end{equation}

\begin{equation}
  \alpha〈 v, \hat{w}〉=\alpha₀〈 v₀, \hat{w}〉
  +η\bigl(\alpha₁〈 v₀, \hat{w}〉+\alpha₀〈 v₁, \hat{w}〉\bigr)
  +o(η).
\end{equation}

On obtient le problème variationnel d'ordre 0
\begin{equation}
  ℋ^\ast(v₀, \hat{w})=\alpha₀〈 v₀, \hat{w}〉
  \quad\text{pour tout}\quad\hat{w}∈ V,
\end{equation}
qui montre que \(v₀\) est le vecteur propre de \(ℋ^\ast\) associé à
la valeur propre \(\alpha₀\). Si \(\alpha₀≠ 0\), \(ℋ^\ast\) étant
positive par hypothèse, on a nécessairement \(\alpha₀>0\), et la valeur propre
de la hessienne est positive.

On considère maintenant le cas où \(\alpha₀\), c'est-à-dire que
\(v₀∈\kerℋ^\ast\). En prenant \(\hat{w}∈\kerℋ^\ast\), on
obtient alors le problème variationnel d'ordre 1
\begin{equation}
  𝒯^\ast(u₁, v₀, \hat{w})
  +λ₁\dot{ℋ}^\ast(v₀, \hat{w})
  =\alpha₁〈v₀, \hat{w}〉
  \quad\text{pour tout}\quad
  \hat{w}∈\kerℋ^\ast.
\end{equation}

En posant \(u₁=ξᵢaᵢ\) et \(v₀=\chi ⱼ aⱼ\), on obtient l'équation
\begin{equation}
  \bigl(𝒯_{ijk}^\astξₖ+λ₁\dot{ℋ}_{ij}^\ast\bigr)\chi ⱼ
  =\alpha₁\chi ᵢ,
\end{equation}
qui est un problème aux valeurs propres pour la matrice symétrique
\((𝒯_{ijk}^\astξₖ+λ₁\dot{ℋ}_{ij}^\ast)_{1≤i,
  j≤m}\)


% \printbibliography
\end{document}

%%% Local Variables:
%%% coding: utf-8
%%% fill-column: 79
%%% mode: latex
%%% TeX-engine: xetex
%%% TeX-master: t
%%% End:
