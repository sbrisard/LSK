\newcommand{\mytitle}{Notes relatives à la méthode asymptotique de Lyapunov–Schmidt–Koiter}
\newcommand{\myauthor}{Sébastien Brisard}
\newcommand{\myaddress}{Navier, Ecole des Ponts, Univ Gustave Eiffel, IFSTTAR, CNRS, Marne-la-Vall\'ee, France}
\newcommand{\mysubject}{Note bibliographique}

\documentclass[12pt, final]{amsart}

\usepackage{polyglossia}
\setdefaultlanguage{french}

\usepackage{amsfonts}
\usepackage{amsmath}
\usepackage{amssymb}

\usepackage{amsthm}
\newtheorem{remark}{Remarque}
\newtheorem{theorem}{Théorème}

\usepackage[backend=biber,bibencoding=utf8,doi=false,giveninits=true,isbn=false,maxnames=10,minnames=5,sortcites=true,style=authoryear,texencoding=utf8,url=false]{biblatex}
\addbibresource{stab.bib}

\usepackage[breaklinks=true, colorlinks=true, pdftitle={\mytitle}, pdfauthor={\myauthor}, pdfsubject={\mysubject}, urlcolor=blue]{hyperref}

\usepackage[color={1 1 0}]{pdfcomment}

\usepackage{unicode-math}
\setmainfont{XITS}
\setmathfont{XITS Math}

\newcommand{\order}[2]{#1_{[#2]}}

\begin{document}
\title{\mytitle}
\author{\myauthor}
\address{\myaddress}
\email{sebastien.brisard@univ-eiffel}

\begin{abstract}
  blabla
\end{abstract}


\maketitle


\section{Notations}

L'espace des champs cinématiquement admissibles est noté \(U\). On suppose
qu'il a la structure d'espace vectoriel. L'énergie du système est noté \(ℰ\).

\section{Analyse de la branche fondamentale}

Soit \(u₀∈ U\) un point d'équilibre du système~: l'énergie \(ℰ\) est
stationnaire en \(u₀\). On s'intéresse à la stabilité de cet
équilibre. L'équilibre est stable si \(ℰ\) est minimale en ce point. On sait
que \(ℰ_{,uu}(u₀)≥0\) est une condition \emph{nécessaire} de stabilité. De
plus, \(ℰ_{,uu}(u₀)>0\) est une condition \emph{suffisante} de stabilité.

On s'intéresse ici au cas où \(ℰ_{,uu}(u₀)\) est positive, sans être définie
positive~; soit \(V\) son noyau, qui forme un sous-espace vectoriel de
\(U\). On suppose que \(V\) est de dimension finie \(m=\dim V\). Soit
\(\bigl(v₁, \ldots, vₘ\bigr)\) une base orthonormée de ce noyau pour le produit
scalaire \(〈•,•〉\) (qui n'est pas précisé pour le moment). On introduit le
sous-espace supplémentaire orthogonal \(W\) de \(V\) dans \(U\)
\begin{equation}
  U=V\stackrel⟂⊕W.
\end{equation}

Pour étudier la stabilité de l'équilibre, on calcule l'énergie dans un état
\(u₀+ξv+ηw\) voisin du point d'équilibre \(u₀\), avec \(ξ, η∈ℝ\)
\guillemotleft{}petits\guillemotright{}, \(v∈V\) and \(w∈W\). On obtient alors,
à l'ordre 4 en \(ξ\) et \(η\)
\begin{equation}
  \begin{aligned}[b]
    Δℰ ={}&ℰ(u₀+ξv+ηw)-ℰ(u₀)\\
    ={}&\tfrac12ℰ_{,uu}(u₀;ξv+ηw, ξv+ηw)
    +\tfrac16ℰ_{,uuu}(u₀;ξv+ηw, ξv+ηw, ξv+ηw)\\
    &+\tfrac1{24}ℰ_{,uuuu}(u₀;ξv+ηw, ξv+ηw, ξv+ηw, ξv+ηw)
    +𝒪\bigl(\bigl(ξ²+η²\bigr)²\bigr),
  \end{aligned}
\end{equation}
où le terme linéaire a été omis puisque \(u\) est un point critique de
l'énergie. En tenant compte de la multilinéarité et de la symétrie des
différentielles successives de l'énergie \(ℰ\), ainsi que du fait que
\(ℰ_{,uu}(v, •)=0\), l'expression précédente s'écrit
\begin{equation}
  \begin{aligned}[b]
    Δℰ={}&\tfrac12η²ℰ_{,uu}(w, w)+\tfrac16ξ³ℰ_{,uuu}(v, v, v)
    +\tfrac12ξ²ηℰ_{,uuu}(v, v, w)\\
    &+\tfrac12ξη²ℰ_{,uuu}(v, w, w)+\tfrac16η³ℰ_{,uuu}(w, w, w)\\
    &+\tfrac1{24}ξ⁴ℰ_{,uuuu}(v, v, v, v)+\tfrac16ξ³ℰ_{,uuuu}(v, v, v, w)\\
    &+\tfrac14ξ²η²ℰ_{,uuuu}(v, v, w, w)+\tfrac16ξη³ℰ_{,uuuu}(v, w, w, w)\\
    &+\tfrac1{24}η⁴ℰ_{,uuuu}(w, w, w, w)+𝒪\bigl(\bigl(ξ²+η²\bigr)²\bigr),
  \end{aligned}
\end{equation}
où l'on convient que toutes les différentielles de \(ℰ\) sont évaluées au point
d'équilibre \(u₀\).

Pour que l'équilibre soit stable, il faut que expression soit positive ou nulle
pour tous \(ξ\) et \(η\) suffisamment petits. En prenant tout d'abord \(η=0\),
on obtient les conditions nécessaires
\begin{equation}
  \label{eq:20211108164416}
  ℰ_{,uuu}(v, v, v)=0\quad\text{et}\quadℰ_{,uuuu}(v, v, v, v)≥0
  \quad\text{pour tout}\quad v∈V.
\end{equation}

En d'autres termes, s'il existe \(v∈ V\) tel que \(ℰ_{,uuu}(v, v, v)≠0\) ou
\(ℰ_{,uuuu}(v, v, v, v)<0\), alors l'équilibre est \emph{instable}. Les
conditions précédentes ne sont pas suffisantes pour assurer la stabilité. En
effet, supposant ces conditions remplies, on prend maintenant \(η=ξ²\)
\begin{equation}
  Δℰ=\tfrac12ξ⁴\bigl[ℰ_{,uu}(w, w)+ℰ_{,uuu}(v, v, w)
  +\tfrac1{12}ℰ_{,uuuu}(v, v, v, v)\bigr]+o(ξ⁴)
\end{equation}
et on obtient la condition nécessaire supplémentaire
\begin{equation}
  \label{eq:20211109145356}
  ℰ_{,uu}(v, v)+ℰ_{,uuu}(v, v, w)+\tfrac1{12}ℰ_{,uuuu}(v, v, v, v)≥0,
\end{equation}
pour tous \(v∈V\) et \(w∈W\). Pour \(v∈\kerℰ₂\) fixé, l'expression précédente
est minimale lorsque \(w\) satisfait le problème variationnel
\begin{equation}
  \label{eq:20211109145224}
  2ℰ_{,uu}(w, \hat{w})+ℰ_{,uuu}(v, v, \hat{w})=0\quad\text{pour tout}\quad\hat{w}∈W.
\end{equation}

Soit \(w_{ij}∈ W\) l'unique solution du problème variationnel suivant
\begin{equation}
  \label{eq:20211221155859}
  2ℰ_{,uu}(w_{ij}, \hat{w})+ℰ_{,uuu}(vᵢ, vⱼ, \hat{w})=0\quad\text{pour tout}\quad\hat{w}∈W.
\end{equation}

Alors, pour \(v=ξ^i vᵢ\), la solution du problème
variationnel~\eqref{eq:20211109145224} est \(w=ξ^iξ^jw_{ij}\). Pour cette
valeur de \(v\), la condition~\eqref{eq:20211109145356} s'écrit
\begin{equation}
  \bigl[\tfrac1{12}ℰ_{,uuuu}(vᵢ, vⱼ, vₖ, v_l)-ℰ_{,uu}(w_{ij}, w_{kl})\bigr]ξ^iξ^jξ^kξ^l≥0,
\end{equation}
pour tous \(ξᵢ, ξⱼ, ξₖ, ξ_l∈ℝ\). On peut montrer que l'inégalité stricte est
une condition \emph{suffisante} de stabilité.

\section{Bifurcations}

Soit \(u^*(λ)\) la branche fondamentale. Par définition
\begin{equation}
  ℰ_{,u}[u^*(λ), λ;\hat{u}]=0\quad\text{pour tout}\quad\hat{u}∈ U.
\end{equation}

Il sera commode d'introduire les notations suivantes
\begin{align}
  ℰ₂(λ)&=ℰ_{,uu}[u^*(λ), λ],\\
  ℰ₃(λ)&=ℰ_{,uuu}[u^*(λ), λ],\\
  ℰ₄(λ)&=ℰ_{,uuuu}[u^*(λ), λ].
\end{align}

Noter que \(ℰ₂\), \(ℰ₃\) et \(ℰ₄\) sont des formes bi-, tri- et quadri-linéaires,
respectivement. L'application de ces formes à des éléments de \(U\) sera notée
\(ℰ₂(λ; u, v)\), \(ℰ₃(λ; u, v, w)\), etc\dots. La dérivée de ces formes par
rapport à \(λ\) sera notée à l'aide d'un point supérieur (\(\dot{ℰ₂}\),
\(\dot{ℰ₃}\), \dots).

On suppose que l'équilibre est stable pour des valeurs suffisamment petites de
\(λ\). Plus précisément, on suppose que \(ℰ₂(λ)\) est définie positive pour tout
\(λ<λ₀\). Pour \(λ=λ₀\), la forme quadratique \(ℰ₂(λ₀)\) n'est plus que
positive. En notant \(u₀=u^*(λ₀)\) la position d'équilibre obtenue pour la
valeur critique \(λ₀\) du paramètre de chargement \(λ\), on s'intéresse à
toutes les courbes d'équilibre qui passent par le point \((u₀, λ₀)\).

Noter que dans ce qui suit, on convient que les formes \(ℰ₂\), \(ℰ₃\) et \(ℰ₄\)
sont implicitement évaluées en \(λ₀\) lorsque \(λ\) n'est pas rappelé : ainsi,
on notera \(ℰ₂(•, •)\) plutôt que \(ℰ₂(λ₀; •, •)\).

On écrit toute courbe d'équilibre passant par le point \((u₀, λ₀)\) sous la
forme paramétrique suivante
\begin{align}
  \label{eq:20211115075817}
  λ&=λ₀+ηλ₁+η²λ₂+η³λ₃+\cdots,\\
  \label{eq:20211115075835}
  u&=u^*(λ)+ηu₁+η²u₂+η³u₃+\cdots,
\end{align}
où \(η\) est un paramètre, non précisé pour le moment. Noter que, dans la
représentation paramétrique de \(u\), \(u^*\) est évalué en \(λ\) et pas en
\(λ₀\).

On se restreindra dans ce qui suit au cas non-dégénéré \(u₁≠0\). On peut alors
toujours supposer que \(〈u₁, u₁〉=1\). \pdfmargincomment{Que se passe-t-il si
  u₁=u₂=…=0 ?} En effet, en posant \(θ=\lVert u₁\rVert η\), les développements
précédents s'écrivent
\begin{align}
  λ&=λ₀+θ\lVert u₁\rVert⁻¹λ₁+θ²\lVert u₁\rVert⁻²λ₂+θ³\lVert u₁\rVert⁻³λ₃+\cdots,\\
  u&=u^*(λ)+θ\lVert u₁\rVert⁻¹u₁+θ²\lVert u₁\rVert⁻²u₂+θ³\lVert u₁\rVert⁻³u₃+
\end{align}
et le terme linéaire en \(θ\) du développement asymptotique de \(u\) est bien
de norme 1.

Les coefficients \(λ ₖ\) et \(uₖ\) des développements~\eqref{eq:20211115075817}
et \eqref{eq:20211115075835} sont identifiés en écrivant que l'énergie est
stationnaire le long de la courbe d'équilibre, c'est-à-dire que le résidu
\(ℰ_{,u}[u(η), λ(η)]\) est nul. Le développement limité du résidu est établi au
voisinage de \(η=0\) dans l'annexe~\ref{sec:20211112182000} [voir
Éq.~\eqref{eq:20220107080901}]. En écrivant que tous ses termes s'annulent, on
trouve successivement, pour tout \(\hat{u}∈ U\)
\begin{gather}
  \label{eq:20211112182917}
  ℰ₂(λ₀; u₁, \hat{u})=0,\\
  \label{eq:20211112183220}
  ℰ₃(λ₀; u₁, u₁, \hat{u})+2λ₁\dot{ℰ₂}(λ₀; u₁, \hat{u})+2ℰ₂(λ₀; u₂, \hat{u})=0,\\
  \label{eq:20220114135717}
  \begin{aligned}[b]
    ℰ₄(λ₀; u₁, u₁, u₁, \hat{u})+6ℰ₃(λ₀; u₁, u₂, \hat{u})+6ℰ₂(λ₀; u₃, \hat{u})&\\
    +3λ₁\bigl[\dot{ℰ₃}(λ₀; u₁, u₁, \hat{u})+2\dot{ℰ₂}(λ₀; u₂, \hat{u})\bigr]&\\
    +3λ₁²\ddot{ℰ₂}(λ₀; u₁, \hat{u})
    +6λ₂\dot{ℰ₂}(λ₀; u₁, \hat{u})&=0.
  \end{aligned}
\end{gather}

On déduit de l'équation~\eqref{eq:20211112182917} que \(u₁∈V\). On pose alors
\begin{equation}
  \label{eq:20220124135236}
  u₁=ξ₁^i vᵢ.
\end{equation}

En prenant \(\hat{u}=vᵢ\), l'équation~\eqref{eq:20211112183220} s'écrit
\begin{equation}
  \label{eq:20220216140121}
  ℰ_{ijk}(λ₀)\,ξ₁^jξ₁^k+2λ₁\dot{ℰ}_{ij}(λ₀)\,ξ₁^j=0.
\end{equation}

Pour le terme d'ordre 2, on introduit la décomposition~:
\(u_2=ξ_2^iv_i+u_2^W\), où \(u_2^W∈W\). On a alors
\(ℰ₂(u_2, \hat{u})=ℰ₂(u_2^W, \hat{u})\) et l'équation~\eqref{eq:20211112183220}
s'écrit
\begin{equation}
  ℰ₃(λ₀; u₁, u₁, \hat{u})+2λ₁\dot{ℰ₂}(λ₀; u₁, \hat{u})
  +2ℰ₂(λ₀; u_2^W, \hat{u})=0,
\end{equation}
pour tout \(\hat{u}∈V\). En prenant la fonction test dans l'espace \(W\), on
obtient le problème variationnel suivant~: trouver \(u_2^W∈W\) tel que
\begin{equation}
  \label{eq:20211210131623}
  ξ₁^iξ₁^jℰ₃(λ₀; vᵢ, vⱼ, \hat{w})
  +2λ₁ξ₁^i\dot{ℰ₂}(λ₀; vᵢ, \hat{w})
  +2ℰ₂(λ₀; u_2^W, \hat{w})=0,
\end{equation}
pour tout \(\hat{w}∈W\). Soient \(wᵢ∈W\) les solutions des problèmes
variationnels suivants
\begin{equation}
  \label{eq:20220208143055}
  ℰ₂(λ₀; wᵢ, \hat{w})+\dot{ℰ₂}(λ₀; vᵢ, \hat{w})=0,
\end{equation}
pour tout \(\hat{w}∈W\). La solution du problème~\eqref{eq:20211210131623}
s'obtient par simple combinaison linéaire des \(wᵢ\) et \(w_{ij}\) introduits
précédemment par le problème variationnel~\eqref{eq:20211221155859}, de sorte
que
\begin{equation}
  \label{eq:20220124135324}
  u_2^W=ξ_1^iξ_1^jw_{ij}+λ_1ξ_1^i w_i
  \quad\text{et}\quad
  u_2=ξ_2^iv_i+ξ_1^iξ_1^jw_{ij}+λ_1ξ_1^i w_i.
\end{equation}

En prenant \(\hat{u}=v_i\) dans l'équation~\eqref{eq:20220114135717}, on
obtient l'équation de bifurcation suivante
\begin{equation}
  \label{eq:20220210143805}
  \begin{aligned}[b]
    6ξ₂^j\bigl[ξ₁^kℰ_{ijk}(λ₀)+λ₁\dot{ℰ}_{ij}(λ₀)\bigr]&\\
    +ξ₁^jξ₁^kξ₁^l\bigl[ℰ_{ijkl}(λ₀)+6ℰ₃(λ₀; v_i, v_j, w_{kl})\bigr]&\\
    +3λ₁ξ₁^jξ₁^k\bigl[\dot{ℰ}_{ijk}(λ₀)+2ℰ₃(λ₀; v_i, v_j, w_k)
    +2\dot{ℰ₂}(λ₀; v_i, w_{jk})\bigr]&\\
    +3λ₁²ξ₁^j\bigl[\ddot{ℰ}_{ij}(λ₀)+2\dot{ℰ₂}(v_i, w_j)\bigr]
    +6λ₂ξ₁^j\dot{ℰ}_{ij}(λ₀)
    &=0.
  \end{aligned}
\end{equation}

On remarque que certains termes peuvent être symétrisés. Ainsi
\begin{equation}
  \begin{aligned}[b]
    ξ₁^jξ₁^kξ₁^lℰ₃(λ₀; v_i, v_j, w_{kl})={}&\tfrac13ξ₁^jξ₁^kξ₁^l\bigl[
    ℰ₃(λ₀; v_i, v_j, w_{kl})\\
    &+ℰ₃(λ₀; v_i, v_k, w_{lj})+ℰ₃(λ₀; v_i, v_l, w_{jk})\bigr],
  \end{aligned}
\end{equation}
de même
\begin{equation}
  2ξ₁^jξ₁^kℰ₃(λ₀; v_i, v_j, w_k)=ξ₁^jξ₁^k\bigl[ℰ₃(λ₀; v_i, v_j, w_k)
  +ℰ₃(λ₀; v_i, w_j, v_k)\bigr]
\end{equation}
et l'équation~\eqref{eq:20220210143805} s'écrit
\begin{equation}
\label{eq:20220216141706}
  6A_{ij}ξ₂^j+E_{ijkl}\,ξ₁^jξ₁^kξ₁^l+3λ₁F_{ijk}ξ₁^jξ₁^k+3λ₁²G_{ij}ξ₁^j
  +6λ₂\mathring{E}_{ij}ξ₁^j=0,
\end{equation}
en posant\pdfmargincomment{C'est l'expression de Bᵢⱼ de Nick, voir
  Éq. (AC-5.14) p. 74}
\begin{equation}
  A_{ij}=ξ₁^kℰ_{ijk}(λ₀)+λ₁\dot{ℰ}_{ij}(λ₀)
\end{equation}
\pdfmargincomment{Cette expression coïncide avec l'expression
  (AC-5.11), page 71, de ℰ\_ijkl de Nick. Le facteur 2 provient du fait que
  dans le problème variationnel (AC-5.9) qui définit les \(v_{ij}\) de Nick, le
  facteur 2 du problème~\eqref{eq:20211221155859} n'est pas présent.}
\begin{equation}
  E_{ijkl}=ℰ₄(λ₀; v_i, v_j, v_k, v_l)+2\bigl[ℰ₃(λ₀; v_i, v_j, w_{kl})
  +ℰ₃(λ₀; v_i, v_k, w_{lj})+ℰ₃(λ₀; v_i, v_l, w_{jk})\bigr].
\end{equation}

\begin{equation}
  F_{ijk}=\dot{ℰ}₃(λ₀; v_i, v_j, v_k)+ℰ₃(λ₀; v_i, v_j, w_k)
  +ℰ₃(λ₀; v_i, w_j, v_k)+2\dot{ℰ₂}(λ₀; v_i, w_{jk})
\end{equation}

\begin{equation}
  G_{ij}=\ddot{ℰ}_{ij}(λ₀)+2\dot{ℰ₂}(v_i, w_j)
\end{equation}

On supposera satisfaite la condition suivante, qui assure que ce système est régulier
\begin{equation}
  \det\bigl(ξ₁^kℰ_{ijk}+λ₁\dot{ℰ}_{ij}\bigr)_{i,j}\neq0.
\end{equation}

Les \(ξ₂^i\) sont donc déterminés de façon unique si \(λ₁\), \(λ₂\) et \(ξ₁^i\)
sont connus.

Le développement limité suivant de l'énergie le long de la branche bifurquée
est établi dans l'annexe~\ref{sec:20220121172919}
\begin{equation}
  \label{eq:20220121172753}
  \begin{aligned}[b]
    ℰ[u(η), λ(η)]={}&ℰ\bigl(u^*[λ(η)], λ(η)\bigr)
    +\tfrac12η²ℰ₂(λ₀; u₁, u₁)\\
    &+\tfrac16η³\bigl[ℰ₃(λ₀;u₁, u₁, u₁)+6ℰ₂(λ₀; u₁, u₂)\\
    &+3λ₁\dot{ℰ}₂(λ₀; u₁, u₁)\bigr]+\tfrac1{24}η⁴\bigl\{ℰ₄(λ₀;u₁, u₁, u₁, u₁)\\
    &+12ℰ₃(λ₀; u₁, u₁, u₂)+12ℰ₂(λ₀; u₂, u₂)\\
    &+18ℰ₂(λ₀; u₁, u₃)+4λ₁\bigl[\dot{ℰ}₃(λ₀; u₁, u₁, u₁)\\
    &+6\dot{ℰ}₂(λ₀; u₁, u₂)\bigr]+6λ₁²\ddot{ℰ}₂(λ₀; u₁, u₁)\\
    &+12λ₂\dot{ℰ}₂(λ₀; u₁, u₁)\bigr\}+o(η⁴).
  \end{aligned}
\end{equation}

La relation~\eqref{eq:20211112182917} montre tout d'abord que les termes en
\(ℰ₂(λ₀; u₁, u₁)\), \(ℰ₂(λ₀; u₁, u₂)\) et \(ℰ₂(λ₀; u₁, u₃)\) sont nuls. Le
premier terme non-nul du développement limité~\eqref{eq:20220121172753} est
donc le terme d'ordre 3. En prenant de plus \(\hat{u}=u₁\) dans la
relation~\eqref{eq:20211112183220}, on trouve finalement\pdfmargincomment{Cette
  expression coïncide avec l'Éq. (AC-5.29) de Tryantafyllidis.}
\begin{equation}
  ℰ[u(η), λ(η)]=ℰ\bigl(u^*[λ(η)], λ(η)\bigr)+\tfrac16λ₁η³\dot{ℰ}₂(u₁, u₁)+o(η³).
\end{equation}

Si \(λ₁=0\), le premier terme non-nul du développement
limité~\eqref{eq:20220121172753} est d'ordre 4. En prenant cette fois
\(\hat{u}=u₂\) dans la relation~\eqref{eq:20211112183220} et \(\hat{u}=u₁\)
dans la relation~\eqref{eq:20220114135717}, on obtient\pdfmargincomment{Cette
  expression coïncide avec l'Éq. (AC-5.30) de Tryantafyllidis.}
\begin{equation}
  ℰ[u(η), λ(η)]=ℰ\bigl(u^*[λ(η)], λ(η)\bigr)+\tfrac1{4}λ₂η⁴\dot{ℰ}₂(λ₀; u₁, u₁)+o(η⁴).
\end{equation}

\begin{center}
  ***
\end{center}

Pour analyser la stabilité de la branche bifurquée ainsi trouvée, il faut
déterminer le signe de la hessienne de l'énergie. On peut d'ores et déjà
remarquer que, sur la branche fondamentale (\(u₁=u₂=0\)), en prenant \(η=λ-λ₀\)
(\(λ₁=1\))
\begin{equation}
  ℰ₂(λ; \hat{u}, \hat{v})
  =ℰ₂(λ₀; \hat{u}, \hat{v})+\bigl(λ-λ₀\bigr)\dot{ℰ}₂(λ₀; \hat{u}, \hat{v})+o(λ-λ₀).
\end{equation}

Dans ce qui suit, on supposera que \(\dot{ℰ}_2(λ₀)≠0\). Pour \(\hat{v}∈V\),
l'égalité précédente s'écrit
\begin{equation}
  ℰ₂(λ₀; \hat{v}, \hat{v})=\bigl(λ-λ₀\bigr)\dot{ℰ}₂(\hat{v}, \hat{v})+o(λ-λ₀).
\end{equation}

Comme la branche fondamentale est stable pour \(λ<λ₀\), on doit avoir
\(\dot{ℰ}₂(λ₀; \hat{v}, \hat{v})<0\). La forme quadratique \(\dot{ℰ}₂(λ₀)\) est
donc définie négative sur \(V\). Le développement limité de la hessienne de
l'énergie le long de la branche bifurquée est établi dans
l'annexe~\ref{sec:20211115081016}. Pour tout \(\hat{u}∈U\), on trouve
\begin{equation}
  \label{eq:20211115082025}
  \begin{aligned}[b]
    ℰ_{,uu}[u(η), λ(η); \hat{u}, \hat{u}]
    ={}&ℰ₂(λ₀; \hat{u}, \hat{u})+η\bigl[ℰ₃(λ₀; u₁, \hat{u}, \hat{u})
    +λ₁\dot{ℰ}₂(λ₀; \hat{u}, \hat{u})\bigr]\\
    &+\tfrac12η²\bigl[ℰ₄(λ₀; u₁, u₁, \hat{u}, \hat{u})
    +2λ₁\dot{ℰ}₃(λ₀; u₁, \hat{u}, \hat{u})\\
    &+λ₁²\ddot{ℰ}₂(λ₀; \hat{u}, \hat{u})
    +ℰ₃(λ₀; u₂, \hat{u}, \hat{u})\\
    &+λ₂\dot{ℰ}₂(λ₀; \hat{u}, \hat{u})\bigr]
    +o(η²).
  \end{aligned}
\end{equation}

On peut décomposer le vecteur \(\hat{u}∈U\) de façon unique sous la forme
\(\hat{u}=\hat{v}+\hat{w}\), avec \(\hat{v}∈V\) et \(\hat{w}∈W\). Le terme
constant du développement précédent vaut alors \(ℰ₂(λ₀; \hat{w}, \hat{w})\). Si
\(\hat{w}≠0\), alors ce terme constant est strictement positif, puisque la
hessienne est définie positive sur \(W\) en \(λ=λ₀\). La hessienne sur la
branche bifurquée est donc positive pour tout \(\hat{u}∈U\) ayant une
composante dans \(W\). Il suffit donc d'étudier le signe de la hessienne sur la
branche bifurquée pour \(\hat{u}∈V\), soit \(\hat{u}=\hat{ξ}^iv_i\). Dans ce
cas, compte-tenu de l'expression~\eqref{eq:20220124135324} de \(u₂\)
\begin{equation}
  \begin{aligned}[b]
    ℰ₃(λ₀; u₂, \hat{u}, \hat{u})={}&
    ξ₁^iξ₁^j\hat{ξ}^k\hat{ξ}^lℰ₃(λ₀; w_{ij}, v_k, v_l)
    +ξ₂^i\hat{ξ}^j\hat{ξ}^kℰ₃(λ₀; v_i, v_j, v_k)\\
    &+λ₁ξ₁^i\hat{ξ}^j\hat{ξ}^kℰ₃(λ₀; w_i, v_j, v_k).
  \end{aligned}
\end{equation}

On peut complètement symétriser le premier terme
\begin{equation}
  ξ₁^iξ₁^j\hat{ξ}^k\hat{ξ}^lℰ₃(λ₀; w_{ij}, v_k, v_l)
  =\tfrac13\bigl[ξ₁^iξ₁^j\hat{ξ}^k\hat{ξ}^lℰ₃(λ₀; w_{ij}, v_k, v_l)
  +ξ₁^iξ₁^j\hat{ξ}^k\hat{ξ}^lℰ₃(λ₀; w_{ij}, v_k, v_l)
  +ξ₁^iξ₁^j\hat{ξ}^k\hat{ξ}^lℰ₃(λ₀; w_{ij}, v_k, v_l)\bigr]
\end{equation}

\begin{equation}
  \label{eq:20220203144500}
  \begin{aligned}[b]
    ℰ_{,uu}[u(η), λ(η); \hat{u}, \hat{u}]
    ={}&η\hat{ξ}^i\hat{ξ}^j\bigl[ξ₁^kℰ_{ijk}(λ₀)+λ₁\dot{ℰ}_{ij}(λ₀)\bigr]\\
    &+\tfrac12η²\hat{ξ}^i\hat{ξ}^j\bigl\{ξ₁^kξ₁^l\bigl[ℰ_{ijkl}(λ₀)
    -2ℰ₂(λ₀; w_{ij}, w_{kl})\bigr]\\
    &+λ₁ξ₁^k\bigl[ℰ₃(λ₀; v_i, v_j, w_k)+\dot{ℰ}_{ijk}(λ₀)\bigr]\\
    &+λ₁²\ddot{ℰ}_{ij}(λ₀)+ξ₂^kℰ_{ijk}(λ₀)+λ₂\dot{ℰ}_{ij}\bigr\}+o(η²).
  \end{aligned}
\end{equation}

Compte-tenu de la relation~\eqref{eq:20211112183220}, on trouve pour
\(\hat{v}=u₁\) (\(\hat{ξ}^i=ξ₁^i\))
\begin{equation}
  ℰ_{,uu}[u(η), λ(η); u₁, u₁]=-λ₁η\dot{ℰ}₂(λ₀; u₁, u₁)+o(η).
\end{equation}

Si \(λ₁≠0\), l'expression précédente peut également s'écrire
\begin{equation}
  ℰ_{,uu}[u(η), λ(η); u₁, u₁]=-\bigl(λ-λ₀\bigr)\dot{ℰ}₂(λ₀; u₁, u₁)+o(λ-λ₀),
\end{equation}
qui est négative pour \(λ<λ₀\): la branche bifurquée est instable sous la
charge critique. Il reste alors à étudier le signe de la hessienne de la
branche bifurquée au-delà de la charge critique (\(λ>λ₀\)).

\section{Cas d'un mode de flambement simple (\(m=1\))}

Lorsque \(m=\dim V=1\), la base \(v₁, …, vₘ\) est réduite au seul vecteur
\(v₁\) et \(u₁\) est parallèle à ce vecteur. Comme \(\lVert u₁\rVert=1\), on a
donc nécessairement \(u₁=v₁\) (quitte à changer \(η\) en \(-η\)). L'équation de
bifurcation~\eqref{eq:20220216140121} s'écrit alors
\begin{equation}
  \label{eq:20220203144712}
  ℰ₁₁₁(λ₀)+2λ₁\dot{ℰ}₁₁(λ₀)=0,
  \quad\text{soit}\quad
  λ₁=-\frac{ℰ₁₁₁(λ₀)}{2\dot{ℰ}₁₁(λ₀)},
\end{equation}
où on remarque que le quotient a un sens, puisque \(\dot{ℰ₂}(λ₀)\) est définie
négative sur \(V\). On trouve donc les développements limités
\begin{equation}
  λ=λ₀+λ₁η+o(η)
  \quad\text{et}\quad
  u=u^*(λ)+ηv₁+o(η),
\end{equation}
soit finalement, en éliminant \(η\)
\begin{equation}
  λ=λ₀-\frac{ξℰ₁₁₁(λ₀)}{2\dot{ℰ}₁₁(λ₀)}+o(ξ),
  \quad\text{avec}\quad
  ξ=〈u(λ)-u^*(λ), v₁〉.
\end{equation}

Pour déterminer la stabilité de la branche bifurquée, on calcule la hessienne
en \((v₁, v₁)\). L'équation~\eqref{eq:20220203144500} s'écrit
\begin{equation}
  ℰ_{,uu}[u(η), λ(η); v₁, v₁]=η\bigl[ℰ₁₁₁(λ₀)+λ₁\dot{ℰ}₁₁(λ₀)\bigr]+o(η),
\end{equation}
soit, en substituant l'équation~\eqref{eq:20220203144712}
\begin{equation}
  ℰ_{,uu}[u(η), λ(η); v₁, v₁]=-λ₁η\dot{ℰ}₁₁(λ₀)+o(η).
\end{equation}

Ce développement ne permet de conclure que si le terme linéaire est non-nul,
soit \(ℰ₁₁₁(λ₀)≠0\) [voir Éq.~\eqref{eq:20220203144712}]. Dans ce cas, le
développement asymptotique précédent s'écrit également
\begin{equation}
  ℰ_{,uu}[u(η), λ(η); v₁, v₁]=-\bigl(λ-λ₀\bigr)\dot{ℰ}₁₁(λ₀)+o(λ-λ₀).
\end{equation}

Comme \(\dot{ℰ}₂(λ₀)\) est définie négative, la branche bifurquée est
\emph{instable} pour \(λ<λ₀\) et \emph{stable} pour \(λ>λ₀\) lorsque
\(ℰ₁₁₁(λ₀)≠0\).

Supposons maintenant que \(ℰ₁₁₁(λ₀)=0\)~; alors \(λ₁=0\) et il faut calculer au
moins un terme supplémentaire dans le développement limité de la
Hessienne. L'équation de bifurcation~\eqref{eq:20220216141706} s'écrit
\begin{equation}
  \label{eq:20220217164528}
  ℰ₁₁₁₁(λ₀)+6ℰ₃(λ₀; v₁, v₁, u₂)+6λ₂\dot{ℰ}₁₁(λ₀)=0.
\end{equation}
En introduisant le développement~\eqref{eq:20220124135324} de \(u₂\) et en
utilisant le problème variationnel~\eqref{eq:20211221155859}
\begin{equation}
  u₂=ξ₂v₁+w₁₁+λ₁w₁,
  \quad\text{donc}\quad
  \begin{aligned}[t]
    ℰ₃(λ₀;v₁, v₁, u₂)&=ℰ₃(λ₀;v₁, v₁, w₁₁)\\&=-2ℰ₂(λ₀;w₁₁, w₁₁)
  \end{aligned}
\end{equation}
soit finalement
\begin{equation*}
  λ₂=-\frac{ℰ₁₁₁₁(λ₀)-12ℰ₂(λ₀;w₁₁, w₁₁)}{6\dot{ℰ}₁₁(λ₀)},
\end{equation*}
le quotient ayant une nouvelle fois un sens. Le développement
asymptotique~\eqref{eq:20211115082025} de la Hessienne s'écrit alors, en tenant
compte de l'Éq.~\eqref{eq:20220217164528}
\begin{equation}
  \begin{aligned}[b]
    ℰ_{,uu}[u(η), λ(η); v₁, v₁]
    ={}&\tfrac12η²\bigl[ℰ₁₁₁₁(λ₀)+2ℰ₃(λ₀; v₁, v₁, u₂)+2λ₂\dot{ℰ}₁₁(λ₀)\bigr]+o(η²)\\
    ={}&\tfrac5{12}η²ℰ₁₁₁₁(λ₀)+o(η²).
  \end{aligned}
\end{equation}

\appendix

\section{Propriétés des formes bilinéaires symétriques, positives}

Soit \(ℬ\) une forme bilinéaire symétrique, positive sur l'espace vectoriel
\(V\). On définit son noyau \(\kerℬ\) de la façon suivante
\begin{equation}
  \kerℬ=\{u∈V|ℬ(u, u)=0\}.
\end{equation}

\begin{theorem}
  Le noyau \(\kerℬ\) d'une forme bilinéaire, symétrique, positive \(ℬ\) sur
  l'espace vectoriel \(V\) est un sous-espace vectoriel de \(V\).
\end{theorem}
\begin{proof}
  Soient \(u, v∈\kerℬ\), \(α∈ℝ\) et \(w=u+α v\). Montrons que \(w∈\kerℬ\). Il
  suffit d'évaluer \(ℬ(w, w)\)
  \begin{equation}
    ℬ(w, w)=ℬ(u+α v, u+α v)=ℬ(u, u)+2αℬ(u, v)+α²ℬ(v, v),
  \end{equation}
  où l'on a tenu compte de la symétrie de \(ℬ\) pour écrire que
  \(ℬ(u, v)=ℬ(v, u)\). Comme \(u, v∈\kerℬ\), le premier et le dernier terme
  sont nuls, soit \(ℬ(w, w)=2αℬ(u, v)\). La forme bilinéaire étant positive,
  cette grandeur est positive, \emph{quelle que soit la valeur de \(α∈ℝ\)}. On
  en déduit donc que \(ℬ(u, v)=0\), puis que \(ℬ(w, w)=0\).
\end{proof}

Soit \(u∈V\). Alors
\begin{equation}
  u∈\kerℬ\quad\text{ssi}\quad\text{pour tout }v∈V, ℬ(u, v)=0.
\end{equation}
\begin{proof}
  Soient \(u∈\kerℬ\), \(v∈V\) et \(α∈ℝ\). Comme précédemment, on écrit que
  \(ℬ(w, w)≥0\), avec \(w=α u+v\)
  \begin{equation}
    ℬ(w, w)=2αℬ(u, v)+ℬ(v, v)≥0,
  \end{equation}
  où l'on a tenu compte de ce que \(ℬ(u, u)=0\). L'expression précédente,
  affine en \(α\), a un signe constant. Le terme linéaire en \(α\) est donc
  nul, soit \(ℬ(u, v)=0\).

  Réciproquement, si \(ℬ(u, v)=0\) pour tout \(v∈V\), alors \(ℬ(u, u)=0\) (en
  prenant \(v=u\)).
\end{proof}

\section{Développements limités le long d'une branche bifurquée du diagramme
  d'équilibre}

\subsection{Principe du calcul}
\label{sec:20220107121442}

On pose dans ce qui suit
\begin{align}
  \label{eq:20211112155446}
  Λ(η)&=λ(η)-λ₀=ηλ₁+η²λ₂+η³λ₃+\cdots,\\
  \label{eq:20211112113028}
  U(η)&=u(η)-u^*[λ(η)]=ηu₁+η²u₂+η³u₃+\cdots.
\end{align}

On considère une quantité \(ℱ\), fonction de \(u\) et \(λ\)~: \(ℱ(u,
λ)\). Cette fonctionnelle est évaluée le long de la branche bifurquée. En
d'autres termes, on considère
\begin{equation}
  f(η)=F\bigl(u^*[λ₀+Λ(η)]+U(η), λ₀+Λ(η)\bigr).
\end{equation}

On souhaite établir un développement limité de \(f\) au voisinage de \(η=0\),
ce qui conduit à calculer les dérivées successives de \(f\) en \(η=0\), puisque
\begin{equation}
  f(η)=f(0)+η f'(0)+\tfrac12η²f''(0)+\cdots
\end{equation}

Pour calculer ces dérivées, il sera commode d'introduire la fonction auxiliaire
\(F\)
\begin{equation}
  F(η, λ)=ℱ[u^*(λ)+U(η), λ],
\end{equation}
dans laquelle les variables \(λ\) et \(η\) sont provisoirement considérées
comme indépendantes. On a
\begin{equation}
  f(η)=F[η, λ₀+Λ(η)],
\end{equation}
d'où l'on déduit successivement que
\begin{align}
  \label{eq:20211112162417}
  f'(η)={}&∂_ηF+Λ'∂_λF,\\
  \label{eq:20211112165810}
  f''(η)={}&∂_{ηη}²F+2Λ'∂_{ηλ}²F+Λ'^2∂_{λλ}²F+Λ''∂_λ F,\\
  \notag
  f'''(η)={}&∂_{ηηη}³F+3Λ'∂_{ηηλ}³F+3Λ'^2∂_{ηλλ}³F+Λ'^3∂_{λλλ}³F\\
  \label{eq:20211112173223}
          &+3Λ''∂_{ηλ}²F+3Λ'Λ''∂_{λλ}²F+Λ'''∂_λF,\\
  \notag
  f''''(η)={}&∂_{ηηηη}⁴F+4Λ'∂_{ηηηλ}⁴F+6Λ'^2∂_{ηηλλ}⁴F+4Λ'^3∂_{ηλλλ}⁴F\\
  \notag
          &+Λ'^4∂_{λλλλ}⁴F+6Λ''∂_{ηηλ}³F+12Λ'Λ''∂_{ηλλ}³F+6Λ'^2Λ''∂_{λλλ}³F\\
          &+4Λ'''∂_{ηλ}²F+\bigl(3Λ''^2+4Λ'Λ'''\bigr)∂_{λλ}²F+Λ''''∂_λF
\end{align}
où \(Λ\) et ses dérivées sont évaluées en \(η\), tandis que \(F\) et ses
dérivées partielles sont évaluées en \([η, λ₀+Λ(η)]\). En \(η=0\), les
relations précédentes s'écrivent
\begin{align}
  \label{eq:20220107060454}
  f'(0)={}&∂_η F+λ₁∂_λ F,\\
  \label{eq:20220107124311}
  f''(0)={}&∂_{ηη}²F+2λ₁∂_{ηλ}²F+2λ₂∂_λ F+λ₁²∂_{λλ}²F,\\
  \notag
  f'''(0)={}&∂_{ηηη}³F+3λ₁∂_{ηηλ}³F+3λ₁²∂_{ηλλ}³F+λ₁³∂_{λλλ}³F\\
  \label{eq:20220107060500}
          &+6λ₂∂_{ηλ}²F+6λ₁λ₂∂_{λλ}²F+6λ₃∂_λF,\\
  \notag
  f''''(0)={}&∂_{ηηηη}⁴F+4λ₁∂_{ηηηλ}⁴F+6λ₁²∂_{ηηλλ}⁴F+4λ₁³∂_{ηλλλ}⁴F\\
  \notag
          &+λ₁⁴∂_{λλλλ}⁴F+12λ₂∂_{ηηλ}³F+24λ₁λ₂∂_{ηλλ}³F+12λ₁²λ₂∂_{λλλ}³F\\
          &+24λ₃∂_{ηλ}²F+\bigl(12λ₂²+24λ₁λ₃\bigr)∂_{λλ}²F+24λ₄∂_λF
\end{align}
où \(F\) et ses dérivées sont évaluées en \((0, λ₀)\).

\subsection{Développement limité du résidu}
\label{sec:20211112182000}

On cherche un développement limité du résidu (c'est-à-dire de la première
variation de l'énergie). La fonction test \(\hat{u}∈U\) étant fixée, la
méthode précédente est donc appliquée avec
\begin{equation}
  \label{eq:20220107054629}
  f(η)=ℰ_{,u}[u(η), λ(η);\hat{u}]
  \quad\text{et}\quad
  F(η, λ)=ℰ_{,u}[u^*(λ)+U(η), λ; \hat{u}].
\end{equation}

On remarque tout d'abord que \(F(0, λ)=ℰ_{,u}[u^*(λ), λ; \hat{u}]=0\), puisque
\(u^*(λ)\) est un point d'équilibre. En dérivant par rapport à \(λ\), on obtient
\begin{equation}
  \label{eq:20211112164240}
  \frac{∂^kF}{∂λ^k}(0, λ)=0.
\end{equation}

En dérivant une première fois l'expression~\eqref{eq:20220107054629} de \(F\),
on obtient
\begin{align}
  ∂_ηF(η, λ)={}
  &ℰ_{,uu}[u^*(λ)+U(η), λ; U'(η), \hat{u}],\\
  \notag
  ∂_{ηη}²F(η, λ)={}
  &ℰ_{,uuu}[u^*(λ)+U(η), λ; U'(η), U'(η), \hat{u}]\\
  &+ℰ_{,uu}[u^*(λ)+U(η), λ; U''(η), \hat{u}],\\
  \notag
  ∂_{ηηη}³F(η, λ)={}
  &ℰ_{,uuuu}[u^*(λ)+U(η), λ;U'(η), U'(η), U'(η), \hat{u}]\\
  \notag
  &+3ℰ_{,uuu}[u^*(λ)+U(η), λ;U'(η), U''(η), \hat{u}]\\
  &+ℰ_{,uu}[u^*(λ)+U(η), λ;U'''(η), \hat{u}],
\end{align}
soit, en \(η=0\)
\begin{align}
  ∂_η F(0, λ)={}
  &ℰ₂(λ; u₁, \hat{u}),\\
  ∂_{ηη}²F(0, λ)={}
  &ℰ₃(λ; u₁, u₁, \hat{u})+2ℰ₂(λ; u₂, \hat{u}),\\
  ∂_{ηηη}³F(0, λ)={}
  &ℰ₄(λ; u₁, u₁, u₁, \hat{u})+6ℰ₃(λ; u₁, u₂, \hat{u})+6ℰ₂(λ; u₃, \hat{u}).
\end{align}

Les dérivées croisées de \(F\) en \((0, λ)\) s'obtiennent par simple dérivation
des relations précédentes par rapport à \(λ\)
\begin{align}
  ∂_{ηλ}²F(0, λ)={}&\dot{ℰ₂}(λ; u₁, \hat{u}),\\
  ∂_{ηηλ}³F(0, λ)={}&\dot{ℰ₃}(λ; u₁, u₁, \hat{u})+2\dot{ℰ₂}(λ; u₂, \hat{u}),\\
  ∂_{ηλλ}³F(0, λ)={}&\ddot{ℰ₂}(λ; u₁, \hat{u}).
\end{align}

En insérant les résultats précédentes dans les relations
générales~\eqref{eq:20220107060454}–\eqref{eq:20220107060500}, on trouve
finalement les expressions suivantes des dérivées successives de \(f\) en
\(η=0\)
\begin{align}
  f'(0)={}
  &ℰ₂(λ₀; u₁, \hat{u}),\\
  f''(0)={}
  &ℰ₃(λ₀; u₁, u₁, \hat{u})
    +2λ₁\dot{ℰ₂}(λ₀; u₁, \hat{u})
    +2ℰ₂(λ₀; u₂, \hat{u}),\\
  \notag
  f'''(0)={}
  &ℰ₄(λ₀; u₁, u₁, u₁, \hat{u})
    +6ℰ₃(λ₀; u₁, u₂, \hat{u})
    +6ℰ₂(λ₀; u₃, \hat{u})\\
  \notag
  &+3λ₁\bigl[\dot{ℰ₃}(λ₀; u₁, u₁, \hat{u})
    +2\dot{ℰ₂}(λ₀; u₂, \hat{u})\bigr]\\
  &+3λ₁²\ddot{ℰ₂}(λ₀; u₁, \hat{u})
    +6λ₂\dot{ℰ₂}(λ₀; u₁, \hat{u}).
\end{align}

On en déduit finalement le développement limité à l'ordre 3 en \(η\) du résidu
\begin{equation}
  \label{eq:20220107080901}
  \begin{aligned}[b]
    ℰ_{,u}[u(η), λ(η)]={}
    &ηℰ₂(λ₀; u₁, \hat{u})
    +\tfrac12η²\bigl[ℰ₃(λ₀; u₁, u₁, \hat{u})\\
    &+2λ₁\dot{ℰ₂}(λ₀; u₁, \hat{u})
    +2ℰ₂(λ₀; u₂, \hat{u})\bigr]\\
    &+\tfrac16η³\bigl\{
    ℰ₄(λ₀; u₁, u₁, u₁, \hat{u})
    +6ℰ₃(λ₀; u₁, u₂, \hat{u})\\
    &+6ℰ₂(λ₀; u₃, \hat{u})
    +3λ₁\bigl[\dot{ℰ₃}(λ₀; u₁, u₁, \hat{u})\\
    &+2\dot{ℰ₂}(λ₀; u₂, \hat{u})\bigr]
    +3λ₁²\ddot{ℰ₂}(λ₀; u₁, \hat{u})\\
    &+6λ₂\dot{ℰ₂}(λ₀; u₁, \hat{u})\bigr\}
    +o(η³).
  \end{aligned}
\end{equation}

\subsection{Développement limité de l'énergie}
\label{sec:20220121172919}

On s'intéresse ici à l'écart d'énergie, pour un chargement \(λ\) donné, entre
la branche bifurquée et la branche fondamentale, soit
\begin{equation}
  F(η, λ) = ℰ[u^*(λ)+U(η), λ]-ℰ[u^*(λ), λ]
\end{equation}
et
\begin{equation}
  f(η) = F[η, λ₀+Λ(η)].
\end{equation}

On observe tout d'abord que \(F(0, λ)=0\) pour tout \(λ\), donc
\begin{equation}
  \frac{∂^kF}{∂λ^k}(0, λ)=0\quad(k≥0),
\end{equation}
tandis que les dérivées de \(F\) par rapport à \(η\) s'écrivent
\begin{align}
  ∂_ηF(η, λ)={}&ℰ_{,u}[u^*(λ)+U(η), λ; U'(η)],\\
  \notag
  ∂_{ηη}²F(η, λ)={}&ℰ_{,uu}[u^*(λ)+U(η), λ; U'(η), U'(η)]\\
               &+ℰ_{,u}[u^*(λ)+U(η), λ; U''(η)],\\
  \notag
  ∂_{ηηη}³F(η, λ)={}&ℰ_{,uuu}[u^*(λ)+U(η), λ; U'(η), U'(η), U'(η)]\\
  \notag
               &+3ℰ_{,uu}[u^*(λ)+U(η), λ; U'(η), U''(η)]\\
               &+ℰ_{,u}[u^*(λ)+U(η), λ; U'''(η)],\\
  ∂_{ηηηη}⁴F(η, λ)={}&ℰ_{,uuuu}[u^*(λ)+U(η), λ; U'(η), U'(η), U'(η), U'(η)]\\
  \notag
               &+6ℰ_{,uuu}[u^*(λ)+U(η), λ; U'(η), U'(η), U''(η)]\\
  \notag
               &+3ℰ_{,uu}[u^*(λ)+U(η), λ; U''(η), U''(η)]\\
  \notag
               &+3ℰ_{,uu}[u^*(λ)+U(η), λ; U'(η), U'''(η)]\\
  \notag
               &+ℰ_{,u}[u^*(λ)+U(η), λ; U''''(η)],
\end{align}
soit, en \(η=0\), en observant que \(ℰ_{,u}[u^*(λ), λ]=0\)
\begin{align}
  ∂_ηF(0, λ)={}&0,\\
  ∂_{ηη}²F(0, λ)={}&ℰ₂(λ; u₁, u₁),\\
  ∂_{ηηη}³F(0, λ)={}&ℰ₃(λ;u₁, u₁, u₁)+6ℰ₂(λ; u₁, u₂),\\
  \notag
  ∂_{ηηηη}⁴F(η, λ)={}&ℰ₄(λ;u₁, u₁, u₁, u₁)+12ℰ₃(λ; u₁, u₁, u₂)\\
               &+12ℰ₂(λ; u₂, u₂)+18ℰ₂(λ; u₁, u₃).
\end{align}

On en déduit que
\begin{align}
  ∂_{ηλ}²F(0, λ)={}&0,\\
  ∂_{ηηλ}³F(0, λ)={}&\dot{ℰ}₂(λ; u₁, u₁),\\
  ∂_{ηλλ}³F(0, λ)={}&0,\\
  ∂_{ηηηλ}⁴F(0, λ)={}&\dot{ℰ}₃(λ; u₁, u₁, u₁)+6\dot{ℰ}₂(λ; u₁, u₂),\\
  ∂_{ηηλλ}⁴F(0, λ)={}&\ddot{ℰ}₂(λ; u₁, u₁),\\
  ∂_{ηλλλ}⁴F(0, λ)={}&0
\end{align}
et finalement
\begin{align}
  f'(0)={}&0,\\
  f''(0)={}&ℰ₂(λ₀; u₁, u₁),\\
  f'''(0)={}&ℰ₃(λ₀;u₁, u₁, u₁)+6ℰ₂(λ₀; u₁, u₂)+3λ₁\dot{ℰ}₂(λ₀; u₁, u₁),\\
  \notag
  f''''(0)={}&ℰ₄(λ₀;u₁, u₁, u₁, u₁)+12ℰ₃(λ₀; u₁, u₁, u₂)\\
  \notag
          &+12ℰ₂(λ₀; u₂, u₂)+18ℰ₂(λ₀; u₁, u₃)\\
  \notag
          &+4λ₁\bigl[\dot{ℰ}₃(λ₀; u₁, u₁, u₁)+6\dot{ℰ}₂(λ₀; u₁, u₂)\bigr]\\
          &+6λ₁²\ddot{ℰ}₂(λ₀; u₁, u₁)+12λ₂\dot{ℰ}₂(λ₀; u₁, u₁)
\end{align}

On en déduit finalement le développement limité de l'énergie~\eqref{eq:20220121172753}.

\subsection{Développement limité de la hessienne}
\label{sec:20211115081016}

On cherche maintenant un développement limité de la hessienne de l'énergie. Les
fonctions test \(\hat{u}, \hat{v}∈U\) étant fixées, on applique la méthode du
§\ref{sec:20220107121442} à la fonction \(f(η)=F[η, λ₀+Λ(η)]\), avec
\begin{equation}
  F(η, λ)=ℰ_{,uu}[u^*(λ)+U(η), λ; \hat{u}, \hat{v}].
\end{equation}

On observe tout d'abord que \(F(0, λ)=ℰ₂(λ; \hat{u}, \hat{v})\), soit, en
dérivant par rapport à \(λ\)
\begin{equation}
  ∂_λ F(0, λ)=\dot{ℰ₂}(λ; \hat{u}, \hat{v})
  \quad\text{et}\quad
  ∂_{λλ}²F(0, λ)=\ddot{ℰ₂}(λ; \hat{u}, \hat{v}).
\end{equation}

On trouve de même successivement
\begin{align}
  ∂_ηF(η, λ)={}&ℰ_{,uuu}[u^\ast(λ)+U(η), λ; U'(η), \hat{u}, \hat{v}],\\
  \notag
  ∂_{ηη}²F(η, λ)={}&ℰ_{,uuuu}[u^\ast(λ)+U(η), λ; U'(η), U'(η), \hat{u}, \hat{v}]\\
               &+ℰ_{,uuu}[u^\ast(λ)+U(η), λ; U''(η), \hat{u}, \hat{v}],
\end{align}
soit, en \(η=0\)
\begin{align}
  ∂_ηF(0, λ)&=ℰ₃(λ; u₁, \hat{u}, \hat{v}),\\
  ∂_{ηη}²F(0, λ)&=ℰ₄(λ; u₁, u₁, \hat{u}, \hat{v})+2ℰ₃(λ; u₂, \hat{u}, \hat{v}),
\end{align}
et en dérivant cette fois par rapport à \(λ\)
\begin{equation}
  ∂_{ηλ}²F(0, λ)=\dot{ℰ₃}(λ; u₁, \hat{u}, \hat{v}).
\end{equation}

En insérant les résultats précédents dans les
expressions~\eqref{eq:20220107060454} et \eqref{eq:20220107124311}, on trouve
\begin{align}
  f'(0)={}&ℰ₃(λ₀; u₁, \hat{u}, \hat{v})
            +λ₁\dot{ℰ₂}(λ₀; \hat{u}, \hat{v}),\\
  \notag
  f''(0)={}&ℰ₄(λ₀; u₁, u₁, \hat{u}, \hat{v})
             +2λ₁\dot{ℰ₃}(λ₀; u₁, \hat{u}, \hat{v})
             +λ₁²\ddot{ℰ₂}(λ₀; \hat{u}, \hat{v})\\
          &+2ℰ₃(λ₀; u₂, \hat{u}, \hat{v})
            +2λ₂\dot{ℰ₂}(λ₀; \hat{u}, \hat{v}).
\end{align}



qui conduisent finalement au développement limité suivant, à l'ordre 2 en \(η\)
\begin{equation}
  \begin{aligned}[b]
    ℰ_{,uu}[u(η), λ(η); \hat{u}, \hat{v}]={}&ℰ₂(λ₀; \hat{u}, \hat{v})
    +η\bigl[ℰ₃(λ₀; u₁, \hat{u}, \hat{v})\\
    &+λ₁\dot{ℰ₂}(λ₀; \hat{u}, \hat{v})\bigr]
    +\tfrac12η²\bigl[ℰ₄(λ₀; u₁, u₁, \hat{u}, \hat{v})\\
    &+2λ₁\dot{ℰ₃}(λ₀; u₁, \hat{u}, \hat{v})
    +λ₁²\ddot{ℰ₂}(λ₀; \hat{u}, \hat{v})\\
    &+2ℰ₃(λ₀; u₂, \hat{u}, \hat{v})
    +2λ₂\dot{ℰ₂}^*(λ₀; \hat{u}, \hat{v})\bigr]
    +o(η²).
  \end{aligned}
\end{equation}

\subsection{Développement limité des valeurs propres et vecteurs propres de la Hessienne}

On cherche les vecteurs propres \(v∈V\) et valeurs propres \(α∈ℝ\) de la
Hessienne
\begin{equation}
  \label{eq:20211115082122}
  ℰ_{,uu}[u(η), λ(η)](v, \hat{u})=α〈v, \hat{u}〉\quad\text{pour tout}\quad\hat{u}∈V.
\end{equation}

On cherche les développements limités à l'ordre 1 en \(η\) de \(v\) et \(α\)
\begin{equation}
  \label{eq:20211115082037}
  v = v₀+η v₁+o(η)\quad\text{et}\quadα=α₀+ηα₁+o(η)
\end{equation}

Les développements limités~\eqref{eq:20211115082025} et
\eqref{eq:20211115082037} sont insérés dans le
problème~\eqref{eq:20211115082122}
\begin{equation}
  \begin{aligned}[b]
    ℰ_{,uu}[u(η), λ(η)](v, \hat{w})={}
    &ℰ₂^*(v₀, \hat{w})+η\bigl[ℰ₃^*(u₁, v₀, \hat{w})+λ₁\dot{ℰ₂}^*(v₀, \hat{w})\\
    &+ℰ₂^*(v₁, \hat{w})\bigr]+o(η)
  \end{aligned}
\end{equation}

\begin{equation}
  α〈 v, \hat{w}〉=α₀〈v₀, \hat{w}〉+η\bigl(α₁〈 v₀, \hat{w}〉+α₀〈 v₁, \hat{w}〉\bigr)+o(η).
\end{equation}

On obtient le problème variationnel d'ordre 0
\begin{equation}
  ℰ₂^*(v₀, \hat{w})=α₀〈v₀, \hat{w}〉\quad\text{pour tout}\quad\hat{w}∈V,
\end{equation}
qui montre que \(v₀\) est le vecteur propre de \(ℰ₂^*\) associé à la valeur
propre \(α₀\). Si \(α₀≠ 0\), \(ℰ₂^*\) étant positive par hypothèse, on a
nécessairement \(α₀>0\), et la valeur propre de la hessienne est positive.

On considère maintenant le cas où \(α₀\), c'est-à-dire que \(v₀∈\kerℰ₂^*\). En
prenant \(\hat{w}∈\kerℰ₂^*\), on obtient alors le problème variationnel d'ordre
1
\begin{equation}
  ℰ₃^*(u₁, v₀, \hat{w})+λ₁\dot{ℰ₂}^*(v₀, \hat{w})=α₁〈v₀, \hat{w}〉
  \quad\text{pour tout}\quad\hat{w}∈\kerℰ₂^*.
\end{equation}

En posant \(u₁=ξᵢaᵢ\) et \(v₀=χⱼaⱼ\), on obtient l'équation
\begin{equation}
  \bigl(ℰ_{3,ijk}^*ξₖ+λ₁\dot{ℰ}_{2, ij}^*\bigr)χⱼ=α₁χᵢ,
\end{equation}
qui est un problème aux valeurs propres pour la matrice symétrique
\((ℰ_{3, ijk}^*ξₖ+λ₁\dot{ℰ}_{2,ij}^*)_{1≤i, j≤m}\)

% \printbibliography
\end{document}

%%% Local Variables:
%%% coding: utf-8
%%% fill-column: 79
%%% mode: latex
%%% TeX-engine: xetex
%%% TeX-master: t
%%% End:
