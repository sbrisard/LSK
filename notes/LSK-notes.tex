\documentclass{article}
\usepackage[french]{babel}
\usepackage{amsmath,amssymb,latexsym,theorem}
\usepackage[tikz]{mdframed}

%%%%%%%%%% Start TeXmacs macros
\newcommand{\textdots}{...}
\newcommand{\tmaffiliation}[1]{\\ #1}
\newcommand{\tmem}[1]{{\em #1\/}}
\newcommand{\tmemail}[1]{\\ \textit{Email:} \texttt{#1}}
\newenvironment{proof}{\noindent\textbf{Proof\ }}{\hspace*{\fill}$\Box$\medskip}
\mdfsetup{linecolor=black,linewidth=0.5pt,skipabove=0.5em,skipbelow=0.5em,hidealllines=true,innerleftmargin=0pt,innerrightmargin=0pt,innertopmargin=0pt,innerbottommargin=0pt}
{\theorembodyfont{\rmfamily}\newtheorem{remark}{Remark}}
\newtheorem{theorem}{Theorem}
\newmdenv[hidealllines=false,innertopmargin=1ex,innerbottommargin=1ex,innerleftmargin=1ex,innerrightmargin=1ex]{tmframed}
%%%%%%%%%% End TeXmacs macros

%


\begin{document}

\title{Notes relatives à la méthode asymptotique de
Lyapunov--Schmidt--Koiter}

\author{
  Sébastien Brisard
  \tmaffiliation{Univ Gustave Eiffel, Ecole des Ponts, IFSTTAR, CNRS, Navier,
  F-77454 Marne-la-Vallée, France}
  \tmemail{sebastien.brisard@univ-eiffel.fr}
}

\maketitle

\begin{abstract}
  blabla
\end{abstract}

\section{Notations}

L'espace des champs cinématiquement admissibles est noté $U$. On
suppose qu'il a la structure d'espace vectoriel. L'énergie du système
est notée $ℰ (u, λ)$, où $λ$ désigne un
paramètre de chargement. Soit $u^{\ast} (λ)$ la branche
fondamentale. Par définition
\begin{equation}
  ℰ_{, u} [u^{\ast} (λ), λ ; \hat{u}] = 0 \quad
  \text{pour tout} \quad \hat{u}∈U.
\end{equation}
Il sera commode d'introduire les notations suivantes
\begin{equation}
  ℰ_2 (λ) =ℰ_{, u  u}  [u^{\ast} (λ),
  λ], \quad ℰ_3 (λ) =ℰ_{, u  u
   u} [u^{\ast} (λ), λ], \quad ℰ_4 (λ)
  =ℰ_{, u  u  u  u} [u^{\ast} (λ),
  λ] .
\end{equation}
Noter que $ℰ_2$, $ℰ_3$ et $ℰ_4$ sont des formes
bi-, tri- et quadri-linéaires, respectivement. L'application de ces formes
à des éléments de $U$ sera notée $ℰ_2 (λ ; u,
v)$, $ℰ_3 (λ ; u, v, w)$, etc.... La dérivée de ces
formes par rapport à $λ$ sera notée à l'aide d'un point
supérieur ($\dot{ℰ_2}$, $\dot{ℰ_3}$, ...).

On suppose que l'équilibre est stable pour des valeurs suffisamment
petites de $λ$. Plus précisément, on suppose que $ℰ_2
(λ)$ est définie positive pour tout $λ < λ_0$. Pour
$λ = λ_0$, la forme quadratique $ℰ_2 (λ_0)$ n'est
plus que positive. En notant $u_0 = u^{\ast} (λ_0)$ la position
d'équilibre obtenue pour la valeur critique $λ_0$ du paramètre
de chargement $λ$, on s'intéresse à toutes les courbes
d'équilibre qui passent par le point $(u_0, λ_0)$.

Noter que dans ce qui suit, on convient que les formes $ℰ_2$,
$ℰ_3$ et $ℰ_4$ sont implicitement évaluées en
$λ_0$ lorsque $λ$ n'est pas rappelé : ainsi, on notera
$ℰ_2 (\bullet, \bullet)$ plutôt que $ℰ_2 (λ_0 ;
\bullet, \bullet)$.

\section{Analyse de la branche fondamentale}

On s'intéresse dans ce paragraphe à la stabilité du point critique
$(u_0, λ_0) .$ Par hypothèse, $ℰ_2 (λ_0)$ est
positive, sans être définie positive~; soit $V$ son noyau, qui forme
un sous-espace vectoriel de $U$. On suppose que $V$ est de dimension finie $m
= \dim V$. Soit $(v_1, \ldots, v_m)$ une base orthonormée de ce noyau pour
le produit scalaire $\langle \bullet, \bullet \rangle$ (qui n'est pas
précisé pour le moment). On introduit le sous-espace
supplémentaire orthogonal $W$ de $V$ dans $U$
\begin{equation}
  U = V \overset{\perp}{\otimes} W.
\end{equation}
Pour étudier la stabilité de l'équilibre, on calcule l'énergie
dans un état $u_0 + ξ v + η w$ voisin du point d'équilibre $u_0$,
avec $ξ, η∈\mathbb{R}$ {\guillemotleft} petits {\guillemotright}, $v
\in V$ and $w∈W$. On obtient alors, à l'ordre 4 en $ξ$ et $η$
\begin{eqnarray}
  \Delta ℰ & = & ℰ (u_0 + ξ v + η w, λ_0)
  -ℰ (u_0, λ_0) \nonumber\\
  & = & \tfrac{1}{2} ℰ_2 (ξ v + η w, ξ v + η w) +
  \tfrac{1}{6} ℰ_3 (ξ v + η w, ξ v + η w, ξ v + η w)
  \nonumber\\
  &  &  + \tfrac{1}{24} ℰ_4 (ξ v + η w, ξ v + η
  w, ξ v + η w, ξ v + η w) +\mathcal{O} [(ξ^2 + η^2)^2],
\end{eqnarray}
où le terme linéaire a été omis puisque $u_0$ est un point
critique de l'énergie. En tenant compte de la multilinéarité et de
la symétrie des différentielles successives de l'énergie
$ℰ$, ainsi que du fait que $ℰ_2 (v, \bullet) = 0$ (puisque
$v∈V$), l'expression précédente s'écrit
\begin{eqnarray}
  \Delta ℰ & = & \tfrac{1}{2} η^2 ℰ_2 (w, w) +
  \tfrac{1}{6} ξ^3 ℰ_3 (v, v, v) + \tfrac{1}{2} ξ^2 η
  ℰ_3 (v, v, w) \nonumber\\
  &  & + \tfrac{1}{2} ξ η^2 ℰ_3 (v, w, w) + \tfrac{1}{6}
  η^3 ℰ_3 (w, w, w) \nonumber\\
  &  & + \tfrac{1}{24} ξ^4 ℰ_4 (v, v, v, v) + \tfrac{1}{6} ξ^3
  η ℰ_4 (v, v, v, w) \nonumber\\
  &  & + \tfrac{1}{4} ξ^2 η^2 ℰ_4 (v, v, w, w) + \tfrac{1}{6}
  ξ η^3 ℰ_4 (v, w, w, w) \nonumber\\
  &  & + \tfrac{1}{24} η^4 ℰ_4 (w, w, w, w) +\mathcal{O} [(ξ^2
  + η^2)^2],
\end{eqnarray}
où l'on convient que toutes les différentielles de $ℰ$ sont
évaluées au point d'équilibre $u_0$.

Pour que l'équilibre soit stable, il faut que cette expression soit
positive ou nulle pour tous $ξ$ et $η$ suffisamment petits. En prenant
tout d'abord $η = 0$, on obtient les conditions nécessaires
\begin{equation}
  \label{eq20211108164416} ℰ_3 (v, v, v) = 0 \quad \text{et} \quad
  ℰ_4 (v, v, v, v) \geq 0 \quad \text{pour tout} \quad v∈V.
\end{equation}
En d'autres termes, s'il existe $v∈V$ tel que $ℰ_3 (v, v, v)
\neq 0$ ou $ℰ_4 (v, v, v, v) < 0$, alors l'équilibre est
{\tmem{instable}}. Les conditions précédentes ne sont pas suffisantes
pour assurer la stabilité. En effet, supposant ces conditions remplies, on
prend maintenant $η = ξ^2$
\begin{equation}
  \Delta ℰ= \tfrac{1}{2} ξ^4  \left[ ℰ_2 (w, w)
  +ℰ_3 (v, v, w) + \tfrac{1}{12} ℰ_4 (v, v, v, v) \right]
  + o (ξ^4)
\end{equation}
et on obtient la condition nécessaire supplémentaire
\begin{equation}
  \label{eq20211109145356} ℰ_2 (w, w) +ℰ_3 (v, v, w) +
  \tfrac{1}{12} ℰ_4 (v, v, v, v) \geq 0,
\end{equation}
pour tous $v∈V$ et $w∈W$. Pour $v∈V$ fixé, l'expression
précédente est minimale lorsque $w$ satisfait le problème
variationnel
\begin{equation}
  \label{eq20211109145224} 2ℰ_2 (w, \hat{w}) +ℰ_3 (v, v,
  \hat{w}) = 0 \quad \text{pour tout} \quad \hat{w}∈W.
\end{equation}
Soit $w_{i  j}∈W$ l'unique solution du problème variationnel
suivant
\begin{equation}
  \label{eq:pbvar wij} ℰ_2 (w_{i  j}, \hat{w})
  +ℰ_3 (v_i, v_j, \hat{w}) = 0 \quad \text{pour tout} \quad \hat{w}
 ∈W.
\end{equation}
Alors, pour $v = ξ^i v_i$, la solution du problème
variationnel~\eqref{eq20211109145224} est $w = = \tfrac{1}{2} ξ^i ξ^j w_{i
 j}$. Pour cette valeur de $v$, la condition~\eqref{eq20211109145356}
s'écrit
\begin{equation}
  [ℰ_4 (v_i, v_j, v_k, v_l) - 3ℰ_2 (w_{i  j}, w_{k
   l})] ξ^i ξ^j ξ^k ξ^l \geq 0,
\end{equation}
pour tous $ξ_i, ξ_j, ξ_k, ξ_l∈\mathbb{R}$. On peut montrer que
l'inégalité stricte est une condition {\tmem{suffisante}} de
stabilité.

\section{Bifurcations}

On écrit toute courbe d'équilibre passant par le point $(u_0,
λ_0)$ sous la forme paramétrique suivante
\begin{eqnarray}
  λ & = & λ_0 + η λ_1 + \tfrac{1}{2} η^2 λ_2 +
  \tfrac{1}{6} η^3 λ_3 + \cdots,  \label{eq20211115075817}\\
  u & = & u^{\ast} (λ) + η u_1 + \tfrac{1}{2} η^2 u_2 +
  \tfrac{1}{6} η^3 u_3 + \cdots,  \label{eq20211115075835}
\end{eqnarray}
où $η$ est un paramètre, non précisé pour le moment. Noter
que, dans la représentation paramétrique de $u$, $u^{\ast}$ est
évalué en $λ$ et pas en $λ_0$.

Les coefficients $λ_k$ et $u_k$ des
développements~\eqref{eq20211115075817} et \eqref{eq20211115075835} sont
identifiés en écrivant que l'énergie est stationnaire le long de
la courbe d'équilibre, c'est-à-dire que le résidu $ℰ_{,
u}  [u (η), λ (η)]$ est nul. Le développement limité du
résidu est établi au voisinage de $η = 0$ dans
l'annexe~\ref{sec20211112182000} [voir Éq.~\eqref{eq20220107080901}]. En
écrivant que tous ses termes s'annulent, on trouve successivement, pour
tout $\hat{u}∈U$
\begin{equation}
  \label{eq20211112182917} ℰ_2 (λ_0 ; u_1, \hat{u}) = 0,
\end{equation}
\begin{equation}
  \label{eq:res2} ℰ_3 (λ_0 ; u_1, u_1, \hat{u}) + 2 λ_1
  \dot{ℰ_2} (λ_0 ; u_1, \hat{u}) +ℰ_2 (λ_0 ;
  u_2, \hat{u}) = 0,
\end{equation}
\begin{eqnarray}
  ℰ_4 (λ_0 ; u_1, u_1, u_1, \hat{u}) + 3ℰ_3
  (λ_0 ; u_1, u_2, \hat{u}) +ℰ_2 (λ_0 ; u_3, \hat{u}) &
  &  \nonumber\\
  + 3 λ_1  \dot{ℰ_3} (λ_0 ; u_1, u_1, \hat{u}) + 3
  λ_1  \dot{ℰ_2} (λ_0 ; u_2, \hat{u}) &  &  \nonumber\\
  + 3 λ_1^2  \ddot{ℰ_2} (λ_0 ; u_1, \hat{u}) + 3
  λ_2  \dot{ℰ_2} (λ_0 ; u_1, \hat{u}) & = & 0.
  \label{eq:res3}
\end{eqnarray}
On déduit de l'équation~\eqref{eq20211112182917} que $u_1∈V$. En
prenant la fonction test également dans $V$, on déduit de
l'équation~\eqref{eq:res2} que $u_1$ est solution du problème suivant
: trouver $u_1∈V$ tel que
\begin{equation}
  \label{eq:bifurcation 1a} \tfrac{1}{2} ℰ_3 (λ_0 ; u_1, u_1,
  \hat{v}) + λ_1  \dot{ℰ_2} (λ_0 ; u_1, \hat{v}) = 0,
\end{equation}
pour tout $\hat{v}∈V$. On remarque d'ores et déjà que si \ l'est
également. Il est commode de transformer l'équation de bifurcation
\eqref{eq:bifurcation 1a}, intrinsèque, en un système d'équations
scalaires. à cet effet, on décompose $u_1∈V$ dans la base
$(v_i)_{1 \leqslant i \leqslant m}$
\begin{equation}
  \label{eq:decomposition u1} u_1 = ξ_1^i v_i .
\end{equation}
En prenant $\hat{v} = v_i$, l'équation~\eqref{eq:bifurcation 1a}
s'écrit
\begin{equation}
  \label{eq:bifurcation 1b} \tfrac{1}{2} ℰ_3 (λ_0 ; v_i, v_j,
  v_k)  \hspace{0.17em} ξ_1^j ξ_1^k + λ_1  \dot{ℰ}_2
  (λ_0 ; v_i, v_j)  \hspace{0.17em} ξ_1^j = 0.
\end{equation}
On obtient ainsi un système de $m$ équations quadratiques à $(m +
1)$ inconnues, qui permet en général de déterminer les valeurs de
$λ_1$ et $u_1$ (voir discussion ci-après ***TODO -- Compléter
référence***).

Afin de déterminer les termes suivants du développement asymptotique
de la branche bifurquée, soit $λ_2$ et $u_2$, on introduit la
décomposition
\begin{equation}
  u_2 = ξ_2^i v_i + \tilde{u}_2,
\end{equation}
où $\tilde{u}_2∈W$ est la projection orthogonale de $u_2$ sur
$W$(notation provisoire). On a alors $ℰ_2 (u_2, \hat{u})
=ℰ_2 (\tilde{u}_2, \hat{u})$ et l'équation~\eqref{eq:res2}
s'écrit
\begin{equation}
  ℰ_3 (λ_0 ; u_1, u_1, \hat{u}) + 2 λ_1
  \dot{ℰ_2} (λ_0 ; u_1, \hat{u}) +ℰ_2 (λ_0 ;
  \tilde{u}_2, \hat{u}) = 0,
\end{equation}
pour tout $\hat{u}∈U$. En prenant cette fois-ci la fonction test dans
l'espace $W$, on obtient le problème variationnel suivant~: trouver
$\tilde{u}_2∈W$ tel que
\begin{equation}
  \label{eq20211210131623} ℰ_2 (λ_0 ; {\tilde{u}_2} , \hat{w})
  + ξ_1^i ξ_1^j ℰ_3 (λ_0 ; v_i, v_j, \hat{w}) + 2
  λ_1 ξ_1^i  \dot{ℰ_2} (λ_0 ; v_i, \hat{w}) = 0,
\end{equation}
pour tout $\hat{w}∈W$. Soient $w_i∈W$ les solutions des problèmes
variationnels suivants
\begin{equation}
  \label{eq:pbvar wi} ℰ_2 (λ_0 ; w_i, \hat{w}) + 2
  \dot{ℰ_2} (λ_0 ; v_i, \hat{w}) = 0,
\end{equation}
pour tout $\hat{w}∈W$. La solution du
problème~\eqref{eq20211210131623} s'obtient par simple combinaison
linéaire des $w_i$ et $w_{ij}$ [on rappelle que ces derniers sont
définis par le problème variationnel~\eqref{eq:pbvar wij}]
\begin{equation}
  \tilde{u}_2 = ξ_1^i ξ_1^j w_{i  j} + λ_1 ξ_1^i w_i,
\end{equation}
de sorte que
\begin{equation}
  \label{eq:decomposition u2} u_2 = ξ_2^i v_i + ξ_1^i ξ_1^j w_{i
   j} + λ_1 ξ_1^i w_i .
\end{equation}
En introduisant les expressions~\eqref{eq:decomposition u1} et
\eqref{eq:decomposition u2} dans l'équation~\eqref{eq:res3} et en prenant
de plus $\hat{u} = v_i$, on obtient alors les équations suivantes
\begin{eqnarray}
  3 [ℰ_3 (λ_0 ; v_i, v_j, v_k) ξ_1^k + λ_1
  \dot{ℰ}_2 (λ_0 ; v_i, v_j)] ξ_2^j + 3 λ_2
  \dot{ℰ}_2 (λ_0 ; v_i, v_j) ξ_1^j &  &  \nonumber\\
  + [ℰ_4 (λ_0 ; v_i, v_j, v_k, v_l) + 3ℰ_3
  (λ_0 ; v_i, v_j, w_{k  l})] ξ_1^j ξ_1^k ξ_1^l &  &
  \nonumber\\
  + 3 λ_1  [\dot{ℰ}_3 (λ_0 ; v_i, v_j, v_k)
  +ℰ_3 (λ_0 ; v_i, v_j, w_k) + \dot{ℰ_2} (λ_0
  ; v_i, w_{j  k})] ξ_1^j ξ_1^k &  &  \nonumber\\
  + 3 λ_1^2  [\ddot{ℰ}_2 (λ_0 ; v_i, v_j) +
  \dot{ℰ_2} (λ_0 ; v_i, w_j)] ξ_1^j & = & 0,
  \label{eq:bifurcation 2a}
\end{eqnarray}
qui permet en principe de déterminer $λ_2$ ainsi que les $ξ_2^i$.
On montre dans le paragraphe \ref{sec:Simplification des équations de
bifurcation} que les équations \eqref{eq:bifurcation 1b} et
\eqref{eq:bifurcation 2a} peuvent s'écrire sous la forme suivante
\begin{equation}
  \label{eq:bifurcation 1c} \tfrac{1}{2} E_{i  j  k}
  (λ_0) ξ_1^j ξ_1^k + λ_1 F_{i  j} (λ_0) ξ_1^j
  = 0,
\end{equation}
\begin{equation}
  \label{eq:bifurcation 2b} \tfrac{1}{3} E_{i  j  k
  l} (λ_0)  \hspace{0.17em} ξ_1^j ξ_1^k ξ_1^l + λ_2 F_{i
   j} (λ_0) ξ_1^j + A_{i  j} (λ_0) ξ_2^j +
  λ_1  \dot{A}_{i  j} (λ_0) ξ_1^j = 0,
\end{equation}
où les tenseurs $E_{i  j  k}$, $E_{i  j  k
 l}$, $F_{i  j}$ et $A_{i  j}$ sont définis comme
suit ***je ne suis pas sûr du terme faisant intervenir $\dot{A}_{i
 j} (λ_0)$***
\begin{equation}
  \label{eq:def Eijk} E_{i  j  k} (λ) =ℰ_3
  (λ ; v_i, v_j, v_k) +ℰ_2 (λ  ; v_i, w_{j  k})
  +ℰ_2 (λ ; v_j, w_{i  k}) +ℰ_2 (λ ;
  v_k, w_{i  j}),
\end{equation}
\begin{equation}
  \label{eq:def Eijkl} E_{i  j  k  l} (λ)
  =ℰ_4 (λ  ; v_i, v_j, v_k, v_l) +ℰ_3 (λ ;
  v_i, v_j, w_{k  l}) +ℰ_3 (λ ; v_i, v_k, w_{l
   j}) +ℰ_3 (λ ; v_i, v_l, w_{j  k}),
\end{equation}
\begin{equation}
  \label{eq:def Fij} F_{i  j} (λ) = \dot{ℰ}_2 (λ
  ; v_i, v_j) + \tfrac{1}{2}  [ℰ_2 (λ  ; v_i, w_j)
  +ℰ_2 (λ  ; v_j, w_i)],
\end{equation}
\begin{equation}
  \label{eq:def Aij} A_{i  j} (λ) = E_{i  j  k}
  (λ) ξ_1^k + λ_1 F_{i  j} (λ) .
\end{equation}
Noter que tous ces tenseurs sont {\tmem{symétriques}}. On remarque que,
puisque $ℰ_2 (λ_0 ; v_i, \bullet) = 0$, on a les
simplifications suivantes en $λ = λ_0$ : $E_{i  j
k} (λ_0) =ℰ_3 (λ_0 ; v_i, v_j, v_k)$ et $F_{i
j} (λ_0) = \dot{ℰ}_2 (λ_0 ; v_i, v_j)$.

\paragraph{Si la forme $ℰ_3 (λ_0)$ n'est pas nulle sur
$V$}L'équation \eqref{eq:bifurcation 1c} admet au plus $(2^m - 1)$ paires
de solutions réelles $(λ_1, u_1)$ et $(- λ_1, - u_1)$.

\begin{remark}
  Je ne sais pas démontrer ce résultat sur le nombre de solutions
  réelles.
\end{remark}

\paragraph{Si la forme $ℰ_3 (λ_0)$ est nulle sur
$V$}L'équation \eqref{eq:bifurcation 1a} conduit nécessairement à
$λ_1 = 0$, puisque $\dot{ℰ}_2 (λ_0)$ est définie
négative. Dès lors, l'équation \eqref{eq:bifurcation 2b}
s'écrit\marginpar{Expliquer pourquoi la forme quadratique
$\dot{ℰ}_2 (λ_0)$ est bien définie négative}
\begin{equation}
  \tfrac{1}{3} E_{i  j  k  l} (λ_0)
  \hspace{0.17em} ξ_1^j ξ_1^k ξ_1^l + λ_2 F_{i  j}
  (λ_0) ξ_1^j = 0.
\end{equation}
Cette équation admet cette fois au plus $\frac{3^m - 1}{2}$ paires de
solutions réelles $(λ_2, u_1)$ et $(- λ_2, - u_1)$.

\begin{remark}
  Je ne sais pas non plus démontrer ce résultat sur le nombre de
  solutions réelles.
\end{remark}

\begin{tmframed}
  \paragraph{Note du 29/04/2022}J'ai relu tous les calculs précédents.
  Il reste à reprendre les calculs des développements asymptotiques de
  l'énergie et de sa hessienne, pour tenir compte en particulier des
  factorielles introduites maintenant dans les développements
  asymptotiques. Il faudrait également introduire les tenseurs
  précédents dans les expressions de l'énergie et de sa hessienne.
\end{tmframed}

Le développement limité suivant de l'énergie le long de la branche
bifurquée est établi dans l'annexe~\ref{sec:DL energie}
\begin{eqnarray}
  ℰ [u (η), λ (η)] & = & ℰ [u^{\ast} [λ
  (η)], λ (η)] + \tfrac{1}{6} λ_1 η^3 F_{i  j}
  (λ_0) ξ_1^i ξ_1^j \nonumber\\
  &  & \tfrac{1}{24} η^4  \{ E_{i  j  k  l}
  (λ_0) ξ_1^i ξ_1^j ξ_1^k ξ_1^l + 4 λ_1  \dot{E}_{i
   j  k} (λ_0) ξ_1^i ξ_1^j ξ_1^k
  \nonumber\\
  &  & +  6 [λ_1^2  \dot{F}_{i  j}
  (λ_0) + λ_2 F_{i  j} (λ_0)] ξ_1^i ξ_1^j \}
  + o (η^4) .  \label{eq:DL energie}
\end{eqnarray}
Si $λ_1 \neq 0$, le premier terme non-nul du développement
limité précédent est d'ordre 3
\begin{equation}
  ℰ [u (η), λ (η)] =ℰ (u^{\ast} [λ
  (η)], λ (η)) + \tfrac{1}{6} λ_1 η^3 F_{i  j}
  (λ_0) ξ_1^i ξ_1^j + o (η^3),
\end{equation}
tandis que si $λ_1 = 0$, le premier terme est d'ordre 4
\begin{equation}
  ℰ [u (η), λ (η)] =ℰ (u^{\ast} [λ
  (η)], λ (η)) + \tfrac{1}{4} λ_2 η^4 F_{i  j}
  (λ_0) ξ_1^i ξ_1^j + o (η^4) .
\end{equation}
\begin{center}
  ***
\end{center}

Pour analyser la stabilité de la branche bifurquée ainsi trouvée,
il faut déterminer le signe de la hessienne de l'énergie. On peut
d'ores et déjà remarquer que, sur la branche fondamentale ($u_1 = u_2
= 0$), en prenant $η = λ - λ_0$ ($λ_1 = 1$)
\begin{equation}
  ℰ_2 (λ ; \hat{u}, \hat{v}) =ℰ_2 (λ_0 ;
  \hat{u}, \hat{v}) + (λ - λ_0)  \dot{ℰ}_2 (λ_0 ;
  \hat{u}, \hat{v}) + o (λ - λ_0) .
\end{equation}
Dans ce qui suit, on supposera que $\dot{ℰ}_2 (λ_0) \neq 0$.
Pour $\hat{v}∈V$, l'égalité précédente s'écrit
\begin{equation}
  ℰ_2 (λ_0 ; \hat{v}, \hat{v}) = (λ - λ_0)
  \dot{ℰ}_2 (\hat{v}, \hat{v}) + o (λ - λ_0) .
\end{equation}
Comme la branche fondamentale est stable pour $λ < λ_0$, on doit
avoir $\dot{ℰ}_2 (λ_0 ; \hat{v}, \hat{v}) < 0$. La forme
quadratique $\dot{ℰ}_2 (λ_0)$ est donc définie
négative sur $V$. Le développement limité de la hessienne de
l'énergie le long de la branche bifurquée est établi dans
l'annexe~\ref{sec:DL hessienne}. Pour tous $\hat{u}, \hat{v}∈U$, on trouve
\begin{eqnarray}
  ℰ_{, u  u} [u (η), λ (η) ; \hat{u}, \hat{v}] &
  = & ℰ_2 (λ_0 ; \hat{u}, \hat{v}) + η [ℰ_3
  (λ_0 ; u_1, \hat{u}, \hat{v})   + λ_1
  \dot{ℰ_2} (λ_0 ; \hat{u}, \hat{v})] \nonumber\\
  &  &  + \tfrac{1}{2} η^2  [ℰ_4 (λ_0 ; u_1, u_1,
  \hat{u}, \hat{v})  +ℰ_3 (λ_0 ; u_2, \hat{u},
  \hat{v}) + λ_2  \dot{ℰ_2} (λ_0 ; \hat{u}, \hat{v})
  \nonumber\\
  &  &  + 2 λ_1  \dot{ℰ_3} (λ_0 ; u_1,
  \hat{u}, \hat{v}) + λ_1^2  \ddot{ℰ_2} (λ_0 ; \hat{u},
  \hat{v}) ] + o (η^2) .  \label{eq:DL hessienne}
\end{eqnarray}
Pour une analyse de stabilité, on doit prendre $\hat{u} = \hat{v}$, soit
\begin{eqnarray}
  ℰ_{, u  u} [u (η), λ (η) ; \hat{u}, \hat{u}] &
  = & ℰ_2 (λ_0 ; \hat{u}, \hat{u}) + η [ℰ_3
  (λ_0 ; u_1, \hat{u}, \hat{u}) + λ_1  \dot{ℰ}_2
  (λ_0 ; \hat{u}, \hat{u})] \nonumber\\
  &  & + \tfrac{1}{2} η^2  [ℰ_4 (λ_0 ; u_1, u_1, \hat{u},
  \hat{u}) +ℰ_3 (λ_0 ; u_2, \hat{u}, \hat{u}) + λ_2
  \dot{ℰ}_2 (λ_0 ; \hat{u}, \hat{u})  \nonumber\\
  &  & +  2 λ_1  \dot{ℰ}_3 (λ_0 ; u_1,
  \hat{u}, \hat{u}) + λ_1^2  \ddot{ℰ}_2 (λ_0 ; \hat{u},
  \hat{u})] + o (η^2) .  \label{eq:DL hessienne diag}
\end{eqnarray}
On peut décomposer le vecteur $\hat{u}∈U$ de fa{\c c}on unique sous la
forme $\hat{u} = \hat{v} + \hat{w}$, avec $\hat{v}∈V$ et $\hat{w}∈W$.
Le terme constant du développement précédent vaut alors
$ℰ_2 (λ_0 ; \hat{w}, \hat{w})$. Si $\hat{w} \neq 0$, alors ce
terme constant est strictement positif, puisque la hessienne est définie
positive sur $W$ en $λ = λ_0$. Au voisinage du point de
bifurcation, la hessienne sur la branche bifurquée est donc positive pour
tout $\hat{u}∈U$ ayant une composante dans $W$. Il suffit donc
d'étudier le signe de la hessienne sur la branche bifurquée pour
$\hat{u}∈V$, soit $\hat{u} = \hat{ξ}^i v_i$. L'expression~\eqref{eq:DL
hessienne diag} se simplifie alors sous la forme suivante

Compte-tenu de l'expression~\eqref{eq:decomposition u2}
\begin{eqnarray}
  ℰ_{, u  u} [u (η), λ (η) ; \hat{u}, \hat{u}] &
  = & η [ℰ_3 (λ_0 ; v_i, v_j, v_k) ξ_1^k + λ_1
  \dot{ℰ}_2 (λ_0 ; v_i, v_j)]  \hat{ξ}^i  \hat{ξ}^j
  \nonumber\\
  &  & + \tfrac{1}{2} η^2  [ℰ_4 (λ_0 ; v_i, v_j, v_k, v_l)
  ξ_1^k ξ_1^l +ℰ_3 (λ_0 ; v_i, v_j, v_k) ξ_2^k
   \nonumber\\
  &  & +ℰ_3 (λ_0 ; v_i, v_j, w_{k  l}) ξ_1^k
  ξ_1^l + λ_1 ℰ_3 (λ_0 ; v_i, v_j, w_k) ξ_1^k +
  λ_2  \dot{ℰ}_2 (λ_0 ; v_i, v_j) \nonumber\\
  &  &  + 2 λ_1  \dot{ℰ}_3 (λ_0 ; v_i, v_j,
  v_k) ξ_1^k + λ_1^2  \ddot{ℰ}_2 (λ_0 ; v_i, v_j)]
  \hat{ξ}^i  \hat{ξ}^j + o (η^2) \nonumber
\end{eqnarray}
Si $λ_1 \neq 0$, il suffit d'étudier le signe de la forme
quadratique $[E_{i  j  k} (λ_0) ξ_1^k + λ_1 F_{i
 j} (λ_0)] .$ Si $λ_1 = 0$ et que $ℰ_3
(λ_0) = 0$ sur $V$, alors le développement limité
précédent s'écrit
\begin{eqnarray}
  ℰ_{, u  u} [u (η), λ (η) ; \hat{u}, \hat{u}] &
  = & \tfrac{1}{2} η^2  \{ [ℰ_4 (λ_0 ; v_i, v_j, v_k, v_l)
   +ℰ_3 (λ_0 ; v_i, v_j, w_{k  l})] ξ_1^k
  ξ_1^l \nonumber\\
  &  & +  λ_2  \dot{ℰ}_2 (λ_0 ; v_i, v_j) \}
  \hat{ξ}^i  \hat{ξ}^j + o (η^2) \nonumber
\end{eqnarray}
\begin{tmframed}
  12/05/2022 Relecture jusqu'à l'égalité précédente. Je
  suis un peu surpris, car je m'attendais à un terme en $3ℰ_3
  (λ_0 ; v_i, v_j, w_{k  l})${\textdots}
\end{tmframed}

Compte-tenu de la relation~\eqref{eq20211112183220}, on trouve pour $\hat{v} =
u_1$ ($\hat{ξ}^i = ξ_1^i$)
\begin{equation}
  ℰ_{, u  u} [u (η), λ (η) ; u_1, u_1] = -
  λ_1 η \dot{ℰ}_2 (λ_0 ; u_1, u_1) + o (η) .
\end{equation}
Si $λ_1 \neq 0$, l'expression précédente peut également
s'écrire
\begin{equation}
  ℰ_{, u  u} [u (η), λ (η) ; u_1, u_1] = -
  (λ - λ_0)  \dot{ℰ}_2 (λ_0 ; u_1, u_1) + o
  (λ - λ_0),
\end{equation}
qui est négative pour $λ < λ_0$: la branche bifurquée est
instable sous la charge critique. Il reste alors à étudier le signe de
la hessienne de la branche bifurquée au-delà de la charge critique
($λ > λ_0$).

\section{Cas d'un mode de flambement simple ($m = 1$)}

Lorsque $m = \dim V = 1$, la base $v_1, \ldots, v_m$ est réduite au seul
vecteur $v_1$ et $u_1$ est parallèle à ce vecteur. Comme $\lVert u_1
\rVert = 1$, on a donc nécessairement $u_1 = v_1$ (quitte à changer
$η$ en $- η$). L'équation de bifurcation~\eqref{eq20220216140121}
s'écrit alors
\begin{equation}
  \label{eq20220203144712} ℰ_{1  1  1} (λ_0) +
  2 λ_1  \dot{ℰ}_{1  1} (λ_0) = 0, \quad
  \text{soit} \quad λ_1 = - \frac{ℰ_{1  1  1}
  (λ_0)}{2 \dot{ℰ}_{1  1} (λ_0)},
\end{equation}
où on remarque que le quotient a un sens, puisque $\dot{ℰ_2}
(λ_0)$ est définie négative sur $V$. On trouve donc les
développements limités
\begin{equation}
  λ = λ_0 + λ_1 η + o (η)  \quad \text{et} \quad u =
  u^{\ast} (λ) + η v_1 + o (η),
\end{equation}
soit finalement, en éliminant $η$
\begin{equation}
  λ = λ_0 - \frac{ξ ℰ_{1  1  1}
  (λ_0)}{2 \dot{ℰ}_{1  1} (λ_0)} + o (ξ),
  \quad \text{avec} \quad ξ = \langle u (λ) - u^{\ast} (λ), v_1
  \rangle .
\end{equation}
Pour déterminer la stabilité de la branche bifurquée, on calcule
la hessienne en $(v_1, v_1)$. L'équation~\eqref{eq20220203144500}
s'écrit
\begin{equation}
  ℰ_{, u  u} [u (η), λ (η) ; v_1, v_1] = η
  [ℰ_{1  1  1} (λ_0) + λ_1
  \dot{ℰ}_{1  1} (λ_0)] + o (η),
\end{equation}
soit, en substituant l'équation~\eqref{eq20220203144712}
\begin{equation}
  ℰ_{, u  u} [u (η), λ (η) ; v_1, v_1] = -
  λ_1 η \dot{ℰ}_{1  1} (λ_0) + o (η) .
\end{equation}
Ce développement ne permet de conclure que si le terme linéaire est
non-nul, soit $ℰ_{1  1  1} (λ_0) \neq 0$ [voir
Éq.~\eqref{eq20220203144712}]. Dans ce cas, le développement
asymptotique précédent s'écrit également
\begin{equation}
  ℰ_{, u  u} [u (η), λ (η) ; v_1, v_1] = -
  (λ - λ_0)  \dot{ℰ}_{1  1} (λ_0) + o
  (λ - λ_0) .
\end{equation}
Comme $\dot{ℰ}_2 (λ_0)$ est définie négative, la
branche bifurquée est {\tmem{instable}} pour $λ < λ_0$ et
{\tmem{stable}} pour $λ > λ_0$ lorsque $ℰ_{1  1
 1} (λ_0) \neq 0$.

Supposons maintenant que $ℰ_{1  1  1} (λ_0) =
0$~; alors $λ_1 = 0$ et il faut calculer au moins un terme
supplémentaire dans le développement limité de la Hessienne.
L'équation de bifurcation~\eqref{eq20220216141706} s'écrit
\begin{equation}
  \label{eq20220217164528} ℰ_{1  1  1  1}
  (λ_0) + 6ℰ_3 (λ_0 ; v_1, v_1, u_2) + 6 λ_2
  \dot{ℰ}_{1  1} (λ_0) = 0.
\end{equation}
En introduisant le développement~\eqref{eq20220124135324} de $u_2$ et en
utilisant le problème variationnel~\eqref{eq20211221155859}
\begin{equation}
  u_2 = ξ_2 v_1 + w_{1  1} + λ_1 w_1,
\end{equation}
donc
\begin{equation}
  ℰ_3 (λ_0 ; v_1, v_1, u_2) =ℰ_3 (λ_0 ; v_1,
  v_1, w_{1  1}) = - 2ℰ_2 (λ_0 ; w_{11}, w_{11})
\end{equation}
soit finalement
\begin{equation} λ_2 = - \frac{ℰ_{1  1  1  1}
   (λ_0) - 12ℰ_2 (λ_0 ; w_{11}, w_{11})}{6
   \dot{ℰ}_{1  1} (λ_0)}, \end{equation}
le quotient ayant une nouvelle fois un sens. Le développement
asymptotique~\eqref{eq20211115082025} de la Hessienne s'écrit alors, en
tenant compte de l'Éq.~\eqref{eq20220217164528}
\begin{eqnarray}
  ℰ_{, u  u} [u (η), λ (η) ; v_1, v_1] & = &
  \tfrac{1}{2} η^2  [ℰ_{1  1  1  1}
  (λ_0) + 2ℰ_3 (λ_0 ; v_1, v_1, u_2) + 2 λ_2
  \dot{ℰ}_{1  1} (λ_0)] + o (η^2) \nonumber\\
  & = & \tfrac{5}{12} η^2 ℰ_{1  1  1  1}
  (λ_0) + o (η^2) .
\end{eqnarray}

\section{Propriétés des formes bilinéaires symétriques,
positives}

Dans ce qui suit, $\mathcal{B}$ désigne une forme bilinéaire
symétrique et positive sur l'espace vectoriel $U$. On définit son
noyau $\ker \mathcal{B}$ de la fa{\c c}on suivante
\begin{equation}
  \ker \mathcal{B}= \{u∈U, \mathcal{B}(u, u) = 0\} .
\end{equation}
\begin{theorem}
  Le noyau d'une forme bilinéaire, symétrique et positive est un
  sous-espace vectoriel.
\end{theorem}

\begin{proof}
  Soient $u, v∈\ker \mathcal{B}$, $α∈\mathbb{R}$ et $w = u +
  α v$. Montrons que $w∈\ker \mathcal{B}$. Il suffit d'évaluer
  $\mathcal{B} (w, w)$
  \begin{equation}
    \mathcal{B} (w, w) =\mathcal{B} (u + α v, u + α v) =\mathcal{B}
    (u, u) + 2 α \mathcal{B} (u, v) + α^2 \mathcal{B} (v, v),
  \end{equation}
  où l'on a tenu compte de la symétrie de $\mathcal{B}$ pour
  écrire que $\mathcal{B} (u, v) =\mathcal{B} (v, u)$. Comme $u, v \in
  \ker \mathcal{B}$, le premier et le dernier terme sont nuls, soit
  $\mathcal{B} (w, w) = 2 α \mathcal{B} (u, v)$. La forme bilinéaire
  étant positive, cette grandeur est positive, {\tmem{quelle que soit la
  valeur de $α∈\mathbb{R}$}}. On en déduit donc que $\mathcal{B}
  (u, v) = 0$, puis que $\mathcal{B} (w, w) = 0$ et donc que $w∈\ker
  \mathcal{B}.$
\end{proof}

\begin{theorem}
  Soit $u∈V$. Alors
  \begin{equation}
    u∈\ker \mathcal{B} \quad \text{ssi} \quad \text{pour tout } v∈V,
    \mathcal{B} (u, v) = 0.
  \end{equation}
\end{theorem}

\begin{proof}
  Soient $u∈\ker \mathcal{B}$, $v∈V$ et $α∈\mathbb{R}$. Comme
  précédemment, on écrit que $\mathcal{B} (w, w) \geq 0$, avec $w
  = α u + v$
  \begin{equation}
    \mathcal{B} (w, w) = 2 α \mathcal{B} (u, v) +\mathcal{B} (v, v) \geq
    0,
  \end{equation}
  où l'on a tenu compte de ce que $\mathcal{B} (u, u) = 0$. L'expression
  précédente, affine en $α$, a un signe constant. Le terme
  linéaire en $α$ est donc nul, soit $\mathcal{B} (u, v) = 0$.
  Réciproquement, si $\mathcal{B} (u, v) = 0$ pour tout $v∈V$, alors
  $\mathcal{B} (u, u) = 0$ (en prenant $v = u$).
\end{proof}

\section{Développements limités le long d'une branche bifurquée du
diagramme d'équilibre}

\subsection{Principe du calcul}\label{sec20220107121442}

On pose dans ce qui suit
\begin{eqnarray}
  λ (η) & = & λ (η) - λ_0 = η λ_1 +
  \tfrac{1}{2} η^2 λ_2 + \tfrac{1}{6} η^3 λ_3 + \cdots,
  \label{eq20211112155446}\\
  U (η) & = & u (η) - u^{\ast} [λ (η)] = η u_1 +
  \tfrac{1}{2} η^2 u_2 + \tfrac{1}{6} η^3 u_3 + \cdots .
  \label{eq20211112113028}
\end{eqnarray}
On considère une fonctionnelle $\mathcal{F}$ de $u$ et $λ$~:
$\mathcal{F} (u, λ)$. Cette fonctionnelle est évaluée le long de
la branche bifurquée. En d'autres termes, on considère
\begin{equation}
  f (η) = F \{ u^{\ast} [λ_0 + λ (η)] + U (η), λ_0
  + λ (η) \} .
\end{equation}
On souhaite établir un développement limité de $f$ au voisinage de
$η = 0$, ce qui conduit à calculer les dérivées successives de
$f$ en $η = 0$, puisque
\begin{equation}
  f (η) = f (0) + η f' (0) + \tfrac{1}{2} η^2 f'' (0) + \cdots
\end{equation}
Pour calculer ces dérivées, il sera commode d'introduire la fonction
auxiliaire $F$
\begin{equation}
  F (η, λ) =\mathcal{F} [u^{\ast} (λ) + U (η), λ],
\end{equation}
dans laquelle les variables $λ$ et $η$ sont provisoirement
considérées comme indépendantes. On a
\begin{equation}
  f (η) = F [η, λ_0 + λ (η)],
\end{equation}
d'où l'on déduit successivement que
\begin{equation}
  \label{eq20211112162417} f' (η) = \partial_{η} F + λ'
  \partial_{λ} F,
\end{equation}
\begin{equation}
  \label{eq20211112165810} f'' (η) = \partial_{η  η}^2 F + 2
  λ' \partial_{η  λ}^2 {F + λ'}^2
  \partial_{λ  λ}^2 F + λ'' \partial_{λ} F,
\end{equation}
\begin{eqnarray}
  \label{eq20211112173223} f''' (η) & = & \partial_{η  η
   η}^3 F + 3 λ' \partial_{η  η
  λ}^3 {F + 3 λ'}^2 \partial_{η  λ
  λ}^3 {F + λ'}^3 \partial_{λ  λ
  λ}^3 F + 3 λ'' \partial_{η  λ}^2 F + 3 λ'
  λ'' \partial_{λ  λ}^2 F \nonumber\\
  &  &  + λ''' \partial_{λ} F
\end{eqnarray}
\begin{eqnarray}
  f'''' (η) & = & \partial_{η  η  η
  η}^4 F + 4 λ' \partial_{η  η  η
  λ}^4 {F + 6 λ'}^2 \partial_{η  η  λ
   λ}^4 {F + 4 λ'}^3 \partial_{η  λ
   λ  λ}^4 {F + λ'}^4 \partial_{λ
   λ  λ  λ}^4 F + 6 λ''
  \partial_{η  η  λ}^3 F \nonumber\\
  &  & + 12 λ' λ'' \partial_{η  λ
  λ}^3 {F + 6 λ'}^2 λ'' \partial_{λ  λ
   λ}^3 F + 4 λ''' \partial_{η  λ}^2 F +
  \left( {3 λ''}^2 + 4 λ' λ''' \right) \partial_{λ
   λ}^2 F \\
  &  & + λ'''' \partial_{λ} F
\end{eqnarray}
où $λ$ et ses dérivées sont évaluées en $η$,
tandis que $F$ et ses dérivées partielles sont évaluées en
$[η, λ_0 + λ (η)]$. En $η = 0$, les relations
précédentes s'écrivent
\begin{equation}
  \label{eq20220107060454} f' (0) = \partial_{η} F + λ_1
  \partial_{λ} F,
\end{equation}
\begin{equation}
  \label{eq20220107124311} f'' (0) = \partial_{η  η}^2 F + 2
  λ_1 \partial_{η  λ}^2 F + λ_2
  \partial_{λ} F + λ_1^2 \partial_{λ  λ}^2 F,
\end{equation}
\begin{eqnarray}
  f''' (0) & = & \partial_{η  η  η}^3 F + 3 λ_1
  \partial_{η  η  λ}^3 F + 3 λ_1^2
  \partial_{η  λ  λ}^3 F + λ_1^3
  \partial_{λ  λ  λ}^3 F + 3 λ_2
  \partial_{η  λ}^2 F + 3 λ_1 λ_2
  \partial_{λ  λ}^2 F \nonumber\\
  &  &  + λ_3 \partial_{λ} F,  \label{eq20220107060500}
\end{eqnarray}
\begin{eqnarray}
  f'''' (0) & = & \partial_{η  η  η  η}^4
  F + 4 λ_1 \partial_{η  η  η
  λ}^4 F + 6 λ_1^2 \partial_{η  η  λ
   λ}^4 F + 4 λ_1^3 \partial_{η  λ
   λ  λ}^4 F + λ_1^4 \partial_{λ
   λ  λ  λ}^4 F + 6 λ_2
  \partial_{η  η  λ}^3 F \nonumber\\
  &  & + 12 λ_1 λ_2 \partial_{η  λ
  λ}^3 F + 6 λ_1^2 λ_2 \partial_{λ  λ
   λ}^3 F + 4 λ_3 \partial_{η  λ}^2 F +
  (3 λ_2^2 + 4 λ_1 λ_3) \partial_{λ
  λ}^2 F \nonumber\\
  &  & + λ_4 \partial_{λ} F,
\end{eqnarray}
où $F$ et ses dérivées sont évaluées en $(0, λ_0)$.

\subsection{Développement limité du
résidu}\label{sec20211112182000}

On cherche un développement limité du résidu (c'est-à-dire de
la première variation de l'énergie). La fonction test $\hat{u}∈U$
étant fixée, la méthode précédente est donc appliquée
avec
\begin{equation}
  \label{eq20220107054629} f (η) =ℰ_{, u} [u (η), λ
  (η) ; \hat{u}] \quad \text{et} \quad F (η, λ) =ℰ_{, u}
  [u^{\ast} (λ) + U (η), λ ; \hat{u}] .
\end{equation}
On remarque tout d'abord que $F (0, λ) =ℰ_{, u} [u^{\ast}
(λ), λ ; \hat{u}] = 0$, puisque $u^{\ast} (λ)$ est un point
d'équilibre. En dérivant par rapport à $λ$, on obtient
\begin{equation}
  \label{eq20211112164240} \frac{\partial^k F}{\partial λ^k} (0,
  λ) = 0.
\end{equation}
En dérivant par rapport à $η$ l'expression~\eqref{eq20220107054629}
de $F$, on obtient successivement
\begin{equation}
  \partial_{η} F (η, λ) =ℰ_{, u  u} [u^{\ast}
  (λ) + U (η), λ ; U' (η), \hat{u}],
\end{equation}
\begin{eqnarray}
  \partial_{η  η}^2 F (η, λ) & = & ℰ_{, u
   u  u} [u^{\ast} (λ) + U (η), λ ; U' (η),
  U' (η), \hat{u}] \nonumber\\
  &  &  +ℰ_{, u  u} [u^{\ast} (λ) + U
  (η), λ ; U'' (η), \hat{u}],
\end{eqnarray}
\begin{eqnarray}
  \partial_{η  η  η}^3 F (η, λ) & = &
  ℰ_{, u  u  u  u} [u^{\ast} (λ) + U
  (η), λ ; U' (η), U' (η), U' (η), \hat{u}] \nonumber\\
  &  &  + 3ℰ_{, u  u  u} [u^{\ast}
  (λ) + U (η), λ ; U' (η), U'' (η), \hat{u}] \nonumber\\
  &  &  +ℰ_{, u  u} [u^{\ast} (λ) + U
  (η), λ ; U''' (η), \hat{u}],
\end{eqnarray}
soit, en $η = 0$
\begin{equation}
  \partial_{η} F (0, λ) =ℰ_2 (λ ; u_1, \hat{u}),
\end{equation}
\begin{equation}
  \partial_{η  η}^2 F (0, λ) =ℰ_3 (λ ;
  u_1, u_1, \hat{u}) +ℰ_2 (λ ; u_2, \hat{u}),
\end{equation}
\begin{equation}
  \partial_{η  η  η}^3 F (0, λ) =ℰ_4
  (λ ; u_1, u_1, u_1, \hat{u}) + 3ℰ_3 (λ ; u_1, u_2,
  \hat{u}) +ℰ_2 (λ ; u_3, \hat{u}) .
\end{equation}
Les dérivées croisées de $F$ en $(0, λ)$ s'obtiennent par
simple dérivation des relations précédentes par rapport à
$λ$
\begin{equation}
  \partial_{η  λ}^2 F (0, λ) = \dot{ℰ_2}
  (λ ; u_1, \hat{u}),
\end{equation}
\begin{equation}
  \partial_{η  η  λ}^3 F (0, λ) =
  \dot{ℰ_3} (λ ; u_1, u_1, \hat{u}) + \dot{ℰ_2}
  (λ ; u_2, \hat{u}),
\end{equation}
\begin{equation}
  \partial_{η  λ  λ}^3 F (0, λ) =
  \ddot{ℰ_2} (λ ; u_1, \hat{u}) .
\end{equation}
En insérant les résultats précédentes dans les relations
générales~\eqref{eq20220107060454}--\eqref{eq20220107060500}, on
trouve alors les expressions suivantes des dérivées successives de $f$
en $η = 0$
\begin{equation}
  f' (0) =ℰ_2 (λ_0 ; u_1, \hat{u}),
\end{equation}
\begin{equation}
  f'' (0) =ℰ_3 (λ_0 ; u_1, u_1, \hat{u}) + 2 λ_1
  \dot{ℰ_2} (λ_0 ; u_1, \hat{u}) +ℰ_2 (λ_0 ;
  u_2, \hat{u}),
\end{equation}
\begin{eqnarray}
  f''' (0) & = & ℰ_4 (λ_0 ; u_1, u_1, u_1, \hat{u}) +
  3ℰ_3 (λ_0 ; u_1, u_2, \hat{u}) +ℰ_2 (λ_0 ;
  u_3, \hat{u}) \nonumber\\
  &  &  + 3 λ_1  [\dot{ℰ_3} (λ_0 ; u_1, u_1,
  \hat{u}) + \dot{ℰ_2} (λ_0 ; u_2, \hat{u})] \nonumber\\
  &  &  + 3 λ_1^2  \ddot{ℰ_2} (λ_0 ; u_1,
  \hat{u}) + 3 λ_2  \dot{ℰ_2} (λ_0 ; u_1, \hat{u}) .
\end{eqnarray}
On en déduit finalement le développement limité à l'ordre 3 en
$η$ du résidu
\begin{eqnarray}
  ℰ_{, u} [u (η), λ (η)] & = & η ℰ_2
  (λ_0 ; u_1, \hat{u}) \nonumber\\
  &  &  + \tfrac{1}{2} η^2  [ℰ_3 (λ_0 ; u_1, u_1,
  \hat{u}) + 2 λ_1  \dot{ℰ_2} (λ_0 ; u_1, \hat{u})
  +ℰ_2 (λ_0 ; u_2, \hat{u})] \nonumber\\
  &  &  + \tfrac{1}{6} η^3  \{ ℰ_4 (λ_0 ; u_1,
  u_1, u_1, \hat{u}) + 3ℰ_3 (λ_0 ; u_1, u_2, \hat{u})
   +ℰ_2 (λ_0 ; u_3, \hat{u}) \nonumber\\
  &  &  + 3 λ_1  [\dot{ℰ_3} (λ_0 ; u_1, u_1,
  \hat{u}) + \dot{ℰ_2} (λ_0 ; u_2, \hat{u})] + 3 λ_1^2
  \ddot{ℰ_2} (λ_0 ; u_1, \hat{u}) \nonumber\\
  &  &   + 3 λ_2  \dot{ℰ_2} (λ_0 ;
  u_1, \hat{u}) \} + o (η^3) .  \label{eq20220107080901}
\end{eqnarray}
\subsection{Développement limité de l'énergie}\label{sec:DL
energie}

On s'intéresse ici à l'écart d'énergie, pour un chargement
$λ$ donné, entre la branche bifurquée et la branche
fondamentale, soit
\begin{equation} F (η, λ) =ℰ [u^{\ast} (λ) + U (η), λ]
   -ℰ [u^{\ast} (λ), λ] \end{equation}
et
\begin{equation} f (η) = F [η, λ_0 + λ (η)] . \end{equation}
On observe tout d'abord que $F (0, λ) = 0$ pour tout $λ$, donc
\begin{equation} \frac{\partial^k F}{\partial λ^k} (0, λ) = 0 \quad (k \geq 0),
\end{equation}
tandis que les dérivées de $F$ par rapport à $η$ s'écrivent
\begin{equation} \partial_{η} F (η, λ) =ℰ_{, u} [u^{\ast} (λ) +
   U (η), λ ; U' (η)], \end{equation}
\begin{equation} \partial_{η  η}^2 F (η, λ) =ℰ_{, u
    u} [u^{\ast} (λ) + U (η), λ ; U' (η), U' (η)]
   +ℰ_{, u} [u^{\ast} (λ) + U (η), λ ; U'' (η)],
\end{equation}
\begin{eqnarray*}
  \partial_{η  η  η}^3 F (η, λ) & = &
  ℰ_{, u  u  u} [u^{\ast} (λ) + U (η),
  λ ; U' (η), U' (η), U' (η)]\\
  &  &    + 3ℰ_{, u  u}
  [u^{\ast} (λ) + U (η), λ ; U' (η), U'' (η)]\\
  &  &    +ℰ_{, u} [u^{\ast} (λ) +
  U (η), λ ; U''' (η)],
\end{eqnarray*}
\begin{eqnarray}
  \partial_{η  η  η  η}^4 F (η,
  λ) & = & ℰ_{, u  u  u  u} [u^{\ast}
  (λ) + U (η), λ ; U' (η), U' (η), U' (η), U' (η)]
  \nonumber\\
  &  &  + 6ℰ_{, u  u  u} [u^{\ast}
  (λ) + U (η), λ ; U' (η), U' (η), U'' (η)]
  \nonumber\\
  &  &  + 3ℰ_{, u  u} [u^{\ast} (λ) + U
  (η), λ ; U'' (η), U'' (η)] \nonumber\\
  &  &  + 3ℰ_{, u  u} [u^{\ast} (λ) + U
  (η), λ ; U' (η), U''' (η)] \nonumber\\
  &  &  +ℰ_{, u} [u^{\ast} (λ) + U (η), λ ;
  U'''' (η)], \nonumber
\end{eqnarray}
soit, en $η = 0$, en observant que $ℰ_{, u} [u^{\ast} (λ),
λ] = 0$
\begin{equation} \partial_{η} F (0, λ) = 0, \end{equation}
\begin{equation} \partial_{η  η}^2 F (0, λ) =ℰ_2 (λ ;
   u_1, u_1), \end{equation}
\begin{equation} \partial_{η  η  η}^3 F (0, λ) =ℰ_3
   (λ ; u_1, u_1, u_1) + 3ℰ_2 (λ ; u_1, u_2), \end{equation}
\begin{eqnarray}
  \partial_{η  η  η  η}^4 F (η,
  λ) & = & ℰ_4 (λ ; u_1, u_1, u_1, u_1) + 6ℰ_3
  (λ ; u_1, u_1, u_2) + 3ℰ_2 (λ ; u_2, u_2) \nonumber\\
  &  &  + 3ℰ_2 (λ ; u_1, u_3) . \nonumber
\end{eqnarray}
On en déduit que
\begin{equation} \partial_{η  λ}^2 F (0, λ) = 0, \end{equation}
\begin{equation} \partial_{η  η  λ}^3 F (0, λ) =
   \dot{ℰ}_2 (λ ; u_1, u_1), \end{equation}
\begin{equation} \partial_{η  λ  λ}^3 F (0, λ) = 0, \end{equation}
\begin{equation} \partial_{η  η  η  λ}^4 F (0,
   λ) = \dot{ℰ}_3 (λ ; u_1, u_1, u_1) + 3
   \dot{ℰ}_2 (λ ; u_1, u_2), \text{} \text{} \end{equation}
\begin{equation} \partial_{η  η  λ  λ}^4 F (0,
   λ) = \ddot{ℰ}_2 (λ ; u_1, u_1), \end{equation}
\begin{equation} \partial_{η  λ  λ  λ}^4 F (0,
   λ) = 0 \end{equation}
et finalement
\begin{equation} f' (0) = 0, \end{equation}
\begin{equation} f'' (0) =ℰ_2 (λ_0 ; u_1, u_1), \end{equation}
\begin{equation} f''' (0) =ℰ_3 (λ_0 ; u_1, u_1, u_1) + 3ℰ_2
   (λ_0 ; u_1, u_2) + 3 λ_1  \dot{ℰ}_2 (λ_0 ; u_1,
   u_1), \end{equation}
\begin{eqnarray}
  f'''' (0) & = & ℰ_4 (λ_0 ; u_1, u_1, u_1, u_1) +
  6ℰ_3 (λ_0 ; u_1, u_1, u_2) \nonumber\\
  &  &  + 3ℰ_2 (λ_0 ; u_2, u_2) + 3ℰ_2
  (λ_0 ; u_1, u_3) \nonumber\\
  &  &  + 4 λ_1  \dot{ℰ}_3 (λ_0 ; u_1, u_1,
  u_1) + 12 λ_1  \dot{ℰ}_2 (λ_0 ; u_1, u_2) \nonumber\\
  &  &  + 6 λ_1^2  \ddot{ℰ}_2 (λ_0 ; u_1, u_1)
  + 6 λ_2  \dot{ℰ}_2 (λ_0 ; u_1, u_1) . \nonumber
\end{eqnarray}
Les relations précédentes se simplifient notamment en tenant compte de
ce que $u_1∈V$ : $ℰ_2 (λ_0 ; u_1, u_i) = 0$ pour $i = 1,
2, 3$. On trouve ainsi, pour $f'' (0)$ et $f''' (0)$
\begin{equation}
  \label{eq:DL energie derivee 2nde} f'' (0) = 0
\end{equation}
et
\begin{eqnarray}
  f''' (0) & = & ℰ_3 (λ_0 ; u_1, u_1, u_1) + 3 λ_1
  \dot{ℰ}_2 (λ_0 ; u_1, u_1) \nonumber\\
  & = & - 2 λ_1  \dot{ℰ_2} (λ_0 ; u_1, \hat{v}) + 3
  λ_1  \dot{ℰ}_2 (λ_0 ; u_1, u_1) \nonumber\\
  & = & λ_1 F_{i  j} (λ_0) ξ_1^i ξ_1^j,  \label{eq:DL
  energie derivee 3ieme}
\end{eqnarray}
en utilisant l'équation de bifurcation~\eqref{eq:bifurcation 1a} dans la
deuxième ligne. En introduisant les décompositions
\eqref{eq:decomposition u1} et \eqref{eq:decomposition u2} de $u_1$ et $u_2$,
on trouve tout d'abord, pour $ℰ_3 (λ_0 ; u_1, u_1, u_2)$
\begin{eqnarray*}
  ℰ_3 (λ_0 ; u_1, u_1, u_2) & = & ℰ_3 (λ_0 ;
  v_i, v_j, v_k) ξ_1^i ξ_1^j ξ_2^k +ℰ_3 (λ_0 ; v_i, v_j,
  w_{k  l}) ξ_1^i ξ_1^j ξ_1^k ξ_1^l\\
  &  &  + λ_1 ℰ_3 (λ_0 ; v_i, v_j, w_k)
  ξ_1^i ξ_1^j ξ_1^k\\
  & = & ℰ_3 (λ_0 ; v_i, v_j, v_k) ξ_1^i ξ_1^j ξ_2^k
  +ℰ_3 (λ_0 ; v_i, v_j, w_{k  l}) ξ_1^i ξ_1^j
  ξ_1^k ξ_1^l\\
  &  &  - λ_1 ℰ_2 (λ_0 ; w_{i  j},
  w_k) ξ_1^i ξ_1^j ξ_1^k,
\end{eqnarray*}
en tenant compte de la définition~\eqref{eq:pbvar wij} des $w_{i
j}$. Dans le dernier terme de l'expression précédente, les indices
$i$, $j$ et $k$ sont muets, donc
\begin{eqnarray*}
  ℰ_3 (λ_0 ; u_1, u_1, u_2) & = & ℰ_3 (λ_0 ;
  v_i, v_j, v_k) ξ_1^i ξ_1^j ξ_2^k +ℰ_3 (λ_0 ; v_i, v_j,
  w_{k  l}) ξ_1^i ξ_1^j ξ_1^k ξ_1^l\\
  &  &  - λ_1 ℰ_2 (λ_0 ; w_{i }, w_{j
   k}) ξ_1^i ξ_1^j ξ_1^k\\
  & = & ℰ_3 (λ_0 ; v_i, v_j, v_k) ξ_1^i ξ_1^j ξ_2^k
  +ℰ_3 (λ_0 ; v_i, v_j, w_{k  l}) ξ_1^i ξ_1^j
  ξ_1^k ξ_1^l\\
  &  &  + 2 λ_1  \dot{ℰ}_2 (λ_0 ; v_{i
  }, w_{j  k}) ξ_1^i ξ_1^j ξ_1^k,
\end{eqnarray*}
en introduisant cette fois-ci la définition~\eqref{eq:pbvar wi} de $w_i .$
On procède de même pour le terme suivant, soit $ℰ_2
(λ_0 ; u_2, u_2)$
\begin{eqnarray*}
  ℰ_2 (λ_0 ; u_2, u_2) & = & ℰ_2 (λ_0 ;
  ξ_1^i v_i + ξ_1^i ξ_1^j w_{i  j} + λ_1 ξ_1^i w_i,
  ξ_1^i v_i + ξ_1^k ξ_1^l w_{k  l} + λ_1 ξ_1^k w_k)\\
  & = & ℰ_2 (λ_0 ; ξ_1^i ξ_1^j w_{i  j} +
  λ_1 ξ_1^i w_i, ξ_1^k ξ_1^l w_{k  l} + λ_1 ξ_1^k
  w_k)\\
  & = & ℰ_2 (λ_0 ; w_{i  j}, w_{k  l})
  ξ_1^i ξ_1^j ξ_1^k ξ_1^l + 2 λ_1 ℰ_2 (λ_0 ;
  w_{i  j}, w_k) ξ_1^i ξ_1^j ξ_1^k\\
  &  &  + λ_1^2 ℰ_2 (λ_0 ; w_i, w_j) ξ_1^i
  ξ_1^j\\
  & = & ℰ_2 (λ_0 ; w_{i  j}, w_{k  l})
  ξ_1^i ξ_1^j ξ_1^k ξ_1^l + 2 λ_1 ℰ_2 (λ_0 ;
  w_i, w_{j  k}) ξ_1^i ξ_1^j ξ_1^k\\
  &  &  + \tfrac{1}{2} λ_1^2  [ℰ_2 (λ_0 ; w_i,
  w_j) +ℰ_2 (λ_0 ; w_j, w_i)] ξ_1^i ξ_1^j\\
  & = & ℰ_2 (λ_0 ; w_{i  j}, w_{k  l})
  ξ_1^i ξ_1^j ξ_1^k ξ_1^l - 4 λ_1  \dot{ℰ}_2
  (λ_0 ; v_i, w_{j  k}) ξ_1^i ξ_1^j ξ_1^k\\
  &  &  - λ_1^2  [\dot{ℰ}_2 (λ_0 ; v_i, w_j) +
  \dot{ℰ}_2 (λ_0 ; v_j, w_i)] ξ_1^i ξ_1^j\\
  & = & ℰ_3 (λ_0 ; v_i, v_j, w_{k  l}) ξ_1^i
  ξ_1^j ξ_1^k ξ_1^l - 4 λ_1  \dot{ℰ}_2 (λ_0 ; v_i,
  w_{j  k}) ξ_1^i ξ_1^j ξ_1^k\\
  &  &  - λ_1^2  [\dot{ℰ}_2 (λ_0 ; v_i, w_j) +
  \dot{ℰ}_2 (λ_0 ; v_j, w_i)] ξ_1^i ξ_1^j
\end{eqnarray*}
et enfin
\begin{eqnarray*}
  \dot{ℰ}_2 (λ_0 ; u_1, u_2) & = & \dot{ℰ}_2
  (λ_0 ; v_i, v_j) ξ_1^i ξ_2^j + \dot{ℰ}_2 (λ_0 ;
  v_i, w_{j  k}) ξ_1^i ξ_1^j ξ_1^k + λ_1
  \dot{ℰ}_2 (λ_0 ; v_i, w_j) ξ_1^i ξ_1^j\\
  & = & \dot{ℰ}_2 (λ_0 ; v_i, v_j) ξ_1^i ξ_2^j +
  \dot{ℰ}_2 (λ_0 ; v_i, w_{j  k}) ξ_1^i ξ_1^j
  ξ_1^k\\
  &  &  + \tfrac{1}{2} λ_1  [\dot{ℰ}_2 (λ_0 ;
  v_i, w_j) + \dot{ℰ}_2 (λ_0 ; v_j, w_i)] ξ_1^i ξ_1^j .
\end{eqnarray*}
En rassemblant les résultats précédents, on trouve pour $f''''
(0)$
\begin{eqnarray*}
  f'''' (0) & = & ℰ_4 (λ_0 ; v_i, v_j, v_k {, v_l} ) ξ_1^i
  ξ_1^j ξ_1^k ξ_1^l + 6ℰ_3 (λ_0 ; v_i, v_j, v_k) ξ_1^i
  ξ_1^j ξ_2^k\\
  &  &  + 6ℰ_3 (λ_0 ; v_i, v_j, w_{k  l})
  ξ_1^i ξ_1^j ξ_1^k ξ_1^l + 12 λ_1  \dot{ℰ}_2
  (λ_0 ; v_{i }, w_{j  k}) ξ_1^i ξ_1^j ξ_1^k\\
  &  &  - 3ℰ_3 (λ_0 ; v_i, v_j, w_{k  l})
  ξ_1^i ξ_1^j ξ_1^k ξ_1^l - 12 λ_1  \dot{ℰ}_2
  (λ_0 ; v_i, w_{j  k}) ξ_1^i ξ_1^j ξ_1^k\\
  &  &  - 3 λ_1^2  [\dot{ℰ}_2 (λ_0 ; v_i, w_j)
  + \dot{ℰ}_2 (λ_0 ; v_j, w_i)] ξ_1^i ξ_1^j + 4 λ_1
  \dot{ℰ}_3 (λ_0 ; v_i, v_j, v_k) ξ_1^i ξ_1^j ξ_1^k\\
  &  &  + 12 λ_1  \dot{ℰ}_2 (λ_0 ; v_i,
  v_j) ξ_1^i ξ_2^j + 12 λ_1  \dot{ℰ}_2 (λ_0 ;
  v_i, w_{j  k}) ξ_1^i ξ_1^j ξ_1^k\\
  &  &  + 6 λ_1^2  [\dot{ℰ}_2 (λ_0 ; v_i,
  w_j) + \dot{ℰ}_2 (λ_0 ; v_j, w_i)] ξ_1^i ξ_1^j + 6
  λ_1^2  \ddot{ℰ}_2 (λ_0 ; v_i, v_j) ξ_1^i ξ_1^j\\
  &  &  + 6 λ_2  \dot{ℰ}_2 (λ_0 ; v_i, v_j)
  ξ_1^i ξ_1^j\\
  & = & \left[ ℰ_4 (λ_0 ; v_i, v_j, v_k {, v_l} ) +
  3ℰ_3 (λ_0 ; v_i, v_j, w_{k  l}) \right] ξ_1^i
  ξ_1^j ξ_1^k ξ_1^l\\
  &  &  + 4 λ_1  [\dot{ℰ}_3 (λ_0 ; v_i, v_j,
  v_k) + 3 \dot{ℰ}_2 (λ_0 ; v_i, w_{j  k})] ξ_1^i
  ξ_1^j ξ_1^k\\
  &  &  + \{ 3 λ_1^2  [\dot{2 \ddot{ℰ}_2
  (λ_0 ; v_i, v_j) + \dot{ℰ}}_2 (λ_0 ; v_i, w_j) +
  \dot{ℰ}_2 (λ_0 ; v_j, w_i)] + 6 λ_2
  \dot{ℰ}_2 (λ_0 ; v_i, v_j) \} ξ_1^i ξ_1^j\\
  &  &  + 6ℰ_3 (λ_0 ; v_i, v_j, v_k) ξ_1^i ξ_1^j
  ξ_2^k + 12 λ_1  \dot{ℰ}_2 (λ_0 ; v_i, v_j)
  ξ_1^i ξ_2^j\\
  & = & E_{i  j  k  l} (λ_0) ξ_1^i ξ_1^j
  ξ_1^k ξ_1^l + 4 λ_1  \dot{E}_{i  j  k}
  (λ_0) ξ_1^i ξ_1^j ξ_1^k + 6 [λ_1^2  \dot{F}_{i
   j} (λ_0) + λ_2 F_{i  j} (λ_0)]
  ξ_1^i ξ_1^j\\
  &  &  + 6 [E_{i  j  k} (λ_0) ξ_1^k + 2
  λ_1 F_{i  j} (λ_0)] ξ_1^i ξ_2^j,
\end{eqnarray*}
et on observe que le dernier terme (en $ξ_1^i ξ_2^j$) est nul, du fait de
l'équation de bifurcation~\eqref{eq:bifurcation 1c}. On obtient donc
\begin{equation}
  \label{eq:DL energie derivee 4ieme} f'''' (0) = E_{i  j  k
   l} (λ_0) ξ_1^i ξ_1^j ξ_1^k ξ_1^l + 4 λ_1
  \dot{E}_{i  j  k} (λ_0) ξ_1^i ξ_1^j ξ_1^k + 6
  [λ_1^2  \dot{F}_{i  j} (λ_0) + λ_2 F_{i
   j} (λ_0)] ξ_1^i ξ_1^j .
\end{equation}
Le développement limité~\eqref{eq:DL energie} est alors obtenu en
rassemblant les résultats~\eqref{eq:DL energie derivee 2nde}, \eqref{eq:DL
energie derivee 3ieme} et \eqref{eq:DL energie derivee 4ieme}.

\begin{remark}
  On peut réécrire $f'''' (0)$ en tenant compte de l'équation de
  bifurcation~\eqref{eq:bifurcation 2b}. En multipliant celle-ci par
  $ξ_1^i$, on trouve en effet
  \begin{eqnarray*}
    E_{i  j  k  l} (λ_0) ξ_1^i
    \hspace{0.17em} ξ_1^j ξ_1^k ξ_1^l & = & - 3 λ_2 F_{i
    j} (λ_0) ξ_1^i ξ_1^j - 3 A_{i  j} (λ_0)
    ξ_1^i ξ_2^j - 3 λ_1  [\dot{E}_{i  j  k}
    (λ_0) ξ_1^k + λ_1  \dot{F}_{i  j} (λ_0)]
    ξ_1^i ξ_1^j\\
    & = & - 3 λ_1  \dot{E}_{i  j  k} (λ_0)
    ξ_1^i ξ_1^j ξ_1^k - 3 [λ_1^2  \dot{F}_{i  j}
    (λ_0) + λ_2 F_{i  j} (λ_0)] ξ_1^i ξ_1^j - 3
    A_{i  j} (λ_0) ξ_1^i ξ_2^j,
  \end{eqnarray*}
  soit
  \begin{equation} f'''' (0) = λ_1  \dot{E}_{i  j  k} (λ_0)
     ξ_1^i ξ_1^j ξ_1^k + 3 [λ_1^2  \dot{F}_{i  j}
     (λ_0) + λ_2 F_{i  j} (λ_0)] ξ_1^i ξ_1^j
     - 3 A_{i  j} (λ_0) ξ_1^i ξ_2^j . \end{equation}
\end{remark}

\subsection{Développement limité de la hessienne}\label{sec:DL
hessienne}

On cherche maintenant un développement limité de la hessienne de
l'énergie. Les fonctions test $\hat{u}, \hat{v}∈U$ étant
fixées, on applique la méthode du
{\textsection}\ref{sec20220107121442} à la fonction $f (η) = F [η,
λ_0 + λ (η)]$, avec
\begin{equation} F (η, λ) =ℰ_{, u  u} [u^{\ast} (λ) + U
   (η), λ ; \hat{u}, \hat{v}] . \end{equation}
On observe tout d'abord que $F (0, λ) =ℰ_2 (λ ; \hat{u},
\hat{v})$, soit, en dérivant par rapport à $λ$
\begin{equation} \partial_{λ} F (0, λ) = \dot{ℰ_2} (λ ; \hat{u},
   \hat{v}) \quad \text{et} \quad \partial_{λ  λ}^2 F (0,
   λ) = \ddot{ℰ_2} (λ ; \hat{u}, \hat{v}) . \end{equation}
On trouve de même successivement
\begin{equation} \partial_{η} F (η, λ) =ℰ_{, u  u  u}
   [u^{\ast} (λ) + U (η), λ ; U' (η), \hat{u}, \hat{v}], \end{equation}
\begin{eqnarray}
  \partial_{η  η}^2 F (η, λ) & = & ℰ_{, u
   u  u  u} [u^{\ast} (λ) + U (η), λ ;
  U' (η), U' (η), \hat{u}, \hat{v}] \nonumber\\
  &  &  +ℰ_{, u  u  u} [u^{\ast} (λ)
  + U (η), λ ; U'' (η), \hat{u}, \hat{v}], \nonumber
\end{eqnarray}
soit, en $η = 0$
\begin{equation} \partial_{η} F (0, λ) =ℰ_3 (λ ; u_1, \hat{u},
   \hat{v}) \text{} \end{equation}
et
\begin{equation} \partial_{η  η}^2 F (0, λ) =ℰ_4 (λ ;
   u_1, u_1, \hat{u}, \hat{v}) +ℰ_3 (λ ; u_2, \hat{u},
   \hat{v}), \end{equation}
et en dérivant cette fois par rapport à $λ$
\begin{equation} \partial_{η  λ}^2 F (0, λ) = \dot{ℰ_3}
   (λ ; u_1, \hat{u}, \hat{v}) . \end{equation}
En insérant les résultats précédents dans les
expressions~\eqref{eq20220107060454} et \eqref{eq20220107124311}, on trouve
\begin{equation} f' (0) =ℰ_3 (λ_0 ; u_1, \hat{u}, \hat{v}) + λ_1
   \dot{ℰ_2} (λ_0 ; \hat{u}, \hat{v}), \end{equation}
\begin{eqnarray}
  f'' (0) & = & ℰ_4 (λ_0 ; u_1, u_1, \hat{u}, \hat{v})
  +ℰ_3 (λ_0 ; u_2, \hat{u}, \hat{v}) + λ_2
  \dot{ℰ_2} (λ_0 ; \hat{u}, \hat{v}) \nonumber\\
  &  &   + 2 λ_1  \dot{ℰ_3} (λ_0 ;
  u_1, \hat{u}, \hat{v}) + λ_1^2  \ddot{ℰ_2} (λ_0 ;
  \hat{u}, \hat{v}) . \nonumber
\end{eqnarray}
qui conduisent finalement au développement limité~\eqref{eq:DL
hessienne}.

\subsection{Développement limité des valeurs propres et vecteurs
propres de la Hessienne}

On cherche les vecteurs propres $x∈U$ et valeurs propres $α \in
\mathbb{R}$ de la hessienne de l'énergie. En d'autre terme, on cherche $x$
et $α$ tels que
\begin{equation}
  ℰ_{, u  u} [u (η), λ (η) ; x, \hat{u}] =
  α \langle x, \hat{u} \rangle \quad \text{pour tout} \quad \hat{u} \in
  V.
\end{equation}
On cherche les développements limités à l'ordre 2 en $η$ de $x$
et $α$
\begin{eqnarray*}
  x & = & x_0 + η x_1 + \tfrac{1}{2} η^2 x_2 + o (η^2),\\
  α & = & α_0 + η α_1 + \tfrac{1}{2} η^2 α_2 + o
  (η^2) .
\end{eqnarray*}
Ces développements limités sont tout d'abord insérés dans le
développement limité \eqref{eq:DL hessienne} de la hessienne de
l'énergie
\begin{eqnarray*}
  ℰ_{, u  u} [u (η), λ (η) ; x, \hat{u}] & = &
  ℰ_2 (λ_0 ; x_0, \hat{u}) + η ℰ_2 (λ_0 ;
  x_1, \hat{u}) + \tfrac{1}{2} η^2 ℰ_2 (λ_0 ; x_2,
  \hat{u})\\
  &  & + η ℰ_3 (λ_0 ; u_1, x_0, \hat{u}) + η^2
  ℰ_3 (λ_0 ; u_1, x_1, \hat{u})\\
  &  & + η λ_1  \dot{ℰ_2} (λ_0 ; x_0, \hat{u}) +
  η^2 λ_1  \dot{ℰ_2} (λ_0 ; x_1, \hat{u})\\
  &  & + \tfrac{1}{2} η^2  [ℰ_4 (λ_0 ; u_1, u_1, x_0,
  \hat{u})  +ℰ_3 (λ_0 ; u_2, x_0, \hat{u})\\
  &  & + λ_2  \dot{ℰ_2} (λ_0 ; x_0, \hat{u}) + 2
  λ_1  \dot{ℰ_3} (λ_0 ; u_1, x, \hat{u})\\
  &  & + λ_1^2  \ddot{ℰ_2} (λ_0 ; x, \hat{u})
  ] + o (η^2)\\
  & = & ℰ_2 (λ_0 ; x_0, \hat{u})\\
  &  & + η [ℰ_3 (λ_0 ; u_1, x_0, \hat{u}) +ℰ_2
  (λ_0 ; x_1, \hat{u}) + λ_1  \dot{ℰ_2} (λ_0 ;
  x_0, \hat{u})]\\
  &  & + \tfrac{1}{2} η^2  [ℰ_2 (λ_0 ; x_2, \hat{u}) +
  2ℰ_3 (λ_0 ; u_1, x_1, \hat{u}) + 2 λ_1
  \dot{ℰ_2} (λ_0 ; x_1, \hat{u})]\\
  &  & + \tfrac{1}{2} η^2  [ℰ_4 (λ_0 ; u_1, u_1, x_0,
  \hat{u}) +ℰ_3 (λ_0 ; u_2, x_0, \hat{u}) + λ_2
  \dot{ℰ_2} (λ_0 ; x_0, \hat{u})]\\
  &  & + \tfrac{1}{2} η^2  [2 λ_1  \dot{ℰ_3} (λ_0 ;
  u_1, x, \hat{u}) + λ_1^2  \ddot{ℰ_2} (λ_0 ; x,
  \hat{u})] + o (η^2)
\end{eqnarray*}
et
\begin{eqnarray*}
  α \langle x, \hat{u} \rangle & = & α_0  \langle x_0, \hat{u}
  \rangle + η (α_1 \langle x_0, \hat{u} \rangle + α_0 \langle
  x_1, \hat{u} \rangle)\\
  &  & + \tfrac{1}{2} η^2  (α_2 \langle x_0, \hat{u} \rangle + 2
  α_1 \langle x_1, \hat{u} \rangle + α_0 \langle x_2, \hat{u}
  \rangle) + o (η^2) .
\end{eqnarray*}
\paragraph{Problème variationnel d'ordre 0}Trouver $x_0∈U$ et
$α_0∈\mathbb{R}$ tels que, pour tout $\hat{u}∈U$
\begin{equation} ℰ_2 (λ_0 ; x_0, \hat{u}) = α_0  \langle x_0, \hat{u}
   \rangle . \end{equation}
On en déduit que $x_0$ est le vecteur propre de $ℰ_2
(λ_0)$ associé à la valeur propre $α_0$. Si $α_0 \neq
0$, $ℰ_2  (λ_0)$ étant positive par hypothèse, on a
nécessairement $α_0 > 0$, et la valeur propre de la hessienne est
positive. On considère donc dans ce qui suit le cas où $α_0 = 0$,
c'est-à-dire que $x_0∈V$
\begin{equation} x_0 = \chi_0^i v_i \end{equation}


\paragraph{Problème variationnel d'ordre 1}Trouver $x_1∈U$ et
$α_1∈\mathbb{R}$ tels que, pour tout $\hat{u}∈U$
\begin{equation} ℰ_3 (λ_0 ; u_1, x_0, \hat{u}) +ℰ_2 (λ_0 ;
   x_1, \hat{u}) + λ_1  \dot{ℰ_2} (λ_0 ; x_0, \hat{u}) =
   α_1  \langle x_0, \hat{u} \rangle, \end{equation}
soit, en rempla{\c c}ant $u_1$ et $x_0$ par leurs décompositions dans la
base $v_i$
\begin{equation} ℰ_3 (λ_0 ; v_j, v_k, \hat{u}) ξ_1^k \chi_0^j
   +ℰ_2 (λ_0 ; x_1, \hat{u}) + λ_1  \dot{ℰ_2}
   (λ_0 ; v_j, \hat{u}) \chi_0^j = α_1 \chi_0^j  \langle v_j,
   \hat{u} \rangle . \end{equation}
En prenant tout d'abord $\hat{u} = v_i$, on obtient
\begin{equation} [ℰ_3 (λ_0 ; v_i, v_j, v_k) ξ_1^k + λ_1
   \dot{ℰ_2} (λ_0 ; v_i, v_j)] \chi_0^j = α_1 \chi_0^i,
\end{equation}
soit encore
\begin{equation} [E_{i  j  k} (λ_0) ξ_1^k + λ_1 F_{i
   j} (λ_0)] \chi_0^j = α_1 \chi_0^i . \end{equation}
Ainsi, le vecteur $\chi_0^i$ apparaît comme le vecteur propre de la
matrice symétrique $[E_{i  j  k} (λ_0) ξ_1^k +
λ_1 F_{i  j} (λ_0)]$ associé à la valeur propre
$α_1$. On doit alors discuter en fonction du type de bifurcation.

\paragraph{Cas d'une bifurcation asymétrique}Dans ce cas, la forme
trilinéaire $E_{i  j  k} (λ_0)$ n'est pas nulle sur
$V$, et $α_1 \neq 0$. Le terme dominant de $α$ est donc d'ordre 1,
tandis que le terme dominant de $x$ est d'ordre 0.

\paragraph{Cas d'une bifurcation symétrique}La forme trilinéaire $E_{i
 j  k} (λ_0)$ est identiquement nulle sur $V$ ; de plus,
$λ_1 = 0$. On trouve alors que $α_1 = 0$, et on ne peut
déterminer les $\chi_0^i$. On prend maintenant $\hat{u} = \hat{w}∈W$
dans le problème variationnel d'ordre 1, et on pose $x_1 = \chi_1^i v_i +
y_1$, avec $y_1∈W$. On obtient alors le problème variationnel suivant
: trouver $y_1∈W$ tel que, pour tout $\hat{w}∈W$,
\begin{equation} ℰ_3 (λ_0 ; v_j, v_k, \hat{w}) ξ_1^k \chi_0^j
   +ℰ_2 (λ_0 ; y_1, \hat{w}) + λ_1  \dot{ℰ_2}
   (λ_0 ; v_j, \hat{w}) \chi_0^j = 0. \end{equation}
La solution de ce problème est exprimée à l'aide des $w_{i
 j}$ et $w_i$ définis respectivement par les problèmes
variationnels auxiliaires \eqref{eq:pbvar wij} et \eqref{eq:pbvar wi}
\begin{equation} y_1 = ξ_1^i \chi_0^j w_{i  j} + λ_1 \chi_0^i w_i, \end{equation}
soit
\begin{equation} x_1 = \chi_1^i v_i + ξ_1^i \chi_0^j w_{i  j} + λ_1 \chi_0^i
   w_i . \end{equation}
Dans le cas d'une bifurcation symétrique, le problème aux valeurs
propres d'ordre 2 s'écrit quant à lui
\begin{equation} ℰ_2 (λ_0 ; x_2, \hat{u}) + 2ℰ_3 (λ_0 ; u_1,
   x_1, \hat{u}) +ℰ_4  (λ_0 ; u_1, u_1, x_0, \hat{u})
   +ℰ_3 (λ_0 ; u_2, x_0, \hat{u}) + λ_2
   \dot{ℰ_2} (λ_0 ; x_0, \hat{u}) = α_2  \langle x_0,
   \hat{u} \rangle \end{equation}
soit, en prenant $\hat{u} = \widehat{v_i}∈V$ et en introduisant les
développements de $u_1$, $u_2$, $x_0 $ et $x_1$
\begin{equation} ℰ_4  (λ_0 ; v_i, v_j, v_k, v_l) \chi_0^j ξ_{1 }^k ξ_1^l
   + 2ℰ_3 (λ_0 ; u_1, x_1, v_i) +ℰ_3 (λ_0 ;
   u_2, x_0, \hat{u}) + λ_2  \dot{ℰ_2} (λ_0 ; x_0,
   \hat{u}) \end{equation}
\section{Simplification des équations de
bifurcation}\label{sec:Simplification des équations de bifurcation}

Dans ce paragraphe, on simplifie les équations de bifurcation
\eqref{eq:bifurcation 1b} et \eqref{eq:bifurcation 2a} pour obtenir les formes
\eqref{eq:bifurcation 1c} et \eqref{eq:bifurcation 2b}. On commence par
symétriser les termes cubique, quadratique et linéaire en $ξ_1^i$ de
l'équation \eqref{eq:bifurcation 2b}.

\paragraph{Terme cubique en $ξ_1^i$}On observe que
\begin{equation} ℰ_3 (λ_0 ; v_i, v_j, w_{k  l}) ξ_1^j ξ_1^k
   ξ_1^l = \text{} \tfrac{1}{3}  [ℰ_3 (λ_0 ; v_i, v_j, w_{k
    l})  +ℰ_3 (λ_0 ; v_i, v_k, w_{j
   l}) +ℰ_3 (λ_0 ; v_i, v_l, w_{j  k}] ξ_1^j ξ_1^k
   ξ_1^l . \end{equation}
On obtient donc l'expression suivante du terme cubique en $ξ_1^i$ dans
l'équation de bifurcation \eqref{eq:bifurcation 2a}
\begin{equation} ℰ_4 (λ_0 ; v_i, v_j, v_k, v_l) +ℰ_3 (λ_0 ;
   v_i, v_j, w_{k  l}) +ℰ_3 (λ_0 ; v_i, v_k, w_{j
    l}) +ℰ_3 (λ_0 ; v_i, v_l, w_{j  k}), \end{equation}
qui suggère d'introduire le \ $E_{i  j  k  l}
(λ)$ défini par l'équation \eqref{eq:def Eijkl}. Le terme
cubique en $ξ_1^i$ dans l'équation de bifurcation \eqref{eq:bifurcation
2a} est alors simplement~: $E_{i  j  k  l}
(λ_0)$.

\paragraph{Terme quadratique en $ξ_1^i$}On observe de même que
\begin{equation} ℰ_3 (λ_0 ; v_i, v_j, w_k) ξ_1^j ξ_1^k = \tfrac{1}{2}
   [ℰ_3 (λ_0 ; v_i, v_j, w_k) +ℰ_3 (λ_0 ; v_i,
   w_j, v_k)] ξ_1^j ξ_1^k . \end{equation}
En prenant tout d'abord $\widehat{w} = w_k$ dans le problème variationnel
\eqref{eq:pbvar wij}, on trouve
\begin{equation} ℰ_3 (λ_0 ; v_i, v_j, w_k) = -ℰ_2 (λ_0 ;
   w_{i  j}, w_k), \end{equation}
puis, en prenant cette fois $\hat{w} = w_{i  j}$ dans le problème
variationnel \eqref{eq:pbvar wi}
\begin{equation} ℰ_2 (λ_0 ; w_k, w_{i  j}) = - 2 \dot{ℰ_2}
   (λ_0 ; v_k, w_{i  j}), \end{equation}
soit finalement
\begin{equation} ℰ_3 (λ_0 ; v_i, v_j, w_k) ξ_1^j ξ_1^k =
   [\dot{ℰ}_2 (λ_0 ; v_j, w_{i  k}) +
   \dot{ℰ}_2 (λ_0 ; v_k, w_{i  j})] ξ_1^j ξ_1^k .
\end{equation}
On obtient donc l'expression suivante du terme quadratique en $ξ_1^i$ dans
l'équation de bifurcation \eqref{eq:bifurcation 2a}
\begin{equation} 3 λ_1  [\dot{ℰ}_3 (λ_0 ; v_i, v_j, v_k) +
   \dot{ℰ_2} (λ_0 ; v_i, w_{j  k}) +
   \dot{ℰ}_2 (λ_0 ; v_j, w_{i  k}) +
   \dot{ℰ}_2 (λ_0 ; v_k, w_{i  j})], \end{equation}
qui suggère d'introduire le tenseur $E_{i  j  k}
(λ)$ défini par l'équation \eqref{eq:def Eijk}. Le terme
quadratique en $ξ_1^i$ dans l'équation de bifurcation
\eqref{eq:bifurcation 2a} est alors simplement~: $3 λ_1  \dot{E}_{i
 j  k} (λ_0)$.

\paragraph{Terme linéaire en $ξ_1^i$}Par des arguments similaires, on
établit également que
\begin{equation} \dot{ℰ_2} (λ_0 ; v_i, w_j) = - \tfrac{1}{2} ℰ_2
   (λ_0 ; w_i, w_j) = - \tfrac{1}{2} ℰ_2 (λ_0 ; w_j,
   w_i) = \dot{ℰ_2} (λ_0 ; v_j, w_i) . \end{equation}
On obtient donc l'expression suivante du terme linéaire en $ξ_1^i$ dans
l'équation de bifurcation \eqref{eq:bifurcation 2a}
\begin{equation} \ddot{ℰ}_2 (λ_0 ; v_i, v_j) + \tfrac{1}{2}
   [\dot{ℰ_2} (λ_0 ; v_i, w_j) + \dot{ℰ_2}
   (λ_0 ; v_j, w_i)], \end{equation}
qui suggère d'introduire le tenseur $F_{i  j} (λ)$
défini par l'équation \eqref{eq:def Fij}. Le terme linéaire en
$ξ_1^i$ dans l'équation de bifurcation \eqref{eq:bifurcation 2a} est
alors simplement~: $3 λ_1^2  \dot{F}_{i  j} (λ_0)$.

\paragraph{Synthèse~: simplification des équations
\eqref{eq:bifurcation 1a} et \eqref{eq:bifurcation 2a}}En rassemblant les
résultats précédents, on obtient tout d'abord pour l'équation
\eqref{eq:bifurcation 2a}
\begin{equation} 3 [E_{i  j  k} (λ_0) + λ_1 F_{i  j}
   (λ_0)] ξ_2^j + 3 λ_2 F_{i  j} (λ_0) ξ_1^j +
   E_{i  j  k  l} (λ_0)  \hspace{0.17em} ξ_1^j
   ξ_1^k ξ_1^l + 3 λ_1  \dot{E}_{i  j  k}
   (λ_0)  \hspace{0.17em} ξ_1^j ξ_1^k + 3 λ_1^2  \dot{F}_{i
    j} (λ_0) ξ_1^j = 0, \end{equation}
qui suggère d'introduire le tenseur $A_{i  j} (λ)$
défini par l'équation \eqref{eq:def Aij}. On obtient alors finalement
l'équation de bifurcation \eqref{eq:bifurcation 2b}. Les tenseurs $F_{i
 j}$ et $E_{i  j  k}$ ainsi introduits permettent
également de réécrire l'équation de bifurcation
\eqref{eq:bifurcation 1b} sous la forme compacte \eqref{eq:bifurcation 1c}.

\end{document}

% Local Variables:
% fill-column: 80
% End:
