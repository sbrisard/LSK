\newcommand{\sbtitle}{Notes relatives à la méthode asymptotique de Lyapunov–Schmidt–Koiter}
\newcommand{\sbauthor}{Sébastien Brisard}
\newcommand{\sbemail}{sebastien.brisard@univ-eiffel.fr}
\newcommand{\sbaddress}{Univ Gustave Eiffel, Ecole des Ponts, IFSTTAR, CNRS, Navier, F-77454 Marne-la-Vall\'ee, France}
\newcommand{\sbsubject}{Note bibliographique}

\documentclass[12pt, final]{scrartcl}
%\setkomafont{disposition}{\rmfamily}

\usepackage{polyglossia}
\setdefaultlanguage{english}

\usepackage{amsfonts}
\usepackage{amsmath}
\usepackage{amssymb}

\usepackage{amsthm}
\theoremstyle{definition}
\renewcommand{\qedsymbol}{}
\newtheorem{remark}{Remark}
\newtheorem{theorem}{Theorem}

\usepackage[backend=biber,bibencoding=utf8,doi=false,giveninits=true,isbn=false,maxnames=10,minnames=5,sortcites=true,style=authoryear,texencoding=utf8,url=false]{biblatex}
\addbibresource{LSK-notes.bib}

\usepackage[breaklinks=true, colorlinks=true, pdftitle={\sbtitle}, pdfauthor={\sbauthor}, pdfsubject={\sbsubject}, urlcolor=blue]{hyperref}

\usepackage[color={1 1 0}]{pdfcomment}

\usepackage{unicode-math}
% \setmainfont{XITS}
% \setmathfont{XITS Math}
\setmainfont{Asana Math}
\setmathfont{Asana Math}

\newcommand{\D}{\mathrm{d}}
\newcommand{\order}[2][1]{#2^{(#1)}}
\newcommand{\reals}{\mathbb{R}}
\newcommand{\wrt}{w.r.t.}

\begin{document}
\title{\sbtitle}
\author{\sbauthor\thanks{\sbaddress~--- \sbemail}}
\maketitle

\begin{abstract}
  These are my notes on the LSK method for the analysis of the stability and
  bifurcation(s) of a conservative system. These notes are based on several
  references: Koiter's initial PhD thesis~\parencite{koit1945} as well as some
  graphical illustrations from his lecture notes~\parencite{koit2009}. I enjoyed
  the concise presentation of \textcite{nguy2000} as well as the lecture notes
  of \textcite{tria2017}. Finally, the chapter by \textcite{poti1987} helped me
  clear some issues.

  These notes by Sébastien Brisard are licensed under a Creative Commons
  Attribution 4.0 International License. To view a copy of this license, visit
  \url{http://creativecommons.org/licenses/by/4.0/}.

  I hope the reader will find these notes useful, even though there are still a
  few points which I do not fully understand (they are clearly indicated in the
  text).
\end{abstract}

\section{Notations}

L'espace des champs cinématiquement admissibles est noté \(U\). On suppose qu'il
a la structure d'espace vectoriel. L'énergie du système est notée \(ℰ(u, λ)\),
où \(λ\) désigne un paramètre de chargement. Soit \(u^{\ast}(λ)\) la branche
fondamentale. Par définition
\begin{equation}
  ℰ_{,u}[u^{\ast}(λ), λ; \hat{u}]=0 \quad \text{pour tout} \quad \hat{u}∈U,
\end{equation}
and, deriving twice \wrt{} \(λ\), we find successively, for all \(\hat{u} ∈ U\)
\begin{equation}
  \label{eq:20220901143843}
  ℰ_{,uu}[u^\ast(λ), λ; \dot{u}^\ast(λ), \hat{u}] + ℰ_{,uλ}[u^\ast(λ), λ; \hat{u}] = 0
\end{equation}
and
\begin{multline}
  \label{eq:20220901143902}
  ℰ_{,uuu}[u^\ast(λ), λ; \dot{u}^\ast(λ), \dot{u}^\ast(λ), \hat{u}] + 2ℰ_{,uuλ}[u^\ast(λ), λ; \dot{u}^\ast(λ), \hat{u}]\\
  + ℰ_{,uλλ}[u^\ast(λ), λ; \hat{u}] + ℰ_{,uu}[u^\ast(λ), λ; \ddot{u}^\ast(λ), \hat{u}] = 0
\end{multline}





Il sera commode d'introduire les notations suivantes
\begin{equation}
  ℰ₂(λ) = ℰ_{,uu}[u^{\ast}(λ), λ], \quad ℰ₃(λ) = ℰ_{,uuu}[u^{\ast}(λ), λ], \quad ℰ₄(λ) = ℰ_{,uuuu} [u^{\ast}(λ), λ].
\end{equation}
Noter que \(ℰ₂\), \(ℰ₃\) et \(ℰ₄\) sont des formes bi-, tri- et
quadri-linéaires, respectivement. L'application de ces formes à des éléments de
\(U\) sera notée \(ℰ₂(λ; u, v)\), \(ℰ₃(λ; u, v, w)\), etc. La dérivée de ces
formes par rapport à \(λ\) sera notée à l'aide d'un point supérieur
(\(\dot{ℰ}_2\), \(\dot{ℰ}_3\), \dots).

On suppose que l'équilibre est stable pour des valeurs suffisamment petites de
\(λ\). Plus précisément, on suppose que \(ℰ₂(λ)\) est définie positive pour tout
\(λ < λ₀\). Pour \(λ = λ₀\), la forme quadratique \(ℰ₂(λ₀)\) n'est plus que
positive. On note \(u₀ = u^{\ast}(λ₀)\), \(\dot{u}₀ = \dot{u}^\ast(λ₀)\) et
\(\ddot{u}₀ = \ddot{u}^\ast(λ₀)\) de sorte que les
Éqs.~\eqref{eq:20220901143843} et \eqref{eq:20220901143902} s'écrivent, en
\(λ = λ₀\)
\begin{gather}
  \label{eq:20220901144331}
  ℰ_{,uu}(u₀, λ₀; \dot{u}₀, \bullet) + ℰ_{,uλ}(u₀, λ₀; \bullet) = 0\\
  \label{eq:20220901144335}
  ℰ_{,uuu}(u₀, λ₀; \dot{u}₀, \dot{u}₀, \bullet) + 2ℰ_{,uuλ}(u₀, λ₀; \dot{u}₀, \bullet) + ℰ_{,uλλ}(u₀, λ₀; \bullet) + ℰ_{,uu}(u₀, λ₀; \ddot{u}₀, \bullet) = 0
\end{gather}

On s'intéresse dans ce qui suit à toutes les courbes d'équilibre qui passent par
le point \((u₀, λ₀)\).

\pdfmarkupcomment{Noter que dans ce qui suit}{Est-ce encore d'actualité ?}, on
convient que les formes \(ℰ₂\), \(ℰ₃\) et \(ℰ₄\) sont implicitement évaluées en
\(λ₀\) lorsque \(λ\) n'est pas rappelé : ainsi, on notera \(ℰ₂(•, •)\) plutôt
que \(ℰ₂(λ₀ ; •, •)\).

Par hypothèse, \(ℰ₂(λ₀)\) est positive, sans être définie
positive~; soit \(V\) son noyau, qui forme un sous-espace vectoriel de \(U\). On
suppose que \(V\) est de dimension finie \(m = \dim V\). Soit
\((v₁, \ldots, v_m)\) une base orthonormée de ce noyau pour le produit scalaire
\(〈 •, • 〉\)(qui n'est pas précisé pour le moment). On introduit le
sous-espace supplémentaire orthogonal \(W\) de \(V\) dans \(U\)
\begin{equation}
  U = V \overset{\perp}{\otimes} W.
\end{equation}

\begin{remark}
  \label{rem:20220902095055}
  The bilinear form \(ℰ₂\) being elliptic over \(W\), variational problems of the
  type: find \(w ∈ W\) such that
  \begin{equation}
    ℰ₂(w, \hat{w})+\ell(\hat{w}) = 0 \quad \text{for all} \quad \hat{w}∈W
  \end{equation}
  are well-posed for any linear form \(\ell\) over \(W\). In particular, for
  \(\ell=0\), the unique solution to the variational problem
  \begin{equation}
    ℰ₂(w, \hat{w}) = 0 \quad \text{for all} \quad \hat{w}∈W
  \end{equation}
  is \(w = 0\).
\end{remark}

For \(1 ≤ i, j ≤ m\), we introduce the solutions \(w_i, w_{ij} ∈ W\) to the
following variational problems
\begin{gather}
  \label{eq:20220524134525}
  ℰ₂(λ₀; w_i, \hat{w}) + \dot{ℰ}₂(λ₀; v_i, \hat{w}) = 0,\\
  \label{eq:20220519164523}
  ℰ₂(λ₀; w_{i j}, \hat{w})+ℰ₃(λ₀; v_i, v_j, \hat{w}) = 0,
\end{gather}
for all \(\hat{w} ∈ W\). Since \(w_{i}\) and \(w_{ij}\) belong to \(W\), we have
\(〈 w_{i}, v 〉 = 〈 w_{ij}, v 〉 = 0\) for all \(v ∈ V\). Since \(ℰ₂(λ₀; •,
•)\) is symmetric, it can be verified that \(w_{ij}=w_{ji}\). We also introduce
the following tensors, defined in \(V\)
\begin{gather}
  E_{ijk} = ℰ₃(λ₀; v_i, v_j, v_k) + ℰ₂(λ₀; v_i, w_{jk}) + ℰ₂(λ₀; v_j, w_{ki}) + ℰ₂(λ₀; v_k, w_{ij}),\\
  E_{ijkl} = ℰ₄(λ₀ ; v_i, v_j, v_k, v_l) + ℰ₃(λ₀ ; v_i, v_j, w_{kl}) + ℰ₃(λ₀ ; v_i, v_k, w_{lj}) + ℰ₃(λ₀ ; v_i, v_l, w_{jk}),\\
  F_{ij} = \dot{ℰ}₂(λ₀; v_i, v_j) + ℰ₂(λ₀; v_i, w_j) + ℰ₂(λ₀; v_j, w_i),
\end{gather}
as well as the derivatives
\begin{gather}
  \label{eq:20220615063626}
  \mathring{E}_{ijk} = \dot{ℰ}₃(λ₀; v_i, v_j, v_k) + \dot{ℰ₂}(λ₀; v_i, w_{jk}) + \dot{ℰ}₂(λ₀; v_j, w_{ki}) + \dot{ℰ}₂(λ₀; v_k, w_{ij}),\\
  \label{eq:20220615063633}
  \mathring{F}_{ij} = \ddot{ℰ}₂(λ₀; v_i, v_j) + \dot{ℰ}₂(λ₀; v_i, w_j) + \dot{ℰ}₂(λ₀; v_j, w_i).
\end{gather}

Note that, since \(ℰ₂(λ₀; v_i, •) = 0\), the above expressions simplify as follows
\begin{gather}
  \label{eq:20220524135619}
  E_{ijk} = ℰ₃(λ₀; v_i, v_j, v_k),\\
  \label{eq:20220524135553}
  E_{ijkl} = ℰ₄(λ₀ ; v_i, v_j, v_k, v_l) + ℰ₃(λ₀ ; v_i, v_j, w_{kl}) + ℰ₃(λ₀ ; v_i, v_k, w_{jl}) + ℰ₃(λ₀ ; v_i, v_l, w_{jk}),\\
  \label{eq:20220524135643}
  F_{ij} = \dot{ℰ}₂(λ₀; v_i, v_j).
\end{gather}

The tensors \(E_{ijk}\), \(F_{ij}\), \(\mathring{E}_{ijk}\) and
\(\mathring{F}_{ij}\) are fully symmetric. Furthermore, the following expression
of \(E_{ijkl}\) result from Eq.~\eqref{eq:20220519164523}
\begin{equation}
  \label{eq:20220802081116}
  E_{ijkl} = ℰ₄(λ₀ ; v_i, v_j, v_k, v_l) - ℰ₂(λ₀ ; w_{ij}, w_{kl}) - ℰ₂(λ₀ ; w_{ik}, w_{jl}) - ℰ₂(λ₀ ; w_{il}, w_{jk}),
\end{equation}
which shows that \(E_{ijkl}\) is also fully symmetric. We close this section,
with two useful identities
\begin{equation}
  \label{eq:20220617084433}
  \begin{aligned}[b]
    \mathring{F}_{ij} ={} & \ddot{ℰ}₂(λ₀; v_i, v_j) + \dot{ℰ}₂(λ₀; v_i, w_j) + \dot{ℰ}₂(λ₀; v_j, w_i)\\
    ={} & \ddot{ℰ}₂(λ₀; v_i, v_j) + \dot{ℰ}₂(λ₀; v_i, w_j) - ℰ₂(λ₀; w_j, w_i) & \text{Eq.~\eqref{eq:20220524134525}, with \(v_i = v_j\) and \(\hat{w} = w_i\)}\\
    ={} & \ddot{ℰ}₂(λ₀; v_i, v_j)  + 2\dot{ℰ}₂(λ₀; v_i, w_j) & \text{Eq.~\eqref{eq:20220524134525}, with \(\hat{w} = w_j\)}\\
    ={} & \ddot{ℰ}₂(λ₀; v_i, v_j) + 2\dot{ℰ}₂(λ₀; v_j, w_i), & \text{symmetry w.r.t. \(i\) and \(j\)}
  \end{aligned}
\end{equation}
and, from Eq.~\eqref{eq:20220519164523}
\begin{equation}
  \label{eq:20220617085256}
  \begin{aligned}[b]
  \mathring{E}_{ijk} ={}& \dot{ℰ}₃(λ₀; v_i, v_j, v_k) + \dot{ℰ}₂(λ₀; v_i, w_{jk}) + \dot{ℰ}₂(λ₀; v_j, w_{ik}) + \dot{ℰ}₂(λ₀; v_k, w_{ij})\\
  ={}& \dot{ℰ}₃(λ₀; v_i, v_j, v_k) - \bigl[ℰ₂(λ₀; w_i, w_{jk}) + ℰ₂(λ₀; w_j, w_{ik}) + ℰ₂(λ₀; w_k, w_{ij})\bigr].
  \end{aligned}
\end{equation}

\section{Analysis of the critical point}
\label{sec:20220802061621}

In this section, we discuss the stability of the critical point \((u₀, λ₀)\). To
this end, we evaluate the potential energy in a neighboring state
\(u₀ + u\), where \(u ∈ U\) is ``small''. We have, to the fourth order
\begin{equation}
  \begin{aligned}[b]
    ℰ(u₀ + u, λ₀) - ℰ(u₀, λ₀) ={}
    &\tfrac{1}{2} ℰ₂(λ₀; u, u) + \tfrac{1}{6} ℰ₃(λ₀; u, u, u)\\
    &+ \tfrac{1}{24} ℰ₄(λ₀; u, u, u, u) + o(〈 u , u 〉²),
  \end{aligned}
\end{equation}
where the linear term has been omitted, \(u₀\) being a critical point of the
energy. Since \(v ∈ V\), we have \(ℰ₂(λ₀; v, •) = 0\). We now expand \(u\) as
\(u = ξ v + η w\), with \(ξ, η ∈ \reals\) and \(v ∈ V\) and \(w ∈ W\) are fixed,
orthogonal directions. Owing to the multi-linearity and symmetry of the
successive differential of \(ℰ\), the above expression expands as
follows
\begin{equation}
  \begin{aligned}[b]
    ℰ(u₀ + u, λ₀) - ℰ(u₀, λ₀) ={}
    & \tfrac{1}{2} η² ℰ₂(λ₀; w, w) + \tfrac{1}{6} ξ³ ℰ₃(λ₀; v, v, v)\\
    & + \tfrac{1}{2} ξ² η ℰ₃(λ₀; v, v, w) + \tfrac{1}{2} ξ η² ℰ₃(λ₀; v, w, w)\\
    & + \tfrac{1}{6} η³ ℰ₃(λ₀; w, w, w) + \tfrac{1}{24} ξ⁴ ℰ₄(λ₀; v, v, v, v)\\
    & + \tfrac{1}{6} ξ³ η ℰ₄(λ₀; v, v, v, w) + \tfrac{1}{4} ξ² η² ℰ₄(λ₀; v, v, w, w)\\
    & + \tfrac{1}{6} ξ η³ ℰ₄(λ₀; v, w, w, w) + \tfrac{1}{24} η⁴ ℰ₄(λ₀; w, w, w, w)\\
    & + o\bigl[\bigl(ξ² + η²\bigr)²\bigr].
  \end{aligned}
\end{equation}

For the equilibrium to be stable, the above expression must be \(≥ 0\) for all
\(ξ\) et \(η\) small enough. Taking first \(η = 0\), we get the following necessary conditions
\begin{equation}
  \label{eq:20211108164416}
  ℰ₃(λ₀; v, v, v) = 0 \quad \text{and} \quad ℰ₄(λ₀; v, v, v, v) \geq 0 \quad \text{for all} \quad v∈V.
\end{equation}

\begin{remark}
  Note that, from Theorem~\ref{thr:20220802112835}, the first of these two
  conditions is equivalent to \(E_{ijk}=0\), for all \(i, j, k = 1, \ldots m\).
\end{remark}

In other words, if there exists \(v ∈ V\) such that \(ℰ₃(λ₀; v, v, v) \neq 0\)
or \(ℰ₄(v, v, v, v) < 0\), then the equilibrium is \emph{unstable} at the
critical point. The above conditions are not sufficient. Indeed, assuming
conditions~\eqref{eq:20211108164416} to hold, we now take \(η = ξ²\)
\begin{equation}
  \begin{aligned}[b]
    ℰ(u₀ + u, λ₀) - ℰ(u₀, λ₀) ={} & \tfrac{1}{2} ξ⁴ \bigl[ ℰ₂(λ₀; w, w) + ℰ₃(λ₀; v, v, w)\\
    & + \tfrac{1}{12} ℰ₄(λ₀; v, v, v, v) \bigr] + o(ξ⁴)
  \end{aligned}
\end{equation}
and we get the further necessary condition
\begin{equation}
  \label{eq:20211109145356}
  ℰ₂(w, w) + ℰ₃(v, v, w) + \tfrac{1}{12} ℰ₄(v, v, v, v) \geq 0 \quad \text{for all} \quad v ∈ V \quad \text{and} \quad w ∈ W.
\end{equation}

The direction \(v∈V\) being fixed, the above expression is minimal when \(w\)
satisfies the following variational problem
\begin{equation}
  \label{eq:20211109145224}
  2ℰ₂(w, \hat{w}) +ℰ₃(v, v, \hat{w}) = 0 \quad \text{for all} \quad \hat{w}∈W.
\end{equation}

Expanding \(v ∈ V\) in the \((v_i)\) basis as follows: \(v = ξ_i v_i\), it is
observed that the solution to the above variational problem is
\(w = \tfrac{1}{2} ξ_i ξ_j w_{ij}\), where \(w_{ij}\) is the solution to the
elementary variational problem \eqref{eq:20220519164523}. For this value of
\(w\), condition~\eqref{eq:20211109145356} reads
\begin{equation}
  \bigl[ℰ₄(v_i, v_j, v_k, v_l) - 3ℰ₂(w_{ij}, w_{kl})\bigr] ξ_i ξ_j ξ_k ξ_l \geq 0 \quad \text{for all} \quad ξ_1, \ldots, ξ_m ∈ \reals,
\end{equation}
which, in view of definition~\eqref{eq:20220802081116} of \(E_{ijkl}\), is equivalent to
\begin{equation}
  E_{ijkl} ξ_i ξ_j ξ_k ξ_l \geq 0 \quad \text{for all} \quad ξ_m, \ldots, ξ_m ∈ \reals.
\end{equation}

Note that Eq.~\eqref{eq:20211109145224} implies \(ℰ₄(λ₀; v, v, v, v) ≥ 0\),
which becomes a redundant necessary condition. Indeed, plugging
\(w= ξ_i ξ_j w_{ij}\) into Eq.~\eqref{eq:20211109145224} cancels the first two
terms. To sum up, we have the following \emph{necessary} conditions for
stability
\begin{equation}
  E_{ijk} ξ_i ξ_j ξ_k = 0 \quad \text{and} \quad E_{ijkl} ξ_i ξ_j ξ_k ξ_l ≥ 0 \quad \text{for all} \quad ξ_m, \ldots, ξ_m ∈ \reals.
\end{equation}

\pdfmarkupcomment{Conversely, the following condition is \emph{sufficient} to ensure stability of the critical point}{\`A d\'emontrer}
\begin{equation}
  E_{ijk} ξ_i ξ_j ξ_k = 0 \quad \text{and} \quad E_{ijkl} ξ_i ξ_j ξ_k ξ_l > 0 \quad \text{for all} \quad ξ_m, \ldots, ξ_m ∈ \reals.
\end{equation}

\section{Analysis of bifurcated branches}
\label{sec:20220617075558}

In this section, we show that, besides the fundamental branch \(u^\ast(λ)\),
other (bifurcated) equilibrium branches may pass through the critical point
\((u₀, λ₀)\). The starting point is the characterization of an equilibrium by
the stationarity of the energy, which defines all equilibrium branches as
implicit functions, which can be expanded with respect to some perturbation
parameter.

The first approach (see Sec.~\ref{sec:20220902091527}) relies on the
Lyapunov–Schmidt decomposition of the equilibrium branch over \(V\) and
\(W\). However, this approach leads to tedious derivations. This approach has
historical and pedagogical value: in particular, it provides a meaning to
\(w_i\) and \(w_{ij}\) defined by Eqs.~\eqref{eq:20220524134525} and
\eqref{eq:20220519164523}. In Sec.~\ref{sec:20220902092109}, a more systematic
approach is developed, that I found in the paper by \textcite[][Appendix
A]{chak2018}.

\subsection{The Lyapunov–Schmidt decomposition}
\label{sec:20220902091527}

The following decomposition of the equilibrium state \(u\) at the load-level
\(λ\)is postulated
\begin{equation}
  \label{eq:20220902174235}
  u = u^\ast(λ) + ξ_i v_i + w, \quad \text{with} \quad w ∈ W.
\end{equation}

It follows from the orthogonality of \(V\) and \(W\) that \(〈v_i, w〉 = 0\) for
all \(i=1, \ldots, m\). Stationarity of the energy is expressed as follows
\begin{equation}
  ℰ_{,u}[u^\ast(λ) + ξ_i v_i + w, λ; \hat{u}] = 0, \quad \text{for all} \quad \hat{u} ∈ U
\end{equation}
or, equivalently
\begin{equation}
  \label{eq:20220901120544}
  ℰ_{,u}[u^\ast(λ) + ξ_i v_i + w, λ; \hat{v}] = 0, \quad \text{for all} \quad \hat{v} ∈ V
\end{equation}
and
\begin{equation}
  \label{eq:20220825143616}
  ℰ_{,u}[u^\ast(λ) + ξ_i v_i + w, λ; \hat{w}] = 0, \quad \text{for all} \quad \hat{w} ∈ W.
\end{equation}
The method proceeds in three steps. In \textbf{Step 1},
Eq.~\eqref{eq:20220825143616} is used to define \(w\) as an implicit function of
\(ξ₁\), \dots, \(ξ_m\) and \(λ\). Then, in \textbf{Step 2},
Eq.~\eqref{eq:20220825143616} is used to define \(λ\) as an implicit function of
\(ξ₁\), \dots, \(ξ_m\). Finally, a parametrization \(η\) of \(ξ₁\), \dots
\(ξ_m\) is introduced in \textbf{Step 3} and the Taylor expansion of \(u\) and
\(λ\) with respect to \(η\) is derived. These steps are presented below.

\paragraph{Step 1: \(w\) as a function of \(ξ_i\) and \(λ\)} In this paragraph,
\(\hat{w}\) denotes an arbitrary test function in \(W\). From the implicit
function theorem, Eq.~\eqref{eq:20220825143616} defines a function
\((ξ_1, \ldots, ξ_m, λ) \mapsto w(ξ_1, \ldots, ξ_m, λ)\) in the neighborhood of
\((ξ₁, \ldots, ξ_m, λ) = (0, \ldots, 0, λ₀)\). Why the theorem applies will be
clarified below. Eq.~\eqref{eq:20220825143616} is first differentiated \wrt{}
\(ξ_i\)
\begin{equation}
  \label{eq:20220826140926}
  ℰ_{,uu}(u^\ast + ξ_k v_k + w, λ; v_i + w_{,i}, \hat{w}) = 0.
\end{equation}

Substituting \(ξ_1 = \cdots = ξ_m = 0, λ = λ₀\) in the above equations and
observing that \(ℰ₂(λ₀; v_i, W) = 0\) since \(v_i ∈ V\), we get
\begin{equation}
\label{eq:20220825150219}
  ℰ₂(λ₀; v_i + w_{,i}, \hat{w}) = ℰ₂(λ₀; w_{,i}, \hat{w}) = 0.
\end{equation}

Since \(w ∈ W\) for all \(ξ^i\) and \(λ\), we have \(w_{,i} ∈ W\) and,
Remark~\ref{rem:20220902095055} leads to \(w_{,i} = 0\) at the point
\(ξ₁ = 0, \ldots, ξ_m = 0\) and \(λ = λ₀\). Eq.~\eqref{eq:20220825143616} is
then differentiated \wrt{} \(λ\)
\begin{equation}
  \label{eq:20220830145945}
  ℰ_{,uu}(u^\ast + ξ_i v_i + w, λ; \dot{u}^\ast + w_{,λ}, \hat{w}) + ℰ_{,uλ}(u^\ast + ξ_i v_i + w, λ; \hat{w}) = 0
\end{equation}
and, at \(ξ₁ = \ldots = ξ_m = 0\)
\begin{equation}
  \label{eq:20220830151513}
  ℰ_{,uu}(u^\ast, λ; w_{,λ}, \hat{w})
  + \underbrace{ℰ_{,uu}(u^\ast, λ; \dot{u}^\ast, \hat{w}) + ℰ_{,uλ}(u^\ast, λ; \hat{w})}_{=0 \quad \text{see Eq.~\eqref{eq:20220901143843}}}
  = ℰ₂(λ; w_{,λ}, \hat{w}) = 0,
\end{equation}
which proves similarly that the derivative of \(w\) with respect to \(λ\)
vanishes at the critical point. We have found so far that
\begin{equation}
  \frac{∂w}{∂ξ_1} \biggr\rvert_{ξ_1 = \cdots = ξ_m = 0, λ = λ₀}
  = \ldots =
  \frac{∂w}{∂ξ_m} \biggr\rvert_{ξ_1 = \cdots = ξ_m = 0, λ = λ₀}
  = \frac{∂w}{∂λ} \biggr\rvert_{ξ_1 = \cdots = ξ_m = 0, λ = λ₀}= 0.
\end{equation}

To express the second-order derivatives of \(w\), Eq.~\eqref{eq:20220826140926}
is differentiated first with respect to \(ξ_j\), then with respect to
\(λ\). This delivers
\begin{equation}
  ℰ_{,uuu}(u^\ast + ξ_k v_k + w, λ; v_i + w_{,i}, v_j + w_{,j}, \hat{w}) + ℰ_{,uu}(u^\ast + ξ_k v_k + w, λ; w_{,ij}, \hat{w}) = 0
\end{equation}
and
\begin{equation}
  \begin{aligned}[b]
    ℰ_{,uuu}(u^\ast + ξ_k v_k + w, λ; v_i + w_{,i}, \dot{u}^\ast + w_{,λ}, \hat{w}) &\\
    + ℰ_{,uuλ}(u^\ast + ξ_k v_k + w, λ; v_i + w_{,i}, \hat{w}) + ℰ_{,uu}(u^\ast + ξ_k v_k + w, λ; w_{,iλ}, \hat{w}) &= 0
  \end{aligned}
\end{equation}
and, at \(ξ_1 = \cdots = ξ_m = 0, λ = λ₀\) (recalling that, at this point,
\(w_{,1} = \cdots = w_{, m} = w_{,λ} = 0\))
\begin{equation}
  ℰ_3(λ₀; v_i, v_j, \hat{w}) + ℰ₂(λ₀; w_{,ij}, \hat{w}) = 0
  \quad \text{and} \quad
  \dot{ℰ}₂(λ₀; v_i, \hat{w}) + ℰ₂(λ₀; w_{,iλ}, \hat{w}) = 0.
\end{equation}
The variational problems~\eqref{eq:20220524134525} and \eqref{eq:20220519164523}
are recognized, leading to
\begin{equation}
  \frac{∂²w}{∂ξ_i ∂ξ_j}\biggr\rvert_{ξ_1 = \cdots = ξ_m = 0, λ = λ₀} = w_{ij}
  \quad\text{and}\quad
  \frac{∂²w}{∂λ ∂ξ_i}\biggr\rvert_{ξ_1 = \cdots = ξ_m = 0, λ = λ₀} = w_{i}.
\end{equation}

The \(w_i\) and \(w_{ij}\) defined by the variational
problems~\eqref{eq:20220524134525} and \eqref{eq:20220519164523} therefore
appear as the second-order derivatives of \(w\) at \(ξ_k = 0\) and \(λ = λ_0\),
with respect to \(λ\), \(ξ_i\) and \(ξ_i\), \(ξ_j\).

Finally, differentiating Eq.~\eqref{eq:20220830151513} \wrt{} \(λ\) leads to
\begin{equation}
  \dot{ℰ}₂(λ; w_{,λ}, \hat{w}) + ℰ₂(λ; w_{,λλ}, \hat{w}) = 0
\end{equation}
and, at \(λ = λ₀\)
\begin{equation}
  \frac{∂²w}{∂λ²}\biggr\rvert_{ξ_1 = \cdots = ξ_m = 0, λ = λ₀} = 0.
\end{equation}

We have obtained the following Taylor expansion of the component \(w\) of the
LSK expansion of \(u\)
\begin{equation}
  w(ξ_1, \ldots, ξ_m, λ) = \tfrac{1}{2} ξ_i ξ_j w_{ij} + \bigl( λ - λ₀ \bigr) ξ_i w_i + o\Bigl(ξ₁² + \cdots + ξ_m² + \bigl(λ - λ₀\bigr)^2\Bigr).
\end{equation}

\paragraph{Step 2: \(λ\) as a function of \(ξ_i\)} We now turn to
Eq.~\eqref{eq:20220901120544}. Since \(w\) is a function of \(λ\) and \(ξ_k\)
(\(k = 1, \ldots, m\)) this equation implicitly defines \(λ\) as a function of
\(ξ_k\), the derivatives of which can be evaluated at \(ξ₁ = \cdots = ξ_m =
0\). In this paragraph, \(\hat{v}\) denotes an arbitrary element of
\(V\). Besides, unless otherwise mentioned, the differentials of the energy
\(ℰ_{,uu}\), \(ℰ_{,uλ}\), \(ℰ_{,λλ}\), \(ℰ_{,uuu}\) \dots{} are evaluated at
\(u = u^\ast(λ) + ξ_k v_k + w(ξ_k, λ)\). Differentiating first
Eq.~\eqref{eq:20220901120544} with respect to \(ξ_i\)
\begin{equation}
  \label{eq:20220901121940}
  ℰ_{,uu}[v_i + w_{,i} + λ_{,i} (\dot{u}^\ast + w_{,λ}), \hat{v}] + λ_{, i} ℰ_{,uλ}(\hat{v}) = 0,
\end{equation}
then with respect to \(ξ_j\)
\begin{equation}
  \label{eq:20220901125230}
  \begin{gathered}[b]
    ℰ_{,uuu}[v_i + w_{,i} + λ_{,i} (\dot{u}^\ast + w_{,λ}), v_j + w_{,j} + λ_{,j} (\dot{u}^\ast + w_{,λ}), \hat{v}]\\
    + λ_{,j}ℰ_{,uuλ}[v_i + w_{,i} + λ_{,i} (\dot{u}^\ast + w_{,λ}), \hat{v}]\\
    + ℰ_{,uu}[w_{,ij} + λ_{,ij} (\dot{u}^\ast + w_{,λ}) + λ_{,i}λ_{,j} (\ddot{u}^\ast + w_{,λλ}), \hat{v}]\\
    + λ_{, ij} ℰ_{,uλ}(\hat{v}) + λ_{, i} ℰ_{,uuλ}[v_j + w_{,j} + λ_{,j} (\dot{u}^\ast + w_{,λ}), \hat{v}] + λ_{,i} λ_{,j} ℰ_{,uλλ}(\hat{v})= 0,
  \end{gathered}
\end{equation}

Eqs.~\eqref{eq:20220901121940} and \eqref{eq:20220901125230} are then evaluated
at \(ξ₁ = \cdots = ξ_m = 0\), delivering
\begin{equation}
  \label{eq:20220901152056}
  \underbrace{ℰ_{,uu}(u₀, λ₀; v_i, \hat{v})}_{=0 \text{ since } \hat{v} ∈ V}
  + λ_{, i} \bigl[ \underbrace{ℰ_{,uu}(u₀, λ₀; \dot{u}₀, \hat{v}) +  ℰ_{,uλ}(u₀, λ₀; \hat{v})}_{ = 0 \text{ from Eq.~\eqref{eq:20220901143843}}} \bigr] = 0,
\end{equation}
and
% \begin{equation}
%   \begin{gathered}[b]
%     ℰ_{,uuu}(u₀, λ₀; v_i + λ_{,i}\dot{u}₀, v_j + λ_{,j} \dot{u}₀, \hat{v}) + λ_{,j}ℰ_{,uuλ}(u₀, λ₀; v_i + λ_{,i} \dot{u}₀, \hat{v})\\
%     + ℰ_{,uu}(u₀, λ₀; w_{ij} + λ_{,ij} \dot{u}₀ + w_{,λ} + λ_{,i}λ_{,j} \ddot{u}₀, \hat{v})\\
%     + λ_{, ij} ℰ_{,uλ}(u₀, λ₀; \hat{v}) + λ_{, i} ℰ_{,uuλ}(u₀, λ₀; v_j + λ_{,j} \dot{u}₀, \hat{v}) + λ_{,i} λ_{,j} ℰ_{,uλλ}(u₀, λ₀; \hat{v}) = 0
%   \end{gathered}
% \end{equation}
\begin{equation}
  \label{eq:20220901152145}
  \begin{gathered}[b]
    ℰ_{,uuu}(u₀, λ₀; v_i , v_j, \hat{v}) + \underbrace{ℰ_{,uu}(u₀, λ₀; w_{ij}, \hat{v})}_{=0 \text{ since } \hat{v} ∈ V}\\
    +λ_{,i} \bigl[ℰ_{,uuu}(u₀, λ₀; v_j , \dot{u}₀, \hat{v}) + ℰ_{,uuλ}[u₀, λ₀; v_j, \hat{v}]\bigr]\\
    +λ_{,j} \bigl[ℰ_{,uuu}(u₀, λ₀; v_i , \dot{u}₀, \hat{v}) + ℰ_{,uuλ}(u₀, λ₀; v_i, \hat{v})\bigr]\\
    +λ_{,ij} \bigl[ \underbrace{ℰ_{,uu}(u₀, λ₀;  \dot{u}₀, \hat{v}) + ℰ_{,uλ}(u₀, λ₀; \hat{v})}_{ = 0 \text{ from Eq.~\eqref{eq:20220901143843}}} \bigr]\\
    +λ_{,i} λ_{,j}\bigl[ \underbrace{ℰ_{,uuu}(u₀, λ₀; \dot{u}₀ , \dot{u}₀, \hat{v}) + 2ℰ_{,uuλ}(u₀, λ₀; \dot{u}₀, \hat{v}) + ℰ_{,uλλ}(u₀, λ₀; \hat{v}) + ℰ_{,uu}(u₀, λ₀; \ddot{u}₀, \hat{v})}_{ = 0 \text{ from Eq.~\eqref{eq:20220901143902}}} \bigr] = 0
  \end{gathered}
\end{equation}

Eq.~\eqref{eq:20220901152056} is non-informative (identically
satisfied), while Eq.~\eqref{eq:20220901152145} simplifies as follows
\begin{equation}
  \begin{aligned}[b]
    ℰ_{,uuu}(u₀, λ₀; v_i , v_j, \hat{v}) + λ_{,i} \bigl[ \underbrace{ℰ_{,uuu}(u₀, λ₀; v_j , \dot{u}₀, \hat{v}) + ℰ_{,uuλ}(u₀, λ₀; v_j, \hat{v})}_{=\dot{ℰ}₂(λ₀; v_j, \hat{v})} \bigr]&\\
    +λ_{,j} \bigl[ \underbrace{ℰ_{,uuu}(u₀, λ₀; v_i , \dot{u}₀, \hat{v}) + ℰ_{,uuλ}(u₀, λ₀; v_i, \hat{v})}_{λ_{,j} \dot{ℰ}₂(λ₀; v_i, \hat{v})} \bigr] &= 0
  \end{aligned}
\end{equation}
and, recognizing derivatives of \(ℰ₂\) with respect to \(λ\), we finally get
\begin{equation}
    ℰ₃(λ₀; v_i , v_j, \hat{v}) + λ_{,i} \dot{ℰ}₂(λ₀; v_j, \hat{v}) + λ_{,j} \dot{ℰ}₂(λ₀; v_i, \hat{v}) = 0.
\end{equation}
Testing with \(v_k ∈ V\), the above equation reads
\begin{equation}
  ℰ₃(λ₀; v_i , v_j, v_k) + λ_{,i} \dot{ℰ}₂(λ₀; v_j, v_k) + λ_{,j} \dot{ℰ}₂(λ₀; v_i, v_k) = 0,
\end{equation}
or, with Eqs.~\eqref{eq:20220524135619} and \eqref{eq:20220524135643}
\begin{equation}
  \label{eq:20220902125031}
  E_{ijk} +  F_{jk} \frac{∂λ}{∂ξ_i} \biggr\rvert_{ξ_1 = \cdots = ξ_m = 0} + F_{ik} \frac{∂λ}{∂ξ_j} \biggr\rvert_{ξ_1 = \cdots = ξ_m = 0} = 0.
\end{equation}

In order to evaluate the second order partial derivatives of \(λ\),
Eq.~\eqref{eq:20220901125230} should be further differentiated with respect to
\(ξ_k\). This leads to extremely tedious derivations, and we will adopt an
alternative approach in Sec.~\ref{sec:20220902092109}.

\paragraph{Step 3: parametrization of the bifurcated branch} The bifurcated
branch is a curve \((u, λ) ∈ ℝ ^ {m + 1}\), which is parametrized by \(η\):
\([u(η), λ(η)]\), with \(u(0) = u₀\) and \(λ(0) = λ₀\); primed quantities
denoting derivatives with respect to \(η\), we introduce
\begin{equation}
  \order[1]{ξ_i} = ξ_i'(0), \quad
  \order[2]{ξ_i} = ξ_i''(0), \quad \ldots, \quad
  \order[1]{λ} = λ'(0), \quad \ldots
\end{equation}
and first observe that
\begin{equation}
  \order[1]{λ} = \order[1]{ξ_i} \frac{∂λ}{∂ξ_i} \biggr\rvert_{ξ_1 = \cdots = ξ_m = 0}
\end{equation}

Multiplying both sides of Eq.~\eqref{eq:20220902125031} by
\(\order[1]{ξ_i} \order{1}{ξ_j}\) therefore results in the following identity
\begin{equation}
  \begin{aligned}[b]
    0 &= E_{ijk} \order[1]{ξ_i} \order[1]{ξ_j} +  F_{jk} \order[1]{ξ_i} \order[1]{ξ_j} \frac{∂λ}{∂ξ_i} \biggr\rvert_{ξ_1 = \cdots = ξ_m = 0} + F_{ik} \order[1]{ξ_i} \order[1]{ξ_j} \frac{∂λ}{∂ξ_j} \biggr\rvert_{ξ_1 = \cdots = ξ_m = 0}\\
    &= E_{ijk} \order[1]{ξ_i} \order[1]{ξ_j} +  F_{jk} \order[1]{λ} \order[1]{ξ_j} + F_{ik} \order[1]{ξ_i} \order[1]{λ}
  \end{aligned}
\end{equation}
and, rearranging
\begin{equation}
  E_{ijk} \order[1]{ξ_j} \order[1]{ξ_k} +  2 \order[1]{λ} F_{ij}  \order[1]{ξ_j} = 0,
\end{equation}
to be compared with Eq.~\eqref{eq:20220524135036}. We now turn to \(w\)
\begin{equation}
  w'(η) = w_{,i} ξ_i' + w_{,λ} λ'
  \quad \text{and} \quad
  w''(η) = w_{,ij} ξ_i' ξ_j' + 2 w_{,iλ} ξ_i' λ' + w_{,i} ξ_i'' + w_{,λλ} λ^{'2} + w_{,λ} λ''
\end{equation}
and, at \(η = 0\)
\begin{equation}
  w'(0) = 0 \quad \text{and} \quad w''(0) = \order[1]{ξ_i} \order[1]{ξ_j} w_{ij}  + 2 \order[1]{λ} \order[1]{ξ_i} w_i
\end{equation}
and we get the Taylor expansion of the bifurcated branch as \(η → 0\)
\begin{equation}
  u(η) = u^\ast[λ(η)] + \order[1]{ξ_i} v_i + \tfrac{1}{2} \bigl( \order[2]{ξ_i} v_i + \order[1]{ξ_i} \order[1]{ξ_j} w_{ij}  + 2\order[1]{λ} \order[1]{ξ_i} w_i\bigr) + o(η²),
\end{equation}
to be compared with Eq.~\eqref{eq:20220524134613}.

\subsection{Alternative route to the asymptotic expansions}
\label{sec:20220902092109}

Following the Appendix A of Ref.~\parencite{chak2018}, we introduce the
following parametrization of the bifurcated branch
\begin{align}
  \label{eq:20211115075817}
  λ &=  λ₀ + η \order[1]{λ} + \tfrac{1}{2} η² \order[2]{λ} + \tfrac{1}{6} η³ \order[3]{λ} + \cdots,\\
  \label{eq:20211115075835}
  u &= u^{\ast}(λ) + η \order[1]{u} + \tfrac{1}{2} η² \order[2]{u} + \tfrac{1}{6} η³ \order[3]{u} + \cdots,
\end{align}
where the parameter \(η\) is not specified, but for the fact that \(η = 0\)
corresponds to the critical point \((u₀, λ₀)\). Note that, in
Eq.~\eqref{eq:20211115075835}, \(u^\ast\) is evaluated at \(λ\) rather than
\(λ_0\).

Les coefficients \(λ_k\) et \(u_k\) des développements~\eqref{eq:20211115075817}
et \eqref{eq:20211115075835} sont identifiés en écrivant que l'énergie est
stationnaire le long de la courbe d'équilibre, c'est-à-dire que le résidu
\(ℰ_{, u} [u(η), λ(η)]\) est nul. Le développement limité du résidu est établi
au voisinage de \(η = 0\) dans l'annexe~\ref{sec:20211112182000} [voir
Éq.~\eqref{eq:20220107080901}]. En écrivant que tous ses termes s'annulent, on
trouve successivement, pour tout \(\hat{u}∈U\)
\begin{equation}
  \label{eq:20211112182917}
  ℰ₂(λ₀; \order[1]u, \hat{u}) = 0,
\end{equation}
\begin{equation}
  \label{eq:20220524133447}
  ℰ₃(λ₀; \order[1]u, \order[1]u, \hat{u}) + 2\order[1]λ\dot{ℰ}₂(λ₀; \order[1]u, \hat{u}) + ℰ₂(λ₀; \order[2]u, \hat{u}) = 0,
\end{equation}
\begin{equation}
  \label{eq:20220708060436}
  \begin{aligned}[b]
    ℰ₄(λ₀; \order[1]u, \order[1]u, \order[1]u, \hat{u}) + 3ℰ₃(λ₀; \order[1]u, \order[2]u, \hat{u}) + ℰ₂(λ₀; \order[3]u, \hat{u})&\\
    + 3\order[1]λ\dot{ℰ}₃(λ₀; \order[1]u, \order[1]u, \hat{u}) + 3\order[1]λ\dot{ℰ}₂(λ₀;  \order[2]u, \hat{u})&\\
    + 3(\order[1]λ)^2\ddot{ℰ}₂(λ₀; \order[1]u, \hat{u}) + 3\order[2]λ\dot{ℰ}₂(λ₀; \order[1]u, \hat{u}) & = 0.
  \end{aligned}
\end{equation}
On déduit de l'équation~\eqref{eq:20211112182917} que \(\order[1]u∈V\). En prenant la
fonction test également dans \(V\), on déduit de
l'équation~\eqref{eq:20220524133447} que \(\order[1]u\) est solution du problème
suivant~: trouver \(\order[1]u∈V\) tel que
\begin{equation}
  \label{eq:20220524133816}
  \tfrac{1}{2} ℰ₃(λ₀; \order[1]u, \order[1]u, \hat{v}) + \order[1]λ\dot{ℰ}₂(λ₀; \order[1]u, \hat{v}) = 0,
\end{equation}
pour tout \(\hat{v}∈V\). The above problem can be transformed into a system of
scalar equations. Indeed, expanding the \(\order[1]u∈V\) in the basis
\((v_i)_{1 ≤ i ≤ m}\) as follows
\begin{equation}
  \label{eq:20220524133944}
  \order[1]u = \order[1]{ξ_i} v_i
\end{equation}
and plugging the definitions~\eqref{eq:20220524135619} and
\eqref{eq:20220524135643} of \(E_{ijk}\) and \(F_{ij}\) into
Eq.~\eqref{eq:20220524133816}
\begin{equation}
  \label{eq:20220524135036}
  \tfrac{1}{2} E_{ijk} \order[1]{ξ_j} \order[1]{ξ_k} + \order[1]λ F_{ij} \order[1]{ξ_j} = 0.
\end{equation}

On obtient ainsi un système de \(m\) équations quadratiques à \((m + 1)\)
inconnues, qui permet en général de déterminer les valeurs de \(\order[1]λ\) et
\(\order[1]u\)(\pdfmarkupcomment{voir discussion ci-après}{Compléter référence}).

Afin de déterminer les termes suivants du développement asymptotique de la
branche bifurquée, soit \(\order[2]λ\) et \(\order[2]u\), on introduit la décomposition
\begin{equation}
  \order[2]u = \order[2]{ξ_i} v_i + \order[2]w,
\end{equation}
où \(\order[2]w ∈ W\) est la projection orthogonale de \(\order[2]u\) sur
\(W\). On a alors \(ℰ₂(\order[2]u, \hat{u}) =ℰ₂(\order[2]{w}, \hat{u})\) et
l'équation~\eqref{eq:20220524133447} s'écrit
\begin{equation}
 ℰ₃(λ₀; \order[1]u, \order[1]u, \hat{u}) + 2\order[1]λ \dot{ℰ}₂(λ₀; \order[1]u, \hat{u}) + ℰ₂(λ₀; \order[2]w, \hat{u}) = 0,
\end{equation}
pour tout \(\hat{u}∈U\). En prenant cette fois-ci la fonction test dans l'espace
\(W\), on obtient le problème variationnel suivant~: trouver \(\order[2]w∈W\)
tel que
\begin{equation}
  \label{eq:20211210131623}
  ℰ₂(λ₀; \order[2]w, \hat{w}) + \order[1]{ξ_i} \order[1]{ξ_j} ℰ₃(λ₀; v_i, v_j, \hat{w}) + 2\order[1]λ \order[1]{ξ_i} \dot{ℰ}₂(λ₀; v_i, \hat{w}) = 0,
\end{equation}
pour tout \(\hat{w}∈W\). The solution to the variational
problem~\eqref{eq:20211210131623} is expressed as a linear combination of the
\(w_i\) and \(w_{ij}\) [defined by the variational
problems~\eqref{eq:20220524134525} and \eqref{eq:20220519164523}]: \(\order[2]w = \order[1]{ξ_i} \order[1]{ξ_j} w_{ij} + 2\order[1]λ \order[1]{ξ_i} w_i\) and
\begin{equation}
  \label{eq:20220524134613}
  \order[2]u = \order[2]{ξ_i} v_i + \order[1]{ξ_i} \order[1]{ξ_j} w_{ij} + 2\order[1]λ \order[1]{ξ_i} w_i.
\end{equation}

Plugging expressions~\eqref{eq:20220524133944} and \eqref{eq:20220524134613}
into Eq.~\eqref{eq:20220708060436} and taking further \(\hat{u} = v_i\)
[remember that \(ℰ₂(λ₀; v_i, •) = 0\)], we then get
% \begin{multline*}
%   ℰ₄(λ₀; v_i, \order[1]{ξ_j} v_j, \order[1]{ξ_k} v_k, \order[1]{ξ_l} v_l) + 3ℰ₃(λ₀; v_i, \order[1]{ξ_j} v_j, \order[2]{ξ_k} v_k + \order[1]{ξ_k} \order[1]{ξ_l} w_{kl} + \order[1]λ \order[1]{ξ_k} w_k)\\
%   + 3\order[1]λ \dot{ℰ}₃(λ₀; v_i, \order[1]{ξ_j} v_j, \order[1]{ξ_k} v_k) + 3\order[1]λ \dot{ℰ}₂(λ₀; v_i, \order[2]{ξ_j} v_j + \order[1]{ξ_j} \order[1]{ξ_k} w_{jk} + \order[1]λ \order[1]{ξ_j} w_j)\\
%   + 3\bigl( \order[1]λ \bigr)^2 \ddot{ℰ}₂(λ₀; v_i, \order[1]{ξ_j} v_j) + 3\order[2]λ \dot{ℰ}₂(λ₀; v_i, \order[1]{ξ_j} v_j) = 0
% \end{multline*}
% \begin{multline*}
%   ℰ₄(λ₀; v_i, v_j, v_k, v_l) \order[1]{ξ_j} \order[1]{ξ_k} \order[1]{ξ_l} + 3ℰ₃(λ₀; v_i, v_j, v_k) \order[1]{ξ_j} \order[2]{ξ_k} + 3ℰ₃(λ₀; v_i, v_j, w_{kl}) \order[1]{ξ_j} \order[1]{ξ_k} \order[1]{ξ_l}\\
%   + 3\order[1]λ ℰ₃(λ₀; v_i, v_j, w_k) \order[1]{ξ_j} \order[1]{ξ_k} + 3\order[1]λ \dot{ℰ}₃(λ₀; v_i, v_j, v_k) \order[1]{ξ_j} \order[1]{ξ_k} + 3\order[1]λ \dot{ℰ}₂(λ₀; v_i, v_j) \order[2]{ξ_j}\\
%   + 3\order[1]λ \dot{ℰ}₂(λ₀; v_i, w_{jk}) \order[1]{ξ_j} \order[1]{ξ_k} + 3\bigl( \order[1]λ \bigr)^2 \dot{ℰ}₂(λ₀; v_i, w_j) \order[1]{ξ_j} + 3\bigl( \order[1]λ \bigr)^2 \ddot{ℰ}₂(λ₀; v_i, v_j) \order[1]{ξ_j} + 3\order[2]λ \dot{ℰ}₂(λ₀; v_i, v_j) \order[1]{ξ_j} = 0
% \end{multline*}
\begin{multline*}
  \bigl[ℰ₄(λ₀; v_i, v_j, v_k, v_l) + 3ℰ₃(λ₀; v_i, v_j, w_{kl})\bigr] \order[1]{ξ_j} \order[1]{ξ_k} \order[1]{ξ_l}\\
  + 3\order[1]λ \bigl[ℰ₃(λ₀; v_i, v_j, w_k) + \dot{ℰ}₃(λ₀; v_i, v_j, v_k) + \dot{ℰ}₂(λ₀; v_i, w_{jk}) \bigr] \order[1]{ξ_j} \order[1]{ξ_k}\\
  + 3\bigl[\bigl(\order[1]λ\bigr)^2 \dot{ℰ}₂(λ₀; v_i, w_j) + \bigl(\order[1]λ\bigr)^2 \ddot{ℰ}₂(λ₀; v_i, v_j) + \order[2]λ \dot{ℰ}₂(λ₀; v_i, v_j)\bigr] \order[1]{ξ_j}\\
  + 3\bigl[ℰ₃(λ₀; v_i, v_j, v_k) \order[1]{ξ_k} + \order[1]λ \dot{ℰ}₂(λ₀; v_i, v_j)\bigr] \order[2]{ξ_j} = 0
\end{multline*}
It results from the variational problems \eqref{eq:20220524134525} and
\eqref{eq:20220519164523} that
\begin{equation*}
  ℰ₃(λ₀; v_i, v_j, w_k) = -ℰ₂(λ₀ ; w_{ij}, w_k) = 2\dot{ℰ}₂(λ₀; v_k, w_{ij}),
\end{equation*}
therefore
\begin{equation*}
  \begin{aligned}[b]
    ℰ₃(λ₀; v_i, v_j, w_k) \order[1]{ξ_j} \order[1]{ξ_k} &= \tfrac{1}{2} \bigl[ ℰ₃(λ₀; v_i, v_j, w_k) + ℰ₃(λ₀; v_i, v_k, w_j)\bigr] \order[1]{ξ_j} \order[1]{ξ_k}\\
                                    &= \bigl[ \dot{ℰ}₂(λ₀; v_k, w_{ij}) + \dot{ℰ}₂(λ₀; v_j, w_{ik}) \bigr] \order[1]{ξ_j} \order[1]{ξ_k}.
  \end{aligned}
\end{equation*}
Similarly,
\begin{equation*}
  \begin{aligned}[b]
    \dot{ℰ}₂(λ₀; v_i, w_j) &= - \tfrac{1}{2} ℰ₂(λ₀; w_i, w_j) = - \tfrac{1}{2} ℰ₂(λ₀; w_j, w_i) = \dot{ℰ}₂(λ₀; v_j, w_i)\\
                           &= \tfrac{1}{2} \bigl[ \dot{ℰ}₂(λ₀; v_i, w_j) + \dot{ℰ}₂(λ₀; v_j, w_i) \bigr].
  \end{aligned}
\end{equation*}
% \begin{multline*}
%   \bigl[ℰ₄(λ₀; v_i, v_j, v_k, v_l) + ℰ₃(λ₀; v_i, v_j, w_{kl}) + ℰ₃(λ₀; v_i, v_k, w_{jl}) + ℰ₃(λ₀; v_i, v_l, w_{jk}) \bigr] \order[1]{ξ_j} \order[1]{ξ_k} \order[1]{ξ_l}\\
%   + 3\order[1]λ \bigl[\dot{ℰ}₃(λ₀; v_i, v_j, v_k) + \dot{ℰ}₂(λ₀; v_i, w_{jk}) + \dot{ℰ}₂(λ₀; v_j, w_{ik}) + \dot{ℰ}₂(λ₀; v_k, w_{ij}) \bigr] \order[1]{ξ_j} \order[1]{ξ_k}\\
%   + 3\bigl( \order[1]λ \bigr)^2 \bigl\{ \ddot{ℰ}₂(λ₀; v_i, v_j) + \tfrac{1}{2} \bigl[ \dot{ℰ}₂(λ₀; v_i, w_j) + \dot{ℰ}₂(λ₀; v_j, w_i) \bigr] \bigr\} \order[1]{ξ_j}\\
%   + 3\bigl[ℰ₃(λ₀; v_i, v_j, v_k) \order[1]{ξ_k} + \order[1]λ \dot{ℰ}₂(λ₀; v_i, v_j)\bigr] \order[2]{ξ_j} + 3\order[2]λ \dot{ℰ}₂(λ₀; v_i, v_j) \order[1]{ξ_j} = 0
% \end{multline*}

Finally, the definitions \eqref{eq:20220615063626}, \eqref{eq:20220615063633},
\eqref{eq:20220524135619}, \eqref{eq:20220524135553} and
\eqref{eq:20220524135643} of \(E_{ijk}\), \(E_{ijkl}\), \(F_{ij}\),
\(\mathring{E}_{ijk}\) and \(\mathring{F}_{ij}\) lead to the following compact
bifurcation equation
\begin{equation}
  \label{eq:20220601070917}
  \tfrac{1}{3} E_{ijkl} \order[1]{ξ_j} \order[1]{ξ_k} \order[1]{ξ_l} + \order[1]λ \bigl( \mathring{E}_{ijk} \order[1]{ξ_k} + \order[1]λ \mathring{F}_{ij} \bigr)\order[1]{ξ_j} + \bigl(E_{ijk} \order[1]{ξ_k} + \order[1]λ F_{ij}\bigr) \order[2]{ξ_j} + \order[2]λ F_{ij} \order[1]{ξ_j} = 0.
\end{equation}

In order to analyse the stability of the bifurcated branches thus found, one
must look at the Hessian of the energy. It is first observed that, on the
fundamental branch
\begin{equation}
 ℰ₂(λ; \hat{u}, \hat{v}) = ℰ₂(λ₀; \hat{u}, \hat{v}) + \bigl(λ - λ₀\bigr) \dot{ℰ}₂(λ₀; \hat{u}, \hat{v}) + o(λ - λ₀).
\end{equation}

In what follows, it will be assumed that \(\dot{ℰ}₂(λ₀)≠0\) and that \(ℰ₂(λ)\)
(which is positive definite over \(V\) for \(λ<λ₀\) and null for \(λ=λ₀\)) is
negative definite for \(λ>λ₀\) sufficiently small (the fundamental branch is
strictly unstable beyond the critical load). From the above expansion, it
results that \(\dot{ℰ}₂(λ₀)\) is negative definite over \(V\). In other words,
\(-F_{ij}\) is a positive definite tensor. The asymptotic expansion of the
Hessian of the energy along the bifurcated branch is derived in
appendix~\ref{sec:20220616055207}. For all \(\hat{u}, \hat{v}∈U\)
\begin{multline}
  \label{eq:20220531054247}
  ℰ_{, uu}[u(η), λ(η); \hat{u}, \hat{v}] = ℰ₂(λ₀ ; \hat{u}, \hat{v}) + η \bigl[ℰ₃(λ₀ ; \order[1]u, \hat{u}, \hat{v})  + \order[1]λ \dot{ℰ}₂(λ₀; \hat{u}, \hat{v})\bigr]\\
  + \tfrac{1}{2} η² \bigl[ℰ₄(λ₀; \order[1]u, \order[1]u, \hat{u}, \hat{v}) + ℰ₃(λ₀; \order[2]u, \hat{u}, \hat{v}) + 2\order[1]λ \dot{ℰ}₃(λ₀; \order[1]u, \hat{u}, \hat{v})\\
  + \bigl(\order[1]λ\bigr)² \ddot{ℰ}₂(λ₀; \hat{u}, \hat{v}) + \order[2]λ \dot{ℰ}₂(λ₀; \hat{u}, \hat{v}) \bigr] + o(η²).
\end{multline}

Stability analysis is performed by means of the eigenvalues \(α ∈ \reals\) and
eigenvectors \(x ∈ U\) of the Hessian
\begin{equation}
  \label{eq:20220617074949}
  ℰ_{, u u} [u(η), λ(η); x, \hat{u}] = α 〈 x, \hat{u} 〉 \quad \text{for all} \quad \hat{u} ∈ V,
\end{equation}
where \(α\) and \(x\) are expanded to second order in \(η\)
\begin{equation}
  \label{eq:20220617064633}
  α = \order[0]α + η \order[1]α + \tfrac{1}{2} η² \order[2]α + o(η²)
  \quad \text{and} \quad
  x = \order[0]x + η \order[1]x + \tfrac{1}{2} η² \order[2]x + o(η²).
\end{equation}

The following results are proved in Appendix~\ref{sec:20220616074108}: first,
\((\order[0]α, x_0)\) is necessarily an eigenpair of \(ℰ₂(λ₀)\). Since \(ℰ₂ (λ₀)\) is
positive, \(\order[0]α ≥ 0\). If \(\order[0]α>0\), then \(α>0\) in the neighborhood of
\(λ₀\). Potentially unstable modes are therefore such that \(\order[0]α=0\). In other
words, \(\order[0]x ∈ V\) and
\begin{equation}
  \label{eq:20220904160057}
  \order[0]x = \order[0]{χ_i} v_i
\end{equation}
furthermore, \((\order[1]α, \order[0]{χ_i})\) is an eigenpair of the symmetric
tensor \((E_{ijk} \order[1]{ξ_k} + \order[1]λ F_{ij})\)
\begin{equation}
  \label{eq:20220609133608}
  \bigl(E_{ijk} \order[1]{ξ_k} + \order[1]λ F_{ij} \bigr) \order[0]{χ_j} = \order[1]α \order[0]{χ_i}.
\end{equation}
As for the higher order terms, it is also found that
\begin{equation}
  \label{eq:20220609133629}
  \order[1]x = \order[1]{χ_i} v_i +  \order[0]{χ_i} \order[1]{ξ_j} w_{i j} + \tfrac{1}{2} \order[1]λ \order[0]{χ_i} w_i
\end{equation}
and
\begin{multline}
  \label{eq:20220616082923}
  \bigl[E_{ijkl} \order[1]{ξ_k} \order[1]{ξ_l} + \order[1]λ\bigl(2 \mathring{E}_{ijk} \order[1]{ξ_k} + \order[1]λ \mathring{F}_{ij}\bigr) + E_{ijk} \order[2]{ξ_k} + \order[2]λ F_{ij} \bigr] \order[0]{χ_j}\\
  + 2\bigl(E_{ijk}  \order[1]{ξ_k} + \order[1]λ F_{ij} \bigr) χ₁^j = 2\order[1]α\order[1]{χ_i} + \order[2]α \order[0]{χ_i}.
\end{multline}

Finally, to close this analysis of the bifurcated branches, the following
asymptotic expansion of the energy is derived in
Appendix~\ref{sec:20220525053434}
\begin{multline}
  \label{eq:20220525053600}
    ℰ[u(η), λ(η)] = ℰ\{u^{\ast}[λ(η)], λ(η)\} + \tfrac{1}{6} \order[1]λ η³ F_{i j} \order[1]{ξ_i} \order[1]{ξ_j}\\
    +\tfrac{1}{24} η⁴ \bigl[E_{ijkl} \order[1]{ξ_i} \order[1]{ξ_j} \order[1]{ξ_k} \order[1]{ξ_l} + 4\order[1]λ \mathring{E}_{ijk} \order[1]{ξ_i} \order[1]{ξ_j} \order[1]{ξ_k} + 6 \bigl(\bigl(\order[1]λ\bigr)^2 \mathring{F}_{ij} + \order[2]λ F_{ij}\bigr) \order[1]{ξ_i} \order[1]{ξ_j}\bigr] + o(η⁴).
\end{multline}

\section{Discussion}

In this section, we discuss the two main cases of bifurcations, namely
\emph{asymmetric} and \emph{symmetric}. In each case, we analyse the stability
of the bifurcated branch.

\begin{remark}
  The boundary case is unclear to me. I think that whether a bifurcation is
  symmetric or asymmetric should depend on the value of \(\order[1]λ\) only. If
  \(\order[1]λ ≠ 0\), the bifurcated branch is \emph{asymmetric}. Conversely, if
  \(\order[1]λ = 0\) and \(\order[2]λ ≠ 0\), then the bifurcated branch is \emph{symmetric}.

  In the literature, the discussion is placed on \(E_{ijk}\). If \(\order[1]λ ≠ 0\),
  surely one of the \(E_{ijk}\) is non-zero also. However, I believe it is
  \emph{not} a sufficient condition: one of the bifurcated branches could be
  symmetric \((\order[1]λ = 0)\), even if all \(E_{ijk}\) are not null. It is true
  however that \emph{all} bifurcated branches are symmetric if, and only if,
  \(E_{ijk}=0\) for all \(i, j, k = 1, \ldots, m\). Therefore, the two cases
  that will be discussed below are: (1) one of the bifurcated branches is
  asymmetric and (2) all bifurcated branches are symmetric. The mixed case ``one
  of the bifurcated branches is symmetric'' will \emph{not} be discussed.
\end{remark}

\subsection{Asymmetric bifurcated branch (\(\order[1]λ ≠ 0\))}

We first consider the situation where \(\order[1]λ ≠ 0\) on the bifurcated
branch. The bifurcation equation~\eqref{eq:20220524135036} shows that
necessarily, \(E_{ijk}\) is not identically nul. This equation has at most
\((2^m - 1)\) pairs of real solutions \((\order[1]λ, \order[1]u)\) et
\((- \order[1]λ, - \order[1]u)\); furthermore, multiplication by
\(\order[1]{ξ_i}\) shows that
\begin{equation}
  \label{eq:20220801085236}
  \order[1]λ = -\frac{E_{ijk} \order[1]{ξ_i} \order[1]{ξ_j} \order[1]{ξ_k}}{2 F_{ij} \order[1]{ξ_i} \order[1]{ξ_j}}.
\end{equation}

\begin{remark}
  I can't prove that the bifurcation equation~\eqref{eq:20220524135036} has at
  most \((2^m - 1)\) pairs of real solutions.
\end{remark}

Along the bifurcated branch, we have \(λ = λ₀ + η \order[1]λ + o(η)\), and \(η\) can be
eliminated. In other words, \(η=λ\) (\(\order[1]λ=1\) and \(\order[2]λ = \order[3]λ = \cdots = 0\)) can
be selected as a parameter. It is therefore possible to express the bifurcated
branch as a function of \(λ\), \(u(λ)\). For example, combining
Eqs.~\eqref{eq:20220524133816} and \eqref{eq:20220531054247}, we find that
\begin{equation}
  \begin{aligned}[b]
    ℰ_{, uu}[u(η), λ(η); \order[1]u, \order[1]u]
    &= η \bigl[ℰ₃(λ₀ ; \order[1]u, \order[1]u, \order[1]u)  + \order[1]λ \dot{ℰ}₂(λ₀; \order[1]u, \order[1]u)\bigr] + o(η)\\
    &= - η \order[1]λ \dot{ℰ}₂(λ₀; \order[1]u, \order[1]u) + o(η),
  \end{aligned}
\end{equation}
or
\begin{equation}
  \label{eq:20220819160235}
  ℰ_{, uu}[u(λ), λ; \order[1]u, \order[1]u] = -\bigl( λ - λ₀ \bigr) \dot{ℰ}₂(λ₀; \order[1]u, \order[1]u) + o(λ - λ₀).
\end{equation}

For \(λ < λ₀\), the above quantity is \emph{negative} (since \(\dot{ℰ}₂\) is
negative definite). In other words

\begin{center}
  \framebox{For asymmetric bifurcations, below the critical load, the bifurcated
    branch is unstable}
\end{center}

To investigate the stability above the critical load, we need to analyse the
sign of the eigenvalues \(α\) of the Hessian. At first order,
\(α = η \order[1]α + o(η)\), where \(\order[1]α\) is an eigenvalue of
\((E_{ijk} \order[1]{ξ_k} + \order[1]λ F_{ij})\). Let \(α_{\min}\) and \(α_{\max}\) be the minimum
and maximum eigenvalues of this second-order tensor. Three cases must be
discussed
\begin{enumerate}
\item If \(α_{\min} α_{\max} > 0\), then \((E_{ijk} \order[1]{ξ_k} + \order[1]λ F_{ij})\) is
  positive or negative definite: all eigenvalues have the same sign,
  \(\epsilon ∈ \{-1, +1\}\). Then the sign of the eigenvalues \(α\) of the
  Hessian is \(\epsilon η\) and there is a stability switch at the critical
  load. Since the bifurcated branch is unstable \emph{below} the critical load,
  this means that it is \emph{stable} above the critical load.
\item If \(α_{\min} α_{\max} < 0\), then the extremal eigenvalues of the Hessian
  are \(η α_{\min}\) and \(η α_{\max}\), the product of which is
  \(η² α_{\min} α_{\max} < 0\). The bifurcated branch is \emph{unstable} for all
  values of \(λ\).
\item If \(α_{\min} α_{\max} = 0\), the analysis is inconclusive.
\end{enumerate}

To close this section, it is observed that the dominant term of the
expansion~\eqref{eq:20220525053600} of the potential energy along the bifurcated
branch is of the third order in \(η\)
\begin{equation}
  ℰ[u(η), λ(η)] = ℰ\{u^{\ast}[λ(η)], λ(η)\} + \tfrac{1}{6} \order[1]λ η³ F_{i j} \order[1]{ξ_i} \order[1]{ξ_j} + o(η³).
\end{equation}

Eliminating \(λ\) and plugging expression~\eqref{eq:20220801085236} of \(\order[1]λ\)
delivers the expression of the potential energy, where \(λ\) is the parameter
\begin{equation}
  \begin{aligned}[b]
    ℰ[u(λ), λ] &= ℰ[u^{\ast}(λ), λ] + \frac{\bigl(λ - λ₀\bigr)³}{6\bigl( \order[1]λ \bigr)^2} F_{i j} \order[1]{ξ_i} \order[1]{ξ_j} + o(λ³)\\
    &= ℰ[u^{\ast}(λ), λ] + \frac{2 \bigl( F_{i j} \order[1]{ξ_i} \order[1]{ξ_j} \bigr)³}{3 \bigl( E_{ijk} \order[1]{ξ_i} \order[1]{ξ_j} \order[1]{ξ_k} \bigr)²} \bigl(λ - λ₀\bigr)³ + o(λ³).
  \end{aligned}
\end{equation}

Recalling that \(F_{i j} \order[1]{ξ_i} \order[1]{ξ_j} < 0\), it is found that, above the critical
load, the potential energy is \emph{smaller} along the bifurcated branch than
along the fundamental branch.

\begin{remark}
  As expected, the above expression does not depend on the scaling of \(\order[1]u\) (of the \(\order[1]{ξ_i}\)).
\end{remark}
\begin{remark}
  It has been shown in Sec.~\ref{sec:20220802061621} that, when \(E_{ijk}\) is
  not identically null, the bifurcation point is \emph{unstable}.
\end{remark}

\subsection{A particular case of symmetric bifurcation}

We now consider the case \(E_{ijk}=0\) for all \(i, j, k = 1, \ldots, m\). Then
[see Eq.~\eqref{eq:20220524135036}] \(\order[1]λ = 0\) on \emph{all} bifurcated
branches. It is assumed that, on the bifurcated branch under consideration, the
next term of the expansion of \(λ\) is non-zero: \(\order[2]λ ≠ 0\). The bifurcation is
\emph{symmetric}, and the bifurcation equation~\eqref{eq:20220601070917} reduces
to
\begin{equation}
  \label{eq:20220801092222}
  \tfrac{1}{3} E_{ijkl} \order[1]{ξ_j} \order[1]{ξ_k} \order[1]{ξ_l}  + \order[2]λ F_{ij} \order[1]{ξ_j} = 0,
\end{equation}
which has at most \((3^m - 1) / 2\) pairs of real solutions \((\order[2]λ, \order[1]u)\) and
\((- \order[2]λ, - \order[1]u)\). Upon multiplication by \(\order[1]{ξ_i}\), the above equation delivers
the following expression of \(\order[2]λ\)
\begin{equation}
  \label{eq:20220801093236}
  \order[2]λ = -\frac{E_{ijkl} \order[1]{ξ_i} \order[1]{ξ_j} \order[1]{ξ_k} \order[1]{ξ_l}}{3 F_{ij} \order[1]{ξ_i} \order[1]{ξ_j}}.
\end{equation}

Since \(F_{ij} \order[1]{ξ_i} \order[1]{ξ_j} < 0\), \(\order[2]λ\) has the same sign as
\(E_{ijkl}\order[1]{ξ_i} \order[1]{ξ_j} \order[1]{ξ_k} \order[1]{ξ_l}\). In other words, if
\(E_{ijkl}\order[1]{ξ_i} \order[1]{ξ_j} \order[1]{ξ_k} \order[1]{ξ_l} > 0\), (resp. \(<0\)) then the bifurcated branch
exists above (resp. below) the critical load \(λ₀\) only.

\begin{remark}
  I can't prove that the bifurcation equation~\eqref{eq:20220801092222} has at
  most \((3^m - 1) / 2\) pairs of real solutions.
\end{remark}

Turning now to the eigenpairs of the Hessian of the energy along
the bifurcated branch, Eq.~\eqref{eq:20220609133608} shows that \(\order[1]α = 0\). Then
\(α = \order[2]α η² / 2 + o(η²)\) and, from Eq.~\eqref{eq:20220616082923}
\begin{equation}
  \bigl(E_{ijkl} \order[1]{ξ_k} \order[1]{ξ_l} + \order[2]λ F_{ij} \bigr) \order[0]{χ_j} = \order[2]α \order[0]{χ_i}.
\end{equation}

If \((E_{ijkl} \order[1]{ξ_k} \order[1]{ξ_l} + \order[2]λ F_{ij} )\) is positive definite, then the
bifurcated branch is stable (note that, in that case, the bifurcated branch
exists above the critical load only). If one of the eigenvalues of this tensor
is \(<0\), then the bifurcated branch is unstable. The stability is undecided
when all eigenvalues are \(≥ 0\).

\begin{remark}
  Note that, from Eq.~\eqref{eq:20220801092222},
  \begin{equation}
    E_{ijkl} \order[1]{ξ_i} \order[1]{ξ_j} \order[1]{ξ_k} \order[1]{ξ_l} + \order[2]λ F_{ij} \order[1]{ξ_i} \order[1]{ξ_j} = \tfrac{2}{3} E_{ijkl} \order[1]{ξ_i} \order[1]{ξ_j} \order[1]{ξ_k} \order[1]{ξ_l}
  \end{equation}

\end{remark}

To conclude this section, it is observed that, when \(\order[1]λ = 0\), the dominant
term of the potential energy along the bifurcated branch is of the fourth order
[see Eq.~\eqref{eq:20220525053600}]. Combining with Eq.~\eqref{eq:20220801093236},
\begin{equation}
  \label{eq:20220801094437}
  \begin{aligned}[b]
    ℰ[u(η), λ(η)]
    &= ℰ\{u^{\ast}[λ(η)], λ(η)\} + \tfrac{1}{24} η⁴ \bigl(E_{ijkl} \order[1]{ξ_i} \order[1]{ξ_j} \order[1]{ξ_k} \order[1]{ξ_l}  + 6  \order[2]λ F_{ij} \order[1]{ξ_i} \order[1]{ξ_j}\bigr) + o(η⁴)\\
    &= ℰ\{u^{\ast}[λ(η)], λ(η)\} - \tfrac{1}{24} η⁴ E_{ijkl} \order[1]{ξ_i} \order[1]{ξ_j} \order[1]{ξ_k} \order[1]{ξ_l} + o(η⁴).
  \end{aligned}
\end{equation}

The expansion \(λ = λ₀ + \order[2]λ η² / 2 + o(η²)\) can be inverted as follows
\begin{equation}
  η⁴ = \frac{4 \bigl(λ - λ₀\bigr)²}{\bigl( \order[2]λ \bigr)^2} + o(λ²) = \frac{36 \bigl( F_{ij} \order[1]{ξ_i} \order[1]{ξ_j} \bigr)²}{\bigl( E_{ijkl} \order[1]{ξ_i} \order[1]{ξ_j} \order[1]{ξ_k} \order[1]{ξ_l} \bigr)²} \bigl( λ - λ₀ \bigr)²
\end{equation}
and expression~\eqref{eq:20220801094437} reads
\begin{equation}
  ℰ[u(η), λ(η)] = ℰ\{u^{\ast}[λ(η)], λ(η)\} - \frac{3 \bigl( F_{ij} \order[1]{ξ_i} \order[1]{ξ_j} \bigr)²}{2 E_{ijkl} \order[1]{ξ_i} \order[1]{ξ_j} \order[1]{ξ_k} \order[1]{ξ_l} } \bigl( λ - λ₀ \bigr)²  + o(λ²).
\end{equation}

Again, the above expression does not depend on the scaling of \(\order[1]u\) (of the
\(\order[1]{ξ_i}\)). Note that, if \(E_{ijkl} \order[1]{ξ_i} \order[1]{ξ_j} \order[1]{ξ_k} \order[1]{ξ_l} > 0\), then only loads
that are greater than the critical load can be reached on the bifurcated branch,
where the energy is lower than the fundamental branch.

\medskip

The above discussion simplifies considerably when there is only one buckling
mode (\(m = 1\)). This is addressed in the next section.

\section{The case of a single mode}

In this section, we discuss the case \(m = 1\); all tensors considered above
(\(F_{ij}\), \(E_{ijk}\), \(E_{ijkl}\)) then reduce to simple scalars. To avoid
ambiguity, indices are kept: \(F_{11}\), \(E_{11}\), \(E_{11}\). Since
\(\dot{ℰ}₂(λ₀)\) is negative definite over \(V\), we have \(F_{11} < 0\).

It is first observed that the following conditions are \emph{necessary} to
ensure stability of the critical point
\begin{equation}
  E_{111} = 0 \quad \text{and} \quad E_{1111} ≥ 0,
\end{equation}
which shows that \emph{asymmetric} bifurcation points are always
\emph{unstable}.

\subsection{Asymmetric bifurcations}

We first consider the case \(E_{111} ≠ 0\). Owing to the discussion above, the
bifurcation point is unstable. Setting \(\order[1]λ = 1\),
Eq.~\eqref{eq:20220524135036} delivers
\begin{equation}
  E_{111} \order[1]ξ_1 + 2F_{11} = 0 \quad \text{and} \quad u(λ) = u^\ast(λ) - \frac{2F_{11}}{E_{111}} \bigl( λ - λ_0 \bigr) v_1 + o(λ - λ_0).
\end{equation}

Furthermore, the hessian of the energy along the bifurcated branch is retrieved
from Eq.~\eqref{eq:20220819160235}
\begin{equation}
  \begin{aligned}[b]
    ℰ_{, uu}[u(η), λ(η), v_1, v_1] &= η \bigl(E_{111} \order[1]{ξ_1} + \order[1]λ F_{11}\bigr) + o(η) = -2 η F_{11} + o(η)\\
    &= -2 F_{11} \bigl( λ - λ₀ \bigr) + o(λ - λ₀).
  \end{aligned}
\end{equation}

Asymmetric bifurcations branches are \emph{unstable} for \(λ ≤ λ_0\) and
\emph{stable} for \(λ > λ₀\) (stability switch).

\subsection{Symmetric bifurcations}

We now consider the case \(E_{111}=0\). From the general discussion of
Sec.~\ref{sec:20220802061621}, the bifurcation point is \emph{stable} if
\(E_{1111} > 0\) and \emph{unstable} if \(E_{1111} < 0\). The bifurcation
equation~\eqref{eq:20220801092222} reduces to
\begin{equation}
  E_{1111} \bigl( \order[1]{ξ_1} \bigr)^2 + 3\order[2]λ F_{11} = 0,
\end{equation}
which in particular shows that \(\order[2]λ\) has the same sign as
\(E_{1111}\). Since the expansion of \(λ\) reads:
\(λ = λ_0 + \order[2]λ η^2 / 2\), the bifurcation branch exists only for loads
\emph{above} the critical load (\(λ ≥ λ_0\)) if \(E_{1111} > 0\) and only for
loads \emph{below} the critical load (\(λ ≤ λ_0\)) if \(E_{1111} < 0\).

From Eq.~\eqref{eq:20220531054247}, the hessian of the energy along the
bifurcated branch reads
\begin{equation*}
  ℰ_{, uu}[u(η), λ(η); v_1, v_1] = \tfrac{1}{2} η² \bigl[ E_{1111}\bigl(ξ_1^1\bigr) ^2 + \order[2]λ F_{11} \bigr] + o(η²) = - η² \order[2]λ F_{11} + o(η²),
\end{equation*}
which has the sign of \(\order[2]λ\). Therefore the Hessian is positive
(resp. negative) definite if \(E_{1111} > 0\) (resp \(< 0\)).

To sum up, if \(E_{1111} > 0\), then the bifurcation branch (including the
critical point) is \emph{stable} and exists only for loads greater than the
critical load. Conversely, if \(E_{1111} < 0\), then the bifurcation branch
(including the critical point) is \emph{unstable} and exists only for loads
lower than the critical load.

\section{Propriétés des formes bilinéaires symétriques, positives}

Dans ce qui suit, \(\mathcal{B}\) désigne une forme bilinéaire symétrique et
positive sur l'espace vectoriel \(U\). On définit son noyau \(\ker \mathcal{B}\)
de la façon suivante
\begin{equation}
 \ker \mathcal{B}= \bigl\{ u ∈ U, \mathcal{B}(u, u) = 0 \bigr\} .
\end{equation}

\begin{theorem}
  Le noyau d'une forme bilinéaire, symétrique et positive est un sous-espace
  vectoriel.
\end{theorem}
\begin{proof}
  Soient \(u, v∈\ker \mathcal{B}\), \(α∈\reals\) et \(w = u + α v\). Montrons
  que \(w ∈ \ker\mathcal{B}\). Il suffit d'évaluer \(\mathcal{B}(w, w)\)
 \begin{equation}
   \mathcal{B}(w, w) = \mathcal{B}(u + α v, u + α v)
   = \mathcal{B}(u, u) + 2 α \mathcal{B}(u, v) + α² \mathcal{B}(v, v),
 \end{equation}
 où l'on a tenu compte de la symétrie de \(\mathcal{B}\) pour écrire que
 \(\mathcal{B}(u, v) =\mathcal{B}(v, u)\). Comme \(u, v ∈ \ker\mathcal{B}\), le
 premier et le dernier terme sont nuls, soit
 \(\mathcal{B}(w, w) = 2α \mathcal{B}(u, v)\). La forme bilinéaire étant
 positive, cette grandeur est positive, \emph{quelle que soit la valeur de
   \(α∈\reals\)}. On en déduit donc que \(\mathcal{B}(u, v) = 0\), puis que
 \(\mathcal{B}(w, w) = 0\) et donc que \(w ∈ \ker\mathcal{B}\).
\end{proof}

\begin{theorem}
 Soit \(u∈V\). Alors
 \begin{equation}
  u ∈ \ker\mathcal{B} \quad \text{ssi} \quad \text{pour tout } v ∈ V, \mathcal{B}(u, v) = 0.
 \end{equation}
\end{theorem}

\begin{proof}
  Soient \(u∈\ker \mathcal{B}\), \(v∈V\) et \(α∈\reals\). Comme précédemment, on
  écrit que \(\mathcal{B}(w, w) ≥ 0\), avec \(w = α u + v\)
 \begin{equation}
  \mathcal{B}(w, w) = 2 α \mathcal{B}(u, v) +\mathcal{B}(v, v) \geq
  0,
 \end{equation}
 où l'on a tenu compte de ce que \(\mathcal{B}(u, u) = 0\). L'expression
 précédente, affine en \(α\), a un signe constant. Le terme linéaire en \(α\)
 est donc nul, soit \(\mathcal{B}(u, v) = 0\).  Réciproquement, si
 \(\mathcal{B}(u, v) = 0\) pour tout \(v∈V\), alors \(\mathcal{B}(u, u) = 0\)(en
 prenant \(v = u\)).
\end{proof}

\begin{theorem}
  \label{thr:20220802112835}
  Let \(𝒯\) be a trilinear, symmetric form, such that
  \begin{equation}
    \label{eq:20220802111745}
    𝒯(u, u, u) = 0 \quad \text{for all} \quad u ∈ U.
  \end{equation}
  Then
  \begin{equation}
    𝒯(u, v, w) = 0 \quad \text{for all} \quad u, v, w ∈ U.
  \end{equation}
\end{theorem}
\begin{proof}
  The form \(𝒯\) being trilinear and symmetric, we have, for all \(u, v, w ∈ U\)
  and \(α, β ∈ ℝ\)
  \begin{multline}
    𝒯(u + αv + βw, u + αv + βw, u + αv + βw) = 𝒯(u, u, u) + 3α 𝒯(u, u, v)\\
    + 3β 𝒯(u, u, w) + 3α² 𝒯(u, u, v) + 6 α β 𝒯(u, v, w) + 3 β² 𝒯(u, u, w)\\
    + α³ 𝒯(v, v, v) + 3 α² β 𝒯(v, v, w) + 3 α β² 𝒯(v, w, w) + β³ 𝒯(w, w, w)
  \end{multline}
  and, upon simplification using Eq.~\eqref{eq:20220802111745}
  \begin{multline}
    \label{eq:20220802112309}
    3α 𝒯(u, u, v) + 3β 𝒯(u, u, w) + 3α² 𝒯(u, v, v) + 6 α β 𝒯(u, v, w)\\
    + 3 β² 𝒯(u, w, w) + 3 α² β 𝒯(v, v, w) + 3 α β² 𝒯(v, w, w) = 0.
  \end{multline}
  In particular taking successively \(α = ±1\), \(β = 0\) and \(w = 0\) delivers
  \begin{equation}
    ±3 𝒯(u, u, v) + 3 𝒯(u, u, v) = 0 \quad \text{for all} \quad u, v ∈ U,
  \end{equation}
  from which it results that
  \begin{equation}
    𝒯(u, u, v) = 0 \quad \text{for all} \quad u, v ∈ U.
  \end{equation}
  Plugging into Eq.~\eqref{eq:20220802112309} with \(α = β = 1\) results in:
  \(𝒯(u, v, w) = 0\) for all \(u, v, w ∈ U\).
\end{proof}

\appendix
\section{Développements limités le long d'une branche bifurquée du diagramme d'équilibre}

\subsection{Principe du calcul}
\label{sec:20220107121442}
% 02/06/2022 — 099042106e938251657847daca64c8fcbaa833c3
%
% Validation des calculs de ce paragraphe

On pose dans ce qui suit
\begin{align}
  \label{eq:20211112155446}
  Λ(η) & = λ(η) - λ₀ = η \order[1]λ + \tfrac{1}{2} η² \order[2]λ + \tfrac{1}{6} η³ \order[3]λ + \cdots,\\
  \label{eq:20211112113028}
  U(η) & = u(η) - u^{\ast}[λ(η)] = η \order[1]u + \tfrac{1}{2} η² \order[2]u + \tfrac{1}{6} η³ \order[3]u + \cdots.
\end{align}

On considère une fonctionnelle \(\mathcal{F}\) de \(u\) et \(λ\)~:
\(\mathcal{F}(u, λ)\). Cette fonctionnelle est évaluée le long de la branche
bifurquée. En d'autres termes, on considère
\begin{equation*}
  f(η) = F\{ u^{\ast} [λ₀ + Λ(η)] + U(η), λ₀ + Λ(η) \}.
\end{equation*}

On souhaite établir un développement limité de \(f\) au voisinage de \(η = 0\),
ce qui conduit à calculer les dérivées successives de \(f\) en \(η = 0\),
puisque
\begin{equation*}
  f(η) = f(0) + η f'(0) + \tfrac{1}{2} η² f''(0) + \cdots.
\end{equation*}

Pour calculer ces dérivées, il sera commode d'introduire la fonction auxiliaire
\(F\)
\begin{equation*}
  F(η, λ) =\mathcal{F}[u^{\ast}(λ) + U(η), λ],
\end{equation*}
dans laquelle les variables \(λ\) et \(η\) sont provisoirement considérées comme
indépendantes. On a \(f(η) = F[η, λ₀ + Λ(η)]\), d'où l'on déduit successivement
que
\begin{gather*}
  f'(η) = ∂_{η} F + Λ' ∂_{λ} F,\\
  f''(η) = ∂_{ηη}² F + 2Λ' ∂_{ηλ}²F + Λ'^2 ∂_{λλ}² F + Λ'' ∂_{λ} F,\\
  \begin{aligned}[b]
    f'''(η) ={}
    & ∂_{ηηη}³ F + 3Λ' ∂_{ηηλ}³F + 3Λ'^2 ∂_{ηλλ}³F + λ'^3 ∂_{λλλ}³ F\\
    & + 3Λ'' ∂_{ηλ}² F + 3Λ' Λ'' ∂_{λ λ}² F + Λ''' ∂_{λ} F,
  \end{aligned}\\
  \begin{aligned}[b]
    f''''(η) ={}
    & ∂_{ηηηη}⁴ F + 4Λ' ∂_{ηηηλ}⁴F + 6Λ'^2 ∂_{ηηλλ}⁴F + 4Λ'^3 ∂_{ηλλλ}⁴F + Λ'^4 ∂_{λλλλ}⁴ F\\
    & + 6Λ'' ∂_{ηηλ}³ F + 12Λ' Λ'' ∂_{ηλλ}³F + 6Λ'^2 Λ'' ∂_{λλλ}³ F\\
    & + 4 Λ''' ∂_{ηλ}² F + \bigl( 3Λ''^2 + 4 Λ' Λ''' \bigr) ∂_{λλ}² F + λ'''' ∂_{λ}F,
  \end{aligned}
\end{gather*}
où \(Λ\) et ses dérivées sont évaluées en \(η\), tandis que \(F\) et ses
dérivées partielles sont évaluées en \([η, λ₀ + Λ(η)]\). En \(η = 0\), les
relations précédentes s'écrivent
\begin{gather}
  \label{eq:20220107060454}
  f'(0) = ∂_{η} F + \order[1]λ ∂_{λ} F,\\
  \label{eq:20220107124311}
  f''(0) = ∂_{ηη}² F + 2 \order[1]λ ∂_{ηλ}² F + \bigl( \order[1]λ \bigr)^2 ∂_{λλ}² F + \order[2]λ ∂_{λ} F,\\
  \label{eq:20220107060500}
  \begin{aligned}[b]
    f'''(0) ={}
    & ∂_{ηηη}³ F + 3 \order[1]λ ∂_{ηηλ}³ F + 3 \bigl( \order[1]λ \bigr)^2 ∂_{ηλλ}³ F + \bigl( \order[1]λ \bigr)^3 ∂_{λλλ}³ F\\
    & + 3 \order[2]λ ∂_{ηλ}² F + 3 \order[1]λ \order[2]λ ∂_{λλ}² F + \order[3]λ ∂_{λ} F,
  \end{aligned}\\
  \label{eq:20220602185935}
  \begin{aligned}[b]
    f''''(0) ={}
    & ∂_{ηηηη}⁴F + 4 \order[1]λ ∂_{ηηηλ}⁴ F + 6 \bigl( \order[1]λ \bigr)^2 ∂_{ηηλλ}⁴ F + 4 \bigl( \order[1]λ \bigr)^3 ∂_{ηλλλ}⁴ F + \bigl( \order[1]λ \bigr)^4 ∂_{λλλλ}⁴ F\\
    & + 6 \order[2]λ ∂_{ηηλ}³ F + 12 \order[1]λ \order[2]λ ∂_{ηλλ}³ F + 6 \bigl( \order[1]λ \bigr)^2 \order[2]λ ∂_{λλλ}³ F\\
    & + 4 \order[3]λ ∂_{ηλ}² F + \bigl(3 \bigl( \order[2]λ \bigr)^2 + 4 \order[1]λ \order[3]λ\bigr) ∂_{λλ}² F + λ₄ ∂_{λ} F,
  \end{aligned}
\end{gather}
où \(F\) et ses dérivées sont maintenant évaluées en \((η = 0, λ = λ₀)\).

\subsection{Développement limité du résidu}
\label{sec:20211112182000}
% 03/06/2022 — b028b234970605720c9022c16c7fc3012997ced7
%
% Validation des calculs de ce paragraphe

On cherche un développement limité du résidu (c'est-à-dire de la première
variation de l'énergie). La fonction test \(\hat{u} ∈ U\) étant fixée, la
méthode précédente est donc appliquée avec
\begin{equation}
  \label{eq:20220107054629}
  f(η) = ℰ_{, u} [u(η), λ(η); \hat{u}]
  \quad \text{et} \quad
  F(η, λ) = ℰ_{, u}[u^{\ast}(λ) + U(η), λ; \hat{u}].
\end{equation}

On remarque tout d'abord que
\(F(0, λ) =ℰ_{, u} [u^{\ast} (λ), λ; \hat{u}] = 0\), puisque \(u^{\ast}(λ)\) est
un point d'équilibre. En dérivant par rapport à \(λ\), on obtient
\begin{equation*}
  \frac{∂^k F}{∂ λ^k}(0, λ) = 0 \quad \text{pour tout} \quad k ≥ 0.
\end{equation*}

En dérivant par rapport à \(η\) l'expression~\eqref{eq:20220107054629} de \(F\),
on obtient successivement
\begin{equation*}
  ∂_{η}F(η, λ) = ℰ_{, u u}[u^{\ast}(λ) + U(η), λ; U'(η), \hat{u}],
\end{equation*}
\begin{equation*}
  \begin{aligned}[b]
    ∂_{η η}² F(η, λ) ={}
    & ℰ_{, uuu}[u^{\ast}(λ) + U(η), λ; U'(η), U'(η), \hat{u}]\\
    & + ℰ_{, uu} [u^{\ast}(λ) + U(η), λ; U''(η), \hat{u}],
  \end{aligned}
\end{equation*}
\begin{equation*}
  \begin{aligned}[b]
    ∂_{ηηη}³ F(η, λ) ={}
    & ℰ_{, uuuu}[u^{\ast}(λ) + U(η), λ; U'(η), U'(η), U'(η), \hat{u}]\\
    & + 3ℰ_{, u u u}[u^{\ast}(λ) + U(η), λ; U'(η), U''(η), \hat{u}]\\
    & + ℰ_{, uu}[u^{\ast}(λ) + U(η), λ; U'''(η), \hat{u}],
  \end{aligned}
\end{equation*}
soit, en \(η = 0\)
\[∂_{η}F(0, λ) = ℰ₂(λ; \order[1]u, \hat{u}),\]
\[∂_{ηη}² F(0, λ) = ℰ₃(λ; \order[1]u, \order[1]u, \hat{u}) + ℰ₂(λ; \order[2]u, \hat{u}),\]
\[∂_{ηηη}³ F(0, λ) = ℰ₄(λ; \order[1]u, \order[1]u, \order[1]u, \hat{u}) + 3ℰ₃(λ; \order[1]u, \order[2]u, \hat{u}) + ℰ₂(λ; \order[3]u, \hat{u}).\]

Les dérivées croisées de \(F\) en \((0, λ)\) s'obtiennent par simple dérivation
des relations précédentes par rapport à \(λ\)
\[∂_{ηλ}² F(0, λ) = \dot{ℰ}₂(λ; \order[1]u, \hat{u}), \quad ∂_{ηλλ}³ F(0, λ) = \ddot{ℰ}₂(λ; \order[1]u, \hat{u}),\]
\[∂_{ηηλ}³ F(0, λ) = \dot{ℰ}₃(λ; \order[1]u, \order[1]u, \hat{u}) + \dot{ℰ₂}(λ; \order[2]u, \hat{u}).\]

En insérant les résultats précédents dans les relations
générales~\eqref{eq:20220107060454}--\eqref{eq:20220602185935}, on trouve alors
les expressions suivantes des dérivées successives de \(f\) en \(η = 0\)
\begin{gather*}
  f'(0) = ℰ₂(λ₀; \order[1]u, \hat{u}),\\
  f''(0) = ℰ₃(λ₀; \order[1]u, \order[1]u, \hat{u}) + ℰ₂(λ₀; \order[2]u, \hat{u}) + 2 \order[1]λ \dot{ℰ}₂(λ₀; \order[1]u, \hat{u}),\\
  \begin{aligned}[b]
    f'''(0) ={}
    & ℰ₄(λ₀; \order[1]u, \order[1]u, \order[1]u, \hat{u}) + 3ℰ₃(λ₀; \order[1]u, \order[2]u, \hat{u}) + ℰ₂(λ₀ ; \order[3]u, \hat{u})\\
    & + 3\order[1]λ \dot{ℰ}₃(λ₀; \order[1]u, \order[1]u, \hat{u}) + 3\order[1]λ \dot{ℰ}₂(λ₀; \order[2]u, \hat{u})\\
    & + 3 \bigl( \order[1]λ \bigr)^2 \ddot{ℰ}₂(λ₀; \order[1]u, \hat{u}) + 3 \order[2]λ \dot{ℰ}₂(λ₀; \order[1]u, \hat{u}).
  \end{aligned}
\end{gather*}

On en déduit finalement le développement limité à l'ordre 3 en \(η\) du résidu
\begin{equation}
  \label{eq:20220107080901}
  \begin{gathered}[b]
    ℰ_{, u}[u(η), λ(η)] ={} η ℰ₂(λ₀; \order[1]u, \hat{u}) + \tfrac{1}{2} η² \bigl[ℰ₃(λ₀; \order[1]u, \order[1]u, \hat{u})  + ℰ₂(λ₀; \order[2]u, \hat{u})\\
    {} + 2 \order[1]λ \dot{ℰ}₂(λ₀; \order[1]u, \hat{u})\bigr] + \tfrac{1}{6} η³ \bigl[ ℰ₄(λ₀; \order[1]u, \order[1]u, \order[1]u, \hat{u}) + 3ℰ₃(λ₀; \order[1]u, \order[2]u, \hat{u})\\
    {} + ℰ₂(λ₀; \order[3]u, \hat{u}) + 3\order[1]λ \dot{ℰ}₃(λ₀; \order[1]u, \order[1]u, \hat{u}) + 3\order[1]λ \dot{ℰ}₂(λ₀; \order[2]u, \hat{u})\\
    {} + 3 \bigl( \order[1]λ \bigr)^2 \ddot{ℰ}₂(λ₀; \order[1]u, \hat{u}) + 3 \order[2]λ \dot{ℰ}₂(λ₀ ; \order[1]u, \hat{u}) \bigr] + o(η³).
  \end{gathered}
\end{equation}

\subsection{Développement limité de l'énergie}
\label{sec:20220525053434}
% 07/06/2022 — dd1a4abf18cd94861d754bf3e19a54b8974bb2e8
%
% Relecture de tous les calculs de ce paragraphe

On s'intéresse ici à l'écart d'énergie, pour un chargement \(λ\) donné, entre la
branche bifurquée et la branche fondamentale, soit
\begin{equation}
  F(η, λ) = ℰ[u^{\ast}(λ) + U(η), λ] - ℰ[u^{\ast}(λ), λ]
  \quad \text{et} \quad
  f(η) = F [η, λ₀ + Λ(η)].
\end{equation}

On observe tout d'abord que \(F(0, λ) = 0\) pour tout \(λ\), donc
\begin{equation*}
  \frac{∂^k F}{∂ λ^k}(0, λ) = 0 \quad \text{pour tout} \quad k ≥ 0,
\end{equation*}
tandis que les dérivées de \(F\) par rapport à \(η\) s'écrivent
\begin{gather*}
  ∂_{η} F(η, λ) = ℰ_{, u}(U'),\\
  ∂_{ηη}² F(η, λ) = ℰ_{, uu} (U', U') + ℰ_{, u} (U''),\\
  ∂_{ηηη}³ F(η, λ) = ℰ_{, uuu}(U', U', U') + 3ℰ_{, uu}(U', U'') + ℰ_{, u}(U'''),\\
  \begin{aligned}[b]
    ∂_{ηηηη}⁴ F ={}
    & ℰ_{, uuuu}(U', U', U', U') + 6ℰ_{,uuu}(U', U', U'')\\
    & + 3ℰ_{, uu}(U'', U'') + 4ℰ_{, uu}(U', U''') + ℰ_{, u}(U''''),
  \end{aligned}\\
\end{gather*}
où les différentielles successives de \(ℰ\) sont évaluées en
\([u^{\ast}(λ) + U(η), λ]\), tandis que les dérivées successives de \(U\) sont
évaluées en \(η\).  Les relations précédentes s'écrivent, en \(η = 0\), en
observant que \(ℰ_{, u}[u^{\ast}(λ), λ] = 0\)
\begin{gather*}
  ∂_{η} F(0, λ) = 0,\\
  ∂_{ηη}² F(0, λ) =ℰ₂(λ ; \order[1]u, \order[1]u),\\
  ∂_{ηηη}³ F(0, λ) = ℰ₃(λ; \order[1]u, \order[1]u, \order[1]u) + 3ℰ₂(λ; \order[1]u, \order[2]u),\\
  ∂_{ηηηη}⁴ F(η, λ) = ℰ₄(λ; \order[1]u, \order[1]u, \order[1]u, \order[1]u) + 6ℰ₃(λ; \order[1]u, \order[1]u, \order[2]u) + 3ℰ₂(λ; \order[2]u, \order[2]u) + 4ℰ₂(λ; \order[1]u, \order[3]u).
\end{gather*}

En dérivant alors par rapport à \(λ\), on en déduit que\\
\begin{equation*}
  \begin{gathered}
    ∂_{ηλ}² F(0, λ) = 0,\\
    ∂_{ηηλ}³ F(0, λ) = \dot{ℰ}₂(λ; \order[1]u, \order[1]u),\\
    ∂_{ηλλ}³ F(0, λ) = 0,\\
  \end{gathered}
  \qquad
  \begin{gathered}
    ∂_{ηηηλ}⁴ F(0, λ) = \dot{ℰ}₃(λ; \order[1]u, \order[1]u, \order[1]u) + 3\dot{ℰ}₂(λ; \order[1]u, \order[2]u),\\
    ∂_{ηηλλ}⁴ F(0, λ) = \ddot{ℰ}₂(λ; \order[1]u, \order[1]u),\\
    ∂_{ηλλλ}⁴ F(0, λ) = 0
  \end{gathered}
\end{equation*}
et finalement
\begin{gather*}
    f'(0) = 0, \qquad f''(0) = ℰ₂(λ₀; \order[1]u, \order[1]u),\\
    f'''(0) =ℰ₃(λ₀; \order[1]u, \order[1]u, \order[1]u) + 3ℰ₂(λ₀; \order[1]u, \order[2]u) + 3\order[1]λ \dot{ℰ}₂(λ₀; \order[1]u, \order[1]u),\\
    \begin{aligned}[b]
      f''''(0) ={}
      & ℰ₄(λ₀; \order[1]u, \order[1]u, \order[1]u, \order[1]u) + 6ℰ₃(λ₀; \order[1]u, \order[1]u, \order[2]u) + 3ℰ₂(λ₀; \order[2]u, \order[2]u)\\
      & + 4ℰ₂(λ₀; \order[1]u, \order[3]u) + 4 \order[1]λ \dot{ℰ}₃(λ₀; \order[1]u, \order[1]u, \order[1]u) + 12 \order[1]λ \dot{ℰ}₂(λ₀; \order[1]u, \order[2]u)\\
      & + 6\bigl( \order[1]λ \bigr)^2 \ddot{ℰ}₂(λ₀; \order[1]u, \order[1]u) + 6\order[2]λ \dot{ℰ}₂(λ₀; \order[1]u, \order[1]u).
    \end{aligned}
\end{gather*}

Les relations précédentes se simplifient notamment en tenant compte de ce que
\(\order[1]u∈V\) : \(ℰ₂(λ₀; \order[1]u, u_i) = 0\) pour \(i = 1, 2, 3\). On trouve ainsi
\(f''(0)=0\) et
\begin{equation}
  \label{eq:20220601055448}
  f'''(0) = -\order[1]λ G_{ij} \order[1]{ξ_i} \order[1]{ξ_j},
\end{equation}
en utilisant l'équation de bifurcation~\eqref{eq:20220524133816}. En
introduisant les décompositions \eqref{eq:20220524133944} et
\eqref{eq:20220524134613} de \(\order[1]u\) et \(\order[2]u\), on trouve tout d'abord, pour
\(ℰ₃(λ₀; \order[1]u, \order[1]u, \order[2]u)\)
\begin{equation*}
  \begin{aligned}[b]
    ℰ₃(λ₀; \order[1]u, \order[1]u, \order[2]u)
    ={} & ℰ₃(v_i, v_j, v_k) \order[1]{ξ_i} \order[1]{ξ_j} \order[2]{ξ_k} + ℰ₃(v_i, v_j, w_{k l}) \order[1]{ξ_i} \order[1]{ξ_j} \order[1]{ξ_k} \order[1]{ξ_l}\\
    & + \order[1]λ ℰ₃(v_i, v_j, w_k) \order[1]{ξ_i} \order[1]{ξ_j} \order[1]{ξ_k} \\
    ={} & ℰ₃(v_i, v_j, v_k) \order[1]{ξ_i} \order[1]{ξ_j} \order[2]{ξ_k} + ℰ₃(v_i, v_j, w_{k l}) \order[1]{ξ_i} \order[1]{ξ_j} \order[1]{ξ_k} \order[1]{ξ_l}\\
    & - \order[1]λ ℰ₂(w_{ij}, w_k) \order[1]{ξ_i} \order[1]{ξ_j} \order[1]{ξ_k},
  \end{aligned}
\end{equation*}
en tenant compte de la définition~\eqref{eq:20220519164523} des \(w_{ij}\). Dans
le dernier terme de l'expression précédente, les indices \(i\), \(j\) et \(k\)
sont muets, donc
\begin{equation*}
  \begin{aligned}[b]
    ℰ₃(λ₀; \order[1]u, \order[1]u, \order[2]u)
    ={} & ℰ₃(v_i, v_j, v_k) \order[1]{ξ_i} \order[1]{ξ_j} \order[2]{ξ_k} + ℰ₃(v_i, v_j, w_{kl}) \order[1]{ξ_i} \order[1]{ξ_j} \order[1]{ξ_k} \order[1]{ξ_l}\\
    & - \order[1]λ ℰ₂(w_{i}, w_{jk}) \order[1]{ξ_i} \order[1]{ξ_j} \order[1]{ξ_k}\\
    ={} & ℰ₃(v_i, v_j, v_k) \order[1]{ξ_i} \order[1]{ξ_j} \order[2]{ξ_k} + ℰ₃(v_i, v_j, w_{kl}) \order[1]{ξ_i} \order[1]{ξ_j} \order[1]{ξ_k} \order[1]{ξ_l}\\
    & + 2 \order[1]λ \dot{ℰ}₂(v_{i}, w_{jk}) \order[1]{ξ_i} \order[1]{ξ_j} \order[1]{ξ_k},
  \end{aligned}
\end{equation*}
en introduisant cette fois-ci la définition~\eqref{eq:20220524134525} de
\(w_i\). On procède de même pour le terme suivant, soit \(ℰ₂(\order[2]u, \order[2]u)\)
\begin{equation*}
  \begin{aligned}[b]
    ℰ₂(\order[2]u, \order[2]u)
    ={} & ℰ₂(\order[2]{ξ_i} v_i + \order[1]{ξ_i} \order[1]{ξ_j} w_{i j} + \order[1]λ \order[1]{ξ_i} w_i, \order[2]{ξ_k} v_k + \order[1]{ξ_k} \order[1]{ξ_l} w_{k l} + \order[1]λ \order[1]{ξ_k} w_k)\\
    ={} & ℰ₂(\order[1]{ξ_i} \order[1]{ξ_j} w_{i j} + \order[1]λ \order[1]{ξ_i} w_i, \order[1]{ξ_k} \order[1]{ξ_l} w_{k l} + \order[1]λ \order[1]{ξ_k} w_k)\\
    ={} & ℰ₂(w_{i j}, w_{k l}) \order[1]{ξ_i} \order[1]{ξ_j} \order[1]{ξ_k} \order[1]{ξ_l} + 2 \order[1]λ ℰ₂(w_i, w_{j k}) \order[1]{ξ_i} \order[1]{ξ_j} \order[1]{ξ_k} + \bigl( \order[1]λ \bigr)^2 ℰ₂(w_i, w_j) \order[1]{ξ_i} \order[1]{ξ_j}\\
    ={} & ℰ₂(w_{i j}, w_{k l}) \order[1]{ξ_i} \order[1]{ξ_j} \order[1]{ξ_k} \order[1]{ξ_l} + 2 \order[1]λ ℰ₂(w_i, w_{j k}) \order[1]{ξ_i} \order[1]{ξ_j} \order[1]{ξ_k}\\
    &+ \tfrac{1}{2} \bigl( \order[1]λ \bigr)^2 \bigl[ℰ₂(w_i, w_j) + ℰ₂(w_j, w_i)\bigr] \order[1]{ξ_i} \order[1]{ξ_j}\\
    ={} & -ℰ₃(v_i, v_j, w_{k l}) \order[1]{ξ_i} \order[1]{ξ_j} \order[1]{ξ_k} \order[1]{ξ_l} - 4 \order[1]λ \dot{ℰ}₂ (v_i, w_{j k}) \order[1]{ξ_i} \order[1]{ξ_j} \order[1]{ξ_k}\\
    & - \bigl( \order[1]λ \bigr)^2 \bigl[\dot{ℰ}₂(v_i, w_j) + \dot{ℰ}₂(v_j, w_i)\bigr] \order[1]{ξ_i} \order[1]{ξ_j}
  \end{aligned}
\end{equation*}
et enfin
\begin{equation*}
  \begin{aligned}[b]
    \dot{ℰ}₂(\order[1]u, \order[2]u)
    ={} & \dot{ℰ}₂ (v_i, v_j) \order[1]{ξ_i} \order[2]{ξ_j} + \dot{ℰ}₂(v_i, w_{j k}) \order[1]{ξ_i} \order[1]{ξ_j} \order[1]{ξ_k} + \order[1]λ \dot{ℰ}₂(v_i, w_j) \order[1]{ξ_i} \order[1]{ξ_j}\\
    ={} & \dot{ℰ}₂(v_i, v_j) \order[1]{ξ_i} \order[2]{ξ_j} + \dot{ℰ}₂(v_i, w_{j k}) \order[1]{ξ_i} \order[1]{ξ_j} \order[1]{ξ_k} + \tfrac{1}{2} \order[1]λ [\dot{ℰ}₂(v_i, w_j) + \dot{ℰ}₂(v_j, w_i)] \order[1]{ξ_i} \order[1]{ξ_j}.
  \end{aligned}
\end{equation*}
En rassemblant les résultats précédents, on trouve pour \(f''''(0)\)
\begin{equation*}
  \begin{aligned}[b]
    f''''(0)
    ={} & \bigl[ ℰ₄(v_i, v_j, v_k , v_l) + 3ℰ₃(v_i, v_j, w_{k l}) \bigr] \order[1]{ξ_i} \order[1]{ξ_j} \order[1]{ξ_k} \order[1]{ξ_l}\\
    & + 4 \order[1]λ \bigl[\dot{ℰ}₃(v_i, v_j, v_k) + 3 \dot{ℰ}₂(v_i, w_{j k})\bigr] \order[1]{ξ_i} \order[1]{ξ_j} \order[1]{ξ_k}\\
    & + \bigl\{3 \bigl( \order[1]λ \bigr)^2 \bigl[ 2\ddot{ℰ}₂ (v_i, v_j) + \dot{ℰ}₂(v_i, w_j) + \dot{ℰ}₂(v_j, w_i) \bigr] + 6\order[2]λ \dot{ℰ}₂(v_i, v_j) \bigr\} \order[1]{ξ_i} \order[1]{ξ_j}\\
    & + 6\bigl[ℰ₃(v_i, v_j, v_k) \order[1]{ξ_k} + 2 \order[1]λ \dot{ℰ}₂(v_i, v_j)\bigr] \order[1]{ξ_i} \order[2]{ξ_j},
  \end{aligned}
\end{equation*}
et on observe que le dernier terme(en \(\order[1]{ξ_i} \order[2]{ξ_j}\)) est nul, du fait de
l'équation de bifurcation~\eqref{eq:20220524135036}. On obtient donc
\begin{equation}
  \label{eq:20220601055512}
  f''''(0) = E_{i j k l} \order[1]{ξ_i} \order[1]{ξ_j} \order[1]{ξ_k} \order[1]{ξ_l} + 4 \order[1]λ \mathring{E}_{i j k} \order[1]{ξ_i} \order[1]{ξ_j} \order[1]{ξ_k} - 6 \bigl(\bigl( \order[1]λ \bigr)^2 \mathring{G}_{i j} + \order[2]λ G_{i j}\bigr) \order[1]{ξ_i} \order[1]{ξ_j} .
\end{equation}

Le développement limité~\eqref{eq:20220525053600} est alors obtenu en
rassemblant les expressions précédentes de \(f'(0)\), \(f''(0)\), \(f'''(0)\) et
\(f''''(0)\).

\subsection{Développement limité de la hessienne}
\label{sec:20220616055207}
% 08/06/2022 — aea0da72c80440d74d38d8ace59f381061f71c3e
%
% Relecture de tous les calculs de ce paragraphe

On cherche maintenant un développement limité de la hessienne de l'énergie. Les
fonctions test \(\hat{u}, \hat{v} ∈ U\) étant fixées, on applique la méthode du
\S\ref{sec:20220107121442} à la fonction \(f(η) = F [η, λ₀ + Λ(η)]\), avec
\begin{equation*}
  F(η, λ) = ℰ_{, u u} [u^{\ast}(λ) + U(η), λ; \hat{u}, \hat{v}].
\end{equation*}

On observe tout d'abord que \(F(0, λ) =ℰ₂(λ; \hat{u}, \hat{v})\), soit, en
dérivant par rapport à \(λ\)
\begin{equation*}
  ∂_{λ} F(0, λ) = \dot{ℰ}₂(λ; \hat{u}, \hat{v})
  \quad \text{et} \quad
  ∂_{λλ}² F(0, λ) = \ddot{ℰ}₂(λ; \hat{u}, \hat{v}).
\end{equation*}

On trouve de même successivement
\begin{gather*}
  ∂_{η} F(η, λ) = ℰ_{, uuu}(U', \hat{u}, \hat{v}),\\
  ∂_{ηη}² F(η, λ) = ℰ_{, uuuu}(U', U', \hat{u}, \hat{v}) + ℰ_{, uuu}(U'', \hat{u}, \hat{v}),
\end{gather*}
où les différentielles successives de \(ℰ\) sont évaluées en
\([u^{\ast}(λ) + U(η), λ]\), tandis que les dérivées successives de \(U\) sont
évaluées en \(η\). Les relations précédentes s'écrivent en \(η = 0\)
\begin{gather*}
  ∂_{η} F(0, λ) = ℰ₃(λ; \order[1]u, \hat{u}, \hat{v}),\\
  ∂_{ηη}² F(0, λ) = ℰ₄(λ ; \order[1]u, \order[1]u, \hat{u}, \hat{v}) + ℰ₃(λ; \order[2]u, \hat{u}, \hat{v}),
\end{gather*}
et en dérivant cette fois par rapport à \(λ\)
\begin{equation*}
  ∂_{η λ}² F(0, λ) = \dot{ℰ}₃(λ; \order[1]u, \hat{u}, \hat{v}).
\end{equation*}

En insérant les résultats précédents dans les
expressions~\eqref{eq:20220107060454} et \eqref{eq:20220107124311}, on trouve
\begin{gather*}
  f'(0) = ℰ₃(\order[1]u, \hat{u}, \hat{v}) + \order[1]λ \dot{ℰ}₂(\hat{u}, \hat{v}),\\
  f''(0) = ℰ₄(\order[1]u, \order[1]u, \hat{u}, \hat{v}) + ℰ₃(\order[2]u, \hat{u}, \hat{v}) + 2\order[1]λ \dot{ℰ}₃(\order[1]u, \hat{u}, \hat{v}) + \bigl( \order[1]λ \bigr)^2 \ddot{ℰ}₂(\hat{u}, \hat{v}) + \order[2]λ \dot{ℰ}₂(\hat{u}, \hat{v}) .
\end{gather*}
qui conduisent finalement au développement limité~\eqref{eq:20220531054247}.

\subsection{Asymptotic expansions of the eigenvalues and eigenvectors of the Hessian}
\label{sec:20220616074108}

In this appendix, Eqs.~\eqref{eq:20220609133608}, \eqref{eq:20220609133629} and
\eqref{eq:20220616082923} are derived. The postulated
expansions~\eqref{eq:20220617064633} are plugged into the asymptotic expansion
\eqref{eq:20220531054247} of the Hessian on the one hand
\begin{multline*}
  ℰ_{, uu} [u(η), λ(η); x, \hat{u}] = ℰ₂(\order[0]x, \hat{u}) + η \bigl[ ℰ₂(\order[1]x, \hat{u}) + ℰ₃(\order[1]u, \order[0]x, \hat{u}) + \order[1]λ \dot{ℰ}₂(\order[0]x, \hat{u})\bigr]\\
  + \tfrac{1}{2} η² \bigl[ℰ₂(\order[2]x, \hat{u}) + 2ℰ₃(\order[1]u, \order[1]x, \hat{u}) + 2 \order[1]λ \dot{ℰ}₂(\order[1]x, \hat{u}) + ℰ₄(\order[1]u, \order[1]u, \order[0]x, \hat{u})\\
  + ℰ₃(\order[2]u, \order[0]x, \hat{u}) + 2\order[1]λ \dot{ℰ}₃(\order[1]u, \order[0]x, \hat{u}) + \bigl( \order[1]λ \bigr)^2 \ddot{ℰ}₂(\order[0]x, \hat{u}) + \order[2]λ \dot{ℰ}₂(\order[0]x, \hat{u}) \bigr] + o(η²)
\end{multline*}
(where the \(ℰ_k\) and \(\dot{ℰ}_k\) are all evaluated at \(λ=λ₀\)) and into the
scalar product \(α 〈 x, \hat{u} 〉\) on the other hand
\begin{multline*}
    α 〈 x, \hat{u} 〉 = \order[0]α 〈 \order[0]x, \hat{u} 〉 + η \bigl(\order[1]α 〈 \order[0]x, \hat{u} 〉 + \order[0]α 〈 \order[1]x, \hat{u} 〉\bigr)\\
    + \tfrac{1}{2} η² \bigl(\order[0]α 〈 \order[2]x, \hat{u} 〉 + 2 \order[1]α 〈 \order[1]x, \hat{u} 〉 + \order[2]α 〈 \order[0]x, \hat{u} 〉\bigr) + o(η²).
\end{multline*}

Equating both expressions for all \(\hat{u} ∈ U\) [see
Eq.~\eqref{eq:20220617074949}] leads to three variational problems (for the
\(η⁰\), \(η¹\) and \(η²\) terms) that are discussed below.

\paragraph{Variational problem of order 0} Find \(\order[0]x∈U\) and \(\order[0]α∈\reals\) such
that, for all \(\hat{u}∈U\)
\begin{equation*}
  ℰ₂(\order[0]x, \hat{u}) = \order[0]α 〈 \order[0]x, \hat{u} 〉.
\end{equation*}

The above equation shows that \((\order[0]α, \order[0]x)\) is an eigenpair of \(ℰ₂(λ₀)\). As
discussed in Sec.~\ref{sec:20220617075558}, only the case \(\order[0]α = 0\) is
relevant. Then \(\order[0]x ∈ V\), which is expressed by the
expansion~\eqref{eq:20220904160057} of \(\order[0]x\).

\paragraph{Variational problem of order 1} Find \(\order[1]x∈U\) and \(\order[1]α∈\reals\) such
that, for all \(\hat{u}∈U\)
\begin{equation}
  \label{eq:20220609131953}
  ℰ₂(\order[1]x, \hat{u}) + ℰ₃(\order[1]u, \order[0]x, \hat{u}) + \order[1]λ \dot{ℰ}₂(\order[0]x, \hat{u}) = \order[1]α 〈 \order[0]x, \hat{u} 〉,
\end{equation}
or, equivalently, plugging the expansions~\eqref{eq:20220524133944} and
\eqref{eq:20220609133608} of \(\order[1]u\) and \(\order[0]x\) in the \(v_i\) basis
\begin{equation}
  \label{eq:20220617080547}
  ℰ₂(\order[1]x, \hat{u}) + ℰ₃(v_j, v_k, \hat{u}) \order[0]{χ_j} \order[1]{ξ_k} + \order[1]λ \dot{ℰ}₂(v_j, \hat{u}) \order[0]{χ_j} = \order[1]α \order[0]{χ_j} 〈 v_j, \hat{u} 〉.
\end{equation}

For \(\hat{u} = v_i\), observing that \(〈 v_i, v_j 〉 = δ_{ij}\) since
\((v_i)\) is orthonormal, the above equation reads
\begin{equation}
  \bigl[ℰ₃(λ₀; v_i, v_j, v_k) \order[1]{ξ_k} + \order[1]λ \dot{ℰ}₂(λ₀; v_i, v_j)\bigr] \order[0]{χ_j} = \order[1]α \order[0]{χ_i},
\end{equation}
which reduces to Eq.~\eqref{eq:20220609133608}.

The test function is now picked in \(W = V^\perp\), and \(\order[1]x\) is decomposed as
the sum of its projections onto \(V\) and \(W\): \(\order[1]x = \order[1]{χ_i} v_i + y₁\), where
\(y₁ ∈ W\). Eq.~\eqref{eq:20220617080547} then delivers the following
variational problem: find \(y₁ ∈ W\) such that, for all \(\hat{w} ∈ W\),
\begin{equation}
  ℰ₂(y₁, \hat{w}) + ℰ₃(v_i, v_j, \hat{w}) \order[0]{χ_i} \order[1]{ξ_j} + \order[1]λ \dot{ℰ}₂(v_i, \hat{w}) \order[0]{χ_i} = 0,
\end{equation}
(observe that \(〈 v_j, \hat{w} 〉\) since \(V\) and \(W\) are orthogonal
subspaces). The solution to the above problem is expressed as a linear
combination of the \(w_i\) and \(w_{ij}\) defined by the variational problems
\eqref{eq:20220524134525} and \eqref{eq:20220519164523}, respectively:
\(y₁ = \order[0]{χ_i} \order[1]{ξ_j} w_{i j} + \tfrac{1}{2} \order[1]λ \order[0]{χ_i} w_i\), and the
decomposition~\eqref{eq:20220609133629} is retrieved.

\paragraph{Variational problem of order 2} For all \(\hat{u} ∈ U\),
\begin{multline*}
  ℰ₂(\order[2]x, \hat{u}) + 2ℰ₃(\order[1]u, \order[1]x, \hat{u}) + 2 \order[1]λ \dot{ℰ}₂(\order[1]x, \hat{u}) + ℰ₄(\order[1]u, \order[1]u, \order[0]x, \hat{u}) + ℰ₃(\order[2]u, \order[0]x, \hat{u})\\
  + 2\order[1]λ \dot{ℰ}₃(\order[1]u, \order[0]x, \hat{u}) + \bigl( \order[1]λ \bigr)^2 \ddot{ℰ}₂(\order[0]x, \hat{u}) + \order[2]λ \dot{ℰ}₂(\order[0]x, \hat{u}) = 2 \order[1]α 〈 \order[1]x, \hat{u} 〉 + \order[2]α 〈 \order[0]x, \hat{u} 〉.
\end{multline*}

For \(\hat{u} = \hat{v}_i\), plugging the decompositions
\eqref{eq:20220524133944}, \eqref{eq:20220524134613}, \eqref{eq:20220609133608}
and \eqref{eq:20220609133629} of \(\order[1]u\), \(\order[2]u\), \(\order[0]x \) et \(\order[1]x\) delivers
% \begin{multline*}
%   2ℰ₃(v_i, \order[1]x, \order[1]u) + 2 \order[1]λ \dot{ℰ}₂(v_i, \order[1]x) + ℰ₄(v_i, \order[0]x, \order[1]u, \order[1]u) + ℰ₃(v_i, \order[0]x, \order[2]u)\\
%   + 2\order[1]λ \dot{ℰ}₃(v_i, \order[0]x, \order[1]u) + \bigl( \order[1]λ \bigr)^2 \ddot{ℰ}₂(v_i, \order[0]x) + \order[2]λ \dot{ℰ}₂(v_i, \order[0]x) = 2\order[1]α 〈 v_i, \order[1]x 〉 + \order[2]α 〈 v_i, \order[0]x 〉,
% \end{multline*}
% \begin{multline*}
%   2ℰ₃(v_i, χ₁^jv_j + \order[0]{χ_j}\order[1]{ξ_k}w_{jk}+\tfrac{1}{2} \order[1]λ \order[0]{χ_j} w_j, \order[1]{ξ_l} v_l) + 2 \order[1]λ \dot{ℰ}₂(v_i, χ₁^jv_j + \order[0]{χ_j}\order[1]{ξ_k}w_{jk}+\tfrac{1}{2} \order[1]λ \order[0]{χ_j} w_j)\\
%   + ℰ₄(v_i, \order[0]{χ_j} v_j, \order[1]{ξ_k} v_k, \order[1]{ξ_l} v_l) + ℰ₃(v_i, \order[0]{χ_j} v_j, \order[2]{ξ_k} v_k + \order[1]{ξ_k} \order[1]{ξ_l} w_{kl} + \order[1]λ \order[1]{ξ_k} w_k)\\
%   + 2\order[1]λ \dot{ℰ}₃(v_i, \order[0]{χ_j} v_j, \order[1]{ξ_k} v_k) + \bigl( \order[1]λ \bigr)^2 \ddot{ℰ}₂(v_i, \order[0]{χ_j} v_j) + \order[2]λ \dot{ℰ}₂(v_i, \order[0]{χ_j} v_j)\\
%   = 2\order[1]α 〈 v_i,  χ₁^jv_j + \order[0]{χ_j}\order[1]{ξ_k}w_{jk}+\tfrac{1}{2} \order[1]λ \order[0]{χ_j} w_j 〉 + \order[2]α 〈 v_i, \order[0]{χ_j} v_j〉,
% \end{multline*}
% \begin{multline*}
%   2ℰ₃(v_i, v_j,  v_k) χ₁^j \order[1]{ξ_k} + 2ℰ₃(v_i, w_{jk}, v_l) \order[0]{χ_j} \order[1]{ξ_k} \order[1]{ξ_l} + \order[1]λ ℰ₃(v_i,  w_j, v_k) \order[0]{χ_j} \order[1]{ξ_k} + 2 \order[1]λ \dot{ℰ}₂(v_i, v_j) χ₁^j\\
%   + 2 \order[1]λ \dot{ℰ}₂(v_i, w_{jk}) \order[0]{χ_j} \order[1]{ξ_k}+ \bigl( \order[1]λ \bigr)^2 \dot{ℰ}₂(v_i, w_j) \order[0]{χ_j} + ℰ₄(v_i, v_j,  v_k, v_l) \order[0]{χ_j} \order[1]{ξ_k} \order[1]{ξ_l}\\
%   + ℰ₃(v_i, v_j, v_k) \order[0]{χ_j} \order[2]{ξ_k} + ℰ₃(v_i, v_j, w_{kl}) \order[0]{χ_j} \order[1]{ξ_k} \order[1]{ξ_l} + \order[1]λ ℰ₃(v_i, v_j, w_k) \order[0]{χ_j} \order[1]{ξ_k}\\
%   + 2\order[1]λ \dot{ℰ}₃(v_i, v_j,  v_k) \order[0]{χ_j} \order[1]{ξ_k} + \bigl( \order[1]λ \bigr)^2 \ddot{ℰ}₂(v_i, v_j) \order[0]{χ_j} + \order[2]λ \dot{ℰ}₂(v_i, v_j) \order[0]{χ_j} = 2\order[1]α\order[1]{χ_i} + \order[2]α \order[0]{χ_i}.
% \end{multline*}
\begin{multline*}
  \bigl[ ℰ₄(v_i, v_j,  v_k, v_l) + 2ℰ₃(v_i, w_{jk}, v_l) + ℰ₃(v_i, v_j, w_{kl})\bigr] \order[0]{χ_j} \order[1]{ξ_k} \order[1]{ξ_l}\\
  + \order[1]λ \bigl[ ℰ₃(v_i,  w_j, v_k) + 2 \dot{ℰ}₂(v_i, w_{jk}) + ℰ₃(v_i, v_j, w_k) + 2 \dot{ℰ}₃(v_i, v_j,  v_k) \bigr] \order[0]{χ_j} \order[1]{ξ_k}\\
  + \bigl( \order[1]λ \bigr)^2 \bigl[\dot{ℰ}₂(v_i, w_j) + \ddot{ℰ}₂(v_i, v_j)\bigr] \order[0]{χ_j} + \bigl[ℰ₃(v_i, v_j, v_k) \order[2]{ξ_k} + \order[2]λ \dot{ℰ}₂(v_i, v_j)\bigr] \order[0]{χ_j} \\
  +2\bigl[ℰ₃(v_i, v_j,  v_k)  \order[1]{ξ_k} + \order[1]λ \dot{ℰ}₂(v_i, v_j)\bigr] χ₁^j = 2\order[1]α\order[1]{χ_i} + \order[2]α \order[0]{χ_i}.
\end{multline*}

The \(\order[0]{χ_j} \order[1]{ξ_k}\) term is transformed with Eqs.~\eqref{eq:20220524134525} and
\eqref{eq:20220519164523}
\begin{multline*}
  \bigl[ ℰ₄(v_i, v_j,  v_k, v_l) + ℰ₃(v_i, w_{jk}, v_l) + ℰ₃(v_i, w_{jl}, v_k) + ℰ₃(v_i, v_j, w_{kl})\bigr] \order[0]{χ_j} \order[1]{ξ_k} \order[1]{ξ_l}\\
  + \order[1]λ \bigl[ -ℰ₂(w_{ik},  w_j) - ℰ₂(w_i, w_{jk}) - ℰ₂(w_{ij}, w_k) + 2 \dot{ℰ}₃(v_i, v_j,  v_k) \bigr] \order[0]{χ_j} \order[1]{ξ_k}\\
  + \bigl( \order[1]λ \bigr)^2 \bigl[\dot{ℰ}₂(v_i, w_j) + \ddot{ℰ}₂(v_i, v_j)\bigr] \order[0]{χ_j} + \bigl[ℰ₃(v_i, v_j, v_k) \order[2]{ξ_k} + \order[2]λ \dot{ℰ}₂(v_i, v_j)\bigr] \order[0]{χ_j} \\
  +2\bigl[ℰ₃(v_i, v_j,  v_k)  \order[1]{ξ_k} + \order[1]λ \dot{ℰ}₂(v_i, v_j)\bigr] χ₁^j = 2\order[1]α\order[1]{χ_i} + \order[2]α \order[0]{χ_i},
\end{multline*}
and Eq.~\eqref{eq:20220616082923} results from the application of
Eqs.~\eqref{eq:20220617084433} and \eqref{eq:20220617085256}.

\end{document}

%%% Local Variables:
%%% coding: utf-8
%%% fill-column: 80
%%% mode: latex
%%% TeX-engine: xetex
%%% TeX-master: t
%%% End:
