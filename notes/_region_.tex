\message{ !name(LSK-notes.tex)}\newcommand{\sbtitle}{Notes relatives à la méthode asymptotique de Lyapunov–Schmidt–Koiter}
\newcommand{\sbauthor}{Sébastien Brisard}
\newcommand{\sbaddress}{Univ Gustave Eiffel, Ecole des Ponts, IFSTTAR, CNRS, Navier, F-77454 Marne-la-Vall\'ee, France}
\newcommand{\sbsubject}{Note bibliographique}

\documentclass[12pt, final]{amsart}

\usepackage{polyglossia}
\setdefaultlanguage{french}

\usepackage{amsfonts}
\usepackage{amsmath}
\usepackage{amssymb}

\usepackage{amsthm}
\theoremstyle{definition}
\renewcommand{\qedsymbol}{}
\newtheorem{remark}{Remarque}
\newtheorem{theorem}{Théorème}

\usepackage[backend=biber,bibencoding=utf8,doi=false,giveninits=true,isbn=false,maxnames=10,minnames=5,sortcites=true,style=authoryear,texencoding=utf8,url=false]{biblatex}
\addbibresource{stab.bib}

\usepackage[breaklinks=true, colorlinks=true, pdftitle={\sbtitle}, pdfauthor={\sbauthor}, pdfsubject={\sbsubject}, urlcolor=blue]{hyperref}

\usepackage[color={1 1 0}]{pdfcomment}

\usepackage{unicode-math}
\setmainfont{XITS}
\setmathfont{XITS Math}

\newcommand{\reals}{\mathbb{R}}

\begin{document}

\message{ !name(LSK-notes.tex) !offset(345) }
    \label{eq:20220531054218}
    ℰ_{, u u}[u(η), λ(η) ; \hat{u}, \hat{u}]
    ={} & ℰ₂(λ₀; \hat{u}, \hat{u}) + η [ℰ(λ₀; u₁, \hat{u}, \hat{u}) + λ₁ \dot{ℰ}₂ (λ₀; \hat{u}, \hat{u})]\\
    & + \tfrac{1}{2} η² [ℰ₄(λ₀; u₁, u₁, \hat{u}, \hat{u}) +ℰ₃(λ₀; u₂, \hat{u}, \hat{u}) + λ₂ \dot{ℰ}₂(λ₀; \hat{u}, \hat{u})\\
    & + 2 λ₁ \dot{ℰ}₃(λ₀; u₁, \hat{u}, \hat{u}) + λ₁^2 \ddot{ℰ}₂(λ₀; \hat{u}, \hat{u})] + o(η^2) .
  \end{aligned}
\end{equation}
On peut décomposer le vecteur \(\hat{u}∈U\) de façon unique sous la forme
\(\hat{u} = \hat{v} + \hat{w}\), avec \(\hat{v}∈V\) et \(\hat{w}∈W\). Le terme
constant du développement précédent vaut alors \(ℰ₂(λ₀; \hat{w}, \hat{w})\). Si
\(\hat{w} ≠ 0\), alors ce terme constant est strictement positif, puisque la
hessienne est définie positive sur \(W\) en \(λ = λ₀\). Au voisinage du point de
bifurcation, la hessienne sur la branche bifurquée est donc positive pour tout
\(\hat{u}∈U\) ayant une composante dans \(W\). Il suffit donc d'étudier le signe
de la hessienne sur la branche bifurquée pour \(\hat{u}∈V\), soit
\(\hat{u} = \hat{ξ}^i v_i\). L'expression~\eqref{eq:20220531054218} se simplifie
alors sous la forme suivante Compte-tenu de
l'expression~\eqref{eq:20220524134613}
\begin{e
\message{ !name(LSK-notes.tex) !offset(1136) }

\end{document}

%%% Local Variables:
%%% coding: utf-8
%%% fill-column: 80
%%% mode: latex
%%% TeX-engine: xetex
%%% TeX-master: t
%%% End:
