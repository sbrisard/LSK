\documentclass[12pt, final]{amsart}

\usepackage{polyglossia}
\setdefaultlanguage{french}

\usepackage{amsfonts}
\usepackage{amsmath}
\usepackage{amssymb}

\begin{document}
\title{Notes relatives à la méthode asymptotique de Lyapunov--Schmidt--Koiter}
\author{Sébastien Brisard}
\address{Navier, Ecole des Ponts, Univ Gustave Eiffel, IFSTTAR, CNRS, Marne-la-Vall\'ee, France}
\email{sebastien.brisard@univ-eiffel}

\begin{abstract}
  blabla
\end{abstract}


\maketitle


\section{Notations}

L'espace des champs cinématiquement admissibles est noté \(U\). On suppose
qu'il a la structure d'espace vectoriel. L'énergie du système est noté \(\mathcal E\).

\section{Analyse de la branche fondamentale}

Soit \(u_0\in U\) un point d'équilibre du système~: l'énergie \(\mathcal E\) est
stationnaire en \(u_0\). On s'intéresse à la stabilité de cet
équilibre. L'équilibre est stable si \(\mathcal E\) est minimale en ce point. On sait
que \(\mathcal E_{,uu}(u_0)\geq0\) est une condition \emph{nécessaire} de stabilité. De
plus, \(\mathcal E_{,uu}(u_0)>0\) est une condition \emph{suffisante} de stabilité.

On s'intéresse ici au cas où \(\mathcal E_{,uu}(u_0)\) est positive, sans être définie
positive~; soit \(V\) son noyau, qui forme un sous-espace vectoriel de
\(U\). On suppose que \(V\) est de dimension finie \(m=\dim V\). Soit
\(\bigl(v_1, \ldots, v_m\bigr)\) une base orthonormée de ce noyau pour le produit
scalaire \(\langle \bullet ,\bullet \rangle \) (qui n'est pas précisé pour le moment). On introduit le
sous-espace supplémentaire orthogonal \(W\) de \(V\) dans \(U\)
\begin{equation}
  U=V\stackrel{\perp}{\otimes}W.
\end{equation}

Pour étudier la stabilité de l'équilibre, on calcule l'énergie dans un état
\(u_0+\xi v+\eta w\) voisin du point d'équilibre \(u_0\), avec \(\xi , \eta \in\mathbb R \)
\guillemotleft{}petits\guillemotright{}, \(v\in V\) and \(w\in W\). On obtient alors,
à l'ordre 4 en \(\xi \) et \(\eta \)
\begin{eqnarray}
  \notag
  \Delta \mathcal E&=&\mathcal E(u_0+\xi v+\eta w)-\mathcal E(u_0)\\
  \notag
    &=&\tfrac12\mathcal E_{,uu}(u_0;\xi v+\eta w, \xi v+\eta w)
    +\tfrac16\mathcal E_{,uuu}(u_0;\xi v+\eta w, \xi v+\eta w, \xi v+\eta w)\\
    &&+\tfrac1{24}\mathcal E_{,uuuu}(u_0;\xi v+\eta w, \xi v+\eta w, \xi v+\eta w, \xi v+\eta w)
    +\mathcal{O}\bigl(\bigl(\xi ²+\eta ²\bigr)²\bigr),
\end{eqnarray}
où le terme linéaire a été omis puisque \(u\) est un point critique de
l'énergie. En tenant compte de la multilinéarité et de la symétrie des
différentielles successives de l'énergie \(\mathcal E\), ainsi que du fait que
\(\mathcal E_{,uu}(v, \bullet )=0\), l'expression précédente s'écrit
\begin{eqnarray}
  \notag
    \Delta \mathcal E&=&\tfrac12\eta ²\mathcal E_{,uu}(w, w)+\tfrac16\xi ^3\mathcal E_{,uuu}(v, v, v)
          +\tfrac12\xi ²\eta \mathcal E_{,uuu}(v, v, w)\\
  \notag
      &&+\tfrac12\xi \eta ²\mathcal E_{,uuu}(v, w, w)+\tfrac16\eta ^3\mathcal E_{,uuu}(w, w, w)\\
  \notag
      &&+\tfrac1{24}\xi ^4\mathcal E_{,uuuu}(v, v, v, v)+\tfrac16\xi ^3\mathcal E_{,uuuu}(v, v, v, w)\\
  \notag
      &&+\tfrac14\xi ²\eta ²\mathcal E_{,uuuu}(v, v, w, w)+\tfrac16\xi \eta ^3\mathcal E_{,uuuu}(v, w, w, w)\\
    &&+\tfrac1{24}\eta ^4\mathcal E_{,uuuu}(w, w, w, w)+\mathcal{O}\bigl(\bigl(\xi ²+\eta ²\bigr)²\bigr),
\end{eqnarray}
où l'on convient que toutes les différentielles de \(\mathcal E\) sont évaluées au point
d'équilibre \(u_0\).

Pour que l'équilibre soit stable, il faut que expression soit positive ou nulle
pour tous \(\xi \) et \(\eta \) suffisamment petits. En prenant tout d'abord \(\eta =0\),
on obtient les conditions nécessaires
\begin{equation}
  \label{eq20211108164416}
  \mathcal E_{,uuu}(v, v, v)=0\quad\text{et}\quad\mathcal E_{,uuuu}(v, v, v, v)\geq0
  \quad\text{pour tout}\quad v\in V.
\end{equation}

En d'autres termes, s'il existe \(v\in V\) tel que \(\mathcal E_{,uuu}(v, v, v)\neq0\) ou
\(\mathcal E_{,uuuu}(v, v, v, v)<0\), alors l'équilibre est \emph{instable}. Les
conditions précédentes ne sont pas suffisantes pour assurer la stabilité. En
effet, supposant ces conditions remplies, on prend maintenant \(\eta =\xi ²\)
\begin{equation}
  \Delta \mathcal E=\tfrac12\xi ^4\bigl[\mathcal E_{,uu}(w, w)+\mathcal E_{,uuu}(v, v, w)
  +\tfrac1{12}\mathcal E_{,uuuu}(v, v, v, v)\bigr]+o(\xi ^4)
\end{equation}
et on obtient la condition nécessaire supplémentaire
\begin{equation}
  \label{eq20211109145356}
  \mathcal E_{,uu}(v, v)+\mathcal E_{,uuu}(v, v, w)+\tfrac1{12}\mathcal E_{,uuuu}(v, v, v, v)\geq0,
\end{equation}
pour tous \(v\in V\) et \(w\in W\). Pour \(v\in\ker\mathcal E_2\) fixé, l'expression précédente
est minimale lorsque \(w\) satisfait le problème variationnel
\begin{equation}
  \label{eq20211109145224}
  2\mathcal E_{,uu}(w, \hat{w})+\mathcal E_{,uuu}(v, v, \hat{w})=0\quad\text{pour tout}\quad\hat{w}\in W.
\end{equation}

Soit \(w_{ij}\in W\) l'unique solution du problème variationnel suivant
\begin{equation}
  \label{eq20211221155859}
  2\mathcal E_{,uu}(w_{ij}, \hat{w})+\mathcal E_{,uuu}(v_i, v_j, \hat{w})=0\quad\text{pour tout}\quad\hat{w}\in W.
\end{equation}

Alors, pour \(v=\xi ^i v_i\), la solution du problème
variationnel~\eqref{eq20211109145224} est \(w=\xi ^i\xi ^jw_{ij}\). Pour cette
valeur de \(v\), la condition~\eqref{eq20211109145356} s'écrit
\begin{equation}
  \bigl[\tfrac1{12}\mathcal E_{,uuuu}(v_i, v_j, v_k, v_l)-\mathcal E_{,uu}(w_{ij}, w_{kl})\bigr]\xi ^i\xi ^j\xi ^k\xi ^l\geq0,
\end{equation}
pour tous \(\xi _i, \xi _j, \xi _k, \xi _l\in\mathbb R \). On peut montrer que l'inégalité stricte est
une condition \emph{suffisante} de stabilité.

\section{Bifurcations}

Soit \(u^*(\lambda )\) la branche fondamentale. Par définition
\begin{equation}
  \mathcal E_{,u}[u^*(\lambda ), \lambda ;\hat{u}]=0\quad\text{pour tout}\quad\hat{u}\in U.
\end{equation}

Il sera commode d'introduire les notations suivantes
\begin{eqnarray}
  \mathcal E_2(\lambda )&=&\mathcal E_{,uu}[u^*(\lambda ), \lambda ],\\
  \mathcal E_3(\lambda )&=&\mathcal E_{,uuu}[u^*(\lambda ), \lambda ],\\
  \mathcal E_4(\lambda )&=&\mathcal E_{,uuuu}[u^*(\lambda ), \lambda ].
\end{eqnarray}

Noter que \(\mathcal E_2\), \(\mathcal E_3\) et \(\mathcal E_4\) sont des formes bi-, tri- et quadri-linéaires,
respectivement. L'application de ces formes à des éléments de \(U\) sera notée
\(\mathcal E_2(\lambda ; u, v)\), \(\mathcal E_3(\lambda ; u, v, w)\), etc\dots. La dérivée de ces formes par
rapport à \(\lambda \) sera notée à l'aide d'un point supérieur (\(\dot{\mathcal E_2}\),
\(\dot{\mathcal E_3}\), \dots).

On suppose que l'équilibre est stable pour des valeurs suffisamment petites de
\(\lambda \). Plus précisément, on suppose que \(\mathcal E_2(\lambda )\) est définie positive pour tout
\(\lambda <\lambda _0\). Pour \(\lambda =\lambda _0\), la forme quadratique \(\mathcal E_2(\lambda _0)\) n'est plus que
positive. En notant \(u_0=u^*(\lambda _0)\) la position d'équilibre obtenue pour la
valeur critique \(\lambda _0\) du paramètre de chargement \(\lambda \), on s'intéresse à
toutes les courbes d'équilibre qui passent par le point \((u_0, \lambda _0)\).

Noter que dans ce qui suit, on convient que les formes \(\mathcal E_2\), \(\mathcal E_3\) et \(\mathcal E_4\)
sont implicitement évaluées en \(\lambda _0\) lorsque \(\lambda \) n'est pas rappelé : ainsi,
on notera \(\mathcal E_2(\bullet , \bullet )\) plutôt que \(\mathcal E_2(\lambda _0; \bullet , \bullet )\).

On écrit toute courbe d'équilibre passant par le point \((u_0, \lambda _0)\) sous la
forme paramétrique suivante
\begin{eqnarray}
  \label{eq20211115075817}
  \lambda &=&\lambda _0+\eta \lambda _1+\eta ²\lambda _2+\eta ^3\lambda _3+\cdots,\\
  \label{eq20211115075835}
  u&=&u^*(\lambda )+\eta u_1+\eta ²u_2+\eta ^3u_3+\cdots,
\end{eqnarray}
où \(\eta \) est un paramètre, non précisé pour le moment. Noter que, dans la
représentation paramétrique de \(u\), \(u^*\) est évalué en \(\lambda \) et pas en
\(\lambda _0\).

On se restreindra dans ce qui suit au cas non-dégénéré \(u_1\neq0\). On peut alors
toujours supposer que \(\langle u_1, u_1\rangle =1\). \textbf{Que se passe-t-il si
  \(u_1=u_2=\ldots=0 ?\)} En effet, en posant \(\theta=\lVert u_1\rVert \eta \), les développements
précédents s'écrivent
\begin{eqnarray}
  \lambda &=&\lambda _0+\theta\lVert u_1\rVert^{-1}\lambda _1+\theta²\lVert u_1\rVert^{-2}\lambda _2+\theta^3\lVert u_1\rVert^{-3}\lambda _3+\cdots,\\
  u&=&u^*(\lambda )+\theta\lVert u_1\rVert^{-1}u_1+\theta²\lVert u_1\rVert^{-2}u_2+\theta^3\lVert u_1\rVert^{-3}u_3+
\end{eqnarray}
et le terme linéaire en \(\theta\) du développement asymptotique de \(u\) est bien
de norme 1.

Les coefficients \(\lambda  _k\) et \(u_k\) des développements~\eqref{eq20211115075817}
et \eqref{eq20211115075835} sont identifiés en écrivant que l'énergie est
stationnaire le long de la courbe d'équilibre, c'est-à-dire que le résidu
\(\mathcal E_{,u}[u(\eta ), \lambda (\eta )]\) est nul. Le développement limité du résidu est établi au
voisinage de \(\eta =0\) dans l'annexe~\ref{sec20211112182000} [voir
Éq.~\eqref{eq20220107080901}]. En écrivant que tous ses termes s'annulent, on
trouve successivement, pour tout \(\hat{u}\in U\)
\begin{equation}
  \label{eq20211112182917}
  \mathcal E_2(\lambda _0; u_1, \hat{u})=0,
\end{equation}
\begin{equation}
  \label{eq20211112183220}
  \mathcal E_3(\lambda _0; u_1, u_1, \hat{u})+2\lambda _1\dot{\mathcal E_2}(\lambda _0; u_1, \hat{u})+2\mathcal E_2(\lambda _0; u_2, \hat{u})=0,
\end{equation}
\begin{eqnarray}
  \notag
  \mathcal E_4(\lambda _0; u_1, u_1, u_1, \hat{u})+6\mathcal E_3(\lambda _0; u_1, u_2, \hat{u})+6\mathcal E_2(\lambda _0; u_3, \hat{u})&&\\
  \notag
  +3\lambda _1\bigl[\dot{\mathcal E_3}(\lambda _0; u_1, u_1, \hat{u})+2\dot{\mathcal E_2}(\lambda _0; u_2, \hat{u})\bigr]&&\\
  \label{eq20220114135717}
  +3\lambda _1²\ddot{\mathcal E_2}(\lambda _0; u_1, \hat{u})
  +6\lambda _2\dot{\mathcal E_2}(\lambda _0; u_1, \hat{u})&=&0.
\end{eqnarray}

On déduit de l'équation~\eqref{eq20211112182917} que \(u_1\in V\). On pose alors
\begin{equation}
  \label{eq20220124135236}
  u_1=\xi _1^i v_i.
\end{equation}

En prenant \(\hat{u}=v_i\), l'équation~\eqref{eq20211112183220} s'écrit
\begin{equation}
  \label{eq20220216140121}
  \mathcal E_{ijk}(\lambda _0)\,\xi _1^j\xi _1^k+2\lambda _1\dot{\mathcal E}_{ij}(\lambda _0)\,\xi _1^j=0.
\end{equation}

Pour le terme d'ordre 2, on introduit la décomposition~:
\(u_2=\xi _2^iv_i+u_2^W\), où \(u_2^W\in W\). On a alors
\(\mathcal E_2(u_2, \hat{u})=\mathcal E_2(u_2^W, \hat{u})\) et l'équation~\eqref{eq20211112183220}
s'écrit
\begin{equation}
  \mathcal E_3(\lambda _0; u_1, u_1, \hat{u})+2\lambda _1\dot{\mathcal E_2}(\lambda _0; u_1, \hat{u})
  +2\mathcal E_2(\lambda _0; u_2^W, \hat{u})=0,
\end{equation}
pour tout \(\hat{u}\in V\). En prenant la fonction test dans l'espace \(W\), on
obtient le problème variationnel suivant~: trouver \(u_2^W\in W\) tel que
\begin{equation}
  \label{eq20211210131623}
  \xi _1^i\xi _1^j\mathcal E_3(\lambda _0; v_i, v_j, \hat{w})
  +2\lambda _1\xi _1^i\dot{\mathcal E_2}(\lambda _0; v_i, \hat{w})
  +2\mathcal E_2(\lambda _0; u_2^W, \hat{w})=0,
\end{equation}
pour tout \(\hat{w}\in W\). Soient \(w_i\in W\) les solutions des problèmes
variationnels suivants
\begin{equation}
  \label{eq20220208143055}
  \mathcal E_2(\lambda _0; w_i, \hat{w})+\dot{\mathcal E_2}(\lambda _0; v_i, \hat{w})=0,
\end{equation}
pour tout \(\hat{w}\in W\). La solution du problème~\eqref{eq20211210131623}
s'obtient par simple combinaison linéaire des \(w_i\) et \(w_{ij}\) introduits
précédemment par le problème variationnel~\eqref{eq20211221155859}, de sorte
que
\begin{equation}
  \label{eq20220124135324}
  u_2^W=\xi _1^i\xi _1^jw_{ij}+\lambda _1\xi _1^i w_i
  \quad\text{et}\quad
  u_2=\xi _2^iv_i+\xi _1^i\xi _1^jw_{ij}+\lambda _1\xi _1^i w_i.
\end{equation}

En prenant \(\hat{u}=v_i\) dans l'équation~\eqref{eq20220114135717}, on
obtient l'équation de bifurcation suivante
\begin{eqnarray}
  \notag
  6\xi _2^j\bigl[\xi _1^k\mathcal E_{ijk}(\lambda _0)+\lambda _1\dot{\mathcal E}_{ij}(\lambda _0)\bigr]&&\\
  \notag
  +\xi _1^j\xi _1^k\xi _1^l\bigl[\mathcal E_{ijkl}(\lambda _0)+6\mathcal E_3(\lambda _0; v_i, v_j, w_{kl})\bigr]&&\\
  \notag
    +3\lambda _1\xi _1^j\xi _1^k\bigl[\dot{\mathcal E}_{ijk}(\lambda _0)+2\mathcal E_3(\lambda _0; v_i, v_j, w_k)
    +2\dot{\mathcal E_2}(\lambda _0; v_i, w_{jk})\bigr]&&\\
    \label{eq20220210143805}
    +3\lambda _1²\xi _1^j\bigl[\ddot{\mathcal E}_{ij}(\lambda _0)+2\dot{\mathcal E_2}(v_i, w_j)\bigr]
    +6\lambda _2\xi _1^j\dot{\mathcal E}_{ij}(\lambda _0)
    &=&0.
\end{eqnarray}

On remarque que certains termes peuvent être symétrisés. Ainsi
\begin{eqnarray}
  \notag
  \xi _1^j\xi _1^k\xi _1^l\mathcal E_3(\lambda _0; v_i, v_j, w_{kl})
  &=&\tfrac13\xi _1^j\xi _1^k\xi _1^l\bigl[\mathcal E_3(\lambda _0; v_i, v_j, w_{kl})\\
  &&+\mathcal E_3(\lambda _0; v_i, v_k, w_{lj})+\mathcal E_3(\lambda _0; v_i, v_l, w_{jk})\bigr],
\end{eqnarray}
de même
\begin{equation}
  2\xi _1^j\xi _1^k\mathcal E_3(\lambda _0; v_i, v_j, w_k)=\xi _1^j\xi _1^k\bigl[\mathcal E_3(\lambda _0; v_i, v_j, w_k)
  +\mathcal E_3(\lambda _0; v_i, w_j, v_k)\bigr]
\end{equation}
et l'équation~\eqref{eq20220210143805} s'écrit
\begin{equation}
\label{eq20220216141706}
  6A_{ij}\xi _2^j+E_{ijkl}\,\xi _1^j\xi _1^k\xi _1^l+3\lambda _1F_{ijk}\xi _1^j\xi _1^k+3\lambda _1²G_{ij}\xi _1^j
  +6\lambda _2\mathring{E}_{ij}\xi _1^j=0,
\end{equation}
en posant \textbf{C'est l'expression de \(B_{ij}\) de Nick, voir
  Éq. (AC-5.14) p. 74}
\begin{equation}
  A_{ij}=\xi _1^k\mathcal E_{ijk}(\lambda _0)+\lambda _1\dot{\mathcal E}_{ij}(\lambda _0)
\end{equation}
\textbf{Cette expression coïncide avec l'expression
  (AC-5.11), page 71, de \(\mathcal E_{ijkl}\) de Nick. Le facteur 2 provient du fait que
  dans le problème variationnel (AC-5.9) qui définit les \(v_{ij}\) de Nick, le
  facteur 2 du problème~\eqref{eq20211221155859} n'est pas présent.}
\begin{equation}
  E_{ijkl}=\mathcal E_4(\lambda _0; v_i, v_j, v_k, v_l)+2\bigl[\mathcal E_3(\lambda _0; v_i, v_j, w_{kl})
  +\mathcal E_3(\lambda _0; v_i, v_k, w_{lj})+\mathcal E_3(\lambda _0; v_i, v_l, w_{jk})\bigr].
\end{equation}

\begin{equation}
  F_{ijk}=\dot{\mathcal E}_3(\lambda _0; v_i, v_j, v_k)+\mathcal E_3(\lambda _0; v_i, v_j, w_k)
  +\mathcal E_3(\lambda _0; v_i, w_j, v_k)+2\dot{\mathcal E_2}(\lambda _0; v_i, w_{jk})
\end{equation}

\begin{equation}
  G_{ij}=\ddot{\mathcal E}_{ij}(\lambda _0)+2\dot{\mathcal E_2}(v_i, w_j)
\end{equation}

On supposera satisfaite la condition suivante, qui assure que ce système est régulier
\begin{equation}
  \det\bigl(\xi _1^k\mathcal E_{ijk}+\lambda _1\dot{\mathcal E}_{ij}\bigr)_{i,j}\neq0.
\end{equation}

Les \(\xi _2^i\) sont donc déterminés de façon unique si \(\lambda _1\), \(\lambda _2\) et \(\xi _1^i\)
sont connus.

Le développement limité suivant de l'énergie le long de la branche bifurquée
est établi dans l'annexe~\ref{sec20220121172919}
\begin{eqnarray}
  \notag
  \mathcal E[u(\eta ), \lambda (\eta )]
  &=&\mathcal E\bigl(u^*[\lambda (\eta )], \lambda (\eta )\bigr)+\tfrac12\eta ²\mathcal E_2(\lambda _0; u_1, u_1)\\
  \notag
  &&+\tfrac16\eta ^3\bigl[\mathcal E_3(\lambda _0;u_1, u_1, u_1)+6\mathcal E_2(\lambda _0; u_1, u_2)\\
  \notag
  &&+3\lambda _1\dot{\mathcal E}_2(\lambda _0; u_1, u_1)\bigr]+\tfrac1{24}\eta ^4\bigl\{\mathcal E_4(\lambda _0;u_1, u_1, u_1, u_1)\\
  \notag
  &&+12\mathcal E_3(\lambda _0; u_1, u_1, u_2)+12\mathcal E_2(\lambda _0; u_2, u_2)\\
  \notag
  &&+18\mathcal E_2(\lambda _0; u_1, u_3)+4\lambda _1\bigl[\dot{\mathcal E}_3(\lambda _0; u_1, u_1, u_1)\\
  \notag
  &&+6\dot{\mathcal E}_2(\lambda _0; u_1, u_2)\bigr]+6\lambda _1²\ddot{\mathcal E}_2(\lambda _0; u_1, u_1)\\
  \label{eq20220121172753}
  &&+12\lambda _2\dot{\mathcal E}_2(\lambda _0; u_1, u_1)\bigr\}+o(\eta ^4).
\end{eqnarray}

La relation~\eqref{eq20211112182917} montre tout d'abord que les termes en
\(\mathcal E_2(\lambda _0; u_1, u_1)\), \(\mathcal E_2(\lambda _0; u_1, u_2)\) et \(\mathcal E_2(\lambda _0; u_1, u_3)\) sont nuls. Le
premier terme non-nul du développement limité~\eqref{eq20220121172753} est
donc le terme d'ordre 3. En prenant de plus \(\hat{u}=u_1\) dans la
relation~\eqref{eq20211112183220}, on trouve finalement \textbf{Cette
  expression coïncide avec l'Éq. (AC-5.29) de Tryantafyllidis.}
\begin{equation}
  \mathcal E[u(\eta ), \lambda (\eta )]=\mathcal E\bigl(u^*[\lambda (\eta )], \lambda (\eta )\bigr)+\tfrac16\lambda _1\eta ^3\dot{\mathcal E}_2(u_1, u_1)+o(\eta ^3).
\end{equation}

Si \(\lambda _1=0\), le premier terme non-nul du développement
limité~\eqref{eq20220121172753} est d'ordre 4. En prenant cette fois
\(\hat{u}=u_2\) dans la relation~\eqref{eq20211112183220} et \(\hat{u}=u_1\)
dans la relation~\eqref{eq20220114135717}, on obtient \textbf{Cette
  expression coïncide avec l'Éq. (AC-5.30) de Tryantafyllidis.}
\begin{equation}
  \mathcal E[u(\eta ), \lambda (\eta )]=\mathcal E\bigl(u^*[\lambda (\eta )], \lambda (\eta )\bigr)+\tfrac1{4}\lambda _2\eta ^4\dot{\mathcal E}_2(\lambda _0; u_1, u_1)+o(\eta ^4).
\end{equation}

\begin{center}
  ***
\end{center}

Pour analyser la stabilité de la branche bifurquée ainsi trouvée, il faut
déterminer le signe de la hessienne de l'énergie. On peut d'ores et déjà
remarquer que, sur la branche fondamentale (\(u_1=u_2=0\)), en prenant \(\eta =\lambda -\lambda _0\)
(\(\lambda _1=1\))
\begin{equation}
  \mathcal E_2(\lambda ; \hat{u}, \hat{v})
  =\mathcal E_2(\lambda _0; \hat{u}, \hat{v})+\bigl(\lambda -\lambda _0\bigr)\dot{\mathcal E}_2(\lambda _0; \hat{u}, \hat{v})+o(\lambda -\lambda _0).
\end{equation}

Dans ce qui suit, on supposera que \(\dot{\mathcal E}_2(\lambda _0)\neq0\). Pour \(\hat{v}\in V\),
l'égalité précédente s'écrit
\begin{equation}
  \mathcal E_2(\lambda _0; \hat{v}, \hat{v})=\bigl(\lambda -\lambda _0\bigr)\dot{\mathcal E}_2(\hat{v}, \hat{v})+o(\lambda -\lambda _0).
\end{equation}

Comme la branche fondamentale est stable pour \(\lambda <\lambda _0\), on doit avoir
\(\dot{\mathcal E}_2(\lambda _0; \hat{v}, \hat{v})<0\). La forme quadratique \(\dot{\mathcal E}_2(\lambda _0)\) est
donc définie négative sur \(V\). Le développement limité de la hessienne de
l'énergie le long de la branche bifurquée est établi dans
l'annexe~\ref{sec20211115081016}. Pour tout \(\hat{u}\in U\), on trouve
\begin{eqnarray}
  \notag
  \mathcal E_{,uu}[u(\eta ), \lambda (\eta ); \hat{u}, \hat{u}]
  &=&\mathcal E_2(\lambda _0; \hat{u}, \hat{u})+\eta \bigl[\mathcal E_3(\lambda _0; u_1, \hat{u}, \hat{u})
      +\lambda _1\dot{\mathcal E}_2(\lambda _0; \hat{u}, \hat{u})\bigr]\\
  \notag
  &&+\tfrac12\eta ²\bigl[\mathcal E_4(\lambda _0; u_1, u_1, \hat{u}, \hat{u})
     +2\lambda _1\dot{\mathcal E}_3(\lambda _0; u_1, \hat{u}, \hat{u})\\
  \notag
  &&+\lambda _1²\ddot{\mathcal E}_2(\lambda _0; \hat{u}, \hat{u})
     +\mathcal E_3(\lambda _0; u_2, \hat{u}, \hat{u})\\
  \label{eq20211115082025}
  &&+\lambda _2\dot{\mathcal E}_2(\lambda _0; \hat{u}, \hat{u})\bigr]+o(\eta ²).
\end{eqnarray}

On peut décomposer le vecteur \(\hat{u}\in U\) de façon unique sous la forme
\(\hat{u}=\hat{v}+\hat{w}\), avec \(\hat{v}\in V\) et \(\hat{w}\in W\). Le terme
constant du développement précédent vaut alors \(\mathcal E_2(\lambda _0; \hat{w}, \hat{w})\). Si
\(\hat{w}\neq0\), alors ce terme constant est strictement positif, puisque la
hessienne est définie positive sur \(W\) en \(\lambda =\lambda _0\). La hessienne sur la
branche bifurquée est donc positive pour tout \(\hat{u}\in U\) ayant une
composante dans \(W\). Il suffit donc d'étudier le signe de la hessienne sur la
branche bifurquée pour \(\hat{u}\in V\), soit \(\hat{u}=\hat{\xi }^iv_i\). Dans ce
cas, compte-tenu de l'expression~\eqref{eq20220124135324} de \(u_2\)
\begin{eqnarray}
  \notag
  \mathcal E_3(\lambda _0; u_2, \hat{u}, \hat{u})
  &=&\xi _1^i\xi _1^j\hat{\xi }^k\hat{\xi }^l\mathcal E_3(\lambda _0; w_{ij}, v_k, v_l)
      +\xi _2^i\hat{\xi }^j\hat{\xi }^k\mathcal E_3(\lambda _0; v_i, v_j, v_k)\\
  &&+\lambda _1\xi _1^i\hat{\xi }^j\hat{\xi }^k\mathcal E_3(\lambda _0; w_i, v_j, v_k).
\end{eqnarray}

On peut complètement symétriser le premier terme
\begin{eqnarray}
  \notag
  \xi _1^i\xi _1^j\hat{\xi }^k\hat{\xi }^l\mathcal E_3(\lambda _0; w_{ij}, v_k, v_l)
  &=&\tfrac13\bigl[\xi _1^i\xi _1^j\hat{\xi }^k\hat{\xi }^l\mathcal E_3(\lambda _0; w_{ij}, v_k, v_l)\\
  \notag
  &&+\xi _1^i\xi _1^j\hat{\xi }^k\hat{\xi }^l\mathcal E_3(\lambda _0; w_{ij}, v_k, v_l)\\
  &&+\xi _1^i\xi _1^j\hat{\xi }^k\hat{\xi }^l\mathcal E_3(\lambda _0; w_{ij}, v_k, v_l)\bigr]
\end{eqnarray}

\begin{eqnarray}
  \notag
  \mathcal E_{,uu}[u(\eta ), \lambda (\eta ); \hat{u}, \hat{u}]
  &=&\eta \hat{\xi }^i\hat{\xi }^j\bigl[\xi _1^k\mathcal E_{ijk}(\lambda _0)+\lambda _1\dot{\mathcal E}_{ij}(\lambda _0)\bigr]\\
  \notag
  &&+\tfrac12\eta ²\hat{\xi }^i\hat{\xi }^j\bigl\{\xi _1^k\xi _1^l\bigl[\mathcal E_{ijkl}(\lambda _0)
     -2\mathcal E_2(\lambda _0; w_{ij}, w_{kl})\bigr]\\
  \notag
  &&+\lambda _1\xi _1^k\bigl[\mathcal E_3(\lambda _0; v_i, v_j, w_k)+\dot{\mathcal E}_{ijk}(\lambda _0)\bigr]\\
  \label{eq20220203144500}
  &&+\lambda _1²\ddot{\mathcal E}_{ij}(\lambda _0)+\xi _2^k\mathcal E_{ijk}(\lambda _0)+\lambda _2\dot{\mathcal E}_{ij}\bigr\}+o(\eta ²).
\end{eqnarray}

Compte-tenu de la relation~\eqref{eq20211112183220}, on trouve pour
\(\hat{v}=u_1\) (\(\hat{\xi }^i=\xi _1^i\))
\begin{equation}
  \mathcal E_{,uu}[u(\eta ), \lambda (\eta ); u_1, u_1]=-\lambda _1\eta \dot{\mathcal E}_2(\lambda _0; u_1, u_1)+o(\eta ).
\end{equation}

Si \(\lambda _1\neq0\), l'expression précédente peut également s'écrire
\begin{equation}
  \mathcal E_{,uu}[u(\eta ), \lambda (\eta ); u_1, u_1]=-\bigl(\lambda -\lambda _0\bigr)\dot{\mathcal E}_2(\lambda _0; u_1, u_1)+o(\lambda -\lambda _0),
\end{equation}
qui est négative pour \(\lambda <\lambda _0\): la branche bifurquée est instable sous la
charge critique. Il reste alors à étudier le signe de la hessienne de la
branche bifurquée au-delà de la charge critique (\(\lambda >\lambda _0\)).

\section{Cas d'un mode de flambement simple (\(m=1\))}

Lorsque \(m=\dim V=1\), la base \(v_1, \ldots, v_m\) est réduite au seul vecteur
\(v_1\) et \(u_1\) est parallèle à ce vecteur. Comme \(\lVert u_1\rVert=1\), on a
donc nécessairement \(u_1=v_1\) (quitte à changer \(\eta \) en \(-\eta \)). L'équation de
bifurcation~\eqref{eq20220216140121} s'écrit alors
\begin{equation}
  \label{eq20220203144712}
  \mathcal E_{111}(\lambda _0)+2\lambda _1\dot{\mathcal E}_{11}(\lambda _0)=0,
  \quad\text{soit}\quad
  \lambda _1=-\frac{\mathcal E_{111}(\lambda _0)}{2\dot{\mathcal E}_{11}(\lambda _0)},
\end{equation}
où on remarque que le quotient a un sens, puisque \(\dot{\mathcal E_2}(\lambda _0)\) est définie
négative sur \(V\). On trouve donc les développements limités
\begin{equation}
  \lambda =\lambda _0+\lambda _1\eta +o(\eta )
  \quad\text{et}\quad
  u=u^*(\lambda )+\eta v_1+o(\eta ),
\end{equation}
soit finalement, en éliminant \(\eta \)
\begin{equation}
  \lambda =\lambda _0-\frac{\xi \mathcal E_{111}(\lambda _0)}{2\dot{\mathcal E}_{11}(\lambda _0)}+o(\xi ),
  \quad\text{avec}\quad
  \xi =\langle u(\lambda )-u^*(\lambda ), v_1\rangle .
\end{equation}

Pour déterminer la stabilité de la branche bifurquée, on calcule la hessienne
en \((v_1, v_1)\). L'équation~\eqref{eq20220203144500} s'écrit
\begin{equation}
  \mathcal E_{,uu}[u(\eta ), \lambda (\eta ); v_1, v_1]=\eta \bigl[\mathcal E_{111}(\lambda _0)+\lambda _1\dot{\mathcal E}_{11}(\lambda _0)\bigr]+o(\eta ),
\end{equation}
soit, en substituant l'équation~\eqref{eq20220203144712}
\begin{equation}
  \mathcal E_{,uu}[u(\eta ), \lambda (\eta ); v_1, v_1]=-\lambda _1\eta \dot{\mathcal E}_{11}(\lambda _0)+o(\eta ).
\end{equation}

Ce développement ne permet de conclure que si le terme linéaire est non-nul,
soit \(\mathcal E_{111}(\lambda _0)\neq0\) [voir Éq.~\eqref{eq20220203144712}]. Dans ce cas, le
développement asymptotique précédent s'écrit également
\begin{equation}
  \mathcal E_{,uu}[u(\eta ), \lambda (\eta ); v_1, v_1]=-\bigl(\lambda -\lambda _0\bigr)\dot{\mathcal E}_{11}(\lambda _0)+o(\lambda -\lambda _0).
\end{equation}

Comme \(\dot{\mathcal E}_2(\lambda _0)\) est définie négative, la branche bifurquée est
\emph{instable} pour \(\lambda <\lambda _0\) et \emph{stable} pour \(\lambda >\lambda _0\) lorsque
\(\mathcal E_{111}(\lambda _0)\neq0\).

Supposons maintenant que \(\mathcal E_{111}(\lambda _0)=0\)~; alors \(\lambda _1=0\) et il faut calculer au
moins un terme supplémentaire dans le développement limité de la
Hessienne. L'équation de bifurcation~\eqref{eq20220216141706} s'écrit
\begin{equation}
  \label{eq20220217164528}
  \mathcal E_{1111}(\lambda _0)+6\mathcal E_3(\lambda _0; v_1, v_1, u_2)+6\lambda _2\dot{\mathcal E}_{11}(\lambda _0)=0.
\end{equation}
En introduisant le développement~\eqref{eq20220124135324} de \(u_2\) et en
utilisant le problème variationnel~\eqref{eq20211221155859}
\begin{equation}
  u_2=\xi _2v_1+w_{11}+\lambda _1w_1,
\end{equation}
donc
\begin{equation}
  \mathcal E_3(\lambda _0;v_1, v_1, u_2)=\mathcal E_3(\lambda _0;v_1, v_1, w_{11})=-2\mathcal E_2(\lambda _0;w_{11}, w_{11})
\end{equation}
soit finalement
\begin{equation*}
  \lambda _2=-\frac{\mathcal E_{1111}(\lambda _0)-12\mathcal E_2(\lambda _0;w_{11}, w_{11})}{6\dot{\mathcal E}_{11}(\lambda _0)},
\end{equation*}
le quotient ayant une nouvelle fois un sens. Le développement
asymptotique~\eqref{eq20211115082025} de la Hessienne s'écrit alors, en tenant
compte de l'Éq.~\eqref{eq20220217164528}
\begin{eqnarray}
  \notag
  \mathcal E_{,uu}[u(\eta ), \lambda (\eta ); v_1, v_1]
  &=&\tfrac12\eta ²\bigl[\mathcal E_{1111}(\lambda _0)+2\mathcal E_3(\lambda _0; v_1, v_1, u_2)+2\lambda _2\dot{\mathcal E}_{11}(\lambda _0)\bigr]+o(\eta ²)\\
  &=&\tfrac5{12}\eta ²\mathcal E_{1111}(\lambda _0)+o(\eta ²).
\end{eqnarray}

\appendix

\section{Propriétés des formes bilinéaires symétriques, positives}

Soit \(\mathcal B\) une forme bilinéaire symétrique, positive sur l'espace vectoriel
\(V\). On définit son noyau \(\ker\mathcal B\) de la façon suivante
\begin{equation}
  \ker\mathcal B=\{u\in V|\mathcal B(u, u)=0\}.
\end{equation}

Le noyau \(\ker\mathcal B\) d'une forme bilinéaire, symétrique, positive \(\mathcal B\) sur
l'espace vectoriel \(V\) est un sous-espace vectoriel de \(V\).


  Soient \(u, v\in\ker\mathcal B\), \(\alpha\in\mathbb R \) et \(w=u+\alpha v\). Montrons que \(w\in\ker\mathcal B\). Il
  suffit d'évaluer \(\mathcal B(w, w)\)
  \begin{equation}
    \mathcal B(w, w)=\mathcal B(u+\alpha v, u+\alpha v)=\mathcal B(u, u)+2\alpha\mathcal B(u, v)+\alpha²\mathcal B(v, v),
  \end{equation}
  où l'on a tenu compte de la symétrie de \(\mathcal B\) pour écrire que
  \(\mathcal B(u, v)=\mathcal B(v, u)\). Comme \(u, v\in\ker\mathcal B\), le premier et le dernier terme
  sont nuls, soit \(\mathcal B(w, w)=2\alpha\mathcal B(u, v)\). La forme bilinéaire étant positive,
  cette grandeur est positive, \emph{quelle que soit la valeur de \(\alpha\in\mathbb R \)}. On
  en déduit donc que \(\mathcal B(u, v)=0\), puis que \(\mathcal B(w, w)=0\).

Soit \(u\in V\). Alors
\begin{equation}
  u\in\ker\mathcal B\quad\text{ssi}\quad\text{pour tout }v\in V, \mathcal B(u, v)=0.
\end{equation}
  Soient \(u\in\ker\mathcal B\), \(v\in V\) et \(\alpha\in\mathbb R \). Comme précédemment, on écrit que
  \(\mathcal B(w, w)\geq0\), avec \(w=\alpha u+v\)
  \begin{equation}
    \mathcal B(w, w)=2\alpha\mathcal B(u, v)+\mathcal B(v, v)\geq0,
  \end{equation}
  où l'on a tenu compte de ce que \(\mathcal B(u, u)=0\). L'expression précédente,
  affine en \(\alpha\), a un signe constant. Le terme linéaire en \(\alpha\) est donc
  nul, soit \(\mathcal B(u, v)=0\).

  Réciproquement, si \(\mathcal B(u, v)=0\) pour tout \(v\in V\), alors \(\mathcal B(u, u)=0\) (en
  prenant \(v=u\)).

\section{Développements limités le long d'une branche bifurquée du diagramme
  d'équilibre}

\subsection{Principe du calcul}
\label{sec20220107121442}

On pose dans ce qui suit
\begin{eqnarray}
  \label{eq20211112155446}
  \Lambda (\eta )&=&\lambda (\eta )-\lambda _0=\eta \lambda _1+\eta ²\lambda _2+\eta ^3\lambda _3+\cdots,\\
  \label{eq20211112113028}
  U(\eta )&=&u(\eta )-u^*[\lambda (\eta )]=\eta u_1+\eta ²u_2+\eta ^3u_3+\cdots.
\end{eqnarray}

On considère une quantité \(\mathcal{F}\), fonction de \(u\) et \(\lambda \)~: \(\mathcal{F}(u,
\lambda )\). Cette fonctionnelle est évaluée le long de la branche bifurquée. En
d'autres termes, on considère
\begin{equation}
  f(\eta )=F\bigl(u^*[\lambda _0+\Lambda (\eta )]+U(\eta ), \lambda _0+\Lambda (\eta )\bigr).
\end{equation}

On souhaite établir un développement limité de \(f\) au voisinage de \(\eta =0\),
ce qui conduit à calculer les dérivées successives de \(f\) en \(\eta =0\), puisque
\begin{equation}
  f(\eta )=f(0)+\eta  f'(0)+\tfrac12\eta ²f''(0)+\cdots
\end{equation}

Pour calculer ces dérivées, il sera commode d'introduire la fonction auxiliaire
\(F\)
\begin{equation}
  F(\eta , \lambda )=\mathcal{F}[u^*(\lambda )+U(\eta ), \lambda ],
\end{equation}
dans laquelle les variables \(\lambda \) et \(\eta \) sont provisoirement considérées
comme indépendantes. On a
\begin{equation}
  f(\eta )=F[\eta , \lambda _0+\Lambda (\eta )],
\end{equation}
d'où l'on déduit successivement que
\begin{eqnarray}
  \label{eq20211112162417}
  f'(\eta )&=&\partial _\eta F+\Lambda '\partial _\lambda F,\\
  \label{eq20211112165810}
  f''(\eta )&=&\partial _{\eta \eta }²F+2\Lambda '\partial _{\eta \lambda }²F+\Lambda '^2\partial _{\lambda \lambda }²F+\Lambda ''\partial _\lambda  F,\\
  \notag
  f'''(\eta )&=&\partial _{\eta \eta \eta }^3F+3\Lambda '\partial _{\eta \eta \lambda }^3F+3\Lambda '^2\partial _{\eta \lambda \lambda }^3F+\Lambda '^3\partial _{\lambda \lambda \lambda }^3F\\
  \label{eq20211112173223}
       &&+3\Lambda ''\partial _{\eta \lambda }²F+3\Lambda '\Lambda ''\partial _{\lambda \lambda }²F+\Lambda '''\partial _\lambda F,\\
  \notag
  f''''(\eta )&=&\partial _{\eta \eta \eta \eta }^4F+4\Lambda '\partial _{\eta \eta \eta \lambda }^4F+6\Lambda '^2\partial _{\eta \eta \lambda \lambda }^4F+4\Lambda '^3\partial _{\eta \lambda \lambda \lambda }^4F\\
  \notag
       &&+\Lambda '^4\partial _{\lambda \lambda \lambda \lambda }^4F+6\Lambda ''\partial _{\eta \eta \lambda }^3F+12\Lambda '\Lambda ''\partial _{\eta \lambda \lambda }^3F+6\Lambda '^2\Lambda ''\partial _{\lambda \lambda \lambda }^3F\\
       &&+4\Lambda '''\partial _{\eta \lambda }²F+\bigl(3\Lambda ''^2+4\Lambda '\Lambda '''\bigr)\partial _{\lambda \lambda }²F+\Lambda ''''\partial _\lambda F
\end{eqnarray}
où \(\Lambda \) et ses dérivées sont évaluées en \(\eta \), tandis que \(F\) et ses
dérivées partielles sont évaluées en \([\eta , \lambda _0+\Lambda (\eta )]\). En \(\eta =0\), les
relations précédentes s'écrivent
\begin{eqnarray}
  \label{eq20220107060454}
  f'(0)&=&\partial _\eta  F+\lambda _1\partial _\lambda  F,\\
  \label{eq20220107124311}
  f''(0)&=&\partial _{\eta \eta }²F+2\lambda _1\partial _{\eta \lambda }²F+2\lambda _2\partial _\lambda  F+\lambda _1²\partial _{\lambda \lambda }²F,\\
  \notag
  f'''(0)&=&\partial _{\eta \eta \eta }^3F+3\lambda _1\partial _{\eta \eta \lambda }^3F+3\lambda _1²\partial _{\eta \lambda \lambda }^3F+\lambda _1^3\partial _{\lambda \lambda \lambda }^3F\\
  \label{eq20220107060500}
       &&+6\lambda _2\partial _{\eta \lambda }²F+6\lambda _1\lambda _2\partial _{\lambda \lambda }²F+6\lambda _3\partial _\lambda F,\\
  \notag
  f''''(0)&=&\partial _{\eta \eta \eta \eta }^4F+4\lambda _1\partial _{\eta \eta \eta \lambda }^4F+6\lambda _1²\partial _{\eta \eta \lambda \lambda }^4F+4\lambda _1^3\partial _{\eta \lambda \lambda \lambda }^4F\\
  \notag
       &&+\lambda _1^4\partial _{\lambda \lambda \lambda \lambda }^4F+12\lambda _2\partial _{\eta \eta \lambda }^3F+24\lambda _1\lambda _2\partial _{\eta \lambda \lambda }^3F+12\lambda _1²\lambda _2\partial _{\lambda \lambda \lambda }^3F\\
       &&+24\lambda _3\partial _{\eta \lambda }²F+\bigl(12\lambda _2²+24\lambda _1\lambda _3\bigr)\partial _{\lambda \lambda }²F+24\lambda _4\partial _\lambda F
\end{eqnarray}
où \(F\) et ses dérivées sont évaluées en \((0, \lambda _0)\).

\subsection{Développement limité du résidu}
\label{sec20211112182000}

On cherche un développement limité du résidu (c'est-à-dire de la première
variation de l'énergie). La fonction test \(\hat{u}\in U\) étant fixée, la
méthode précédente est donc appliquée avec
\begin{equation}
  \label{eq20220107054629}
  f(\eta )=\mathcal E_{,u}[u(\eta ), \lambda (\eta );\hat{u}]
  \quad\text{et}\quad
  F(\eta , \lambda )=\mathcal E_{,u}[u^*(\lambda )+U(\eta ), \lambda ; \hat{u}].
\end{equation}

On remarque tout d'abord que \(F(0, \lambda )=\mathcal E_{,u}[u^*(\lambda ), \lambda ; \hat{u}]=0\), puisque
\(u^*(\lambda )\) est un point d'équilibre. En dérivant par rapport à \(\lambda \), on obtient
\begin{equation}
  \label{eq20211112164240}
  \frac{\partial ^kF}{\partial \lambda ^k}(0, \lambda )=0.
\end{equation}

En dérivant une première fois l'expression~\eqref{eq20220107054629} de \(F\),
on obtient
\begin{eqnarray}
  \partial _\eta F(\eta , \lambda )
  &=&\mathcal E_{,uu}[u^*(\lambda )+U(\eta ), \lambda ; U'(\eta ), \hat{u}],\\
  \notag
  \partial _{\eta \eta }²F(\eta , \lambda )
  &=&\mathcal E_{,uuu}[u^*(\lambda )+U(\eta ), \lambda ; U'(\eta ), U'(\eta ), \hat{u}]\\
  &&+\mathcal E_{,uu}[u^*(\lambda )+U(\eta ), \lambda ; U''(\eta ), \hat{u}],\\
  \notag
  \partial _{\eta \eta \eta }^3F(\eta , \lambda )
  &=&\mathcal E_{,uuuu}[u^*(\lambda )+U(\eta ), \lambda ;U'(\eta ), U'(\eta ), U'(\eta ), \hat{u}]\\
  \notag
  &&+3\mathcal E_{,uuu}[u^*(\lambda )+U(\eta ), \lambda ;U'(\eta ), U''(\eta ), \hat{u}]\\
  &&+\mathcal E_{,uu}[u^*(\lambda )+U(\eta ), \lambda ;U'''(\eta ), \hat{u}],
\end{eqnarray}
soit, en \(\eta =0\)
\begin{eqnarray}
  \partial _\eta  F(0, \lambda )
  &=&\mathcal E_2(\lambda ; u_1, \hat{u}),\\
  \partial _{\eta \eta }²F(0, \lambda )
  &=&\mathcal E_3(\lambda ; u_1, u_1, \hat{u})+2\mathcal E_2(\lambda ; u_2, \hat{u}),\\
  \partial _{\eta \eta \eta }^3F(0, \lambda )
  &=&\mathcal E_4(\lambda ; u_1, u_1, u_1, \hat{u})+6\mathcal E_3(\lambda ; u_1, u_2, \hat{u})+6\mathcal E_2(\lambda ; u_3, \hat{u}).
\end{eqnarray}

Les dérivées croisées de \(F\) en \((0, \lambda )\) s'obtiennent par simple dérivation
des relations précédentes par rapport à \(\lambda \)
\begin{eqnarray}
  \partial _{\eta \lambda }²F(0, \lambda )&=&\dot{\mathcal E_2}(\lambda ; u_1, \hat{u}),\\
  \partial _{\eta \eta \lambda }^3F(0, \lambda )&=&\dot{\mathcal E_3}(\lambda ; u_1, u_1, \hat{u})+2\dot{\mathcal E_2}(\lambda ; u_2, \hat{u}),\\
  \partial _{\eta \lambda \lambda }^3F(0, \lambda )&=&\ddot{\mathcal E_2}(\lambda ; u_1, \hat{u}).
\end{eqnarray}

En insérant les résultats précédentes dans les relations
générales~\eqref{eq20220107060454}--\eqref{eq20220107060500}, on trouve
finalement les expressions suivantes des dérivées successives de \(f\) en
\(\eta =0\)
\begin{eqnarray}
  f'(0)
  &=&\mathcal E_2(\lambda _0; u_1, \hat{u}),\\
  f''(0)
  &=&\mathcal E_3(\lambda _0; u_1, u_1, \hat{u})
    +2\lambda _1\dot{\mathcal E_2}(\lambda _0; u_1, \hat{u})
    +2\mathcal E_2(\lambda _0; u_2, \hat{u}),\\
  \notag
  f'''(0)
  &=&\mathcal E_4(\lambda _0; u_1, u_1, u_1, \hat{u})
      +6\mathcal E_3(\lambda _0; u_1, u_2, \hat{u})
    +6\mathcal E_2(\lambda _0; u_3, \hat{u})\\
  \notag
  &&+3\lambda _1\bigl[\dot{\mathcal E_3}(\lambda _0; u_1, u_1, \hat{u})
     +2\dot{\mathcal E_2}(\lambda _0; u_2, \hat{u})\bigr]\\
  &&+3\lambda _1²\ddot{\mathcal E_2}(\lambda _0; u_1, \hat{u})
     +6\lambda _2\dot{\mathcal E_2}(\lambda _0; u_1, \hat{u}).
\end{eqnarray}

On en déduit finalement le développement limité à l'ordre 3 en \(\eta \) du résidu
\begin{eqnarray}
    \mathcal E_{,u}[u(\eta ), \lambda (\eta )]
    &=&\eta \mathcal E_2(\lambda _0; u_1, \hat{u})
    +\tfrac12\eta ²\bigl[\mathcal E_3(\lambda _0; u_1, u_1, \hat{u})\\
    \notag
    &&+2\lambda _1\dot{\mathcal E_2}(\lambda _0; u_1, \hat{u})
    +2\mathcal E_2(\lambda _0; u_2, \hat{u})\bigr]\\
    \notag
    &&+\tfrac16\eta ^3\bigl\{
    \mathcal E_4(\lambda _0; u_1, u_1, u_1, \hat{u})
    +6\mathcal E_3(\lambda _0; u_1, u_2, \hat{u})\\
    \notag
    &&+6\mathcal E_2(\lambda _0; u_3, \hat{u})
    +3\lambda _1\bigl[\dot{\mathcal E_3}(\lambda _0; u_1, u_1, \hat{u})\\
    \notag
    &&+2\dot{\mathcal E_2}(\lambda _0; u_2, \hat{u})\bigr]
    +3\lambda _1²\ddot{\mathcal E_2}(\lambda _0; u_1, \hat{u})\\
    \label{eq20220107080901}
    &&+6\lambda _2\dot{\mathcal E_2}(\lambda _0; u_1, \hat{u})\bigr\}
    +o(\eta ^3).
\end{eqnarray}

\subsection{Développement limité de l'énergie}
\label{sec20220121172919}

On s'intéresse ici à l'écart d'énergie, pour un chargement \(\lambda \) donné, entre
la branche bifurquée et la branche fondamentale, soit
\begin{equation}
  F(\eta , \lambda ) = \mathcal E[u^*(\lambda )+U(\eta ), \lambda ]-\mathcal E[u^*(\lambda ), \lambda ]
\end{equation}
et
\begin{equation}
  f(\eta ) = F[\eta , \lambda _0+\Lambda (\eta )].
\end{equation}

On observe tout d'abord que \(F(0, \lambda )=0\) pour tout \(\lambda \), donc
\begin{equation}
  \frac{\partial ^kF}{\partial \lambda ^k}(0, \lambda )=0\quad(k\geq0),
\end{equation}
tandis que les dérivées de \(F\) par rapport à \(\eta \) s'écrivent
\begin{eqnarray}
  \partial _\eta F(\eta , \lambda )&=&\mathcal E_{,u}[u^*(\lambda )+U(\eta ), \lambda ; U'(\eta )],\\
  \notag
  \partial _{\eta \eta }²F(\eta , \lambda )&=&\mathcal E_{,uu}[u^*(\lambda )+U(\eta ), \lambda ; U'(\eta ), U'(\eta )]\\
               &&+\mathcal E_{,u}[u^*(\lambda )+U(\eta ), \lambda ; U''(\eta )],\\
  \notag
  \partial _{\eta \eta \eta }^3F(\eta , \lambda )&=&\mathcal E_{,uuu}[u^*(\lambda )+U(\eta ), \lambda ; U'(\eta ), U'(\eta ), U'(\eta )]\\
  \notag
            &&+3\mathcal E_{,uu}[u^*(\lambda )+U(\eta ), \lambda ; U'(\eta ), U''(\eta )]\\
               &&+\mathcal E_{,u}[u^*(\lambda )+U(\eta ), \lambda ; U'''(\eta )],\\
  \notag
  \partial _{\eta \eta \eta \eta }^4F(\eta , \lambda )&=&\mathcal E_{,uuuu}[u^*(\lambda )+U(\eta ), \lambda ; U'(\eta ), U'(\eta ), U'(\eta ), U'(\eta )]\\
  \notag
               &&+6\mathcal E_{,uuu}[u^*(\lambda )+U(\eta ), \lambda ; U'(\eta ), U'(\eta ), U''(\eta )]\\
  \notag
               &&+3\mathcal E_{,uu}[u^*(\lambda )+U(\eta ), \lambda ; U''(\eta ), U''(\eta )]\\
  \notag
               &&+3\mathcal E_{,uu}[u^*(\lambda )+U(\eta ), \lambda ; U'(\eta ), U'''(\eta )]\\
               &&+\mathcal E_{,u}[u^*(\lambda )+U(\eta ), \lambda ; U''''(\eta )],
\end{eqnarray}
soit, en \(\eta =0\), en observant que \(\mathcal E_{,u}[u^*(\lambda ), \lambda ]=0\)
\begin{eqnarray}
  \partial _\eta F(0, \lambda )&=&0,\\
  \partial _{\eta \eta }²F(0, \lambda )&=&\mathcal E_2(\lambda ; u_1, u_1),\\
  \partial _{\eta \eta \eta }^3F(0, \lambda )&=&\mathcal E_3(\lambda ;u_1, u_1, u_1)+6\mathcal E_2(\lambda ; u_1, u_2),\\
  \notag
  \partial _{\eta \eta \eta \eta }^4F(\eta , \lambda )&=&\mathcal E_4(\lambda ;u_1, u_1, u_1, u_1)+12\mathcal E_3(\lambda ; u_1, u_1, u_2)\\
            &&+12\mathcal E_2(\lambda ; u_2, u_2)+18\mathcal E_2(\lambda ; u_1, u_3).
\end{eqnarray}

On en déduit que
\begin{eqnarray}
  \partial _{\eta \lambda }²F(0, \lambda )&=&0,\\
  \partial _{\eta \eta \lambda }^3F(0, \lambda )&=&\dot{\mathcal E}_2(\lambda ; u_1, u_1),\\
  \partial _{\eta \lambda \lambda }^3F(0, \lambda )&=&0,\\
  \partial _{\eta \eta \eta \lambda }^4F(0, \lambda )&=&\dot{\mathcal E}_3(\lambda ; u_1, u_1, u_1)+6\dot{\mathcal E}_2(\lambda ; u_1, u_2),\\
  \partial _{\eta \eta \lambda \lambda }^4F(0, \lambda )&=&\ddot{\mathcal E}_2(\lambda ; u_1, u_1),\\
  \partial _{\eta \lambda \lambda \lambda }^4F(0, \lambda )&=&0
\end{eqnarray}
et finalement
\begin{eqnarray}
  f'(0)&=&0,\\
  f''(0)&=&\mathcal E_2(\lambda _0; u_1, u_1),\\
  f'''(0)&=&\mathcal E_3(\lambda _0;u_1, u_1, u_1)+6\mathcal E_2(\lambda _0; u_1, u_2)+3\lambda _1\dot{\mathcal E}_2(\lambda _0; u_1, u_1),\\
  \notag
  f''''(0)&=&\mathcal E_4(\lambda _0;u_1, u_1, u_1, u_1)+12\mathcal E_3(\lambda _0; u_1, u_1, u_2)\\
  \notag
          &&+12\mathcal E_2(\lambda _0; u_2, u_2)+18\mathcal E_2(\lambda _0; u_1, u_3)\\
  \notag
          &&+4\lambda _1\bigl[\dot{\mathcal E}_3(\lambda _0; u_1, u_1, u_1)+6\dot{\mathcal E}_2(\lambda _0; u_1, u_2)\bigr]\\
          &&+6\lambda _1²\ddot{\mathcal E}_2(\lambda _0; u_1, u_1)+12\lambda _2\dot{\mathcal E}_2(\lambda _0; u_1, u_1)
\end{eqnarray}

On en déduit finalement le développement limité de l'énergie~\eqref{eq20220121172753}.

\subsection{Développement limité de la hessienne}
\label{sec20211115081016}

On cherche maintenant un développement limité de la hessienne de l'énergie. Les
fonctions test \(\hat{u}, \hat{v}\in U\) étant fixées, on applique la méthode du
§\ref{sec20220107121442} à la fonction \(f(\eta )=F[\eta , \lambda _0+\Lambda (\eta )]\), avec
\begin{equation}
  F(\eta , \lambda )=\mathcal E_{,uu}[u^*(\lambda )+U(\eta ), \lambda ; \hat{u}, \hat{v}].
\end{equation}

On observe tout d'abord que \(F(0, \lambda )=\mathcal E_2(\lambda ; \hat{u}, \hat{v})\), soit, en
dérivant par rapport à \(\lambda \)
\begin{equation}
  \partial _\lambda  F(0, \lambda )=\dot{\mathcal E_2}(\lambda ; \hat{u}, \hat{v})
  \quad\text{et}\quad
  \partial _{\lambda \lambda }²F(0, \lambda )=\ddot{\mathcal E_2}(\lambda ; \hat{u}, \hat{v}).
\end{equation}

On trouve de même successivement
\begin{eqnarray}
  \partial _\eta F(\eta , \lambda )&=&\mathcal E_{,uuu}[u^\ast(\lambda )+U(\eta ), \lambda ; U'(\eta ), \hat{u}, \hat{v}],\\
  \notag
  \partial _{\eta \eta }²F(\eta , \lambda )&=&\mathcal E_{,uuuu}[u^\ast(\lambda )+U(\eta ), \lambda ; U'(\eta ), U'(\eta ), \hat{u}, \hat{v}]\\
            &&+\mathcal E_{,uuu}[u^\ast(\lambda )+U(\eta ), \lambda ; U''(\eta ), \hat{u}, \hat{v}],
\end{eqnarray}
soit, en \(\eta =0\)
\begin{eqnarray}
  \partial _\eta F(0, \lambda )&=&\mathcal E_3(\lambda ; u_1, \hat{u}, \hat{v}),\\
  \partial _{\eta \eta }²F(0, \lambda )&=&\mathcal E_4(\lambda ; u_1, u_1, \hat{u}, \hat{v})+2\mathcal E_3(\lambda ; u_2, \hat{u}, \hat{v}),
\end{eqnarray}
et en dérivant cette fois par rapport à \(\lambda \)
\begin{equation}
  \partial _{\eta \lambda }²F(0, \lambda )=\dot{\mathcal E_3}(\lambda ; u_1, \hat{u}, \hat{v}).
\end{equation}

En insérant les résultats précédents dans les
expressions~\eqref{eq20220107060454} et \eqref{eq20220107124311}, on trouve
\begin{eqnarray}
  f'(0)&=&\mathcal E_3(\lambda _0; u_1, \hat{u}, \hat{v})
            +\lambda _1\dot{\mathcal E_2}(\lambda _0; \hat{u}, \hat{v}),\\
  \notag
  f''(0)&=&\mathcal E_4(\lambda _0; u_1, u_1, \hat{u}, \hat{v})
             +2\lambda _1\dot{\mathcal E_3}(\lambda _0; u_1, \hat{u}, \hat{v})
             +\lambda _1²\ddot{\mathcal E_2}(\lambda _0; \hat{u}, \hat{v})\\
          &&+2\mathcal E_3(\lambda _0; u_2, \hat{u}, \hat{v})
            +2\lambda _2\dot{\mathcal E_2}(\lambda _0; \hat{u}, \hat{v}).
\end{eqnarray}
qui conduisent finalement au développement limité suivant, à l'ordre 2 en \(\eta \)
\begin{eqnarray}
  \notag
  \mathcal E_{,uu}[u(\eta ), \lambda (\eta ); \hat{u}, \hat{v}]
  &=&\mathcal E_2(\lambda _0; \hat{u}, \hat{v})+\eta \bigl[\mathcal E_3(\lambda _0; u_1, \hat{u}, \hat{v})\\
  \notag
  &&+\lambda _1\dot{\mathcal E_2}(\lambda _0; \hat{u}, \hat{v})\bigr]+\tfrac12\eta ²\bigl[\mathcal E_4(\lambda _0; u_1, u_1, \hat{u}, \hat{v})\\
  \notag
  &&+2\lambda _1\dot{\mathcal E_3}(\lambda _0; u_1, \hat{u}, \hat{v})+\lambda _1²\ddot{\mathcal E_2}(\lambda _0; \hat{u}, \hat{v})\\
  &&+2\mathcal E_3(\lambda _0; u_2, \hat{u}, \hat{v})+2\lambda _2\dot{\mathcal E_2}^*(\lambda _0; \hat{u}, \hat{v})\bigr]+o(\eta ²).
\end{eqnarray}

\subsection{Développement limité des valeurs propres et vecteurs propres de la Hessienne}

On cherche les vecteurs propres \(v\in V\) et valeurs propres \(\alpha\in\mathbb R \) de la
Hessienne
\begin{equation}
  \label{eq20211115082122}
  \mathcal E_{,uu}[u(\eta ), \lambda (\eta )](v, \hat{u})=\alpha\langle v, \hat{u}\rangle \quad\text{pour tout}\quad\hat{u}\in V.
\end{equation}

On cherche les développements limités à l'ordre 1 en \(\eta \) de \(v\) et \(\alpha\)
\begin{equation}
  \label{eq20211115082037}
  v = v_0+\eta  v_1+o(\eta )\quad\text{et}\quad\alpha=\alpha_0+\eta \alpha_1+o(\eta )
\end{equation}

Les développements limités~\eqref{eq20211115082025} et
\eqref{eq20211115082037} sont insérés dans le
problème~\eqref{eq20211115082122}
\begin{eqnarray}
  \notag
  \mathcal E_{,uu}[u(\eta ), \lambda (\eta )](v, \hat{w})
  &=&\mathcal E_2^*(v_0, \hat{w})+\eta \bigl[\mathcal E_3^*(u_1, v_0, \hat{w})+\lambda _1\dot{\mathcal E_2}^*(v_0, \hat{w})\\
  &&+\mathcal E_2^*(v_1, \hat{w})\bigr]+o(\eta )
\end{eqnarray}

\begin{equation}
  \alpha\langle  v, \hat{w}\rangle =\alpha_0\langle v_0, \hat{w}\rangle +\eta \bigl(\alpha_1\langle  v_0, \hat{w}\rangle +\alpha_0\langle  v_1, \hat{w}\rangle \bigr)+o(\eta ).
\end{equation}

On obtient le problème variationnel d'ordre 0
\begin{equation}
  \mathcal E_2^*(v_0, \hat{w})=\alpha_0\langle v_0, \hat{w}\rangle \quad\text{pour tout}\quad\hat{w}\in V,
\end{equation}
qui montre que \(v_0\) est le vecteur propre de \(\mathcal E_2^*\) associé à la valeur
propre \(\alpha_0\). Si \(\alpha_0\neq 0\), \(\mathcal E_2^*\) étant positive par hypothèse, on a
nécessairement \(\alpha_0>0\), et la valeur propre de la hessienne est positive.

On considère maintenant le cas où \(\alpha_0\), c'est-à-dire que \(v_0\in\ker\mathcal E_2^*\). En
prenant \(\hat{w}\in\ker\mathcal E_2^*\), on obtient alors le problème variationnel d'ordre
1
\begin{equation}
  \mathcal E_3^*(u_1, v_0, \hat{w})+\lambda _1\dot{\mathcal E_2}^*(v_0, \hat{w})=\alpha_1\langle v_0, \hat{w}\rangle
  \quad\text{pour tout}\quad\hat{w}\in\ker\mathcal E_2^*.
\end{equation}

En posant \(u_1=\xi _ia_i\) et \(v_0=\chi _ja_j\), on obtient l'équation
\begin{equation}
  \bigl(\mathcal E_{3,ijk}^*\xi _k+\lambda _1\dot{\mathcal E}_{2, ij}^*\bigr)\chi _j=\alpha_1\chi _i,
\end{equation}
qui est un problème aux valeurs propres pour la matrice symétrique
\((\mathcal E_{3, ijk}^*\xi _k+\lambda _1\dot{\mathcal E}_{2,ij}^*)_{1\leq i, j\leq m}\)

% \printbibliography
\end{document}

%%% Local Variables:
%%% coding: utf-8
%%% fill-column: 79
%%% mode: latex
%%% TeX-engine: xetex
%%% TeX-master: t
%%% End:
