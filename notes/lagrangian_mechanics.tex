\newcommand{\sbtitle}{Notes relatives à la mécanique lagrangienne}
\newcommand{\sbauthor}{Sébastien Brisard}
\newcommand{\sbaddress}{Univ Gustave Eiffel, Ecole des Ponts, IFSTTAR, CNRS, Navier, F-77454 Marne-la-Vall\'ee, France}
\newcommand{\sbsubject}{Note bibliographique}

\documentclass[12pt, final]{amsart}

\usepackage{polyglossia}
\setdefaultlanguage{french}

\usepackage{amsfonts}
\usepackage{amsmath}
\usepackage{amssymb}

\usepackage[backend=biber,bibencoding=utf8,doi=false,giveninits=true,isbn=false,maxnames=10,minnames=5,sortcites=true,style=authoryear,texencoding=utf8,url=false]{biblatex}
\addbibresource{stab.bib}

\usepackage[breaklinks=true, colorlinks=true, pdftitle={\sbtitle}, pdfauthor={\sbauthor}, pdfsubject={\sbsubject}, urlcolor=blue]{hyperref}

\usepackage[color={1 1 0}]{pdfcomment}

\usepackage{unicode-math}
\setmainfont{XITS}
\setmathfont{XITS Math}

\newcommand{\D}{\mathup{d}}
\newcommand{\mat}[1]{\mathsf{#1}}
\newcommand{\T}{\mathsf{T}}

\begin{document}
\title{\sbtitle}
\author{\sbauthor}
\address{\sbaddress}
\email{sebastien.brisard@univ-eiffel}

\begin{abstract}
  blabla
\end{abstract}

\maketitle

On considère un système mécanique dont la configuration actuelle est entièrement
définie par un nombre fini de paramètres, \(q=[q_1, \ldots, q_n]^\T\)
\begin{equation}
  \vec x=\vec x(\vec X, q, t),
\end{equation}
où \(\vec x\) désigne la position à l'instant \(t\) du point matériel situé en
\(\vec X\) à \(t=0\). La vitesse \(\vec v\) et l'accélération \(\vec a\) de ce
point matériel sont données par les relations suivantes
\begin{equation}
  \vec v=\frac{\partial\vec x}{\partial q_i}\,\dot{q}_i
  +\frac{\partial\vec x}{\partial t}
  \quad\text{et}\quad
  \vec a=\frac{\partial\vec x}{\partial q_i}\,\ddot{q}_i
  +\frac{\partial^2\vec x}{\partial q_i\,\partial q_j}\,\dot{q}_i\,\dot{q}_j
  +2\,\frac{\partial^2\vec x}{\partial q_i\,\partial t}\,\dot{q}_i
  +\frac{\partial^2\vec x}{\partial t^2}.
\end{equation}

L'énergie cinétique est obtenue en intégrant sur le système entier la densité
massique \(\vec v\cdot\vec v\). Noter que dans l'intégrale ci-dessous,
\(\D m\) est constant (conservation de la masse)
\begin{equation}
  \Kappa=\tfrac12\,\int\vec v\cdot\vec v\,\D m
  =\tfrac12\bigl(\dot{q}^\T\cdot M\cdot\dot{q}+H\cdot\dot{q}+K\bigr)
\end{equation}
avec
\begin{equation}
  M_{ij}=\int\frac{\partial\vec x}{\partial q_i}
  \cdot\frac{\partial\vec x}{\partial q_j}\,\D m,\quad
  H_i=\int\frac{\partial\vec x}{\partial q_i}
  \cdot\frac{\partial\vec x}{\partial t}\,\D m
  \quad\text{et}\quad
  K=\int\frac{\partial\vec x}{\partial t}
  \cdot\frac{\partial\vec x}{\partial t}\,\D m
\end{equation}

\end{document}

%%% Local Variables:
%%% coding: utf-8
%%% fill-column: 80
%%% mode: latex
%%% TeX-engine: xetex
%%% TeX-master: t
%%% End:
